%TODO 等的使用
%%TODO 内容问题
%%HACK 粗鄙技巧
%%CODE 代码改进

%TODO——注意事项

\documentclass[oneside]{book}

%TODO——宏包
%%方便定义命令
%\usepackage{suffix}

%%页面尺寸
\usepackage{geometry}
	\geometry{
		a4paper,
		left = 2.54 cm, right = 2.54 cm, top = 3.18 cm, bottom = 3.18 cm,
		headheight = 3 cm
	}

%%设置标题
\usepackage{titlesec}

%%交叉引用,超链接等
\usepackage[hyperindex]{hyperref}
	\hypersetup{
		%  PDF 书签
		bookmarksopen = true,
		bookmarksopenlevel = 1,
		bookmarksnumbered = true,
		%  PDF 标题作者
%		pdftitle = {},
%		pdfauthor = {},
		%脚注
%		hyperfootnotes = false,
		%目录  只引用页码
		linktoc = page,
		%超链接颜色
		colorlinks,
		linkcolor = {red!60!black},
		citecolor = {green!50!black},
		urlcolor = {blue!70!black}
	}

%%常规字体选择
\usepackage[no-math]{fontspec}
	\setmainfont[
		Extension = .otf,
		BoldFont = xits-bold,
		ItalicFont = xits-italic,
		BoldItalicFont = xits-bolditalic,
		SlantedFont = xits-italic
	]{xits-regular}
%	\setsansfont[
%		Extension = .otf,
%		BoldFont = texgyreheros-bold,
%		ItalicFont = texgyreheros-italic,
%		SlantedFont = texgyreheros-italic
%	]{texgyreheros-regular}
%	\setmonofont[
%		Extension = .otf,
%		BoldFont = texgyrecursor-bold,
%		ItalicFont = texgyrecursor-italic,
%		SlantedFont = texgyrecursor-italic,
%		Ligatures = NoCommon
%	]{texgyrecursor-regular}
%	\newfontfamily{\SourceSans}{Source Sans Pro}
%	\newfontfamily{\Songti}{方正书宋_GBK}

%%AMS 数学支持
\usepackage{amsmath}
\usepackage{amssymb}

%%Pi 符号
\usepackage{pifont}
	% 正确 √
	\newcommand{\cmark}{\ding{51}}
	% 错误 ×
	\newcommand{\xmark}{\ding{55}}

%%调用 Unicode OpenType 数学字体
\usepackage{unicode-math}
	\setmathfont[
		math-style = ISO,
		bold-style = ISO
	]{xits-math.otf}
%加粗使用 \symbf{}
%直立希腊字母:\uppi 等

%%中文文字处理
\usepackage[UTF8, heading = true]{ctex}
	\pagestyle{plain}
\usepackage{xeCJK}
	\setCJKmainfont[
		BoldFont = 方正黑体_GBK,
		ItalicFont = 方正楷体_GBK,
		Mapping = fullwidth-stop
	]{方正书宋_GBK}
	\setCJKsansfont[
		BoldFont = 方正黑体_GBK,
		ItalicFont = 方正黑体_GBK,
		Mapping = fullwidth-stop
	]{方正黑体_GBK}
%	\setCJKsansfont[
%		BoldFont = 思源黑体 CN Regular,
%		ItalicFont = 思源黑体 CN Regular,
%		Mapping = fullwidth-stop
%	]{思源黑体 CN Regular}
	\setCJKmonofont[
		BoldFont = 方正仿宋_GBK,
		ItalicFont = 方正楷体_GBK,
		Mapping = fullwidth-stop
	]{方正仿宋_GBK}
	\setCJKfamilyfont{宋体}{方正书宋_GBK}
	\setCJKfamilyfont{楷体}{方正楷体_GBK}
	\setCJKfamilyfont{黑体}{方正黑体_GBK}
	\setCJKfamilyfont{仿宋}{方正仿宋_GBK}

%%脚注增强版
\usepackage[stable, perpage, bottom]{footmisc}
	%需要调用 pifont 宏包
	%衬线加圈阳文数字:\ding{172}~\ding{181} (1~10)
	%无衬线加圈阳文数字:\ding{192}~\ding{201} (1~10)
	\renewcommand{\thefootnote}{\ding{\numexpr191+\value{footnote} } }
	%TODO:20160705 脚注不用上标
	%HACK:20160709 见 http://tex.stackexchange.com/questions/19844/how-to-set-superscript-footnote-mark-in-the-text-body-but-normalsized-in-the-foo
	\makeatletter
	\newlength{\fnBreite}
	\renewcommand{\@makefntext}[1]{%
		\settowidth{\fnBreite}{\footnotesize\@thefnmark.i}%
		\protect\footnotesize\upshape%
		\setlength{\@tempdima}{\columnwidth}%
		\addtolength{\@tempdima}{-\fnBreite}%
		\makebox[\fnBreite][l]{\@thefnmark\phantom{  }}%
		\parbox[t]{\@tempdima}%
		{\everypar{\hspace*{1em}}\hspace*{-1em}\upshape#1}%
	}
	\makeatother

%%颜色
\usepackage[svgnames]{xcolor}

%%图形
%\usepackage{graphicx}

%%绘图
%\usepackage{tikz}
%	\usepgflibrary{arrows.meta}
%	\tikzset{>=Stealth}

%%数学工具
\usepackage{mathtools}

%%强调公式
\usepackage[ntheorem]{empheq}

%%张量
\usepackage{tensor}

%%物理、数学符号
\usepackage{physics}

%%单位
%\usepackage{siunitx}

%%定制列表环境
%\usepackage{enumitem}
%	%定义带缩进的左对齐格式
%	%前一个参数 2.1 em 确定标签的位置
%	%后一个参数 1.2 em 确定标签与文字的距离
%	%HACK:20160709 可能与字体、字号、标签内容有关
%	\SetLabelAlign{leftalignwithindent}{\hspace{2.1 em} \makebox[1.2 em][l]{#1}}

%%定理类环境
\usepackage[thmmarks, amsmath]{ntheorem}
	\theoremstyle{nonumberplain} %带编号
	\theoremheaderfont{\bfseries}
	\theorembodyfont{\normalfont}
	\theoremsymbol{\mbox{$\Box$}}
		\newtheorem{myProof}{证明:}

%%索引

%TODO——环境
%%定制列表(编号)
%\newenvironment{myEnumerate}
%	{\begin{enumerate} [
%%		label=\bullet, %标签样式:圆点
%		align = leftalignwithindent, %对齐(见上)
%		listparindent = 2 em, %条目段落缩进
%		leftmargin = 0 pt, %文字左边距
%		topsep = 0 pt,
%		itemsep = 0 pt,
%		parsep = 0 pt
%	]}
%	{\end{enumerate}}
%子公式
\newenvironment{mySubEq}
	{\subequations \renewcommand{\theequation}%
		{\theparentequation-\alph{equation}}}%
	{\endsubequations%
		\ignorespacesafterend}
%大括号公式
\newenvironment{braceEq}[1][align]
	{\mySubEq%
		\setkeys{EmphEqEnv}{#1}%
		\setkeys{EmphEqOpt}{left = \empheqlbrace}%
		\EmphEqMainEnv}%
	{\endEmphEqMainEnv \endmySubEq}

%TODO——命令
%% 空行
\newcommand{\blankline}{\mbox{}}
%% 强调
\newcommand{\emphA}[1]{{\bfseries #1}}
\newcommand{\emphB}[1]{{\itshape #1}}
%% 数集黑板粗体
\newcommand{\setbb}[1]{\symbb{#1}}
\newcommand{\realR}{\setbb{R}}
%% 哥特体
%\newcommand{\gothic}[1]{\symfrak{#1}}
%\newcommand{\domainB}{\gothic{B}} %邻域
%\newcommand{\domainD}{\gothic{D}} %定义域

%% 向量
\newcommand{\V}[1]{\symbf{#1}}
%% 张量(Tensor)
\newcommand{\Tens}[1]{\symbf{#1}}
%% 矩阵(Matrix)
\newcommand{\Mat}[1]{\symbf{#1}}

%% Kronecker Delta
\newcommand{\kroneckerDelta}[2]{\delta^{#1}_{#2}}
%% 内积括号形式(Inner Product Braket)
\newcommand{\ipb}[3]{\qty(#1,\,#2)_{#3}}
%% 定义等号
%\newcommand{\defeq}{\coloneq}
\newcommand{\defeq}{\triangleq}

%% m 维实空间
\newcommand{\Rm}{\realR^m}
%% 以 m 维 Euclid 空间为底空间的 r 阶张量全体
\newcommand{\Tensors}[1]{\symcal{J}^{#1}\qty(\Rm)}

%% 公式用标点、文字
\newcommand{\comma}{\text{,}}
\newcommand{\fullstop}{\text{.}}
\newcommand{\semicomma}{\text{;}}
%\newcommand{\const}{\text{const.}}

%TODO——标题页
\title{
	\vspace{-4 cm} \color{Sienna} \Huge 张量分析
}
\author{
	\CJKfamily{楷体} \color{DarkRed} \Large 复旦大学\phantom{空格}谢锡麟
}
\date{
	\CJKfamily{楷体} \color{Goldenrod} \Large \today
}


\begin{document}
%	\frontmatter
%	\maketitle
%	
%%	\tableofcontents
%	
%	\mainmatter
	\chapter{张量的定义及表示}
		\section{对偶基,度量}
\subsection{对偶基}
	$\Rm$ 空间中的基可分为两类:指标写在\emphB{下面}的基
	\begin{equation*}
		\qty{\V{g}_i}^m_{i=1} \subset \Rm
	\end{equation*}
	称为\emphA{协变基},指标写在\emphB{上面}的基
	\begin{equation*}
		\qty{\V{g}^i}^m_{i=1} \subset \Rm
	\end{equation*}
	称为\emphA{逆变基}。
	它们满足\emphA{对偶关系}:
	\begin{equation}
		\qty(\V{g}^i,\,\V{g}_j)_{\Rm}=\KroneckerDelta{i}{j}
		=\left\{\begin{aligned}
			1,\quad i&=j \semicomma\\
			0,\quad i&\neq j \fullstop
		\end{aligned}\right.
		\label{eq:对偶关系}
	\end{equation}
	这里的 $\KroneckerDelta{i}{j}$ 是 \emphA{Kronecker δ 函数}。
	
\subsection{度量}
	下面引入\emphA{度量}的概念。其定义为
	\begin{braceEq}
		g_{ij} &\defeq \qty(\V{g}_i,\,\V{g}_j)_{\Rm} \comma 
		\label{eq:度量的定义_协变} \\
		g^{ij} &\defeq \qty(\V{g}^i,\,\V{g}^j)_{\Rm} \fullstop
		\label{eq:度量的定义_逆变}
	\end{braceEq}
	
	下面证明
	\begin{equation}
		g_{ik}\,g^{kj} = \KroneckerDelta{j}{i} \fullstop
		\label{eq:度量之积}
	\end{equation}
	它也可以写成矩阵的形式:
	\begin{equation}
		\qty[g_{ik}]\qty[g^{kj}]
		=\qty[\KroneckerDelta{j}{i}]=\Mat{I}_m \comma
	\end{equation}
	其中的 $\Mat{I}_m$ 是 $m$ 阶单位阵。
	\begin{myProof}
		\begin{equation}
			g_{ik}\,g^{kj}
			=\qty(\V{g}_i,\,\V{g}_k)_{\Rm} \, g^{kj}
			=\qty(\V{g}_i,\,g^{kj}\V{g}_k)_{\Rm}
		\end{equation}
		后文将说明 $g^{kj}\V{g}_k=\V{g}^j$,因此可得
		\begin{equation}
			g_{ik}\,g^{kj}=\qty(\V{g}_i,\,\V{g}^j)_{\Rm}
			=\KroneckerDelta{j}{i} \fullstop
		\end{equation}
		
		要注意的是,这里的指标 $k$ 是\emphB{哑标}。
		根据\emphA{Einstein 求和约定},
		重复指标并且一上一下时,就表示对它求和。后文除非特殊说明,也均是如此。
	\end{myProof}
	
	现在澄清\emphA{基向量转换关系}。第 $i$ 个协变基向量 $\V{g}_i$ 既然是向量,
	就必然可以用协变基或逆变基来表示。
	根据对偶关系式~\eqref{eq:对偶关系} 和度量的定义
	式~\eqref{eq:度量的定义_协变}、\eqref{eq:度量的定义_逆变},可知
	\begin{braceEq}
		\V{g}_i &=\qty(\V{g}_i,\,\V{g}_k)_{\Rm} \,\V{g}^k
		=g_{ik}\,\V{g}^k \comma \label{eq:基向量转换关系1} \\
		\V{g}_i &=\qty(\V{g}_i,\,\V{g}^k)_{\Rm} \,\V{g}_k
		=\KroneckerDelta{k}{i}\V{g}_k \label{eq:基向量转换关系2}
	\end{braceEq}
	以及
	\begin{braceEq}
		\V{g}^i &=\qty(\V{g}^i,\,\V{g}_k)_{\Rm} \,\V{g}^k
		=\KroneckerDelta{i}{k}\,\V{g}^k \comma \label{eq:基向量转换关系3} \\
		\V{g}^i &=\qty(\V{g}^i,\,\V{g}^k)_{\Rm} \,\V{g}_k
		=g^{ik}\V{g}_k \fullstop \label{eq:基向量转换关系4}
	\end{braceEq}
	这四个式子中,式~\eqref{eq:基向量转换关系2} 和
	\eqref{eq:基向量转换关系3} 是平凡的,
	而式~\eqref{eq:基向量转换关系1} 和 \eqref{eq:基向量转换关系4}
	则通过\emphB{度量}建立起了协变基与逆变基之间的关系。
	这就称为基向量转换关系,也可以叫做“指标升降游戏”。
	
\subsection{向量的分量}
	对于任意的向量 $\V{\xi} \in \Rm$,它可以用协变基表示:
	\begin{equation}
		\V{\xi}=\ipb{\V{\xi}}{\V{g}^k}\,\V{g}_k
		=\xi^k \V{g}_k \comma
	\end{equation}
	也可以用逆变基表示:
	\begin{equation}
		\V{\xi}=\ipb{\V{\xi}}{\V{g}_k}\,\V{g}^k
		=\xi_k \V{g}^k \comma
	\end{equation}
	式中,$\xi^k$ 是 $\V{\xi}$ 与第 $k$ 个\emphB{逆变基}做内积的结果,
	称为 $\V{\xi}$ 的第 $k$ 个\emphA{逆变分量};
	而 $\xi_k$ 是 $\V{\xi}$ 与第 $k$ 个\emphB{协变基}做内积的结果,
	称为 $\V{\xi}$ 的第 $k$ 个\emphA{协变分量}。
	
	以后凡是指标在下的(下标),均称为\emphB{协变}某某;
	指标在上的(上标),称为\emphB{逆变}某某。
	
\section{张量的表示}
\subsection{张量的表示与简单张量}\label{subsec:张量的表示与简单张量}
	所谓\emphA{张量},即\emphA{多重线性函数}。
	
	首先用三阶张量举个例子。考虑任意的 $\T{\Phi}\in\Tensors{3}$,
	其中的 $\Tensors{3}$ 表示以 $\Rm$ 为底空间的三阶张量全体。
	所谓三阶(或三重)线性函数,指“吃掉”三个向量之后变成数,
	并且“吃法”具有线性性。
	
	一般地,$r$ 阶张量的定义如下:
	\begin{equation}
		\mdef{\T{\Phi}}{\underbrace{\Rm\times\Rm\times\cdots\times\Rm}_
			{\text{$r$ 个 $\Rm$}}
			\ni\qty{\V{u}_1,\,\V{u}_2,\,\cdots,\,\V{u}_r}}
		{\T{\Phi}\qty(\V{u}_1,\,\V{u}_2,\,\cdots,\,\V{u}_r)} \comma
	\end{equation}
	式中的 $\T{\Phi}$ 满足
	\begin{align}
		\forall\,\alpha,\,\beta\in\realR,\,
		&\mathrel{\phantom{=}}\T{\Phi}\qty(\V{u}_1,\,\cdots,\,
			\alpha\tilde{\V{u}}_i+\beta\hat{\V{u}}_i,\,\cdots,\,\V{u}_r)
			\notag \\
		&=\alpha\,\T{\Phi}\qty(\V{u}_1,\,\cdots,\,
			\tilde{\V{u}}_i,\,\cdots,\,\V{u_r})
		+\beta\,\T{\Phi}\qty(\V{u}_1,\,\cdots,\,
			\hat{\V{u}}_i,\,\cdots,\,\V{u_r}) \comma
	\end{align}
	即所谓“\emphB{对第 $i$ 个变元的线性性}”。

	在张量空间 $\Tensors{r}$ 上,我们引入线性结构:
	\begin{align}
		\forall\,\alpha,\,\beta\in\realR,\,
		\T{\Phi},\,\T{\Psi}\in\Tensors{r},
		&\mathrel{\phantom{=}}\qty(\alpha\,\T{\Phi}+\beta\,\T{\Psi})
		\qty(\V{u}_1,\,\V{u}_2,\,\cdots,\,\V{u}_r) \notag \\
		&\mathrel{\defeq}
			\alpha\,\T{\Phi}\qty(\V{u}_1,\,\V{u}_2,\,\cdots,\,\V{u}_r)
			+\beta\,\T{\Psi}\qty(\V{u}_1,\,\V{u}_2,\,\cdots,\,\V{u}_r)
		\comma
	\end{align}
	于是
	\begin{equation}
		\alpha\,\T{\Phi}+\beta\,\T{\Psi} \in \Tensors{r} \fullstop
	\end{equation}
	
	下面我们要获得 $\T{\Phi}$ 的表示。
	根据之前任意向量用协变基或逆变基的表示,有
	\begin{align}
		\forall\,\V{u},\,\V{v},\,\V{w}\in\Rm,
		&\mathrel{\phantom{=}}
		\T{\Phi}\qty(\V{u},\,\V{v},\,\V{w}) \notag\\
		&=\T{\Phi}\qty(u^i\V{g}_i,\,v_j\V{g}^j,\,w^k\V{g}_k) \notag
		\intertext{考虑到 $\T{\Phi}$ 对第一变元的线性性,可得}
		&=u^i\,\T{\Phi}\qty(\V{g}_i,\,v_j\V{g}^j,\,w^k\V{g}_k) \notag
		\intertext{同理,}
		&=u^i v_j w^k\,\T{\Phi}\qty(\V{g}_i,\,\V{g}^j,\,\V{g}_k)
		\fullstop
		\label{eq:张量的表示1}
	\end{align}
	注意这里自然需要满足 Einstein 求和约定。
	
	上式中的 $\T{\Phi}\qty(\V{g}_i,\,\V{g}^j,\,\V{g}_k)$ 是一个数。
	它是张量 $\T{\Phi}$ “吃掉”三个基向量的结果。
	至于 $u^i v_j w^k$ 部分,三项分别是 $\V{u}$ 的第 $i$ 个逆变分量、
	$\V{v}$ 的第 $j$ 个协变分量和 $\V{w}$ 的第 $k$ 个逆变分量。
	根据向量分量的定义,可知
	\begin{equation}
		u^i v_j w^k
		= \ipb{\V{u}}{\V{g}^i}
		\cdotp \ipb{\V{v}}{\V{g}_j}
		\cdotp \ipb{\V{w}}{\V{g}^k} \fullstop
		\label{eq:张量的表示2}
	\end{equation}
	
	\blankline
	
	暂时中断一下思路,先给出\emphA{简单张量}的定义。
	\begin{equation}
		\forall\,\V{u},\,\V{v},\,\V{w}\in\Rm,\quad
		\V{\xi}\tp\V{\eta}\tp\V{\zeta}\qty(\V{u},\,\V{v},\,\V{w})
		\defeq \ipb{\V{\xi}}{\V{u}}
		\cdotp \ipb{\V{\eta}}{\V{v}}
		\cdotp \ipb{\V{\zeta}}{\V{w}} \in\realR \comma
	\end{equation}
	式中 $\V{\xi},\,\V{\eta},\,\V{\zeta}\in\Rm$,
	而暂时把 $\V{\xi}\tp\V{\eta}\tp\V{\zeta}$ 理解为一种记号。
	简单张量作为一个映照,组成它的三个向量分别与它们“吃掉”的第一、二、三个变元
	做内积并相乘,结果为一个实数。
	
	考虑到内积的线性性,便有(以第二个变元为例)
	\begin{align}
		\V{\xi}\tp\V{\eta}\tp\V{\zeta}
		\qty(\V{u},\,\alpha\tilde{\V{v}}+\beta\hat{\V{v}},\,\V{w})
		&\defeq \ipb{\V{\xi}}{\V{u}}
		\cdotp \ipb{\V{\eta}}{\alpha\tilde{\V{v}}+\beta\hat{\V{v}}}
		\cdotp \ipb{\V{\zeta}}{\V{w}} \in\realR \notag
		\intertext{注意到
			$\ipb{\V{\eta}}{\alpha\tilde{\V{v}}+\beta\hat{\V{v}}}
				=\alpha\ipb{\V{\eta}}{\tilde{\V{v}}}
				+\beta\ipb{\V{\eta}}{\hat{\V{v}}}$,
			同时再次利用简单张量的定义,可得}
		&= \alpha \V{\xi}\tp\V{\eta}\tp\V{\zeta}
			\qty(\V{u},\,\tilde{\V{v}},\,\V{w})
			+\beta \V{\xi}\tp\V{\eta}\tp\V{\zeta}
			\qty(\V{u},\,\hat{\V{v}},\,\V{w}) \fullstop
	\end{align}
	类似地,对第一变元和第三变元,同样具有线性性。因此,可以知道
	\begin{equation}
		\V{\xi}\tp\V{\eta}\tp\V{\zeta}
		\in\Tensors{3} \fullstop
	\end{equation}
	可见,“简单张量”的名字是名副其实的,它的确是一个特殊的张量。
	
	回过头来看 \eqref{eq:张量的表示2}~式。很明显,它可以用简单张量来表示。
	要注意,由于内积的对称性,可以有两种\footnote{%
		这里只考虑把 $\V{u}$、$\V{v}$、$\V{w}$%
		和 $\V{g}^i$、$\V{g}_j$、$\V{g}^k$ 分别放在一起的情况。}表示方法:
	\begin{gather}
		\V{g}^i\tp\V{g}_j\tp\V{g}^k
		\qty(\V{u},\,\V{v},\,\V{w})
		\intertext{或者}
		\V{u}\tp\V{v}\tp\V{w}
		\qty(\V{g}^i,\,\V{g}_j,\,\V{g}^k) \comma
	\end{gather}
	我们这里取上面一种。代入式~\eqref{eq:张量的表示1},得
	\begin{align}
		&\mathrel{\phantom{=}}
			\T{\Phi}\qty(\V{u},\,\V{v},\,\V{w}) \notag\\
		&=\T{\Phi}\qty(\V{g}_i,\,\V{g}^j,\,\V{g}_k)
			\cdotp\V{g}^i\tp\V{g}_j\tp\V{g}^k
			\qty(\V{u},\,\V{v},\,\V{w}) \notag
		\intertext{由于
			$\T{\Phi}\qty(\V{g}_i,\,\V{g}^j,\,\V{g}_k) \in\Rm$,因此}
		&=\qty[\T{\Phi}\qty(\V{g}_i,\,\V{g}^j,\,\V{g}_k)
			\V{g}^i\tp\V{g}_j\tp\V{g}^k]
			\qty(\V{u},\,\V{v},\,\V{w}) \fullstop
	\end{align}
	方括号里的部分,就是根据 Einstein 求和约定,
	用 $\T{\Phi}\qty(\V{g}_i,\,\V{g}^j,\,\V{g}_k)$
	对 $\V{g}^i\tp\V{g}_j\tp\V{g}^k$ 进行线性组合。
	
	由于 $\V{u}$、$\V{v}$、$\V{w}$ 选取的任意性,可以引入如下记号:
	\begin{equation}
		\T{\Phi}
		=\T{\Phi}\qty(\V{g}_i,\,\V{g}^j,\,\V{g}_k) \,
			\V{g}^i\tp\V{g}_j\tp\V{g}^k
		\eqcolon \tensor{\Phi}{_i^j_k} \,
			\V{g}^i\tp\V{g}_j\tp\V{g}^k \comma
	\end{equation}
	即
	\begin{equation}
		\tensor{\Phi}{_i^j_k}
		\coloneq \T{\Phi}\qty(\V{g}_i,\,\V{g}^j,\,\V{g}_k) \comma
	\end{equation}
	这称为张量的\emphA{分量}。
	它说明一个张量可以用\emphB{张量分量}和基向量组成的\emphB{简单张量}来表示。
	
	指标 $i$、$j$、$k$ 的上下是任意的。这里,
	它有赖于式~\eqref{eq:张量的表示1} 中基向量的选取。
	实际上,对于这里的三阶张量,指标的上下一共有 8 种可能。
	指标全部在下面的,称为\emphA{协变分量}:
	\begin{equation}
		\tensor{\Phi}{^i^j^k} \coloneq
		\T{\Phi}\qty(\V{g}^i,\,\V{g}^j,\,\V{g}^k) \semicomma
	\end{equation}
	指标全部在上面的,称为\emphA{逆变分量}:
	\begin{equation}
		\tensor{\Phi}{_i_j_k} \coloneq
		\T{\Phi}\qty(\V{g}_i,\,\V{g}_j,\,\V{g}_k) \semicomma
	\end{equation}
	其余 6 种,称为\emphA{混合分量}。
	对于一个 $r$ 阶张量,显然共有 $2^r$ 种分量表示,
	其中协变分量与逆变分量各一种,混合分量 $2^r-2$ 种。
	
\subsection{张量分量之间的关系}
	我们已经知道,
	对于任意一个向量 $\V{\xi}\in\Rm$,它可以用协变基或逆变基表示:
	\begin{equation}
		\V{\xi}=\left\{\begin{aligned}
			\xi^i\V{g}_i \comma \\
			\xi_i\V{g}^i \fullstop
		\end{aligned}\right.
	\end{equation}
	式中,协变分量与逆变分量满足\emphB{坐标转换关系}:
	\begin{braceEq}
		\xi^i &=\ipb{\V{\xi}}{\V{g}^i}
		=\ipb{\V{\xi}}{g^{ik}\V{g}_k}
		=g^{ik}\ipb{\V{\xi}}{\V{g}_k}
		=g^{ik}\xi_k \comma \\
		\xi_i &=\ipb{\V{\xi}}{\V{g}_i}
		=\ipb{\V{\xi}}{g_{ik}\V{g}^k}
		=g_{ik}\ipb{\V{\xi}}{\V{g}^k}
		=g_{ik}\xi^k \fullstop
	\end{braceEq}
	每一式的第二个等号都用到了\emphB{基向量转换关系},
	见式~\eqref{eq:基向量转换关系1} 和 \eqref{eq:基向量转换关系4}。
	
	现在再来考虑张量的分量。仍以上文中的张量 $\tensor{\Phi}{_i^j_k}
		\coloneq \T{\Phi}\qty(\V{g}_i,\,\V{g}^j,\,\V{g}_k)$ 为例,
	我们想要知道它与张量 $\tensor{\Phi}{^p_q^r} \coloneq
		\T{\Phi}\qty(\V{g}^p,\,\V{g}_q,\,\V{g}^r)$ 之间的关系。
	利用基向量转换关系,可有
	\begin{align}
		\tensor{\Phi}{_i^j_k}
		&\coloneq\T{\Phi}\qty(\V{g}_i,\,\V{g}^j,\,\V{g}_k) \notag \\
		&=\T{\Phi}
			\qty(g_{ip}\V{g}^p,\,g^{jq}\V{g}_q,\,g_{kr}\V{g}^r) \notag
		\intertext{又利用张量的线性性,得}
		&=g_{ip}g^{jq}g_{kr}
			\T{\Phi}\qty(\V{g}^p,\,\V{g}_q,\,\V{g}^r) \notag \\
		&=g_{ip}g^{jq}g_{kr} \tensor{\Phi}{^p_q^r} \fullstop
	\end{align}
	可见,张量的分量与向量的分量类似,其指标升降可通过\emphB{度量}来实现。
	用同样的手法,还可以得到诸如
	$\tensor{\Phi}{^i^j^k}=g^{jp}\tensor{\Phi}{^i_p^k}$、
	$\tensor{\Phi}{^i_j^k}=g_{jp}g^{kq}\tensor{\Phi}{^i^p_k}$
	这样的关系式。
	
\subsection{相对不同基的张量分量之间的关系}
	$\Rm$ 空间中,除了 $\qty{\V{g}_i}^m_{i=1}$ 和相应的
	对偶基 $\qty{\V{g}^i}^m_{i=1}$ 之外,当然还可以有其他的基,
	比如带括号的 $\qty{\V{g}_{(i)}}^m_{i=1}$ 以及对应的
	对偶基 $\qty{\V{g}^{(i)}}^m_{i=1}$。
	前者对应形如 $\tensor{\Phi}{^i_j^k}
		\coloneq \T{\Phi}\qty(\V{g}^i,\,\V{g}_j,\,\V{g}^k)$ 的张量,
	后者则对应带括号的张量,如 $\tensor{\Phi}{^{(p)}_{(q)}^{(r)}} \coloneq
		\T{\Phi}\qty(\V{g}^{(p)},\,\V{g}_{(q)},\,\V{g}^{(r)})$。
	下面我们来探讨这两个张量的关系。
	
	首先来建立基之间的关系。带括号的第 $i$ 个基向量
	$\V{g}_{(i)}$,作为 $\Rm$ 空间中的一个向量,自然可以用另一组基来表示:
	\begin{equation}
		\V{g}_{(i)}=\left\{\begin{aligned}
			\ipb{\V{g}_{(i)}}{\V{g}_k}\,\V{g}^k \comma \\
			\ipb{\V{g}_{(i)}}{\V{g}^k}\,\V{g}_k \fullstop
		\end{aligned}\right.
	\end{equation}
	同理,自然还有它的对偶基:
	\begin{equation}
		\V{g}^{(i)}=\left\{\begin{aligned}
			\ipb{\V{g}^{(i)}}{\V{g}_k}\,\V{g}^k \comma \\
			\ipb{\V{g}^{(i)}}{\V{g}^k}\,\V{g}_k \fullstop
		\end{aligned}\right.
	\end{equation}
	引入记号 $c^k_{(i)} \coloneq \ipb{\V{g}_{(i)}}{\V{g}^k}$
	和 $c^{(i)}_k \coloneq \ipb{\V{g}^{(i)}}{\V{g}_k}$,那么有
	\begin{braceEq}
		\V{g}_{(i)} &= c^k_{(i)}\V{g}_k \comma \\
		\V{g}^{(i)} &= c^{(i)}_k\V{g}^k \fullstop
	\end{braceEq}
	
	容易看出,这两个系数具有如下性质:
	\begin{equation}
		c^{(i)}_k c^k_{(j)} = \KroneckerDelta{i}{j} \fullstop
	\end{equation}
	写成矩阵形式\footnote{%
		通常我们约定上面的标号作为行号,下面的标号作为列号。},为
	\begin{equation}
		\qty[c^{(i)}_k]\qty[c^k_{(j)}]
		=\qty[\KroneckerDelta{j}{i}]=\Mat{I}_m \fullstop
	\end{equation}
	换句话说,两个系数矩阵是互逆的。
	\begin{myProof}
		\begin{align}
			c^{(i)}_k c^k_{(j)}
			&=\ipb{\V{g}^{(i)}}{\V{g}_k}\,c^k_{(j)} \notag
			\intertext{利用内积的线性性,有}
			&=\ipb{\V{g}^{(i)}}{c^k_{(j)} \V{g}_k} \notag
			\intertext{根据 $c^k_{(j)}$ 的定义,得到}
			&=\ipb{\V{g}^{(i)}}{\V{g}_{(j)}} \fullstop
		\end{align}
		带括号的基同样满足对偶关系 \eqref{eq:对偶关系}~式,于是得证。
	\end{myProof}
	
	上面我们用不带括号的基表示了带括号的基。反之也是可以的:
	\begin{braceEq}
		\V{g}_i &= \ipb{\V{g}_i}{\V{g}^{(k)}}{\Rm}\,\V{g}_{(k)}
			=c^{(k)}_i\V{g}_{(k)} \comma \\
		\V{g}^i &= \ipb{\V{g}^i}{\V{g}_{(k)}}{\Rm}\,\V{g}^{(k)}
			=c^i_{(k)}\V{g}^{(k)} \fullstop
	\end{braceEq}
	这样一来,就建立起了不同基之间的转换关系。
	
	现在我们回到张量。根据张量分量的定义,
	\begin{align}
		\tensor{\Phi}{^i_j^k}
		&\coloneq \T{\Phi}\qty(\V{g}^i,\,\V{g}_j,\,\V{g}^k) \notag
		\intertext{利用之前推导的不同基向量之间的转换关系,得}
		&=\T{\Phi}\qty(
			c^i_{(p)}\V{g}^{(p)},\,c^{(q)}_j\V{g}_{(q)},\,
			c^k_{(r)}\V{g}^{(r)}) \notag
		\intertext{由张量的线性性,提出系数:}
		&=c^i_{(p)} c^{(q)}_j c^k_{(r)} \,
			\T{\Phi}\qty(\V{g}^{(p)},\,\V{g}_{(q)},\V{g}^{(r)}) \notag\\
		&=c^i_{(p)} c^{(q)}_j c^k_{(r)} \,
			\tensor{\Phi}{^{(p)}_{(q)}^{(r)}} \fullstop
	\end{align}
	完全类似,还可以有
	\begin{equation}
		\tensor{\Phi}{^{(i)}_{(j)}^{(k)}}
		=c^{(i)}_p c^g_{(j)} c^{(k)}_r \tensor{\Phi}{^p_q^r} \fullstop
	\end{equation}
	
	\blankline
	
	总结一下这两小节得到的结果。
	对于同一组基下的张量分量,其指标升降通过\emphB{度量}来实现;
	对于不同基下的张量分量,其指标转换则通过不同基之间的转换系数来完成。
	
%	\backmatter
%	{
%		\small
%		\bibliography{Reference}
%	}
%	\printindex[pkg]
%	\printindex[cmd]
\end{document}