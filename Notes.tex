%TODO 等的使用
%%TODO 内容问题
%%HACK 粗鄙技巧
%%CODE 代码改进

%TODO——注意事项

\documentclass[oneside]{book}

%TODO——宏包
%% 方便定义命令(星号版本)
\usepackage{suffix}

%% 页面尺寸
\usepackage{geometry}
	\geometry{
		a4paper,
		left = 2.54 cm, right = 2.54 cm, top = 3.18 cm, bottom = 3.18 cm,
		headheight = 3 cm
	}

%% 设置标题
\usepackage{titlesec}

%% 交叉引用,超链接等
\usepackage[hyperindex]{hyperref}
	\hypersetup{
		%% PDF 书签
		bookmarksopen = true,
		bookmarksopenlevel = 1,
		bookmarksnumbered = true,
		%% PDF 标题作者
%		pdftitle = {},
%		pdfauthor = {},
		%% 脚注
%		hyperfootnotes = false,
		%% 目录  只引用页码
		linktoc = page,
		%% 超链接颜色
		colorlinks,
		linkcolor = {red!60!black},
		citecolor = {green!50!black},
		urlcolor = {blue!70!black}
	}

%% 常规字体选择
% 与 `graphicx' 冲突
\PassOptionsToPackage{draft}{graphicx}
\usepackage[no-math]{fontspec}
	\setmainfont[
		Extension = .otf,
		BoldFont = xits-bold,
		ItalicFont = xits-italic,
		BoldItalicFont = xits-bolditalic,
		SlantedFont = xits-italic
	]{xits-regular}
	\setsansfont[
		Extension = .otf,
		BoldFont = texgyreheros-bold,
		ItalicFont = texgyreheros-italic,
		BoldItalicFont = texgyreheros-bolditalic,
		SlantedFont = texgyreheros-italic
	]{texgyreheros-regular}
%	\setmonofont[
%		Extension = .otf,
%		BoldFont = texgyrecursor-bold,
%		ItalicFont = texgyrecursor-italic,
%		SlantedFont = texgyrecursor-italic,
%		Ligatures = NoCommon
%	]{texgyrecursor-regular}
%	\newfontfamily{\SourceSans}{Source Sans Pro}
%	\newfontfamily{\Songti}{方正书宋_GBK}

%% AMS 数学支持
\usepackage{amsmath}
	% 允许多行公式中间分页
	\allowdisplaybreaks
	% 调整多行公式间距
	\setlength{\jot}{6pt}
\usepackage{amssymb}

%% 数学工具(Must before `unicode-math')
\usepackage{mathtools}

%% Pi 符号
\usepackage{pifont}
	% 正确 √
	\newcommand{\cmark}{\ding{51}}
	% 错误 ×
	\newcommand{\xmark}{\ding{55}}

%% 调用 Unicode OpenType 数学字体
\usepackage{unicode-math}
	\setmathfont[
		math-style = ISO,
		bold-style = ISO
	]{xits-math.otf}
%	]{libertinusmath-regular.otf}
%	]{texgyreschola-math.otf}
%	]{texgyredejavu-math.otf}
%	]{texgyrepagella-math.otf}
%	]{latinmodern-math.otf}
%	]{MinionMath-Regular.otf}
%	]{Asana-math.otf}
% 加粗使用 \symbf{}
% 直立希腊字母:\uppi 等

%% 中文文字处理
\usepackage[UTF8, heading = true]{ctex}
	\pagestyle{plain}
	\ctexset{
		section/format+ = {\normalfont\sffamily},
		subsection/format+ = {\normalfont\sffamily}
	}
\usepackage{xeCJK}
	\setCJKmainfont[
		BoldFont = 方正黑体_GBK,
		ItalicFont = 方正楷体_GBK,
		Mapping = fullwidth-stop
	]{方正书宋_GBK}
	\setCJKsansfont[
		BoldFont = 方正黑体_GBK,
		ItalicFont = 方正黑体_GBK,
		Mapping = fullwidth-stop
	]{方正黑体_GBK}
%	\setCJKsansfont[
%		BoldFont = 思源黑体 CN Regular,
%		ItalicFont = 思源黑体 CN Regular,
%		Mapping = fullwidth-stop
%	]{思源黑体 CN Regular}
	\setCJKmonofont[
		BoldFont = 方正仿宋_GBK,
		ItalicFont = 方正楷体_GBK,
		Mapping = fullwidth-stop
	]{方正仿宋_GBK}
	\setCJKfamilyfont{宋体}{方正书宋_GBK}
	\setCJKfamilyfont{楷体}{方正楷体_GBK}
	\setCJKfamilyfont{黑体}{方正黑体_GBK}
	\setCJKfamilyfont{仿宋}{方正仿宋_GBK}

%% 脚注增强版
\usepackage[stable, perpage, bottom]{footmisc}
	% 需要调用 pifont 宏包
	% 衬线加圈阳文数字:\ding{172}~\ding{181} (1~10)
	% 无衬线加圈阳文数字:\ding{192}~\ding{201} (1~10)
	\renewcommand{\thefootnote}{\ding{\numexpr191+\value{footnote} } }
	%TODO:20160705 脚注不用上标
	%HACK:20160709 见 http://tex.stackexchange.com/q/19844
	\makeatletter
	\newlength{\fnBreite}
	\renewcommand{\@makefntext}[1]{%
		\settowidth{\fnBreite}{\footnotesize\@thefnmark.i}%
		\protect\footnotesize\upshape%
		\setlength{\@tempdima}{\columnwidth}%
		\addtolength{\@tempdima}{-\fnBreite}%
		\makebox[\fnBreite][l]{\@thefnmark\phantom{  }}%
		\parbox[t]{\@tempdima}%
		{\everypar{\hspace*{1em}}\hspace*{-1em}\upshape#1}%
	}
	\makeatother

%% 颜色
\usepackage[svgnames]{xcolor}

%% 表格虚线
\usepackage{arydshln}

%% 图形
% 已被`fontspec' 调用
%\usepackage{graphicx}

%% 绘图
%\usepackage{tikz}
%	\usepgflibrary{arrows.meta}
%	\tikzset{>=Stealth}

%% 强调公式
\usepackage[ntheorem]{empheq}

%% 张量
\usepackage{tensor}

%% 物理、数学符号
\usepackage{physics}

%% 单位
%\usepackage{siunitx}

%% 定制列表环境
\usepackage{enumitem}
	% 定义带缩进的左对齐格式
	% 前一个参数 2.1 em 确定标签的位置
	% 后一个参数 0.55 em 确定标签与文字的距离
	%HACK:20160924 与字体、字号、标签内容有关
	\SetLabelAlign{leftalignwithindent}%
		{\hspace{2.1 em} \makebox[0.55 em][l]{#1}}

%% 定理类环境
\usepackage[thmmarks, amsmath]{ntheorem}
	\theoremstyle{nonumberplain} %带编号
	\theoremheaderfont{\bfseries}
	\theorembodyfont{\normalfont}
	\theoremsymbol{\mbox{$\Box$}}
		\newtheorem{myProof}{证明:}

%% 索引

%TODO——环境
%% 定制列表(编号)
\newenvironment{myEnum}
	{\begin{enumerate}[
%		label=\bullet, %标签样式:圆点
		align = leftalignwithindent, %对齐(见上)
		listparindent = 2 em, %条目段落缩进
		leftmargin = 0 pt, %文字左边距
		topsep = 0 pt,
		itemsep = 0 pt,
		parsep = 0 pt
	]}
	{\end{enumerate}}
%% 子公式
\newenvironment{mySubEq}
	{\subequations \renewcommand{\theequation}%
		{\theparentequation-\alph{equation}}}%
	{\endsubequations%
		\ignorespacesafterend}
%% 大括号公式
\newenvironment{braceEq}[1][align]
	{\mySubEq%
		\setkeys{EmphEqEnv}{#1}%
		\setkeys{EmphEqOpt}{left=\empheqlbrace}%
		\EmphEqMainEnv}%
	{\endEmphEqMainEnv \endmySubEq}
%% 大括号公式2(分情况讨论)
\newenvironment{braceEq*}[2][align]
	{\mySubEq%
		\setkeys{EmphEqEnv}{#1}%
		\setkeys{EmphEqOpt}{left={{\displaystyle #2}\empheqlbrace}}%
		\EmphEqMainEnv}%
	{\endEmphEqMainEnv \endmySubEq}

%TODO——命令
%% 空行
%\newcommand{\blankline}{\mbox{}\par\mbox{}}
\newcommand{\blankline}{\mbox{}}
%% 强调
\newcommand{\emphA}[1]{{\bfseries #1}}
\newcommand{\emphB}[1]{{\itshape #1}}
%% 高亮(Highlight)
\newcommand{\hl}[2][yellow!60!white]{\mathchoice%
	{\colorbox{#1}{$\displaystyle #2$}}%
	{\colorbox{#1}{$\textstyle #2$}}%
	{\colorbox{#1}{$\scriptstyle #2$}}%
	{\colorbox{#1}{$\scriptscriptstyle #2$}}}
\newcommand{\hlw}[1]{\mathchoice%
	{\colorbox{white}{$\displaystyle #1$}}%
	{\colorbox{white}{$\textstyle #1$}}%
	{\colorbox{white}{$\scriptstyle #1$}}%
	{\colorbox{white}{$\scriptscriptstyle #1$}}}
\newcommand{\myPROBLEM}[1]{\colorbox{red}{#1}}

%TODO——数学命令
%--------数学字体--------%
%% 数集 黑板粗体
\let\SET=\symbb
%% 空间 花体
\let\SPACE=\symcal
%% 特殊空间 无衬线直立
\let\SPACEX=\symsfup
%% 域(领域,定义域) 哥特体
\let\DOMAIN=\symfrak
%% 符号 倾斜粗体
\let\SYMBOL=\symbf
%% 一般算子 无衬线直立
\let\OPERATOR=\symsfup
%% 特殊算子 无衬线直立粗体
\let\OPERATORX=\symbfsfup

%--------数集--------%
\newcommand{\realR}{\SET{R}}
\newcommand{\intN}{\SET{N}}
%% m 维实空间
\newcommand{\Rm}{\realR^m}
%% 自然数集(正整数集)
\newcommand{\natN}{\intN^{*}}

%--------空间--------%
%% 以 m 维实空间为底空间的(r)阶张量全体
\newcommand{\Tensors}[2][\Rm]{\SPACE{T}^{#2}\qty(#1)}
%% (r)阶置换全体
\newcommand{\Permutations}[1]{\SPACE{P}_{#1}}
%% 以 m 维实空间为底空间的(r)阶对称张量全体
\newcommand{\SymTensors}[2][\Rm]{\SPACE{S}^{#2}\qty(#1)}
%% 以 m 维实空间为底空间的(r)阶反对称张量全体
\newcommand{\SkwTensors}[2][\Rm]{\Lambda^{#2}\qty(#1)}
%% 线性变换全体
\newcommand{\LinearT}[2]{\SPACE{L}\qty(#1,\,#2)}

%--------特殊空间--------%
%% 正交矩阵全体(Orthogonal Matrices)
\newcommand{\Orth}{\SPACEX{Orth}}
%% 对称张量全体(Symmetric Tensors)
\newcommand{\Sym}{\SPACEX{Sym}}
%% 反对称张量全体(Anti-symmetric Tensors)
\newcommand{\Skw}{\SPACEX{Skw}}

%--------区域--------%
%% 邻域(Neighborhood)
\newcommand{\domB}[2]{\DOMAIN{B}_{#1}\qty(#2)}
%\WithSuffix\newcommand\domB*[2]{\DOMAIN{B}_{#1}\qty(#2)}
%% 定义域(Domain)
\newcommand{\domD}[1]{\DOMAIN{D}_{#1}}

%--------量(Quantity)--------%
%% 向量(Vector)
\newcommand{\V}[1]{\SYMBOL{#1}}
%% 张量(Tensor)
\renewcommand{\T}[1]{\SYMBOL{#1}}
%% 矩阵(Matrix)
\newcommand{\Mat}[1]{\SYMBOL{#1}}
%% 置换(Permutation)
\newcommand{\Perm}[1]{\SYMBOL{#1}}

%--------特殊量--------%
%% Kronecker Delta
\newcommand{\KroneckerDelta}[2]{\delta^{#1}_{#2}}
%% Kronecker Delta (全下标形式)
\WithSuffix\newcommand\KroneckerDelta*[1]{\delta_{#1}}
%% Levi-Civita 记号
\newcommand{\LeviCivita}[1]{\tensor{\epsilon}{#1}}
%% Eddington 张量
\newcommand{\EdTensor}{\T{\epsilon}}
%% 第一类 Christoffel 符号
\newcommand{\ChristoffelA}[3]{\Gamma_{#1#2,\,#3}}
%% 第二类 Christoffel 符号
\newcommand{\ChristoffelB}[3]{\tensor{\Gamma}{^#3_{#1#2}}}

%--------算子--------%
%% 置换算子(Permutation Operator)
\newcommand{\opPerm}[1][\Perm{\sigma}]{\OPERATOR{I}_{#1}}
%% 对称化算子(Symmetrizer)
\newcommand{\opSym}{\symcal{S}}
%% 反对称化算子(Antisymmetrizer)
\newcommand{\opSkw}{\symcal{A}}
%% Jacobi 矩阵(Differential Operator)
\DeclareMathOperator{\JacobiD}{\OPERATOR{D}\!}

%--------特殊算子--------%
%% 恒等映照(Identity)
\newcommand{\Id}{\OPERATORX{Id}}

%--------数学符号--------%
%% 映射(X: x → X)(Math Map)
\newcommand{\mmap}[3]{{#1}:\,{#2}\mapsto{#3}}
%% 带描述集合
\DeclarePairedDelimiterX\set[2]{\{}{\}}{#1 \;\delimsize\vert\; #2}
%% 中标
%\makeatletter
%\protected\def\xvcenter{%
%	\hbox\bgroup$\everyvbox{\everyvbox{}\aftergroup%
%	\m@th\aftergroup$\aftergroup\egroup}%
%	\vcenter%
%}
%\newcommand{\mid@script}[2]{\vcenter{\hbox{$\m@th#1#2$}}}
%\DeclareRobustCommand{\midscript}[1]{\mathchoice%
%	{\mid@script\scriptstyle{#1}}%
%	{\mid@script\scriptstyle{#1}}%
%	{\mid@script\scriptscriptstyle{#1}}%
%	{\mid@script\scriptscriptstyle{#1}}%
%}
%\makeatother
\newcommand\midscript[1]{{\scriptstyle #1}}
%% 单位正交基指标(Orthonormal Index)
\newcommand{\orthIdx}[1]{\langle{#1}\rangle}
%% 范数
\let\PHYSICSNORM=\norm
\renewcommand{\norm}[2][\Rm]{\PHYSICSNORM{#2}_{#1}}
%% 内积(括号形式)(Inner Product Braket)
%\newcommand{\ipb}[3][\Rm]{\qty(#2,\,#3)_{#1}}
\newcommand{\ipb}[3][\Rm]{\left\langle#2,\,#3\right\rangle_{#1}}
%% 张量积(Tensor Product)
\newcommand{\tp}{\otimes}
%% e-点积(e Dot Product)
\newcommand{\edp}[1][e]{\mathbin{\dbinom{#1}{\mathord{\cdot}}}}
%% 全点积(Full Dot Product)
\newcommand{\fdp}{\odot}
%% 点乘(Dot Product)
%% 使用 \cdot
%% \cdotp 作为 ordinary 符号
\newcommand{\cdotord}{\mathord{\cdotp}}
%% 叉乘(Cross Product)(has been defined in `physics' package)
\renewcommand{\cp}{\vectimes}
%% C^p 微分同胚(Diffeomorphism)
\newcommand{\DiffMorp}[1][p]{\symcal{C}^{#1}}
%% 直至 p 阶偏导数连续的函数(Continuous Function)
\newcommand{\cf}[3][p]{\symcal{C}^{#1}\qty(#2;\,#3)}
%% 大 O 记号
\newcommand{\bO}[1]{\mathop{\symcal{O}}\qty(#1)}
%% 小 O 记号
\newcommand{\sO}[1]{\mathop{\symcal{o}}\qty(#1)}
%% 复合(Composite)
\newcommand{\comp}{\circ}
%% 定义等号(Definition Equal)
\newcommand{\defeq}{\triangleq}
%% 转置(Transpose)
\newcommand{\trans}{^{\mkern-1.5mu\mathsf{T}}}
%% 取逆并转置(Inverse and Transpose)
\newcommand{\invTrans}{^{-\mkern-1.5mu\mathsf{T}}}

%--------函数名称--------%
%% 符号函数(Sign)
\DeclareMathOperator{\sgn}{sgn}

%--------公式杂项--------%
%% 带圈数字(Circle digit)
\newcommand{\circNum}[1]{%
	\ifnum #1=0%
		\mathord{\symbol{9450}}%
	\else%
		\mathord{\symbol{\numexpr9311+#1}}%
	\fi}
%% align 空格
\newcommand{\alspace}{\mathrel{\phantom{=}}}
%% 标点、文字
\newcommand{\comma}{\text{,}}
\newcommand{\fullstop}{\text{.}}
\newcommand{\semicomma}{\text{;}}
\newcommand{\const}{\text{const.}}
%\renewcommand{\intertext}[1]{\shortintertext{\textit{#1}}}


%TODO——标题页
\title{
	\vspace{-4 cm} \color{Sienna} \Huge 张量分析
}
\author{
	\CJKfamily{楷体} \color{DarkRed} \Large
	复旦大学\phantom{空格}谢锡麟
}
\date{
	\CJKfamily{楷体} \color{Goldenrod} \Large \today
}


\begin{document}
%	\frontmatter
%	\maketitle
%	
%%	\tableofcontents
%	
%	\mainmatter
%	\chapter{张量的定义及表示}
%		\section{对偶基,度量}
\subsection{对偶基}
	$\Rm$ 空间中的基可分为两类:指标写在\emphB{下面}的基
	\begin{equation*}
		\qty{\V{g}_i}^m_{i=1} \subset \Rm
	\end{equation*}
	称为\emphA{协变基},指标写在\emphB{上面}的基
	\begin{equation*}
		\qty{\V{g}^i}^m_{i=1} \subset \Rm
	\end{equation*}
	称为\emphA{逆变基}。
	它们满足\emphA{对偶关系}:
	\begin{equation}
		\qty(\V{g}^i,\,\V{g}_j)_{\Rm}=\KroneckerDelta{i}{j}
		=\left\{\begin{aligned}
			1,\quad i&=j \semicomma\\
			0,\quad i&\neq j \fullstop
		\end{aligned}\right.
		\label{eq:对偶关系}
	\end{equation}
	这里的 $\KroneckerDelta{i}{j}$ 是 \emphA{Kronecker δ 函数}。
	
\subsection{度量}
	下面引入\emphA{度量}的概念。其定义为
	\begin{braceEq}
		g_{ij} &\defeq \qty(\V{g}_i,\,\V{g}_j)_{\Rm} \comma 
		\label{eq:度量的定义_协变} \\
		g^{ij} &\defeq \qty(\V{g}^i,\,\V{g}^j)_{\Rm} \fullstop
		\label{eq:度量的定义_逆变}
	\end{braceEq}
	
	下面证明
	\begin{equation}
		g_{ik}\,g^{kj} = \KroneckerDelta{j}{i} \fullstop
		\label{eq:度量之积}
	\end{equation}
	它也可以写成矩阵的形式:
	\begin{equation}
		\qty[g_{ik}]\qty[g^{kj}]
		=\qty[\KroneckerDelta{j}{i}]=\Mat{I}_m \comma
	\end{equation}
	其中的 $\Mat{I}_m$ 是 $m$ 阶单位阵。
	\begin{myProof}
		\begin{equation}
			g_{ik}\,g^{kj}
			=\qty(\V{g}_i,\,\V{g}_k)_{\Rm} \, g^{kj}
			=\qty(\V{g}_i,\,g^{kj}\V{g}_k)_{\Rm}
		\end{equation}
		后文将说明 $g^{kj}\V{g}_k=\V{g}^j$,因此可得
		\begin{equation}
			g_{ik}\,g^{kj}=\qty(\V{g}_i,\,\V{g}^j)_{\Rm}
			=\KroneckerDelta{j}{i} \fullstop
		\end{equation}
		
		要注意的是,这里的指标 $k$ 是\emphB{哑标}。
		根据\emphA{Einstein 求和约定},
		重复指标并且一上一下时,就表示对它求和。后文除非特殊说明,也均是如此。
	\end{myProof}
	
	现在澄清\emphA{基向量转换关系}。第 $i$ 个协变基向量 $\V{g}_i$ 既然是向量,
	就必然可以用协变基或逆变基来表示。
	根据对偶关系式~\eqref{eq:对偶关系} 和度量的定义
	式~\eqref{eq:度量的定义_协变}、\eqref{eq:度量的定义_逆变},可知
	\begin{braceEq}
		\V{g}_i &=\qty(\V{g}_i,\,\V{g}_k)_{\Rm} \,\V{g}^k
		=g_{ik}\,\V{g}^k \comma \label{eq:基向量转换关系1} \\
		\V{g}_i &=\qty(\V{g}_i,\,\V{g}^k)_{\Rm} \,\V{g}_k
		=\KroneckerDelta{k}{i}\V{g}_k \label{eq:基向量转换关系2}
	\end{braceEq}
	以及
	\begin{braceEq}
		\V{g}^i &=\qty(\V{g}^i,\,\V{g}_k)_{\Rm} \,\V{g}^k
		=\KroneckerDelta{i}{k}\,\V{g}^k \comma \label{eq:基向量转换关系3} \\
		\V{g}^i &=\qty(\V{g}^i,\,\V{g}^k)_{\Rm} \,\V{g}_k
		=g^{ik}\V{g}_k \fullstop \label{eq:基向量转换关系4}
	\end{braceEq}
	这四个式子中,式~\eqref{eq:基向量转换关系2} 和
	\eqref{eq:基向量转换关系3} 是平凡的,
	而式~\eqref{eq:基向量转换关系1} 和 \eqref{eq:基向量转换关系4}
	则通过\emphB{度量}建立起了协变基与逆变基之间的关系。
	这就称为基向量转换关系,也可以叫做“指标升降游戏”。
	
\subsection{向量的分量}
	对于任意的向量 $\V{\xi} \in \Rm$,它可以用协变基表示:
	\begin{equation}
		\V{\xi}=\ipb{\V{\xi}}{\V{g}^k}\,\V{g}_k
		=\xi^k \V{g}_k \comma
	\end{equation}
	也可以用逆变基表示:
	\begin{equation}
		\V{\xi}=\ipb{\V{\xi}}{\V{g}_k}\,\V{g}^k
		=\xi_k \V{g}^k \comma
	\end{equation}
	式中,$\xi^k$ 是 $\V{\xi}$ 与第 $k$ 个\emphB{逆变基}做内积的结果,
	称为 $\V{\xi}$ 的第 $k$ 个\emphA{逆变分量};
	而 $\xi_k$ 是 $\V{\xi}$ 与第 $k$ 个\emphB{协变基}做内积的结果,
	称为 $\V{\xi}$ 的第 $k$ 个\emphA{协变分量}。
	
	以后凡是指标在下的(下标),均称为\emphB{协变}某某;
	指标在上的(上标),称为\emphB{逆变}某某。
	
\section{张量的表示}
\subsection{张量的表示与简单张量}\label{subsec:张量的表示与简单张量}
	所谓\emphA{张量},即\emphA{多重线性函数}。
	
	首先用三阶张量举个例子。考虑任意的 $\T{\Phi}\in\Tensors{3}$,
	其中的 $\Tensors{3}$ 表示以 $\Rm$ 为底空间的三阶张量全体。
	所谓三阶(或三重)线性函数,指“吃掉”三个向量之后变成数,
	并且“吃法”具有线性性。
	
	一般地,$r$ 阶张量的定义如下:
	\begin{equation}
		\mdef{\T{\Phi}}{\underbrace{\Rm\times\Rm\times\cdots\times\Rm}_
			{\text{$r$ 个 $\Rm$}}
			\ni\qty{\V{u}_1,\,\V{u}_2,\,\cdots,\,\V{u}_r}}
		{\T{\Phi}\qty(\V{u}_1,\,\V{u}_2,\,\cdots,\,\V{u}_r)} \comma
	\end{equation}
	式中的 $\T{\Phi}$ 满足
	\begin{align}
		\forall\,\alpha,\,\beta\in\realR,\,
		&\mathrel{\phantom{=}}\T{\Phi}\qty(\V{u}_1,\,\cdots,\,
			\alpha\tilde{\V{u}}_i+\beta\hat{\V{u}}_i,\,\cdots,\,\V{u}_r)
			\notag \\
		&=\alpha\,\T{\Phi}\qty(\V{u}_1,\,\cdots,\,
			\tilde{\V{u}}_i,\,\cdots,\,\V{u_r})
		+\beta\,\T{\Phi}\qty(\V{u}_1,\,\cdots,\,
			\hat{\V{u}}_i,\,\cdots,\,\V{u_r}) \comma
	\end{align}
	即所谓“\emphB{对第 $i$ 个变元的线性性}”。

	在张量空间 $\Tensors{r}$ 上,我们引入线性结构:
	\begin{align}
		\forall\,\alpha,\,\beta\in\realR,\,
		\T{\Phi},\,\T{\Psi}\in\Tensors{r},
		&\mathrel{\phantom{=}}\qty(\alpha\,\T{\Phi}+\beta\,\T{\Psi})
		\qty(\V{u}_1,\,\V{u}_2,\,\cdots,\,\V{u}_r) \notag \\
		&\mathrel{\defeq}
			\alpha\,\T{\Phi}\qty(\V{u}_1,\,\V{u}_2,\,\cdots,\,\V{u}_r)
			+\beta\,\T{\Psi}\qty(\V{u}_1,\,\V{u}_2,\,\cdots,\,\V{u}_r)
		\comma
	\end{align}
	于是
	\begin{equation}
		\alpha\,\T{\Phi}+\beta\,\T{\Psi} \in \Tensors{r} \fullstop
	\end{equation}
	
	下面我们要获得 $\T{\Phi}$ 的表示。
	根据之前任意向量用协变基或逆变基的表示,有
	\begin{align}
		\forall\,\V{u},\,\V{v},\,\V{w}\in\Rm,
		&\mathrel{\phantom{=}}
		\T{\Phi}\qty(\V{u},\,\V{v},\,\V{w}) \notag\\
		&=\T{\Phi}\qty(u^i\V{g}_i,\,v_j\V{g}^j,\,w^k\V{g}_k) \notag
		\intertext{考虑到 $\T{\Phi}$ 对第一变元的线性性,可得}
		&=u^i\,\T{\Phi}\qty(\V{g}_i,\,v_j\V{g}^j,\,w^k\V{g}_k) \notag
		\intertext{同理,}
		&=u^i v_j w^k\,\T{\Phi}\qty(\V{g}_i,\,\V{g}^j,\,\V{g}_k)
		\fullstop
		\label{eq:张量的表示1}
	\end{align}
	注意这里自然需要满足 Einstein 求和约定。
	
	上式中的 $\T{\Phi}\qty(\V{g}_i,\,\V{g}^j,\,\V{g}_k)$ 是一个数。
	它是张量 $\T{\Phi}$ “吃掉”三个基向量的结果。
	至于 $u^i v_j w^k$ 部分,三项分别是 $\V{u}$ 的第 $i$ 个逆变分量、
	$\V{v}$ 的第 $j$ 个协变分量和 $\V{w}$ 的第 $k$ 个逆变分量。
	根据向量分量的定义,可知
	\begin{equation}
		u^i v_j w^k
		= \ipb{\V{u}}{\V{g}^i}
		\cdotp \ipb{\V{v}}{\V{g}_j}
		\cdotp \ipb{\V{w}}{\V{g}^k} \fullstop
		\label{eq:张量的表示2}
	\end{equation}
	
	\blankline
	
	暂时中断一下思路,先给出\emphA{简单张量}的定义。
	\begin{equation}
		\forall\,\V{u},\,\V{v},\,\V{w}\in\Rm,\quad
		\V{\xi}\tp\V{\eta}\tp\V{\zeta}\qty(\V{u},\,\V{v},\,\V{w})
		\defeq \ipb{\V{\xi}}{\V{u}}
		\cdotp \ipb{\V{\eta}}{\V{v}}
		\cdotp \ipb{\V{\zeta}}{\V{w}} \in\realR \comma
	\end{equation}
	式中 $\V{\xi},\,\V{\eta},\,\V{\zeta}\in\Rm$,
	而暂时把 $\V{\xi}\tp\V{\eta}\tp\V{\zeta}$ 理解为一种记号。
	简单张量作为一个映照,组成它的三个向量分别与它们“吃掉”的第一、二、三个变元
	做内积并相乘,结果为一个实数。
	
	考虑到内积的线性性,便有(以第二个变元为例)
	\begin{align}
		\V{\xi}\tp\V{\eta}\tp\V{\zeta}
		\qty(\V{u},\,\alpha\tilde{\V{v}}+\beta\hat{\V{v}},\,\V{w})
		&\defeq \ipb{\V{\xi}}{\V{u}}
		\cdotp \ipb{\V{\eta}}{\alpha\tilde{\V{v}}+\beta\hat{\V{v}}}
		\cdotp \ipb{\V{\zeta}}{\V{w}} \in\realR \notag
		\intertext{注意到
			$\ipb{\V{\eta}}{\alpha\tilde{\V{v}}+\beta\hat{\V{v}}}
				=\alpha\ipb{\V{\eta}}{\tilde{\V{v}}}
				+\beta\ipb{\V{\eta}}{\hat{\V{v}}}$,
			同时再次利用简单张量的定义,可得}
		&= \alpha \V{\xi}\tp\V{\eta}\tp\V{\zeta}
			\qty(\V{u},\,\tilde{\V{v}},\,\V{w})
			+\beta \V{\xi}\tp\V{\eta}\tp\V{\zeta}
			\qty(\V{u},\,\hat{\V{v}},\,\V{w}) \fullstop
	\end{align}
	类似地,对第一变元和第三变元,同样具有线性性。因此,可以知道
	\begin{equation}
		\V{\xi}\tp\V{\eta}\tp\V{\zeta}
		\in\Tensors{3} \fullstop
	\end{equation}
	可见,“简单张量”的名字是名副其实的,它的确是一个特殊的张量。
	
	回过头来看 \eqref{eq:张量的表示2}~式。很明显,它可以用简单张量来表示。
	要注意,由于内积的对称性,可以有两种\footnote{%
		这里只考虑把 $\V{u}$、$\V{v}$、$\V{w}$%
		和 $\V{g}^i$、$\V{g}_j$、$\V{g}^k$ 分别放在一起的情况。}表示方法:
	\begin{gather}
		\V{g}^i\tp\V{g}_j\tp\V{g}^k
		\qty(\V{u},\,\V{v},\,\V{w})
		\intertext{或者}
		\V{u}\tp\V{v}\tp\V{w}
		\qty(\V{g}^i,\,\V{g}_j,\,\V{g}^k) \comma
	\end{gather}
	我们这里取上面一种。代入式~\eqref{eq:张量的表示1},得
	\begin{align}
		&\mathrel{\phantom{=}}
			\T{\Phi}\qty(\V{u},\,\V{v},\,\V{w}) \notag\\
		&=\T{\Phi}\qty(\V{g}_i,\,\V{g}^j,\,\V{g}_k)
			\cdotp\V{g}^i\tp\V{g}_j\tp\V{g}^k
			\qty(\V{u},\,\V{v},\,\V{w}) \notag
		\intertext{由于
			$\T{\Phi}\qty(\V{g}_i,\,\V{g}^j,\,\V{g}_k) \in\Rm$,因此}
		&=\qty[\T{\Phi}\qty(\V{g}_i,\,\V{g}^j,\,\V{g}_k)
			\V{g}^i\tp\V{g}_j\tp\V{g}^k]
			\qty(\V{u},\,\V{v},\,\V{w}) \fullstop
	\end{align}
	方括号里的部分,就是根据 Einstein 求和约定,
	用 $\T{\Phi}\qty(\V{g}_i,\,\V{g}^j,\,\V{g}_k)$
	对 $\V{g}^i\tp\V{g}_j\tp\V{g}^k$ 进行线性组合。
	
	由于 $\V{u}$、$\V{v}$、$\V{w}$ 选取的任意性,可以引入如下记号:
	\begin{equation}
		\T{\Phi}
		=\T{\Phi}\qty(\V{g}_i,\,\V{g}^j,\,\V{g}_k) \,
			\V{g}^i\tp\V{g}_j\tp\V{g}^k
		\eqcolon \tensor{\Phi}{_i^j_k} \,
			\V{g}^i\tp\V{g}_j\tp\V{g}^k \comma
	\end{equation}
	即
	\begin{equation}
		\tensor{\Phi}{_i^j_k}
		\coloneq \T{\Phi}\qty(\V{g}_i,\,\V{g}^j,\,\V{g}_k) \comma
	\end{equation}
	这称为张量的\emphA{分量}。
	它说明一个张量可以用\emphB{张量分量}和基向量组成的\emphB{简单张量}来表示。
	
	指标 $i$、$j$、$k$ 的上下是任意的。这里,
	它有赖于式~\eqref{eq:张量的表示1} 中基向量的选取。
	实际上,对于这里的三阶张量,指标的上下一共有 8 种可能。
	指标全部在下面的,称为\emphA{协变分量}:
	\begin{equation}
		\tensor{\Phi}{^i^j^k} \coloneq
		\T{\Phi}\qty(\V{g}^i,\,\V{g}^j,\,\V{g}^k) \semicomma
	\end{equation}
	指标全部在上面的,称为\emphA{逆变分量}:
	\begin{equation}
		\tensor{\Phi}{_i_j_k} \coloneq
		\T{\Phi}\qty(\V{g}_i,\,\V{g}_j,\,\V{g}_k) \semicomma
	\end{equation}
	其余 6 种,称为\emphA{混合分量}。
	对于一个 $r$ 阶张量,显然共有 $2^r$ 种分量表示,
	其中协变分量与逆变分量各一种,混合分量 $2^r-2$ 种。
	
\subsection{张量分量之间的关系}
	我们已经知道,
	对于任意一个向量 $\V{\xi}\in\Rm$,它可以用协变基或逆变基表示:
	\begin{equation}
		\V{\xi}=\left\{\begin{aligned}
			\xi^i\V{g}_i \comma \\
			\xi_i\V{g}^i \fullstop
		\end{aligned}\right.
	\end{equation}
	式中,协变分量与逆变分量满足\emphB{坐标转换关系}:
	\begin{braceEq}
		\xi^i &=\ipb{\V{\xi}}{\V{g}^i}
		=\ipb{\V{\xi}}{g^{ik}\V{g}_k}
		=g^{ik}\ipb{\V{\xi}}{\V{g}_k}
		=g^{ik}\xi_k \comma \\
		\xi_i &=\ipb{\V{\xi}}{\V{g}_i}
		=\ipb{\V{\xi}}{g_{ik}\V{g}^k}
		=g_{ik}\ipb{\V{\xi}}{\V{g}^k}
		=g_{ik}\xi^k \fullstop
	\end{braceEq}
	每一式的第二个等号都用到了\emphB{基向量转换关系},
	见式~\eqref{eq:基向量转换关系1} 和 \eqref{eq:基向量转换关系4}。
	
	现在再来考虑张量的分量。仍以上文中的张量 $\tensor{\Phi}{_i^j_k}
		\coloneq \T{\Phi}\qty(\V{g}_i,\,\V{g}^j,\,\V{g}_k)$ 为例,
	我们想要知道它与张量 $\tensor{\Phi}{^p_q^r} \coloneq
		\T{\Phi}\qty(\V{g}^p,\,\V{g}_q,\,\V{g}^r)$ 之间的关系。
	利用基向量转换关系,可有
	\begin{align}
		\tensor{\Phi}{_i^j_k}
		&\coloneq\T{\Phi}\qty(\V{g}_i,\,\V{g}^j,\,\V{g}_k) \notag \\
		&=\T{\Phi}
			\qty(g_{ip}\V{g}^p,\,g^{jq}\V{g}_q,\,g_{kr}\V{g}^r) \notag
		\intertext{又利用张量的线性性,得}
		&=g_{ip}g^{jq}g_{kr}
			\T{\Phi}\qty(\V{g}^p,\,\V{g}_q,\,\V{g}^r) \notag \\
		&=g_{ip}g^{jq}g_{kr} \tensor{\Phi}{^p_q^r} \fullstop
	\end{align}
	可见,张量的分量与向量的分量类似,其指标升降可通过\emphB{度量}来实现。
	用同样的手法,还可以得到诸如
	$\tensor{\Phi}{^i^j^k}=g^{jp}\tensor{\Phi}{^i_p^k}$、
	$\tensor{\Phi}{^i_j^k}=g_{jp}g^{kq}\tensor{\Phi}{^i^p_k}$
	这样的关系式。
	
\subsection{相对不同基的张量分量之间的关系}
	$\Rm$ 空间中,除了 $\qty{\V{g}_i}^m_{i=1}$ 和相应的
	对偶基 $\qty{\V{g}^i}^m_{i=1}$ 之外,当然还可以有其他的基,
	比如带括号的 $\qty{\V{g}_{(i)}}^m_{i=1}$ 以及对应的
	对偶基 $\qty{\V{g}^{(i)}}^m_{i=1}$。
	前者对应形如 $\tensor{\Phi}{^i_j^k}
		\coloneq \T{\Phi}\qty(\V{g}^i,\,\V{g}_j,\,\V{g}^k)$ 的张量,
	后者则对应带括号的张量,如 $\tensor{\Phi}{^{(p)}_{(q)}^{(r)}} \coloneq
		\T{\Phi}\qty(\V{g}^{(p)},\,\V{g}_{(q)},\,\V{g}^{(r)})$。
	下面我们来探讨这两个张量的关系。
	
	首先来建立基之间的关系。带括号的第 $i$ 个基向量
	$\V{g}_{(i)}$,作为 $\Rm$ 空间中的一个向量,自然可以用另一组基来表示:
	\begin{equation}
		\V{g}_{(i)}=\left\{\begin{aligned}
			\ipb{\V{g}_{(i)}}{\V{g}_k}\,\V{g}^k \comma \\
			\ipb{\V{g}_{(i)}}{\V{g}^k}\,\V{g}_k \fullstop
		\end{aligned}\right.
	\end{equation}
	同理,自然还有它的对偶基:
	\begin{equation}
		\V{g}^{(i)}=\left\{\begin{aligned}
			\ipb{\V{g}^{(i)}}{\V{g}_k}\,\V{g}^k \comma \\
			\ipb{\V{g}^{(i)}}{\V{g}^k}\,\V{g}_k \fullstop
		\end{aligned}\right.
	\end{equation}
	引入记号 $c^k_{(i)} \coloneq \ipb{\V{g}_{(i)}}{\V{g}^k}$
	和 $c^{(i)}_k \coloneq \ipb{\V{g}^{(i)}}{\V{g}_k}$,那么有
	\begin{braceEq}
		\V{g}_{(i)} &= c^k_{(i)}\V{g}_k \comma \\
		\V{g}^{(i)} &= c^{(i)}_k\V{g}^k \fullstop
	\end{braceEq}
	
	容易看出,这两个系数具有如下性质:
	\begin{equation}
		c^{(i)}_k c^k_{(j)} = \KroneckerDelta{i}{j} \fullstop
	\end{equation}
	写成矩阵形式\footnote{%
		通常我们约定上面的标号作为行号,下面的标号作为列号。},为
	\begin{equation}
		\qty[c^{(i)}_k]\qty[c^k_{(j)}]
		=\qty[\KroneckerDelta{j}{i}]=\Mat{I}_m \fullstop
	\end{equation}
	换句话说,两个系数矩阵是互逆的。
	\begin{myProof}
		\begin{align}
			c^{(i)}_k c^k_{(j)}
			&=\ipb{\V{g}^{(i)}}{\V{g}_k}\,c^k_{(j)} \notag
			\intertext{利用内积的线性性,有}
			&=\ipb{\V{g}^{(i)}}{c^k_{(j)} \V{g}_k} \notag
			\intertext{根据 $c^k_{(j)}$ 的定义,得到}
			&=\ipb{\V{g}^{(i)}}{\V{g}_{(j)}} \fullstop
		\end{align}
		带括号的基同样满足对偶关系 \eqref{eq:对偶关系}~式,于是得证。
	\end{myProof}
	
	上面我们用不带括号的基表示了带括号的基。反之也是可以的:
	\begin{braceEq}
		\V{g}_i &= \ipb{\V{g}_i}{\V{g}^{(k)}}{\Rm}\,\V{g}_{(k)}
			=c^{(k)}_i\V{g}_{(k)} \comma \\
		\V{g}^i &= \ipb{\V{g}^i}{\V{g}_{(k)}}{\Rm}\,\V{g}^{(k)}
			=c^i_{(k)}\V{g}^{(k)} \fullstop
	\end{braceEq}
	这样一来,就建立起了不同基之间的转换关系。
	
	现在我们回到张量。根据张量分量的定义,
	\begin{align}
		\tensor{\Phi}{^i_j^k}
		&\coloneq \T{\Phi}\qty(\V{g}^i,\,\V{g}_j,\,\V{g}^k) \notag
		\intertext{利用之前推导的不同基向量之间的转换关系,得}
		&=\T{\Phi}\qty(
			c^i_{(p)}\V{g}^{(p)},\,c^{(q)}_j\V{g}_{(q)},\,
			c^k_{(r)}\V{g}^{(r)}) \notag
		\intertext{由张量的线性性,提出系数:}
		&=c^i_{(p)} c^{(q)}_j c^k_{(r)} \,
			\T{\Phi}\qty(\V{g}^{(p)},\,\V{g}_{(q)},\V{g}^{(r)}) \notag\\
		&=c^i_{(p)} c^{(q)}_j c^k_{(r)} \,
			\tensor{\Phi}{^{(p)}_{(q)}^{(r)}} \fullstop
	\end{align}
	完全类似,还可以有
	\begin{equation}
		\tensor{\Phi}{^{(i)}_{(j)}^{(k)}}
		=c^{(i)}_p c^g_{(j)} c^{(k)}_r \tensor{\Phi}{^p_q^r} \fullstop
	\end{equation}
	
	\blankline
	
	总结一下这两小节得到的结果。
	对于同一组基下的张量分量,其指标升降通过\emphB{度量}来实现;
	对于不同基下的张量分量,其指标转换则通过不同基之间的转换系数来完成。
%	
%	\chapter{张量的运算性质}
%		\section{张量积}
	\emphA{张量积}也叫\emphA{张量并},用符号“$\tp$”表示。
	在 \ref{subsec:张量的表示与简单张量}~小节给出简单张量的定义时,
	实际上就用到了张量积。张量积的定义为:
	\begin{align}
		\forall\,\T{\Phi}\in\Tensors{p},\,\T{\Psi}\in\Tensors{q},
		&\mathrel{\phantom{=}} \T{\Phi}\tp\T{\Psi}
			\in\Tensors{p+q} \notag \\
		&=\qty(\Phi^{i_1 \cdots i_p} \,
				\V{g}_{i_1}\tp\cdots\tp\V{g}_{i_p})
			\tp \qty(\Psi_{j_1 \cdots j_q} \,
				\V{g}^{j_1}\tp\cdots\tp\V{g}^{j_q}) \notag \\
		&\defeq \Phi^{i_1 \cdots i_p} \,
			\Psi_{j_1 \cdots j_q}\,
			\qty(\V{g}_{i_1}\tp\cdots\tp\V{g}_{i_p})
			\tp \qty(\V{g}^{j_1}\tp\cdots
				\tp\V{g}^{j_q}_{\phantom{i_p}}) \fullstop
	\end{align}
	由该定义可以知道,关于简单张量 $\qty(\V{g}_{i_1}\tp\cdots
		\tp\V{g}_{i_p}) \tp \qty(\V{g}^{j_1}\tp\cdots
		\tp\V{g}^{j_q}_{\phantom{i_p}})$,相应的张量分量为
	\begin{equation}
		\tensor{\qty\big(\Phi\tp\Psi)}
			{^{i_1 \cdots i_p}_{j_1 \cdots j_q}} \fullstop
	\end{equation}
	
\section{\texorpdfstring{$e$ 点积}{e 点积}}
	张量的 \emphA{$e$ 点积}可以用符号“$\edp$”表示。
	从这个符号可以看出 $e$ 点积的作用:前 $e$ 个指标缩并,后面的点乘。
	
	对于任意的 $\T{\Phi}\in\Tensors{p},\,
		\T{\Psi}\in\Tensors{q},\,
		e\leqslant\min\qty{p,\,q}\in\natN$,$e$ 点积是这样定义的:
	\begin{align}
		&\mathrel{\phantom{=}} \T{\Phi}\edp\T{\Psi} \notag \\
		&=\qty(\Phi^{i_1 \cdots i_{p-e} i_{p-e+1} \cdots i_p} \,
			\V{g}_{i_1}\tp\cdots\tp\V{g}_{i_{p-e}}
			\tp\hl{\V{g}_{i_{p-e+1}}\tp\cdots\tp\V{g}_{i_p}}
			) \notag \\
		&\mathrel{\phantom{=}}\quad\edp
			\qty(\Psi^{j_1 \cdots j_e j_{e+1} \cdots j_q} \,
			\hl{\V{g}_{j_1}\tp\cdots\tp\V{g}_{j_e}}
			\tp\V{g}_{j_{e+1}}\tp\cdots\tp\V{g}_{j_q}) \notag
		\intertext{把高亮的部分做内积,得到\emphB{度量}:}
		&\defeq\Phi^{i_1 \cdots i_{p-e} i_{p-e+1} \cdots i_p} \,
			\Psi^{j_1 \cdots j_e j_{e+1} \cdots j_q} \notag \\
		&\mathrel{\phantom{=}}\quad\cdotp
			g_{i_{p-e+1} j_1} \cdots g_{i_p j_e} \,
			\qty(\V{g}_{i_1}\tp\cdots\tp\V{g}_{i_{p-e}})
			\tp\qty(\V{g}_{j_{e+1}}\tp\cdots\tp\V{g}_{j_q}) \notag
		\intertext{玩一下“指标升降游戏”(注意有两种结合方式:
			与 $\Phi$ 或 $\Psi$),可得}
		&=\left\{\begin{lgathered}
				\tensor{\Phi}{^{i_1 \cdots i_{p-e}}_{\hl{j_1 \cdots j_e}}} \,
				\Psi^{\hl{j_1 \cdots j_e} j_{e+1} \cdots j_q} \\
				\Phi^{i_1 \cdots i_{p-e} \hl{i_{p-e+1} \cdots i_p}} \,
				\tensor{\Psi}{_{\hl{i_{p-e+1} \cdots i_p}}^{j_{e+1}
					\cdots j_q}}
			\end{lgathered}\right\}
			\qty(\V{g}_{i_1}\tp\cdots\tp\V{g}_{i_{p-e}})
			\tp\qty(\V{g}_{j_{e+1}}\tp\cdots\tp\V{g}_{j_q}) \fullstop
	\end{align}
	最后一步的大括号中,高亮的 $j_1 \cdots j_e$
	和 $i_{p-e+1} \cdots i_p$ 都是哑标,可以通过求和求掉。因此有
	\begin{equation}
		\T{\Phi}\edp\T{\Psi} \in \Tensors{p+q-2e} \fullstop
	\end{equation}
	换句话说,$e$ 点积的作用就是将指标\emphB{哑标化}。
	
	作为一个特殊的应用,接下来我们介绍\emphA{全点积},用符号“\fdp”表示。
	对于任意的 $\T{\Phi},\,\T{\Psi}\in\Tensors{p}$,有
	\begin{align}
		&\mathrel{\phantom{=}} \T{\Phi}\fdp\T{\Psi}
			\defeq \T{\Phi}\edp[p]\T{\Psi} \notag \\
		&=\qty(\Phi^{i_1 \cdots i_p}\,\V{g}_{i_1}\tp\cdots\tp\V{g}_{i_p})
			\edp[p]
			\qty(\Psi^{j_1 \cdots j_p}\,\V{g}_{j_1}\tp\cdots\tp\V{g}_{j_p})
			\notag \\
		&=\Phi^{i_1 \cdots i_p} \, \Psi^{j_1 \cdots j_p} \,
			g_{i_1 j_1} \cdots g_{i_p j_p} \notag \\
		&=\left\{\begin{lgathered}
				\Phi_{j_1 \cdots j_p} \, \Psi^{j_1 \cdots j_p} \\
				\Phi^{i_1 \cdots i_p} \, \Psi_{i_1 \cdots i_p}
			\end{lgathered}\right.
			\in\realR \fullstop
	\end{align}
	可见,全点积将\emphB{全部}指标哑标化。
	
	张量自身和自身的全点积,定义为它的\emphA{范数}:
	\begin{equation}
		\T{\Phi}\fdp\T{\Phi}
		=\Phi^{i_1 \cdots i_p} \, \Phi_{i_1 \cdots i_p}
		\eqcolon \qty|\T{\Phi}|^2_{\Tensors{p}} \fullstop
	\end{equation}
	
\section{叉乘}
	张量的\emphA{叉乘}要求底空间为 $\realR^3$。
	对于任意的 $\T{\Phi}\in\Tensors[\realR^3]{p},\,
	\T{\Psi}\in\Tensors[\realR^3]{q}$,叉乘的定义如下:
	\begin{align}
		&\mathrel{\phantom{=}} \T{\Phi}\cp\T{\Psi} \notag \\
		&=\qty(\Phi^{i_1 \cdots i_{p-1} i_p} \,
				\V{g}_{i_1}\tp\cdots\tp\V{g}_{i_{p-1}}\tp\V{g}_{i_p})
			\cp \qty(\Psi_{j_1 j_2 \cdots j_q} \,
				\V{g}^{j_1}\tp\V{g}^{j_2}\cdots\tp\V{g}^{j_q}) \notag \\
		&\defeq \Phi^{i_1 \cdots i_p} \, \Psi_{j_1 \cdots j_p} \,
			\V{g}_{i_1}\tp\cdots\tp\V{g}_{i_{p-1}}
			\tp\qty(\V{g}_{i_p}\cp\V{g}^{j_1})
			\tp\V{g}^{j_2}\cdots\tp\V{g}^{j_q}
			\in\Tensors[\realR^3]{p+q-1} \fullstop
	\end{align}
	注意到,此时简单张量的维数已经降了一阶。
	
	利用\emphA{Levi-Civita 记号},可以进一步展开上式。
	\begin{align}
		\V{g}_{i_p}\cp\V{g}^{j_1}
		=\LeviCivita{_{i_p}^{j_1}_s}\,\V{g}^s \comma
	\end{align}
	式中的
	\begin{equation}
		\LeviCivita{_{i_p}^{j_1}_s}
		=\det[\V{g}_{i_p},\,\V{g}^{j_1},\,\V{g}_s] \fullstop
	\end{equation}
	于是
	\begin{equation}
		\T{\Phi}\cp\T{\Psi} \,
		=\LeviCivita{_{i_p}^{j_1}_s}\,
			\Phi^{i_1 \cdots i_p} \, \Psi_{j_1 \cdots j_p}
			\V{g}_{i_1}\tp\cdots\tp\V{g}_{i_{p-1}} \tp\V{g}^s
			\tp\V{g}^{j_2}\cdots\tp\V{g}^{j_q} \fullstop
	\end{equation}
	
	下面我们再来类比地定义一种混合积“$\edp[\cp]$”。
	对于任意的 $\T{\Phi},\,\T{\Psi}\in\Tensors{3}$,定义
	\begin{align}
		\T{\Phi}\edp[\cp]\T{\Psi}
		&=\qty(\Phi^{ijk}\,\V{g}_i\tp\V{g}_j\tp\V{g}_k)
			\edp[\cp]\qty(\Psi_{pqr}\,\V{g}^p\tp\V{g}^q\tp\V{g}^r)\notag \\
		&\defeq \Phi^{ijk}\,\Psi_{pqr} \,
			\KroneckerDelta{q}{j} \,
			\V{g}_i\tp\qty(\V{g}_k\cp\V{g}^p)\tp\V{g}^r \notag
		\intertext{缩并掉 Kronecker δ,
			同时利用 Levi-Civita 记号展开叉乘项,可有}
		&=\LeviCivita{_k^p_s}\,\Phi^{ijk}\,\Psi_{pjr}\,
			\V{g}_i\tp\V{g}^s\tp\V{g}^r \comma
	\end{align}
	式中的
	\begin{equation}
		\LeviCivita{_k^p_s}=\det[\V{g}_k,\,\V{g}^p,\,\V{g}_s] \fullstop
	\end{equation}
	
	对于这种混合积,并没有一般的约定。不同的研究者往往会采用不同的写法及表示。
	
\section{置换(一)}
	本节主要介绍\emphA{置换运算}的定义及相关概念,
	这将使我们暂时离开张量运算的主线。
	
	置换运算实际上是一种交换位置或者改变次序的运算。
	之后我们还将引入针对张量的\emph{置换算子},它是外积运算和外微分运算的基础。
	这些运算是现代张量分析与微分几何的支柱。
	
\subsection{置换的定义}
	我们从一个例子开始。下面是一个 $2 \times 7$ 的“矩阵”:
	\begin{equation}
		\Perm{\sigma}=\mqty[
			\circNum{1} & \circNum{2} & \circNum{3} & \circNum{4} &
				\circNum{5} & \circNum{6} & \circNum{7} \\
			\circNum{7} & \circNum{4} & \circNum{5} & \circNum{1} &
				\circNum{6} & \circNum{2} & \circNum{3}
		] \fullstop
		\label{eq:置换序号定义}
	\end{equation}
	矩阵里面的每一个数字表示一个位置。可以想象成 7 把椅子,
	先是按第一行的顺序依次排列,再按照第二行的顺序打乱,重新排列。
	于是这就成为一个\emphA{7 阶置换}。这个定义等价于
	\begin{mySubEq}
		\begin{gather}
			\Perm{\sigma}=\mqty*(
				4 & 9 & 2 & 7 & 5 & 8 & 3 \\
				3 & 7 & 5 & 4 & 8 & 9 & 2
			) \comma \label{eq:置换元素表示_数字}
			\intertext{自然也等价于}
			\Perm{\sigma}=\mqty*(
				\spadesuit & \heartsuit & \diamondsuit & \clubsuit &
					\varspadesuit & \varheartsuit & \vardiamondsuit \\
				\vardiamondsuit & \clubsuit & \varspadesuit & \spadesuit &
					\varheartsuit & \heartsuit & \diamondsuit
			) \comma \label{eq:置换元素表示_符号}
		\end{gather}
	\end{mySubEq}
	当然,换用任何元素也都是可以的。
	
	通常我们用方括号表示置换的\emphA{序号定义},即标号的排列轮换;
	用圆括号表示\emphA{元素定义},即标号对应元素的轮换。
	
\subsection{置换的符号}
	接着来定义置换的\emphA{符号} $\sgn\Perm{\sigma}$。
	这里我们把每次交换两个数字称为一次“操作”。
	如果经过\emphB{偶数次}“操作”,可以把经置换后的序列恢复为原来的顺序,
	那么该置换的符号 $\sgn\Perm{\sigma} = 1$;
	而如果经过\emphB{奇数次}“操作”才可以复原,则 $\sgn\Perm{\sigma}=-1$。
	若用一个式子表示,则为
	\begin{equation}
		\sgn\Perm{\sigma} = (-1)^n \comma
	\end{equation}
	其中的 $n$ 是恢复原本顺序所需“操作”的次数.
	
	下面我们以式~\eqref{eq:置换序号定义} 所定义的 $\Perm{\sigma}$ 为例,
	演示求置换符号的过程。这里的关键是通过两两交换,
	按如下步骤把式~\eqref{eq:置换元素表示_符号} 的第二行变换成第一行:
	\begin{gather*}
		\mqty{
			\hl{\vardiamondsuit} & \hlw{\clubsuit} & \hlw{\varspadesuit} &
				\hl{\spadesuit} & \hlw{\varheartsuit} & \hlw{\heartsuit} &
				\hlw{\diamondsuit}
		} \\
		\mqty{ & & & \Downarrow & & & } \\
		\mqty{
			\hl[pink]{\spadesuit} & \hl{\clubsuit} & \hlw{\varspadesuit} &
				\hl[pink]{\vardiamondsuit} & \hlw{\varheartsuit} &
				\hl{\heartsuit} & \hlw{\diamondsuit}
		} \\
		\mqty{ & & & \Downarrow & & & } \\
		\mqty{
			\hlw{\spadesuit} & \hl[pink]{\heartsuit} & \hl{\varspadesuit} &
				\hlw{\vardiamondsuit} & \hlw{\varheartsuit} &
				\hl[pink]{\clubsuit} & \hl{\diamondsuit}
		} \\
		\mqty{ & & & \Downarrow & & & } \\
		\mqty{
			\hlw{\spadesuit} & \hlw{\heartsuit} & \hl[pink]{\diamondsuit} &
				\hl{\vardiamondsuit} & \hlw{\varheartsuit} & \hl{\clubsuit} &
				\hl[pink]{\varspadesuit}
		} \\
		\mqty{ & & & \Downarrow & & & } \\
		\mqty{
			\hlw{\spadesuit} & \hlw{\heartsuit} & \hlw{\diamondsuit} &
				\hl[pink]{\clubsuit} & \hl{\varheartsuit} &
				\hl[pink]{\vardiamondsuit} & \hl{\varspadesuit}
		} \\
		\mqty{ & & & \Downarrow & & & } \\
		\phantom{\mspace{10mu}}\mqty{
			\hlw{\spadesuit} & \hlw{\heartsuit} & \hlw{\diamondsuit} &
				\hlw{\clubsuit} & \hl[pink]{\varspadesuit} &
				\hl{\vardiamondsuit} &
				\hl{\hl[pink]{\varheartsuit}}
		} \\
		\mqty{ & & & \Downarrow & & & } \\
		\mqty{
			\hlw{\spadesuit} & \hlw{\heartsuit} & \hlw{\diamondsuit} &
				\hlw{\clubsuit} & \hlw{\varspadesuit} &
				\hl[pink]{\varheartsuit} & \hl[pink]{\vardiamondsuit}
		}
	\end{gather*}
	一共进行了 6 次两两交换,因此 $\sgn\Perm{\sigma}=1$。
	
\subsection{置换的复合}
	再定义一个置换
	\begin{equation}
		\Perm{\tau}=\mqty[
			1 & 2 & 3 & 4 & 5 & 6 & 7 \\
			5 & 1 & 7 & 3 & 6 & 4 & 2
		] \fullstop
	\end{equation}
	注意这里用了方括号,因此它是一个\emphB{序号定义}。
	方便起见,以后的序号我们都只用不带圈的普通数字表示。
	考虑之前定义的置换
	\begin{equation}
		\Perm{\sigma}=\mqty[
			1 & 2 & 3 & 4 & 5 & 6 & 7 \\
			7 & 4 & 5 & 1 & 6 & 2 & 3
		] \comma
	\end{equation}
	则 $\Perm{\tau}$ 与 $\Perm{\sigma}$ 的复合
	\begin{equation}
		\Perm{\tau}\comp\Perm{\sigma}=
		\qty(\begin{array}{@{}ccccccc@{}}
			\dicei & \diceii & \diceiii & \diceiv & \dicev &
				\dicevi & \circledtwodots \\
			\circledtwodots & \diceiv & \dicev & \dicei & \dicevi &
				\diceii & \diceiii \\
			\hdashline
			\dicevi & \circledtwodots & \diceiii & \dicev & \diceii &
				\dicei & \diceiv
		\end{array})
		\quad\mqty{\\ \leftarrow\Perm{\sigma} \\ \leftarrow\Perm{\tau}}
	\end{equation}
	与函数、线性变换等的复合类似,这里也用小圆圈“$\comp$”表示置换的复合。
	
	假设经过置换 $\Perm{\sigma}$、$\Perm{\tau}$ 作用后得到的序列,
	分别需要 $p$ 次和 $q$ 次两两交换才能复原为原来的序列。
	那么很显然,经过复合置换 $\Perm{\tau}\comp\Perm{\sigma}$ 作用后的序列,
	经过 $q+p$ 次两两交换也一定可以复原。因此,复合置换的符号
	\begin{equation}
		\sgn\qty(\Perm{\tau}\comp\Perm{\sigma})
		=(-1)^{q+p}=(-1)^q \cdotp (-1)^p
		=\sgn\Perm{\tau}\cdotp\sgn\Perm{\sigma} \fullstop
	\end{equation}
	
\subsection{逆置换}
	逆置换 $\Perm{\sigma}^{-1}$ 的定义为
	\begin{equation}
		\Perm{\sigma}^{-1}\comp\Perm{\sigma} = \Id \comma
	\end{equation}
	其中的“$\Id$”是\emphA{恒等映照}。
	
	仍然使用式~\eqref{eq:置换元素表示_符号}:
	\begin{equation}
		\Perm{\sigma}=\mqty*(
			\spadesuit & \heartsuit & \diamondsuit & \clubsuit &
				\varspadesuit & \varheartsuit & \vardiamondsuit \\
			\vardiamondsuit & \clubsuit & \varspadesuit & \spadesuit &
				\varheartsuit & \heartsuit & \diamondsuit
		) \comma
	\end{equation}
	那么自然有
	\begin{equation}
		\Perm{\sigma}^{-1}=\mqty*(
			\vardiamondsuit & \clubsuit & \varspadesuit & \spadesuit &
				\varheartsuit & \heartsuit & \diamondsuit \\
			\spadesuit & \heartsuit & \diamondsuit & \clubsuit &
				\varspadesuit & \varheartsuit & \vardiamondsuit
		) \fullstop
	\end{equation}
	显然,我们有 $\Perm{\sigma}^{-1}\comp\Perm{\sigma} = \Id$。
	
	回忆一下逆矩阵的定义。矩阵 $\Mat{A}$ 的逆 $\Mat{A}^{-1}$ 既要满足
	$\Mat{A}^{-1}\Mat{A}=\Mat{I}$,又要满足
	$\Mat{A}\Mat{A}^{-1}=\Mat{I}$。对于置换也是如此,
	因此我们需要检查 $\Perm{\sigma}\comp\Perm{\sigma}^{-1}$:\footnote{%
		该式中的数字角标用来澄清原始序号。}
	\begin{equation}
		\Perm{\sigma}\comp\Perm{\sigma}^{-1}=
		\qty(\begin{array}{@{}ccccccc@{}}
			\vardiamondsuit & \clubsuit & \varspadesuit & \spadesuit &
				\varheartsuit & \heartsuit & \diamondsuit \\
			\spadesuit_1 & \heartsuit_2 & \diamondsuit_3 & \clubsuit_4 &
				\varspadesuit_5 & \varheartsuit_6 & \vardiamondsuit_7 \\
			\hdashline
			\vardiamondsuit_7 & \clubsuit_4 & \varspadesuit_5 &
				\spadesuit_1 & \varheartsuit_6 &
				\heartsuit_2 & \diamondsuit_3
		\end{array})
		\quad\mqty{
			\\ \leftarrow\Perm{\sigma}^{-1} \\
			\leftarrow\Perm{\sigma}\phantom{^{-1}}
		}
	\end{equation}
	可见的确有 $\Perm{\sigma}\comp\Perm{\sigma}^{-1}=\Id$。
	
	另外,由于恒等映照 $\Id$ 作用后序列不发生变化,
	复原所需的交换次数为 0,因此
	\begin{equation}
		\sgn\Id=(-1)^0=1 \fullstop
	\end{equation}
	而根据定义,
	\begin{equation}
		\Id=\Perm{\sigma}^{-1}\comp\Perm{\sigma} \comma
	\end{equation}
	故有
	\begin{equation}
		\sgn\Perm{\sigma} \cdotp \sgn\Perm{\sigma}^{-1} = 1 \fullstop
	\end{equation}
	由此,可以推知
	\begin{equation}
		\sgn\Perm{\sigma}=\sgn\Perm{\sigma}^{-1} \comma
	\end{equation}
	即置换与它的逆具有\emphB{相同}的符号。
	
\section{置换(二)}
	本节将介绍置换运算的基本性质。
	
\subsection{置换的穷尽}
	先要做一点铺垫。设有序数组
	\begin{equation*}
		\qty{i_1,\,i_2,\,\cdots,\,i_r}
	\end{equation*}
	经置换 $\Perm{\sigma}$ 作用后成为
	\begin{equation*}
		\qty{\Perm{\sigma}(i_1),\,\Perm{\sigma}(i_2),\,
			\cdots,\,\Perm{\sigma}(i_r)} \comma
	\end{equation*}
	则根据之前的元素定义(圆括号),可以把 $\Perm{\sigma}$ 记为
	\begin{equation}
		\Perm{\sigma}=\mqty*(
			i_1 & i_2 & \cdots & i_r \\
			\Perm{\sigma}(i_1) & \Perm{\sigma}(i_2) &
				\cdots & \Perm{\sigma}(i_r)
		)\fullstop
	\end{equation}
	每次置换都将得到一个有序数组。把它们组合到一起,就可以得到集合
	\begin{equation}
		\set[\bigg]
			{\qty(\Perm{\sigma}(i_1),\,\Perm{\sigma}(i_2),\,
				\cdots,\,\Perm{\sigma}(i_r))}
			{\forall\,\Perm{\sigma}\in\Permutations{r}} \fullstop
	\end{equation}
	其中的 $\Permutations{r}$ 表示 $r$ 阶置换的全体。
	根据排列组合原理,$r$ 阶置换的总数等于 $r$ 个元素的\emphB{全排列数}。
	即该集合共有 $r!$ 个元素。
	
	下面我们要证明
	\begin{mySubEq}
		\begin{align}
			&\mathrel{\phantom{=}}\set[\bigg]
				{\qty(\Perm{\sigma}(i_1),\,\Perm{\sigma}(i_2),\,
					\cdots,\,\Perm{\sigma}(i_r))}
				{\forall\,\Perm{\sigma}\in\Permutations{r}} \notag \\
			%
			&=\set[\bigg]
				{\qty(
					\Perm{\tau}\comp\Perm{\sigma}(i_1),\,
					\Perm{\tau}\comp\Perm{\sigma}(i_2),\,\cdots,\,
					\Perm{\tau}\comp\Perm{\sigma}(i_r) )}
				{\forall\,\Perm{\sigma},\,\Perm{\tau}\in\Permutations{r}}
				\label{eq:置换的穷尽_复合1} \\
			&=\set[\bigg]
				{\qty(
					\Perm{\sigma}\comp\Perm{\tau}(i_1),\,
					\Perm{\sigma}\comp\Perm{\tau}(i_2),\,\cdots,\,
					\Perm{\sigma}\comp\Perm{\tau}(i_r) )}
				{\forall\,\Perm{\sigma},\,\Perm{\tau}\in\Permutations{r}}
				\label{eq:置换的穷尽_复合2} \\
			&=\set[\bigg]
				{\qty(
					\Perm{\sigma}^{-1}(i_1),\,
					\Perm{\sigma}^{-1}(i_2),\,\cdots,\,
					\Perm{\sigma}^{-1}(i_r))}
				{\forall\,\Perm{\sigma}\in\Permutations{r}} 
				\label{eq:置换的穷尽_逆} \fullstop
		\end{align}
	\end{mySubEq}
	\colorbox{pink}{这说明置换构成了置换群。?}
	
	\begin{myProof}
		证明的思路是说明集合互相包含。
		
		对于式~\eqref{eq:置换的穷尽_复合1},
		右边的 $\Perm{\tau}\comp\Perm{\sigma}$ 也是一个 $r$ 阶置换,
		自然符合左边集合的定义,因此 $\text{右边}\subset\text{左边}$。
		由于这一步是相当显然的,以下的几个证明我们将略去该步。
		另一方面,左边的 $\Perm{\sigma}$ 可以表示成
		\begin{equation}
			\Perm{\sigma}
			=\Id\comp\Perm{\sigma}
			=\qty(\Perm{\tau}\comp\Perm{\tau}^{-1}) \comp\Perm{\sigma}
			=\Perm{\tau}\comp \qty(\Perm{\tau}^{-1}\comp\Perm{\sigma})
			\comma
		\end{equation}
		这就是右边集合的定义,因此 $\text{左边}\subset\text{右边}$。
		故可证得等式成立。
		
		对于式~\eqref{eq:置换的穷尽_复合2},我们有
		\begin{equation}
			\Perm{\sigma}
			=\Perm{\sigma}\comp\Id
			=\Perm{\sigma}\comp \qty(\Perm{\tau}^{-1}\comp\Perm{\tau})
			=\qty(\Perm{\sigma}\comp\Perm{\tau}^{-1}) \comp\Perm{\tau}
			\comma
		\end{equation}
		它符合了右边集合的定义,因此 $\text{左边}\subset\text{右边}$。
		于是等式成立。
		
		对于式~\eqref{eq:置换的穷尽_逆},我们有
		\begin{equation}
			\Perm{\sigma}=\qty(\Perm{\sigma}^{-1})^{-1} \comma
		\end{equation}
		它符合了右边集合的定义,因此 $\text{左边}\subset\text{右边}$。
		于是等式成立。
	\end{myProof}
	
\subsection{数组元素的乘积} \label{subsec:数组元素的乘积}
	设有序数组 $\qty{i_1,\,i_2,\,\cdots,\,i_r}$、
	$\qty{j_1,\,j_2,\,\cdots,\,j_r}$ 和 $\qty{k_1,\,k_2,\,\cdots,\,k_r}$
	经 $r$ 阶置换 $\Perm{\sigma}$ 作用后分别成为
	$\qty{\Perm{\sigma}(i_1),\,\Perm{\sigma}(i_2),\,
		\cdots,\,\Perm{\sigma}(i_r)}$、
	$\qty{\Perm{\sigma}(j_1),\,\Perm{\sigma}(j_2),\,
		\cdots,\,\Perm{\sigma}(j_r)}$ 和
	$\qty{\Perm{\sigma}(k_1),\,\Perm{\sigma}(k_2),\,
		\cdots,\,\Perm{\sigma}(k_r)}$,也就是说
	\begin{equation}
		\Perm{\sigma}=\mqty*(
			i_1 & i_2 & \cdots & i_r \\
			\Perm{\sigma}(i_1) & \Perm{\sigma}(i_2) &
				\cdots & \Perm{\sigma}(i_r) )
		=\mqty*(
			j_1 & j_2 & \cdots & j_r \\
			\Perm{\sigma}(j_1) & \Perm{\sigma}(j_2) &
				\cdots & \Perm{\sigma}(j_r) )
		=\mqty*(
			k_1 & k_2 & \cdots & k_r \\
			\Perm{\sigma}(k_1) & \Perm{\sigma}(k_2) &
				\cdots & \Perm{\sigma}(k_r) ) \fullstop
	\end{equation}
	我们有如下结论:
	\begin{equation}
		\forall\,\Perm{\sigma}\in\Permutations{r}\, ,\quad
		A_{i_1 j_1 k_1} A_{i_2 j_2 k_2} \cdots
			A_{i_r j_r k_r}
		=A_{\Perm{\sigma}(i_1)\,\Perm{\sigma}(j_1)\,\Perm{\sigma}(k_1)}
			A_{\Perm{\sigma}(i_2)\,\Perm{\sigma}(j_2)\,\Perm{\sigma}(k_2)}
			\cdots
			A_{\Perm{\sigma}(i_r)\,\Perm{\sigma}(j_r)\,\Perm{\sigma}(k_r)}
			\comma
	\end{equation}
	式中的 $A_{ijk}$ 表示三维数组 $\Mat{A}$ 的一个元素,其指标为 $ijk$。
	
	下面通过一个例子来说明这一条性质。还是用式~\eqref{eq:置换元素表示_数字} 和
	\eqref{eq:置换元素表示_符号} 所定义的置换 $\Perm{\sigma}$:
	\begin{equation}
		\Perm{\sigma}=\mqty*(
			4 & 9 & 2 & 7 & 5 & 8 & 3 \\
			3 & 7 & 5 & 4 & 8 & 9 & 2 )
		=\mqty*(
			\spadesuit & \heartsuit & \diamondsuit & \clubsuit &
				\varspadesuit & \varheartsuit & \vardiamondsuit \\
			\vardiamondsuit & \clubsuit & \varspadesuit & \spadesuit &
				\varheartsuit & \heartsuit & \diamondsuit ) \fullstop
				\label{eq:置换元素表示_数组元素的乘积举例}
	\end{equation}
	随意写出一个数组元素乘积:
	\begin{equation}
		A_{379}A_{264}A_{157}A_{483}A_{698}
		A_{\diamondsuit\clubsuit\heartsuit}
		A_{\vardiamondsuit\varspadesuit\varheartsuit} \fullstop
		\label{eq:数组元素乘积举例}
	\end{equation}
	三组下标分别为
	\begin{equation}
		\left\{\begin{lgathered}
			3,\,2,\,1,\,4,\,6,\,\diamondsuit,\,\vardiamondsuit;\\
			7,\,6,\,5,\,8,\,9,\,\clubsuit,\,\varspadesuit;\\
			9,\,4,\,7,\,3,\,8,\,\heartsuit,\,\varheartsuit.\\
		\end{lgathered}
		\right.
	\end{equation}
	考虑 $\Perm{\sigma}$ 的\emphB{序号定义}式~\eqref{eq:置换序号定义}:
	\begin{equation}
		\Perm{\sigma}=\mqty[
			1 & 2 & 3 & 4 & 5 & 6 & 7 \\
			7 & 4 & 5 & 1 & 6 & 2 & 3
		] \fullstop
	\end{equation}
	所谓序号只是位置的抽象表示,而不代表任何真实的元素。
	请记住:置换始终是\emphB{位置}的变换,而非\emphB{元素}的变换,
	不要被式~\eqref{eq:置换元素表示_数组元素的乘积举例} 给迷惑了。
	把 $\Perm{\sigma}$ 作用在这三组下标上,可得
	\begin{equation}
		\left\{\begin{lgathered}
			\vardiamondsuit,\,4,\,6,\,3,\,\diamondsuit,\,2,\,1;\\
			\varspadesuit,\,8,\,9,\,7,\,\clubsuit,\,6,\,5;\\
			\varheartsuit,\,3,\,8,\,9,\,\heartsuit,\,4,\,7.\\
		\end{lgathered}
		\right.
	\end{equation}
	于是之前的数组元素乘积就变成了
	\begin{equation}
		A_{\vardiamondsuit\varspadesuit\varheartsuit}
		A_{483}A_{698}A_{379}
		A_{\diamondsuit\clubsuit\heartsuit}
		A_{264}A_{157} \fullstop
	\end{equation}
	比对一下各元素,可见与式~\eqref{eq:数组元素乘积举例} 的确是完全一样的。
	
\subsection{哑标的穷尽} \label{subsec:哑标的穷尽}
	考虑如下集合:
	\begin{equation}
		\set[\bigg]
		{\qty(i_1,\,i_2,\,\cdots,\,i_r)}
		{\qty{i_1,\,i_2,\,\cdots,\,i_r}
			\text{\ 可取\ } 1,\,2,\,\cdots,\,m}
		\fullstop
	\end{equation}
	每个 $i_k$ 都有 $m$ 种取法,而 $i_k$ 又有 $r$ 个,
	因此该集合一共有 $m^r$ 元素。我们有
	\begin{mySubEq}
		\begin{align}
			\forall\,\Perm{\sigma}\in\Permutations{r}\, ,
			&\mathrel{\phantom{=}}\set[\bigg]
			{\qty(i_1,\,i_2,\,\cdots,\,i_r)}
			{\qty{i_1,\,i_2,\,\cdots,\,i_r}
				\text{\ 可取\ } 1,\,2,\,\cdots,\,m} \notag \\
			&=\set[\bigg]
			{\qty(\Perm{\sigma}(i_1),\,\Perm{\sigma}(i_2),\,
				\cdots,\,\Perm{\sigma}(i_r) )}
			{\qty{i_1,\,i_2,\,\cdots,\,i_r}
				\text{\ 可取\ } 1,\,2,\,\cdots,\,m}
			\label{eq:哑标的穷尽_置换} \\
			&=\set[\bigg]
			{\qty(\Perm{\sigma}^{-1}(i_1),\,\Perm{\sigma}^{-1}(i_2),\,
				\cdots,\,\Perm{\sigma}^{-1}(i_r) )}
			{\qty{i_1,\,i_2,\,\cdots,\,i_r}
				\text{\ 可取\ } 1,\,2,\,\cdots,\,m} \fullstop
			\label{eq:哑标的穷尽_逆置换}
		\end{align}
	\end{mySubEq}
	这里,$i_k$ 起的就是\emphB{哑标}的作用。
	
	\begin{myProof}
		无论怎样置换,$\Perm{\sigma}(i_k)$ 都是 $1,\,2,\,\cdots,\,m$ 中的数。
		因此,对于 $\forall\,\Perm{\sigma}\in\Permutations{r}$,
		\begin{equation}
			\qty(\Perm{\sigma}(i_1),\,\Perm{\sigma}(i_2),\,
				\cdots,\,\Perm{\sigma}(i_r) )
			\in\set[\bigg]
				{\qty(i_1,\,i_2,\,\cdots,\,i_r)}
				{\qty{i_1,\,i_2,\,\cdots,\,i_r}
					\text{\ 可取\ } 1,\,2,\,\cdots,\,m} \comma
		\end{equation}
		即
		\begin{align}
			&\mathrel{\phantom{\subset}}\set[\bigg]
			{\qty(\Perm{\sigma}(i_1),\,\Perm{\sigma}(i_2),\,
				\cdots,\,\Perm{\sigma}(i_r) )}
			{\qty{i_1,\,i_2,\,\cdots,\,i_r}
				\text{\ 可取\ } 1,\,2,\,\cdots,\,m} \notag \\
			&\subset\set[\bigg]
			{\qty(i_1,\,i_2,\,\cdots,\,i_r)}
			{\qty{i_1,\,i_2,\,\cdots,\,i_r}
				\text{\ 可取\ } 1,\,2,\,\cdots,\,m} \fullstop
		\end{align}
		另一方面,由于 $\Id=\Perm{\sigma}^{-1}\comp\Perm{\sigma}$,即
		\begin{equation}
			\qty(i_1,\,i_2,\,\cdots,\,i_r)
			=\qty(\Perm{\sigma}^{-1}\comp\Perm{\sigma}(i_1),\,
				\Perm{\sigma}^{-1}\comp\Perm{\sigma}(i_2),\,
				\cdots,\,\Perm{\sigma}^{-1}\comp\Perm{\sigma}(i_r) ) \comma
		\end{equation}
		而进行一次逆置换仍然使得元素不离开原有的范围,也就是说
		\begin{equation}
			\qty(i_1,\,i_2,\,\cdots,\,i_r)
			\in\set[\bigg]
				{\qty(\Perm{\sigma}(i_1),\,\Perm{\sigma}(i_2),\,
					\cdots,\,\Perm{\sigma}(i_r) )}
				{\qty{i_1,\,i_2,\,\cdots,\,i_r}
					\text{\ 可取\ } 1,\,2,\,\cdots,\,m} \comma
		\end{equation}
		即
		\begin{align}
			&\mathrel{\phantom{\subset}}\set[\bigg]
			{\qty(i_1,\,i_2,\,\cdots,\,i_r)}
			{\qty{i_1,\,i_2,\,\cdots,\,i_r}
				\text{\ 可取\ } 1,\,2,\,\cdots,\,m} \notag \\
			&\subset\set[\bigg]
			{\qty(\Perm{\sigma}(i_1),\,\Perm{\sigma}(i_2),\,
				\cdots,\,\Perm{\sigma}(i_r) )}
			{\qty{i_1,\,i_2,\,\cdots,\,i_r}
				\text{\ 可取\ } 1,\,2,\,\cdots,\,m} \fullstop
		\end{align}
		两个集合互相包含,也就证得了式~\eqref{eq:哑标的穷尽_置换}。
		
		用相同的方法也可证得关于逆置换的 \eqref{eq:哑标的穷尽_逆置换}~式,
		此处从略。
	\end{myProof}
	
\section{置换(三)}
	本节将给出
	
\section{置换(四)}
	本节将回归张量运算的主线,引入\emphA{置换算子}。
	
\subsection{置换算子;对称张量与反对称张量}
	对于任意的置换 $\Perm{\sigma}\in\Permutations{r}$,定义\emphA{置换算子}
	\begin{equation}
		\mdef{\opPerm}
			{\Tensors{r}\ni\T{\Phi}}
			{\opPerm(\T{\Phi})\in\Tensors{r}} \comma
	\end{equation}
	式中
	\begin{equation}
		\opPerm(\T{\Phi})\qty(\V{u}_1,\,\V{u}_2,\,\cdots,\,\V{u}_r)
		\defeq\T{\Phi}\qty(\V{u}_{\Perm{\sigma}(1)},\,
			\V{u}_{\Perm{\sigma}(2)},\,\cdots,\,
			\V{u}_{\Perm{\sigma}(r)})
		\in\realR \fullstop
	\end{equation}
	这里的“$\cdots\in\realR$”是由于张量的定义:\emphB{多重线性函数}。
	
	如果我们的置换
	\begin{equation}
		\Perm{\sigma}=\mqty*(
			i_1 & i_2 & \cdots & i_r \\
			\Perm{\sigma}(i_1) & \Perm{\sigma}(i_1) & \cdots
				& \Perm{\sigma}(i_1)
		) \comma
	\end{equation}
	那么对应的置换算子将满足
	\begin{equation}
		\opPerm(\T{\Phi})\qty(\V{u}_{i_1},\,\V{u}_{i_2},\,
			\cdots,\,\V{u}_{i_r})
		\defeq\T{\Phi}\qty(\V{u}_{\Perm{\sigma}(i_1)},\,
			\V{u}_{\Perm{\sigma}(i_2)},\,\cdots,\,
			\V{u}_{\Perm{\sigma}(i_r)}) \fullstop
	\end{equation}
	
	\blankline
	
	有了置换算子,我们就可以来定义\emphA{对称张量}和\emphA{反对称张量}。
	对称张量的全体记为 $\Sym$,反对称张量的全体记为 $\Skw$。
	如果以 $\Rm$ 为底空间,
	又分别可以记为 $\SymTensors{r}$ 和 $\SkwTensors{r}$。
	
	对于任意的 $\T{\Phi}\in\Tensors{r}$,如果
	\begin{equation}
		\opPerm(\T{\Phi})=\T{\Phi} \comma
	\end{equation}
	则称 $\T{\Phi}$ 为\emphB{对称张量},
	即 $\T{\Phi}\in\Sym \text{\ 或\ } \SymTensors{r}$;如果
	\begin{equation}
		\opPerm(\T{\Phi})=\sgn\Perm{\sigma}\cdotp\T{\Phi} \comma
	\end{equation}
	则称 $\T{\Phi}$ 为\emphB{反对称张量},
	即 $\T{\Phi}\in\Skw \text{\ 或\ } \SkwTensors{r}$。
	
\subsection{置换算子的表示}
	根据上文给出的定义,我们有
	\begin{equation}
		\opPerm(\T{\Phi})\qty(\V{u}_{i_1},\,\cdots,\,\V{u}_{i_r})
		\defeq\T{\Phi}\qty(\V{u}_{\Perm{\sigma}(i_1)},\,\cdots,\,
			\V{u}_{\Perm{\sigma}(i_r)}) \fullstop
	\end{equation}
	首先回忆一下 \ref{subsec:张量的表示与简单张量}~小节中张量的表示:
	选一组基(协变、逆变均可),然后把张量用这组基表示。于是
	\begin{align}
		&\mathrel{\phantom{=}}\opPerm(\T{\Phi})
			\qty(\V{u}_{i_1},\,\cdots,\,\V{u}_{i_r})
		=\T{\Phi}\qty(\V{u}_{\Perm{\sigma}(i_1)},\,\cdots,\,
			\V{u}_{\Perm{\sigma}(i_r)}) \notag
		\intertext{把向量用协变基表示:}
		&=\T{\Phi}\qty(u^{i_1}_{\Perm{\sigma}(i_1)}\,\V{g}_{i_1},\,
			\cdots,\,u^{i_r}_{\Perm{\sigma}(i_r)}\,\V{g}_{i_r}) \notag
		\intertext{根据张量的线性性,提出系数:}
		&=\T{\Phi}\qty(\V{g}_{i_1},\,\cdots,\,\V{g}_{i_r}) \cdotp
			\qty(u^{i_1}_{\Perm{\sigma}(i_1)} \cdots
				u^{i_r}_{\Perm{\sigma}(i_r)}) \notag
		\intertext{前半部分可以用张量分量表示;
			而后半部分是一组逆变分量,可以写成内积的形式}
		&=\tensor{\Phi}{_{i_1 \cdots i_p}}
			\qty[\ipb{\V{u}_{\Perm{\sigma}(i_1)}}{\V{g}^{i_1}}\cdots
				\ipb{\V{u}_{\Perm{\sigma}(i_r)}}{\V{g}^{i_r}}]
		\addtocounter{equation}{1}
		\tag{\theequation*}
		\label{eq:置换算子的表示_中间步骤}
		\intertext{注意到方括号中的其实是简单张量的定义,这就有}
		&=\tensor{\Phi}{_{i_1 \cdots i_p}}
			\V{g}^{i_1}\tp\cdots\tp\V{g}^{i_r}
			\qty(\V{u}_{\Perm{\sigma}(i_1)},\,\cdots,\,
				\V{u}_{\Perm{\sigma}(i_r)}) \fullstop
		\addtocounter{equation}{-1}
	\end{align}
	最后一步仍然没能回到 $\qty(\V{u}_{i_1},\,\cdots,\,\V{u}_{i_r})$,
	因此以上推导只是简单地展开了 $\T{\Phi}$,并没有获得实质性的结果。
	
	然而,只要稍作改动,情况就会大不相同。
	考虑一下 \ref{subsec:数组元素的乘积}~小节中置换运算%
	有关\emphB{数组元素乘积}的性质:
	\begin{equation}
		\forall\,\Perm{\tau}\in\Permutations{r},\quad
		A_{i_1 j_1} \cdots A_{i_r j_r}
		=A_{\Perm{\tau}(i_1) \Perm{\tau}(j_1)} \cdots
			A_{\Perm{\tau}(i_r) \Perm{\tau}(j_r)} \comma
		\label{eq:置换算子的表示推导_置换性质}
	\end{equation}
	式中
	\begin{equation}
		\Perm{\tau}
		=\mqty*(
			i_1 & \cdots & i_r \\
			\Perm{\tau}(i_1) & \cdots & \Perm{\tau}(i_r) )
		=\mqty*(
			j_1 & \cdots & j_r \\
			\Perm{\tau}(j_1) & \cdots & \Perm{\tau}(j_r) ) \fullstop
	\end{equation}
	由此可以看出,式~\eqref{eq:置换算子的表示_中间步骤} 方括号中的部分
	其实是由 $\Perm{\sigma}(i_k)$ 和 $i_k$ 两套指标确定的一组数:
	\begin{equation}
		A_{\Perm{\sigma}(i_k)\,i_k}
		=\ipb{\V{u}_{\Perm{\sigma}(i_k)}}{\V{g}^{i_k}} \semicomma
	\end{equation}
	另一方面,显然有 $\Perm{\sigma}^{-1}\in\Permutations{r}$。于是
	\begin{align}
		&\mathrel{\phantom{=}}\opPerm(\T{\Phi})
			\qty(\V{u}_{i_1},\,\cdots,\,\V{u}_{i_r}) \notag \\
		&=\tensor{\Phi}{_{i_1 \cdots i_r}}
			\qty[\ipb{\V{u}_{\Perm{\sigma}(i_1)}}{\V{g}^{i_1}}\cdots
				\ipb{\V{u}_{\Perm{\sigma}(i_r)}}{\V{g}^{i_r}}] \notag
		\intertext{应用置换的性质 \eqref{eq:置换算子的表示推导_置换性质}~式:}
		&=\tensor{\Phi}{_{i_1 \cdots i_r}}
			\qty[
				\ipb{\V{u}_{\Perm{\sigma}^{-1}\comp\Perm{\sigma}(i_1)}}
					{\V{g}^{\Perm{\sigma}^{-1}(i_1)}} \cdots
				\ipb{\V{u}_{\Perm{\sigma}^{-1}\comp\Perm{\sigma}(i_r)}}
					{\V{g}^{\Perm{\sigma}^{-1}(i_r)}}
			] \notag \\
		&=\tensor{\Phi}{_{i_1 \cdots i_r}}
			\qty[
				\ipb{\V{u}_{i_1}}{\V{g}^{\Perm{\sigma}^{-1}(i_1)}} \cdots
				\ipb{\V{u}_{i_2}}{\V{g}^{\Perm{\sigma}^{-1}(i_r)}}
			] \notag
		\intertext{同样,用简单张量表示,可得}
		&=\tensor{\Phi}{_{i_1 \cdots i_r}}
			\V{g}^{\Perm{\sigma}^{-1}(i_1)}\tp\cdots
				\tp\V{g}^{\Perm{\sigma}^{-1}(i_r)}
			\qty(\V{u}_{i_1},\,\cdots,\,\V{u}_{i_r}) \fullstop
	\end{align}
	这样,我们就得到了置换算子的一种表示:
	\begin{align}
		\opPerm(\T{\Phi})
		&=\opPerm\qty(\tensor{\Phi}{_{i_1 \cdots i_r}}
			\V{g}^{i_1}\tp\cdots\tp\V{g}^{i_r}) \notag \\
		&=\tensor{\Phi}{_{i_1 \cdots i_r}}
			\V{g}^{\Perm{\sigma}^{-1}(i_1)}\tp\cdots
				\tp\V{g}^{\Perm{\sigma}^{-1}(i_r)} \fullstop
		\label{eq:置换算子的表示_逆置换在简单张量上}
	\end{align}
	
	在式~\eqref{eq:置换算子的表示_逆置换在简单张量上} 中,
	$i_1,\,\cdots,\,i_r$ 都是哑标,要被求和求掉。
	张量 $\T{\Phi}$ 的底空间是 $\Rm$,所以每个 $i_k$ 都有 $m$ 个取值。
	考虑一下 \ref{subsec:哑标的穷尽}~小节中置换运算%
	有关\emphB{哑标穷尽}的性质,有
	\begin{align}
		\forall\,\Perm{\sigma}\in\Permutations{r}\, ,
		&\mathrel{\phantom{=}}\set[\bigg]
		{\qty(i_1,\,i_2,\,\cdots,\,i_r)}
		{\qty{i_1,\,i_2,\,\cdots,\,i_r}
			\text{\ 可取\ } 1,\,2,\,\cdots,\,m} \notag \\
		&=\set[\bigg]
		{\qty(\Perm{\sigma}(i_1),\,\Perm{\sigma}(i_2),\,
			\cdots,\,\Perm{\sigma}(i_r) )}
		{\qty{i_1,\,i_2,\,\cdots,\,i_r}
			\text{\ 可取\ } 1,\,2,\,\cdots,\,m} \fullstop
	\end{align}
	因此,我们可以把式~\eqref{eq:置换算子的表示_逆置换在简单张量上} 中的指标
	$i_k$ 换成 $\Perm{\sigma}(i_k)$:
	\begin{align}
		\opPerm(\T{\Phi})
		&=\tensor{\Phi}{_{i_1 \cdots i_r}}
			\V{g}^{\Perm{\sigma}^{-1}(i_1)}\tp\cdots
				\tp\V{g}^{\Perm{\sigma}^{-1}(i_r)} \notag \\
		&=\tensor{\Phi}{_{\Perm{\sigma}(i_1) \cdots \Perm{\sigma}(i_r)}}
			\V{g}^{\Perm{\sigma}^{-1}\comp\Perm{\sigma}(i_1)}\tp\cdots
				\tp\V{g}^{\Perm{\sigma}^{-1}\comp
					\Perm{\sigma}(i_r)} \notag \\
		&=\tensor{\Phi}{_{\Perm{\sigma}(i_1) \cdots \Perm{\sigma}(i_r)}}
			\V{g}^{i_1}\tp\cdots\tp\V{g}^{i_r} \fullstop
	\end{align}
	这是置换算子的另一种表示。
	
	综上,要获得置换算子的表示,
	若是对\emphB{张量分量}进行操作,就直接使用对分量指标使用置换;
	若是对\emphB{简单张量}进行操作,则要对其指标使用逆置换:\footnote{%
		这里稍有改动,用了张量的逆变分量,不过实质都是一样的。%
		使用协变分量还是逆变分量,这个嘛,悉听尊便。}
	\begin{mySubEq}
		\begin{align}
			\opPerm(\T{\Phi})
			&=\opPerm\qty(\tensor{\Phi}{^{i_1 \cdots i_r}}
				\V{g}_{i_1}\tp\cdots\tp\V{g}_{i_r}) \notag \\
			&=\tensor{\Phi}{^{\Perm{\sigma}(i_1)\cdots\Perm{\sigma}(i_r)}}
				\V{g}_{i_1}\tp\cdots\tp\V{g}_{i_r} \\
			&=\tensor{\Phi}{^{i_1 \cdots i_r}}
				\V{g}_{\Perm{\sigma}^{-1}(i_1)}\tp\cdots
					\tp\V{g}_{\Perm{\sigma}^{-1}(i_r)} \fullstop
		\end{align}
	\end{mySubEq}

%	
%	\chapter{微分同胚(曲线坐标系)}
%		\section{微分同胚}
\subsection{双射}
设 $f$ 是集合 $A$ 到 $B$ 的映照。
如果 $A$ 中不同的元素有不同的像,则称 $f$ 为\emphA{单射}(也叫“一对一”);
如果 $B$ 中每个元素都是 $A$ 中元素的像,则称 $f$ 为\emphA{满射};
如果 $f$ 既是单射又是满射,则称 $f$ 为\emphA{双射}(也叫“一一对应”)。
三种情况的示意见图~\ref{fig:单射满射双射}。

\begin{figure}[h]
	\centering
	\includegraphics{Images/Three_Mappings.PNG}
	\caption{单射、满射与双射}
	\label{fig:单射满射双射}
\end{figure}

设开集 $\domD{\V{X}},\,\domD{\V{x}}\subset\Rm$,
它们之间存在\emphB{双射},即\emphB{一一对应}关系:
\begin{equation}
	\mmap{\V{X}(\V{x})}
		{\domD{\V{x}}\ni\V{x}=\mqty[x^1 \\ \vdots \\ x^m]}
		{\V{X}(\V{x})=\mqty[X^1 \\ \vdots \\ X^m](\V{x})
			\in\domD{\V{X}}} \fullstop
\end{equation}
由于该映照实现了 $\domD{\V{x}}$ 到 $\domD{\V{X}}$ 之间的双射,
因此它存在逆映照:
\begin{equation}
	\mmap{\V{x}(\V{X})}
		{\domD{\V{X}}\ni\V{X}=\mqty[X^1 \\ \vdots \\ X^m]}
		{\V{x}(\V{X})=\mqty[x^1 \\ \vdots \\ x^m](\V{X})
			\in\domD{\V{x}}} \fullstop
\end{equation}

我们把 $\domD{\V{X}}$ 称为\emphA{物理域},它是实际物理事件发生的区域;
$\domD{\V{x}}$ 则称为\emphA{参数域}。
由于物理域通常较为复杂,因此我们常把参数域取为规整的形状,以便之后的处理。

设物理量 $f(\V{X})$ 定义在物理域
$\domD{\V{X}}\subset\Rm$ 上\footnote{
	实际的物理事件当然只会发生在三维 Euclid 空间中(只就“空间”而言),
	但在数学上也可以推广到 $m$ 维。
},则 $f$ 就定义了一个\emphA{场}:
\begin{equation}
	\mmap{f}
		{\domD{\V{X}}\ni\V{X}}
		{f(\V{X})} \fullstop
\end{equation}
所谓的“场”,就是自变量用\emphB{位置}刻画的映照。
它可以是\emphA{标量场},如温度、压强、密度等,此时 $f(\V{X})\in\realR$;
也可以是\emphA{向量场},如速度、加速度、力等,此时 $f(\V{X})\in\Rm$;
对于更深入的物理、力学研究,往往还需引入\emphA{张量场},
此时 $f(\V{X})\in\Tensors{r}$。

$\V{X}$ 存在于物理域 $\domD{\V{X}}$ 中,我们称它为\emphA{物理坐标}。
由于上文已经定义了 $\domD{\V{x}}$ 到 $\domD{\V{X}}$ 之间的双射
(不是 $f$!),因此 $\domD{\V{x}}$ 中就有\emphB{唯一的}
$\V{x}$ 与 $\V{X}$ 相对应,
它称为\emphA{参数坐标}(也叫\emphA{曲线坐标})。
又因为物理域 $\domD{\V{X}}$ 上已经定义了场 $f(\V{X})$,
参数域中必然\emphB{唯一}存在场 $\tilde{f}(\V{x})$ 与之对应:
\begin{equation}
	\mmap{\tilde{f}}
		{\domD{\V{x}}\ni\V{x}}
		{\tilde{f}(\V{x})=f\comp\V{X}(\V{x})
			=f\qty\big(\V{X}(\V{x}))} \fullstop
\end{equation}
$\V{x}$ 与 $\V{X}$ 是完全等价的,因而 $\tilde{f}$ 与 $f$ 也是完全等价的,
所以同样有
\begin{equation}
	f(\V{X})=\tilde{f}\qty\big(\V{x}(\V{X})) \fullstop
\end{equation}

物理域中的场要满足\emphB{守恒定律},如质量守恒、动量守恒、能量守恒等。
从数学上看,这些守恒定律就是 $f(\V{X})$ 需要满足的一系列\emphB{偏微分方程}。
将场变换到参数域后,它仍要满足这些方程。
但我们已经设法将参数域取得较为规整,故在其上进行数值求解就会相当方便。

\subsection{参数域方程} \label{subsec:参数域方程}
上文已经提到,物理域中的场 $f(\V{X})$ 需满足守恒定律,
这等价于一系列偏微分方程(PDE)。
在物理学和力学中,用到的 PDE 通常是\emphB{二阶}的,它们可以写成
\begin{equation}
	\forall\,\V{X}\in\domD{\V{X}},\quad
	\sum_{\alpha=1}^{m} A_\alpha(\V{X}) \pdv{f}{X^\alpha} (\V{X})
	+\sum_{\alpha=1}^{m}\sum_{\beta=1}^{m}
		B_{\alpha\beta}(\V{X}) \pdv{f}{X^\beta}{X^\alpha} (\V{X}) = 0
\end{equation}
的形式。我们的目标是把该\emphB{物理域}方程转化为\emphB{参数域}方程,
即关于 $\tilde{f}(\V{x})$ 的 PDE。
多元微积分中已经提供了解决方案:\emphA{链式求导法则}。

考虑到
\begin{equation}
	f(\V{X})=\tilde{f}\qty\big(\V{x}(\V{X}))
	=\tilde{f}\qty(x^1(\V{X}),\,\cdots,\,x^m(\V{X})) \comma
\end{equation}
于是有
\begin{equation}
	\pdv{f}{X^\alpha} (\V{X})
	=\sum_{s=1}^{m} \pdv{\tilde{f}}{\displaystyle x^s}
		\qty\big(\V{x}(\V{X})) \cdotp
		\pdv{x^s}{X^\alpha} (\V{X}) \fullstop
	\label{eq:参数域方程_一阶偏导项}
\end{equation}
这里用到的链式法则,由\emphB{复合映照可微性定理}驱动,
它要求 $\tilde{f}$ 关于 $\V{x}$ 可微,同时 $\V{x}$ 关于 $\V{X}$ 可微。

\myPROBLEM{对于更高阶的项,往往需要更强的条件。}一般地,我们要求
\begin{braceEq}
	\V{X}(\V{x})&\in\cf{\domD{\V{x}}}{\Rm} \semicomma \\
	\V{x}(\V{X})&\in\cf{\domD{\V{X}}}{\Rm} \fullstop
\end{braceEq}
这里的 $\DiffMorp$ 指\emphB{直至 $p$ 阶偏导数(存在且)连续}的映照全体;
$p=1$ 时,它就等价于可微。至于 $p$ 的具体取值,则由 PDE 的阶数所决定。

通常情况下,已知条件所给定的往往都是
$\domD{\V{x}}$ 到 $\domD{\V{X}}$ 的映照
\begin{equation}
	\mmap{\V{X}(\V{x})}
		{\domD{\V{x}}\ni\V{x}=\mqty[x^1 \\ \vdots \\ x^m]}
		{\V{X}(\V{x})=\mqty[X^1 \\ \vdots \\ X^m](\V{x})
			\in\domD{\V{X}}} \comma
\end{equation}
用它不好直接得到式~\eqref{eq:参数域方程_一阶偏导项} 中的
$\pdv*{x^s}{X^\alpha}$ 项,但获得它的“倒数”
$\pdv*{X^\alpha}{x^s}$ 却很容易,只需利用 \emphA{Jacobi 矩阵}:
\begin{equation}
	\JacobiD{\V{X}}(\V{x})
	\defeq\mqty[
		\displaystyle\pdv{X^1}{\displaystyle x^1} & \cdots &
			\displaystyle\pdv{X^1}{\displaystyle x^m} \\[1ex]
		\vdots & \ddots & \vdots \\[0.5ex]
		\displaystyle\pdv{X^m}{\displaystyle x^1} & \cdots &
			\displaystyle\pdv{X^m}{\displaystyle x^m}
		]\, (\V{x}) \in\realR^{m \times m} \comma
	\label{eq:参数域方程_Jacobi矩阵}
\end{equation}
它是一个方阵。

有了 Jacobi 矩阵,施加一些手法就可以得到所需要的 $\pdv*{x^s}{X^\alpha}$ 项。
考虑到
\begin{equation}
	\forall\,\V{X}\in\domD{\V{X}},\quad
	\V{X}\qty\big(\V{x}(\V{X}))=\V{X} \comma
\end{equation}
并且其中的 $\V{X}(\V{x})$ 和 $\V{x}(\V{X})$ 均可微,可以得到
\begin{equation}
	\JacobiD{\V{X}}\qty\big(\V{x}(\V{X}))
	\cdotp \JacobiD{\V{x}}(\V{X})
	=\Mat{I}_m \comma
\end{equation}
其中的 $\Mat{I}_m$ 是单位阵。因此
\begin{equation}
	\JacobiD{\V{x}}(\V{X})
	\defeq\mqty[
		\displaystyle\pdv{x^1}{\displaystyle X^1} & \cdots &
			\displaystyle\pdv{x^1}{\displaystyle X^m} \\[1ex]
		\vdots & \ddots & \vdots \\[0.5ex]
		\displaystyle\pdv{x^m}{\displaystyle X^1} & \cdots &
			\displaystyle\pdv{x^m}{\displaystyle X^m} ]\, (\V{X})
	=(\JacobiD{\V{X}})^{-1}(\V{x})
	=\mqty[
		\displaystyle\pdv{X^1}{\displaystyle x^1} & \cdots &
			\displaystyle\pdv{X^1}{\displaystyle x^m} \\[1ex]
		\vdots & \ddots & \vdots \\[0.5ex]
		\displaystyle\pdv{X^m}{\displaystyle x^1} & \cdots &
			\displaystyle\pdv{X^m}{\displaystyle x^m}
		]^{-1} (\V{x}) \fullstop
\end{equation}
用代数的方法总可以求出
\begin{equation}
	\varphi^s_\alpha\coloneq\pdv{x^s}{X^\alpha}(\V{X}) \comma
	\label{eq:Jacobi矩阵的元素}
\end{equation}
它是通过求逆运算确定的函数,
即位于矩阵 $\JacobiD{\V{x}}$ 第 $s$ 行第 $\alpha$ 列的元素。这样就有
\begin{equation}
	\pdv{f}{X^\alpha} (\V{X})
	=\sum_{s=1}^{m} \pdv{\tilde{f}}{\displaystyle x^s}
		\qty\big(\V{x}(\V{X})) \cdotp
		\varphi^s_\alpha \qty\big(\V{x}(\V{X})) \fullstop
\end{equation}

接下来处理二阶偏导数。由上式,
\begin{align}
	\pdv{f}{X^\beta}{X^\alpha} (\V{X})
	&=\sum_{s=1}^{m} \Bigg[ \qty\Bigg(\sum_{k=1}^{m}
			\pdv{\tilde{f}}{\displaystyle x^k}{\displaystyle x^s}
			\qty\big(\V{x}(\V{X})) \cdotp \pdv{x^s}{X^\beta} (\V{X}) )
		\cdotp \varphi^s_\alpha \qty\big(\V{x}(\V{X})) \notag \\
	&\alspace\phantom{\sum_{s=1}^{m} \Bigg[ }+
		\pdv{\tilde{f}}{\displaystyle x^s}
		\qty\big(\V{x}(\V{X})) \cdotp \qty\Bigg(\sum_{k=1}^{m}
			\pdv{\displaystyle \varphi^s_\alpha}{\displaystyle x^k}
			\qty\big(\V{x}(\V{X}))
			\cdotp \pdv{\displaystyle x^k}{X^\beta} (\V{X}) )
		\Bigg] \notag
	\intertext{继续利用式~\eqref{eq:Jacobi矩阵的元素},有}
	&=\sum_{s=1}^{m} \Bigg[ \qty\Bigg(\sum_{k=1}^{m}
			\pdv{\tilde{f}}{\displaystyle x^k}{\displaystyle x^s}
			\qty\big(\V{x}(\V{X})) \cdotp
			\varphi^s_\beta \qty\big(\V{x}(\V{X})) )
		\cdotp \varphi^s_\alpha \qty\big(\V{x}(\V{X})) \notag \\
	&\alspace\phantom{\sum_{s=1}^{m} \Bigg[ }+
		\pdv{\tilde{f}}{\displaystyle x^s}
		\qty\big(\V{x}(\V{X})) \cdotp \qty\Bigg(\sum_{k=1}^{m}
			\pdv{\displaystyle \varphi^s_\alpha}{\displaystyle x^k}
			\qty\big(\V{x}(\V{X}))
			\cdotp \varphi^k_\beta \qty\big(\V{x}(\V{X})) )
		\Bigg] \fullstop
\end{align}
这样,就把一阶和二阶偏导数项全部用关于 $\V{x}$ 的函数\footnote{%
	当然它仍然是 $\V{X}$ 的\emphB{隐}函数:
	$\V{x}=\V{x}(\V{X})$。}表达了出来。
换句话说,我们已经把\emphB{物理域}中 $f$ 关于 $\V{X}$ 的 PDE,
转化成了\emphB{参数域}中 $\tilde{f}$ 关于 $\V{x}$ 的 PDE。
这就是上文要实现的目标。

\subsection{微分同胚的定义}
上文已经指出了 $\domD{\V{x}}$ 到 $\domD{\V{X}}$ 的映照
$\V{X}(\V{x})$ 所需满足的一些条件。这里再次罗列如下:

\begin{myEnum}
\item $\domD{\V{X}},\,\domD{\V{x}}\subset\Rm$
均为\emphA{开集}\footnote{%
	用形象化的语言来说,如果在区域中的任意一点都可以吹出一个球,
	并能使球上的每个点都落在区域内,那么这个区域就是\emphA{开集}。
	这是\emphB{复合映照可微性定理}的一个要求。};

\item 存在 $\domD{\V{x}}$ 同 $\domD{\V{X}}$ 之间的\emphA{双射}
$\V{X}(\V{x})$,即存在\emphA{一一对应}关系;

\item $\V{X}(\V{x})$ 和它的逆映照 $\V{x}(\V{X})$
满足一定的\emphA{正则性}要求。
\end{myEnum}

\myPROBLEM{对第3点要稍作说明。}

如果满足这三点,则称 $\V{X}(\V{x})$ 为
$\domD{\V{x}}$ 与 $\domD{\V{X}}$ 之间的 \emphA{$\DiffMorp$-微分同胚},
记为 $\V{X}(\V{x})\in\cf{\domD{\V{x}}}{\domD{\V{X}}}$。
把物理域中的一个部分对应到参数域上的一个部分,需要的仅仅是\emphB{双射}这一条件;
而要使得物理域中所满足的 PDE 能够转换到参数域上,
就需要“过去”和“回来”都满足 $p$ 阶偏导数连续的条件(即\emphB{正则性}要求)。

\section{向量值映照的可微性}
\subsection{可微性的定义}
设 $\V{x}_0$ 是参数域 $\domD{\V{x}}$ 中的一个内点。
在映照 $\V{X}(\V{x})$ 的作用下,
它对应到物理域 $\domD{\V{X}}$ 中的点 $\V{X}\qty(\V{x}_0)$。
参数域是一个\emphB{开集}。根据开集的定义,必然存在一个实数 $\lambda>0$,
使得以 $\V{x}_0$ 为球心、$\lambda$ 为半径的球%
能够完全落在定义域 $\domD{\V{x}}$ 内,即
\begin{equation}
	\domB{\lambda}{\V{x}_0}\subset\domD{\V{x}} \comma
\end{equation}
其中的 $\domB{\lambda}{\V{x}_0}$ 表示 $\V{x}_0$ 的 $\lambda$ 邻域。

如果 $\exists\,\JacobiD{\V{X}}\qty(\V{x}_0)\in\LinearT{\Rm}{\Rm}$
\footnote{正如之前已经定义的,$\JacobiD{\V{X}}$ 已经用来表示 Jacobi 矩阵。
	这里还是请先暂时将它视为一种记号,其具体形式将在下一小节给出。},满足
\begin{equation}
	\forall\,\V{x}_0+\V{h}\in\domB{\lambda}{\V{x}_0},\quad
	\V{X}\qty(\V{x}_0+\V{h})-\V{X}\qty(\V{x}_0)
	=\JacobiD{\V{X}}\qty(\V{x}_0)(\V{h})+\sO{\norm{\V{h}}}
	\in\Rm \comma
	\label{eq:向量值映照可微性的定义}
\end{equation}
则称向量值映照 $\V{X}(\V{x})$ 在 $\V{x}_0$ 点\emphA{可微}。
其中,$\LinearT{\Rm}{\Rm}$ 表示从 $\Rm$ 到 $\Rm$ 的\emphA{线性变换}全体。

根据这个定义,所谓\emphB{可微性},指由自变量变化所引起的因变量变化,
可以用一个\emphB{线性变换}近似,而误差为\emphB{一阶}无穷小量。
自变量可见到因变量空间最简单的映照形式就是线性映照(线性变换),
因而具有可微性的向量值映照具有至关重要的作用。

\subsection{Jacobi 矩阵}
下面我们研究 $\JacobiD{\V{X}}\qty(\V{x}_0)\in\LinearT{\Rm}{\Rm}$
的表达形式。由于 $\V{h}\in\Rm$,所以
\begin{equation}
	\V{h}=\mqty[h^1 \\ \vdots \\ h^m]
	=h^1\V{e}_1+\cdots+h^i\V{e}_i+\cdots+h^m\V{e}_m \fullstop
\end{equation}
另一方面,$\JacobiD{\V{X}}\qty(\V{x}_0)\in\LinearT{\Rm}{\Rm}$
具有\emphB{线性性}:
\begin{equation}
	\forall\,\alpha,\,\beta \in\realR
		\text{\ 和\ } \tilde{\V{h}},\,\hat{\V{h}}\in\Rm,\quad
	\JacobiD{\V{X}}\qty(\V{x}_0)
		\qty(\alpha\tilde{\V{h}}+\beta\hat{\V{h}})
	=\alpha \JacobiD{\V{X}}\qty(\V{x}_0)\qty(\tilde{h})
		+\beta \JacobiD{\V{X}}\qty(\V{x}_0)\qty(\hat{h}) \fullstop
\end{equation}
这样就有
\begin{align}
	\JacobiD{\V{X}}\qty(\V{x}_0)\qty(\V{h})
	&=\JacobiD{\V{X}}\qty(\V{x}_0)
		\qty(h^1\V{e}_1+\cdots+h^i\V{e}_i+\cdots+h^m\V{e}_m) \notag \\
	&=h^1\JacobiD{\V{X}}\qty(\V{x}_0)\qty(\V{e}_1) + \cdots
		+h^i\JacobiD{\V{X}}\qty(\V{x}_0)\qty(\V{e}_i) + \cdots
		+h^m\JacobiD{\V{X}}\qty(\V{x}_0)\qty(\V{e}_m)
		\label{eq:推导Jacobi矩阵表达形式_Part1}
	\intertext{注意到 $h^i\in\realR$ 以及
		$\JacobiD{\V{X}}\qty(\V{x}_0)\qty(\V{e}_i)\in\Rm$,
		因而该式可以用矩阵形式表述:}
	&=\mqty[\JacobiD{\V{X}}\qty(\V{x}_0)\qty(\V{e}_1),\,\cdots,\,
			\JacobiD{\V{X}}\qty(\V{x}_0)\qty(\V{e}_m)]
		\mqty[h^1 \\ \vdots \\ h^m] \fullstop
\end{align}
最后一步要用到\emphB{分块矩阵}的思想:左侧的矩阵为 1“行” $m$ 列,
每一“行”是一个 $m$ 维列向量;右侧的矩阵(向量)则为 $m$ 行 1 列。
两者相乘,得到 1“行” 1 列的矩阵(当然实际为 $m$ 行),
即之前的 \eqref{eq:推导Jacobi矩阵表达形式_Part1}~式。
在线性代数中,$m \times m$ 的矩阵
$\mqty[\JacobiD{\V{X}}\qty(\V{x}_0)\qty(\V{e}_1) & \cdots
& \JacobiD{\V{X}}\qty(\V{x}_0)\qty(\V{e}_m)]$
通常称为\emphA{变换矩阵}(也叫\emphA{过渡矩阵})。

接下来要搞清楚变换矩阵的具体形式。取
\begin{equation}
	\V{h}=\mqty[0,\,\cdots,\,\lambda,\,\cdots,\,0]\trans
	=\lambda\,\V{e}_i\in\Rm \comma
\end{equation}
即除了 $\V{h}$ 的第 $i$ 各元素为 $\lambda$ 外,其余元素均为 0
($\lambda \neq 0$)。因而有 $\norm{\V{h}}=\lambda$。
代入可微性的定义 \eqref{eq:向量值映照可微性的定义}~式,可得
\begin{align}
	&\alspace\V{X}\qty(\V{x}_0+\V{h})-\V{X}\qty(\V{x}_0)
	=\V{X}\qty(\V{x}_0+\lambda\,\V{e}_i)
		-\V{X}\qty(\V{x}_0) \notag \\
	&=\mqty[\JacobiD{\V{X}}\qty(\V{x}_0)\qty(\V{e}_1),\,\cdots,\,
			\JacobiD{\V{X}}\qty(\V{x}_0)\qty(\V{e}_i),\,\cdots,\,
			\JacobiD{\V{X}}\qty(\V{x}_0)\qty(\V{e}_m)]
		\mqty[0,\,\cdots,\,\lambda,\,\cdots,\,0]\trans
		+\sO{\lambda} \notag \\
	&=\lambda\cdotp\JacobiD{\V{X}}\qty(\V{x}_0)\qty(\V{e}_i)
		+\sO{\lambda} \fullstop
\end{align}
由于 $\lambda$ 是非零实数,故可以在等式两边同时除以 $\lambda$ 并取极限:
\begin{equation}
	\lim\limits_{\lambda\to 0}
	\frac{\V{X}\qty(\V{x}_0+\lambda\,\V{e}_i)
		-\V{X}\qty(\V{x}_0)}{\lambda}
	=\JacobiD{\V{X}}\qty(\V{x}_0)\qty(\V{e}_i) \comma
\end{equation}
这里的 $\sO{\lambda}$ 根据其定义自然趋于 0。
该式左侧极限中的分子部分,是自变量 $\V{x}$ 第 $i$ 个分量的变化%
所引起因变量的变化;而分母,则是自变量第 $i$ 个分量的变化大小。
我们引入下面的记号:
\begin{equation}
	\pdv{\V{X}}{x^i}\qty(\V{x}_0)
	\coloneq \lim\limits_{\lambda\to 0}
	\frac{\V{X}\qty(\V{x}_0+\lambda\,\V{e}_i)
		-\V{X}\qty(\V{x}_0)}{\lambda} \in\Rm \comma
\end{equation}
它表示因变量 $\V{X}\in\Rm$ 作为一个\emphB{整体},
相对于自变量 $\V{x}\in\Rm$ 第 $i$ 个\emphB{分量} $x^i\in\realR$
的“变化率”,即 $\V{X}$ 关于 $x^i$(在 $\V{x}_0$ 处)%
的\emphA{偏导数}。由于我们没有定义向量的除法,
因此自变量作为\emphB{整体}所引起因变量的变化,是没有意义的。
利用偏导数的定义,可有
\begin{align}
	&\alspace\mqty[\JacobiD{\V{X}}\qty(\V{x}_0)\qty(\V{e}_1),\,\cdots,
		\,\JacobiD{\V{X}}\qty(\V{x}_0)\qty(\V{e}_i),\,\cdots,\,
		\JacobiD{\V{X}}\qty(\V{x}_0)\qty(\V{e}_m)] \notag \\
	&=\mqty[\displaystyle \pdv{\V{X}}{x^1}\qty(\V{x}_0),\,\cdots,\,
		\displaystyle \pdv{\V{X}}{x^i}\qty(\V{x}_0),\,\cdots,\,
		\displaystyle \pdv{\V{X}}{x^m}\qty(\V{x}_0)]
		\in\realR^{m \times m} \fullstop
\end{align}

\blankline

下面给出 $\pdv*{\V{X}}{x^i}\qty(\V{x}_0)$ 的计算式。根据定义,有
\begin{align}
	\pdv{\V{X}}{x^i}\qty(\V{x}_0)
	&\coloneq \lim\limits_{\lambda\to 0}
		\frac{\V{X}\qty(\V{x}_0+\lambda\,\V{e}_i)
		-\V{X}\qty(\V{x}_0)}{\lambda} \in\Rm \notag \\
	&=\lim\limits_{\lambda\to 0} \frac{1}{\lambda}\cdotp
		\left(\vphantom{\mqty{0\\[0.8ex]0}}\right.\!
			\mqty[X^1 \\ \vdots \\ X^m] \qty(\V{x}_0+\lambda\,\V{e}_i)
			-\mqty[X^1 \\ \vdots \\ X^m] \qty(\V{x}_0)
		\!\!\left.\vphantom{\mqty{0\\[0.8ex]0}}\right) \notag \\
	&=\lim\limits_{\lambda\to 0}\,
		\mqty[
			\dfrac{X^1\qty(\V{x}_0+\lambda\,\V{e}_i)-X^1\qty(\V{x}_0)}
				{\lambda} \\[1ex] \vdots \\[0.5ex]
			\dfrac{X^m\qty(\V{x}_0+\lambda\,\V{e}_i)-X^m\qty(\V{x}_0)}
				{\lambda} ] \fullstop
\end{align}
向量极限存在的\emphB{充要条件}是各分量极限均存在,即存在
\begin{equation}
	\pdv{X^\alpha}{x^i}\qty(\V{x}_0) \coloneq
	\lim\limits_{\lambda\to 0}
	\frac{X^\alpha\qty(\V{x}_0+\lambda\,\V{e}_i)-X^\alpha\qty(\V{x}_0)}
		{\lambda\vphantom{X^1\qty(\V{x}_0)}}\in\realR \comma
\end{equation}
其中的 $\alpha=1,\,\cdots,\,m$。
这其实就是我们熟知的\emphB{多元函数}偏导数的定义。
用它来表示\emphB{向量值映照}的偏导数,可有
\begin{equation}
	\pdv{\V{X}}{x^i}\qty(\V{x}_0)
	=\mqty[\displaystyle \pdv{X^1}{x^i}\qty(\V{x}_0) \\[1ex]
		\vdots \\[0.7ex] \displaystyle \pdv{X^m}{x^i}\qty(\V{x}_0)]
	=\sum_{\alpha=1}^{m} \pdv{X^\alpha}{x^i}\qty(\V{x}_0) \,
		\V{e}_\alpha \fullstop
\end{equation}

向量值映照 $\V{X}$ 关于 $x^i$ 的偏导数,
从代数的角度来看,是 Jacobi 矩阵的第 $i$ 列;
从几何的角度来看,则是物理域中 $x^i$ 线的切向量;
从计算的角度来看,又是(该映照)每个分量偏导数的组合。

\blankline

现在我们重新回到 Jacobi 矩阵。情况已经十分明了:
只需把之前获得的各列并起来,就可以得到完整的 Jacobi 矩阵。于是
\begin{align}
	\JacobiD{\V{X}}\qty(\V{x}_0)\qty(\V{h})
	&=\mqty[\displaystyle \pdv{\V{X}}{x^1},\,\cdots,\,\pdv{\V{X}}{x^m}]
		\qty(\V{x}_0)\qty(\V{h}) \notag \\
	&=\,\mqty[
		\displaystyle\pdv{X^1}{\displaystyle x^1} & \cdots &
			\displaystyle\pdv{X^1}{\displaystyle x^m} \\[1ex]
		\vdots & \ddots & \vdots \\[0.5ex]
		\displaystyle\pdv{X^m}{\displaystyle x^1} & \cdots &
			\displaystyle\pdv{X^m}{\displaystyle x^m}
		]\, \qty(\V{x}_0) \cdotp
		\mqty[h^1 \\ \vdots \\ h^m] \fullstop
\end{align}
这与 \ref{subsec:参数域方程}~小节中
\eqref{eq:参数域方程_Jacobi矩阵}~式给出的定义是完全一致的。

\subsection{偏导数的几何意义} \label{subsec:偏导数的几何意义}
这一小节中,我们要回过头来,澄清向量值映照偏导数的几何意义。

如图~\ref{fig:偏导数的几何意义},$\V{X}(\V{x})$ 是定义域空间
$\domD{\V{x}}\subset\Rm$ 到值域空间 $\domD{\V{X}}\subset\Rm$
的向量值映照。在定义域空间 $\domD{\V{x}}$ 中,
过点 $\V{x}_0$ 作一条平行于 $x^i$ 轴的直线,称为 \emphB{$x^i$-线}。
$x^i$ 轴定义了向量 $\V{e}_i$,因而 $x^i$-线上的任意一点均可表示为
$\V{x}_0+\lambda\,\V{e}_i$,其中 $\lambda\in\realR$。

\begin{figure}[h]
	\centering
	\includegraphics{Images/Vector-Value_Mapping.PNG}
	\caption{向量值映照偏导数的几何意义}
	\label{fig:偏导数的几何意义}
\end{figure}

在 $\V{X}(\V{x})$ 的作用下,点 $\V{x}_0$ 被映照到 $\V{X}\qty(\V{x}_0)$,
而 $\V{x}_0+\lambda\,\V{e}_i$ 则被映照到了
$\V{X}\qty(\V{x}_0+\lambda\,\V{e}_i)$。这样一来,
$x^i$-线也就被映照到了值域空间 $\domD{\V{X}}$ 中,成为一条曲线。

根据前面的定义,当 $\lambda\to 0$ 时,
\begin{equation}
	\frac{\V{X}\qty(\V{x}_0+\lambda\,\V{e}_i) - \V{X}\qty(\V{x}_0)}
	{\lambda} \to \pdv{\V{X}}{x^i}\qty(\V{x}_0) \fullstop
\end{equation}
对应到图~\ref{fig:偏导数的几何意义} 中,
就是 $x^i$-线(值域空间中)在 $\V{X}\qty(\V{x}_0)$ 处的\emphA{切向量}。

完全类似,在定义域空间 $\domD{\V{x}}$ 中,过点 $\V{x}_0$
作出 \emphB{$x^j$-线}(自然是平行于 $x^j$ 轴),
其上的点可以表示为 $\V{x}_0+\lambda\,\V{e}_j$。
映射到值域空间 $\domD{\V{X}}$ 上,
则成为 $\V{X}\qty(\V{x}_0+\lambda\,\V{e}_j)$。很显然,
\begin{equation}
	\pdv{\V{X}}{x^j}\qty(\V{x}_0)
	=\frac{\V{X}\qty(\V{x}_0+\lambda\,\V{e}_j) - \V{X}\qty(\V{x}_0)}
	{\lambda}
\end{equation}
就是 $x^j$-线在 $\V{X}\qty(\V{x}_0)$ 处的切向量。在定义域空间中,
$x^i$-线作为直线共有 $m$ 条,它们之间互相垂直。作用到值域空间后,
这样的 $x^i$-线尽管变为了曲线,但仍为 $m$ 条。相应的切向量,自然也有 $m$ 个。

\section{局部基}
这里的讨论基于曲线坐标系(即微分同胚)
$\V{X}(\V{x})\in\cf{\domD{\V{x}}}{\domD{\V{X}}}$。

我们已经知道,$\V{X}(\V{x})$ 的 Jacobi 矩阵可以表示为
\begin{equation}
	\JacobiD{\V{X}}(\V{x})
	=\mqty[\displaystyle \pdv{\V{X}}{x^1},\,\cdots,\,
		\displaystyle \pdv{\V{X}}{x^i},\,\cdots,\,
		\displaystyle \pdv{\V{X}}{x^m}] (\V{x})
		\in\realR^{m \times m} \comma
\end{equation}
式中的
\begin{equation}
	\pdv{\V{X}}{x^i} (\V{x})
	=\lim\limits_{\lambda\to 0}
		\frac{\V{X}\qty(\V{x}+\lambda\,\V{e}_i) - \V{X}(\V{x})}
		{\lambda} \fullstop
	\label{eq:局部基_偏导数}
\end{equation}
在参数域 $\domD{\V{x}}$ 中作出 $x^i$-线。映照到物理域后,它变成一条曲线,
我们仍称之为 $x^i$-线。\ref{subsec:偏导数的几何意义}~小节已经说明,
\eqref{eq:局部基_偏导数}~式表示物理域中 $x^i$-线的\emphB{切向量}。
在张量分析中,我们通常把它记作 $\V{g}_i(\V{x})$。

由于微分同胚要求是\emphB{双射},因而 Jacobi 矩阵
\begin{equation}
	\JacobiD{\V{X}}(\V{x})
	=\mqty[\V{g}_1,\,\cdots,\,\V{g}_i,\,\cdots,\,\V{g}_m](\V{x})
	\in\realR^{m \times m}
\end{equation}
必须是\emphB{非奇异}的。这等价于
\begin{equation}
	\qty{\V{g}_i(\V{x})=\pdv{\V{X}}{x^i} (\V{x})}^m_{i=1}
	\subset\Rm
\end{equation}
\emphB{线性无关}。由此,它们可以构成 $\Rm$ 上的一组\emphA{基}。

用任意的 $\V{x}\in\domD{\V{x}}$ 均可构建一组基。但选取不同的 $\V{x}$,
将会使所得基的取向有所不同。因而这种基称为\emphA{局部协变基}。
和之前一样,我们用“协变”表示指标在下方。

\blankline

有了局部协变基 $\qty{\V{g}_i(\V{x})}^m_{i=1}$,根据
\ref{subsec:对偶基}~小节中的讨论,必然唯一存在与之对应的\emphA{局部逆变基}
$\qty{\V{g}^i(\V{x})}^m_{i=1}$,满足
\begin{equation}
	\mqty[\V{g}^1(\V{x}),\,\cdots,\,\V{g}^m(\V{x})]\trans
		\mqty[\V{g}_1(\V{x}),\,\cdots,\,\V{g}_m(\V{x})]
	=\mqty[\qty(\V{g}^1)\trans \\ \vdots \\ \qty(\V{g}^m)\trans]\,
		(\V{x}) \cdotp \JacobiD{\V{X}}(\V{x})
	=\Mat{I}_m \fullstop
\end{equation}

下面我们来寻找逆变基 $\qty{\V{g}^i(\V{x})}^m_{i=1}$ 的具体表示。考虑到
\footnote{这里的几步推导在 \ref{subsec:参数域方程}~小节中也有所涉及。}
\begin{equation}
	\V{X}\qty\big(\V{x}(\V{X}))=\V{X}\in\Rm \comma
\end{equation}
并利用复合映照可微性定理,可知
\begin{equation}
	\JacobiD{\V{X}}\qty\big(\V{x}(\V{X}))
	\cdotp \JacobiD{\V{x}}(\V{X})
	=\Mat{I}_m \comma
\end{equation}
即有
\begin{equation}
	\JacobiD{\V{x}}(\V{X})
	=(\JacobiD{\V{X}})^{-1}\qty\big(\V{x}(\V{X})) \fullstop
\end{equation}
于是
\begin{equation}
	\mqty[\qty(\V{g}^1)\trans \\ \vdots \\ \qty(\V{g}^m)\trans]\,
		(\V{x})
	=(\JacobiD{\V{X}})^{-1}(\V{x})
	=\JacobiD{\V{x}}(\V{X})
	=\mqty[
		\displaystyle\pdv{x^1}{\displaystyle X^1} & \cdots &
			\displaystyle\pdv{x^1}{\displaystyle X^m} \\[1ex]
		\vdots & \ddots & \vdots \\[0.5ex]
		\displaystyle\pdv{x^m}{\displaystyle X^1} & \cdots &
			\displaystyle\pdv{x^m}{\displaystyle X^m} ]\,(\V{X}) \fullstop
\end{equation}
这样我们就得到了局部逆变基的具体表示(注意转置):
\begin{equation}
	\V{g}^i(\V{x})
	=\mqty[\displaystyle \pdv{x^i}{X^1} \\[1ex]
		\vdots \\[0.5ex] \displaystyle \pdv{x^i}{X^m}]\,(\V{X})
	=\sum_{\alpha=1}^{m} \pdv{x^i}{X^\alpha} (\V{X})\,\V{e}_\alpha
	\fullstop
\end{equation}
定义标量场 $f(\V{x})$ 的\emphA{梯度}为
\begin{equation}
	\nabla f(\V{x}) \defeq
	\sum_{\alpha=1}^{m} \pdv{f}{x^\alpha} (\V{x})\,\V{e}_\alpha \comma
\end{equation}
则局部逆变基又可以表示成
\begin{equation}
	\V{g}^i(\V{x})=\nabla x^i(\V{X}) \fullstop
\end{equation}
这里的梯度实际上就是我们熟知的三维情况在 $m$ 维下的推广。
%	
	\chapter{张量场可微性}
		\section{张量的范数}
\subsection{赋范线性空间}
对于一个\emphA{线性空间} $\SPACE{V}$,它总是定义了\emphA{线性结构}:
\begin{equation}
	\forall\, \V{x},\,\V{y}\in\SPACE{V}
	\text{\ 和\ } \forall\, \alpha,\,\beta\in\realR,\quad
	\alpha\,\V{x}+\beta\,\V{y} \in \SPACE{V} \fullstop
\end{equation}
为了进一步研究的需要,我们还要引入\emphA{范数}的概念。
所谓“范数”,就是对线性空间中任意元素\emphB{大小}的一种刻画。
举个我们熟悉的例子, $m$ 维 Euclid 空间 $\Rm$ 中某个向量的范数,
就定义为该向量在 Descartes 坐标下各分量的平方和的平方根。

一般而言,线性空间 $\SPACE{V}$ 中的范数
$\norm[\SPACE{V}]{\cdotord}$ 是从 $\SPACE{V}$ 到 $\realR$
的一个映照,并且需要满足以下三个条件:

\begin{myEnum}
\item \emphA{非负性}
\begin{equation}
	\forall\,\V{x}\in\SPACE{V},\quad
	\norm[\SPACE{V}]{\V{x}} \geqslant 0
\end{equation}
以及\emphA{非退化性}
\begin{equation}
	\forall\,\V{x}\in\SPACE{V},\quad
	\norm[\SPACE{V}]{\V{x}}=0
	\iff \V{x}=\V{0}\in\SPACE{V} \comma
\end{equation}
这里的 $\V{0}$ 是线性空间 $\SPACE{V}$ 中的\emphA{零元素},
它是唯一存在的。

\blankline

\item 由于零元是唯一的,因此线性空间中的元素 $\V{x}$
就与从 $\V{0}$ 指向它的向量一一对应。
因此,线性空间中的元素也常被称为“向量”。

考虑线性空间中的数乘运算。从几何上看, $\V{x}$ 乘上 $\lambda$,
就是将 $\V{x}$ 沿着原来的指向进行伸缩。显然有
\begin{equation}
	\forall\,\V{x}\in\SPACE{V}
	\text{\ 和\ } \forall\,\lambda\in\realR,\quad
	\norm[\SPACE{V}]{\lambda\,\V{x}}
	=\abs{\lambda}\cdot\norm[\SPACE{V}]{\V{x}} \comma
\end{equation}
这称为\emphA{正齐次性}。

\myPROBLEM{想要图吗?}

\item 线性空间中的加法满足\emphB{平行四边形法则}。直观地看,就有
\begin{equation}
	\forall\, \V{x},\,\V{y}\in\SPACE{V},\quad
	\norm[\SPACE{V}]{\V{x}+\V{y}} \leqslant
	\norm[\SPACE{V}]{\V{x}}+\norm[\SPACE{V}]{\V{y}} \comma
\end{equation}
这称为\emphA{三角不等式}。
\end{myEnum}

定义了范数的线性空间称为\emphA{赋范线性空间}。

\subsection{张量范数的定义}
考虑 $p$ 阶张量 $\T{\Phi}\in\Tensors{p}$,
它可以用\emphB{逆变分量}或\emphB{协变分量}来表示:
\begin{braceEq*}{\T{\Phi}=}
	\Phi^{i_1 \cdots i_p}\,
		\V{g}_{i_1}\tp\cdots\tp\V{g}_{i_p} \comma \\
	\Phi_{i_1 \cdots i_p}\,
		\V{g}^{i_1}\tp\cdots\tp\V{g}^{i_p} \comma
\end{braceEq*}
其中
\begin{braceEq}
	\Phi^{i_1 \cdots i_p}&=
		\T{\Phi}\qty(\V{g}^{i_1},\,\cdots,\,\V{g}^{i_p}) \fullstop \\
	\Phi_{i_1 \cdots i_p}&=
		\T{\Phi}\qty\big(\V{g}_{i_1},\,\cdots,\,\V{g}_{i_p}) \comma
\end{braceEq}
张量的\emphA{范数}定义为
\begin{equation}
	\norm[\Tensors{p}]{\T{\Phi}}\defeq
	\sqrt{\Phi^{i_1 \cdots i_p} \, \Phi_{i_1 \cdots i_p}}
	\in\realR \fullstop
\end{equation}
$i_1 \cdots i_p$ 可独立取值,每个又有 $m$ 种取法,
所以根号下共有 $m^p$ 项。
注意 $\Phi^{i_1 \cdots i_p}$ 与 $\Phi_{i_1 \cdots i_p}$
未必相等,因而根号下的部分未必是平方和,
这与 Euclid 空间中向量的模是不同的。

复习一下 \ref{subsec:相对不同基的张量分量之间的关系}~小节,
我们可以用另一组(带括号的)基表示张量 $\T{\Phi}$:
\begin{braceEq}
	\Phi^{i_1 \cdots i_p} &=
		c^{i_1}_{(\xi_1)} \cdots c^{i_p}_{(\xi_p)} \,
		\Phi^{(\xi_1)\cdots(\xi_p)} \comma \\
	\Phi_{i_1 \cdots i_p} &=
		c^{(\eta_1)}_{i_1} \cdots c^{(\eta_p)}_{i_p} \,
		\Phi_{(\eta_1)\cdots(\eta_p)} \comma
\end{braceEq}
其中的 $c^i_{(\xi)}=\ipb{\V{g}_{(\xi)}}{\V{g}^i}$,
$c^{(\eta)}_i=\ipb{\V{g}^{(\eta)}}{\V{g}_i}$,它们满足
\begin{equation}
	c^i_{(\xi)}\,c^{(\eta)}_i
	=\KroneckerDelta{(\eta)}{(\xi)} \fullstop
\end{equation}
于是
\begin{align}
	&\alspace\Phi^{i_1 \cdots i_p} \, \Phi_{i_1 \cdots i_p} \notag \\
	&=\qty(c^{i_1}_{(\xi_1)} \cdots c^{i_p}_{(\xi_p)} \,
			\Phi^{(\xi_1)\cdots(\xi_p)})
		\qty(c^{(\eta_1)}_{i_1} \cdots c^{(\eta_p)}_{i_p} \,
			\Phi_{(\eta_1)\cdots(\eta_p)}) \notag \\
	&=\qty(c^{i_1}_{(\xi_1)} c^{(\eta_1)}_{i_1}) \cdots
		\qty(c^{i_p}_{(\xi_p)} c^{(\eta_p)}_{i_p}) \,
		\Phi^{(\xi_1)\cdots(\xi_p)} \Phi_{(\eta_1)\cdots(\eta_p)}
		\notag \\
	&=\KroneckerDelta{(\eta_1)}{(\xi_1)} \cdots
		\KroneckerDelta{(\eta_p)}{(\xi_p)} \,
		\Phi^{(\xi_1)\cdots(\xi_p)} \Phi_{(\eta_1)\cdots(\eta_p)}
		\notag \\
	&=\Phi^{(\xi_1)\cdots(\xi_p)} \Phi_{(\xi_1)\cdots(\xi_p)}
	\fullstop
\end{align}
它是 $\T{\Phi}$ 在另一组基下的逆变分量与协变分量乘积之和。

以上结果说明,张量的范数不依赖于基的选取,
这就好比用不同的秤来称同一个人的体重,都将获得相同的结果。
既然如此,不妨采用\emphB{单位正交基}来表示张量的范数:
\begin{align}
	\norm[\Tensors{p}]{\T{\Phi}}
	&\defeq\sqrt{\Phi^{i_1 \cdots i_p} \, \Phi_{i_1 \cdots i_p}}
	\notag \\
	&=\sqrt{\Phi^{\orthIdx{i_1}\cdots\orthIdx{i_p}} \,
		\Phi_{\orthIdx{i_1}\cdots\orthIdx{i_p}}} \notag \\
	&\eqcolon\sqrt{\sum\nolimits_{i_1,\,\cdots,\,i_p=1}^{m}
		\qty\big(\Phi\midscript{\orthIdx{i_1,\,\cdots,\,i_p}})^2}
	\fullstop
\end{align}
这里的 $\Phi\midscript{\orthIdx{i_1,\,\cdots,\,i_p}}$
表示张量 $\T{\Phi}$ 在单位正交基下的分量,它的指标不区分上下。

有了这样的表示,很容易就可以验证张量范数符合之前的三个要求。
一组数的平方和开根号,必然是\emphB{非负}的。
至于\emphB{非退化性},若范数为零,则所有分量均为零,自然成为零张量;
反之,对于零张量,所有分量为零,范数也为零。
将 $\T{\Phi}$ 乘上 $\lambda$,则有
\begin{align}
	\norm[\Tensors{p}]{\lambda\,\T{\Phi}}
	&=\sqrt{\sum\nolimits_{i_1,\,\cdots,\,i_p=1}^{m}
		\qty\big(\lambda\,
			\Phi\midscript{\orthIdx{i_1,\,\cdots,\,i_p}})^2} \notag \\
	&=\sqrt{\lambda^2 \sum\nolimits_{i_1,\,\cdots,\,i_p=1}^{m}
			\qty\big(\Phi\midscript{\orthIdx{i_1,\,\cdots,\,i_p}})^2}
		\notag \\
	&=\abs{\lambda} \sqrt{\sum\nolimits_{i_1,\,\cdots,\,i_p=1}^{m}
			\qty\big(\Phi\midscript{\orthIdx{i_1,\,\cdots,\,i_p}})^2}
		\notag \\
	&=\abs{\lambda}\cdot\norm[\Tensors{p}]{\T{\Phi}} \comma
\end{align}
于是\emphB{正齐次性}也得以验证。
最后,利用 Cauchy--Schwarz 不等式,可有
\begin{align}
	&\alspace\norm[\Tensors{p}]{\T{\Phi}+\T{\Psi}}^2 \notag \\
	&=\sum \qty\Big(\Phi\midscript{\orthIdx{i_1,\,\cdots,\,i_p}}
			+\Psi\midscript{\orthIdx{i_1,\,\cdots,\,i_p}})^2 \notag \\
	&=\sum \qty[
			\qty\big(\Phi\midscript{\orthIdx{i_1,\,\cdots,\,i_p}})^2
			+2\, \Phi\midscript{\orthIdx{i_1,\,\cdots,\,i_p}} \,
				\Psi\midscript{\orthIdx{i_1,\,\cdots,\,i_p}}
			+\qty\big(\Psi\midscript{\orthIdx{i_1,\,\cdots,\,i_p}})^2]
		\notag \\
	&=\sum \qty\big(\Phi\midscript{\orthIdx{i_1,\,\cdots,\,i_p}})^2
		+2\sum \Phi\midscript{\orthIdx{i_1,\,\cdots,\,i_p}} \,
			\Psi\midscript{\orthIdx{i_1,\,\cdots,\,i_p}}
		+\sum \qty\big(\Psi\midscript{\orthIdx{i_1,\,\cdots,\,i_p}})^2
		\notag \\
	&\leqslant\norm[\Tensors{p}]{\T{\Phi}}^2
		+2 \sqrt{\sum \qty\big(
				\Phi\midscript{\orthIdx{i_1,\,\cdots,\,i_p}})^2}
			\sqrt{\sum \qty\big(
				\Psi\midscript{\orthIdx{i_1,\,\cdots,\,i_p}})^2}
		+\norm[\Tensors{p}]{\T{\Phi}}^2 \notag \\
	&=\norm[\Tensors{p}]{\T{\Phi}}^2
		+2\norm[\Tensors{p}]{\T{\Phi}}\cdot\norm[\Tensors{p}]{\T{\Psi}}
		+\norm[\Tensors{p}]{\T{\Phi}}^2 \notag \\
	&=\qty\Big(\norm[\Tensors{p}]{\T{\Phi}}
		+\norm[\Tensors{p}]{\T{\Phi}})^2 \fullstop
\end{align}
两边开方,即为\emphB{三角不等式}。

\blankline

由此,我们就完整地给出了张量大小的刻画手段。
可以看出,它实际上就是 Euclid 空间中向量模的直接推广。

\subsection{简单张量的范数}
根据 \ref{subsec:张量的表示与简单张量}~小节中的定义,
简单张量是形如 $\V{\xi}\tp\V{\eta}\tp\V{\zeta}$ 的张量,
其中的 $\V{\xi},\,\V{\eta},\,\V{\zeta}\in\Rm$,
它是三个向量的张量积。
简单张量的范数为
\begin{equation}
	\norm[\Tensors{3}]{\V{\xi}\tp\V{\eta}\tp\V{\zeta}}
	=\norm{\V{\xi}}\cdot\norm{\V{\eta}}\cdot\norm{\V{\zeta}}
	\fullstop \label{eq:简单张量的范数}
\end{equation}

\begin{myProof}
$\V{\xi}\tp\V{\eta}\tp\V{\zeta}$ 的逆变分量为
\begin{equation}
	\qty(\V{\xi}\tp\V{\eta}\tp\V{\zeta})^{ijk}
	\defeq \V{\xi}\tp\V{\eta}\tp\V{\zeta}
		\qty(\V{g}^i,\,\V{g}^j,\,\V{g}^k)
	=\xi^i \eta^j \zeta^k \fullstop
\end{equation}
同理,它的协变分量为
\begin{equation}
	\qty(\V{\xi}\tp\V{\eta}\tp\V{\zeta})_{ijk}
	\defeq \V{\xi}\tp\V{\eta}\tp\V{\zeta}
		\qty(\V{g}_i,\,\V{g}_j,\,\V{g}_k)
	=\xi_i \eta_j \zeta_k \fullstop
\end{equation}
二者相乘,有
\begin{align}
	&\alspace\qty(\V{\xi}\tp\V{\eta}\tp\V{\zeta})^{ijk}
		\cdot \qty(\V{\xi}\tp\V{\eta}\tp\V{\zeta})_{ijk} \notag \\
	&=\qty(\xi^i \eta^j \zeta^k)
		\cdot \qty(\xi_i \eta_j \zeta_k) \notag \\
	&=\qty(\xi^i \xi_i) \cdot \qty(\eta^j \eta_j)
		\cdot \qty(\zeta^k \zeta_k) \fullstop
\end{align}
注意到
\begin{align}
	\norm{\xi}^2
	&=\ipb{\xi}{\xi} \notag
	\intertext{分别把二者用协变和逆变分量表示:}
	&=\ipb{\xi^i\,\V{g}_i}{\xi_j\,\V{g}^j} \notag \\
	&=\xi^i \xi_j \ipb{\V{g}_i}{\V{g}^j} \notag \\
	&=\xi^i \xi_j \KroneckerDelta{j}{i}
	=\xi^i \xi_i \comma
\end{align}
于是
\begin{equation}
	\qty(\V{\xi}\tp\V{\eta}\tp\V{\zeta})^{ijk}
		\cdot \qty(\V{\xi}\tp\V{\eta}\tp\V{\zeta})_{ijk}
	=\norm{\xi}^2 \cdot \norm{\V{\eta}}^2 \cdot \norm{\V{\zeta}}^2
	\fullstop
\end{equation}
两边开方,即得 \eqref{eq:简单张量的范数}~式。
\end{myProof}

\section{张量场沿坐标曲线的变化率}
在区域 $\domD{\V{x}}\subset\Rm$ 上,
若存在一个自变量用\emphB{位置}刻画的映照
\begin{equation}
	\mmap{\T{\Phi}}{\domD{\V{x}}\ni\V{x}}
		{\T{\Phi}(\V{x})\in\Tensors{r}} \comma
\end{equation}
则称张量 $\T{\Phi}(\V{x})$ \footnote{%
	类似“$\T{\Phi}(\V{x})$”的记号在前文也表示张量 $\T{\Phi}$
	\emphB{作用}在向量 $\V{x}$ 上(“吃掉”了 $\V{x}$),此时有
	$\T{\Phi}(\V{x})\in\realR$,注意不要混淆。
	符号有限,难免如此,还望诸位体谅。}是定义在 $\domD{\V{x}}$
上的一个\emphA{张量场}。

下面我们以三阶张量为例。设在物理域 $\domD{\V{X}}\subset\Rm$
和参数域 $\domD{\V{x}}\subset\Rm$ 之间已经建立了微分同胚
$\V{X}(\V{x})\in\cf{\domD{\V{x}}}{\domD{\V{X}}}$。
在 $\V{X}(\V{x})$ 处,张量场 $\T{\Phi}(\V{x})$
可以用分量形式表示为
\begin{equation}
	\T{\Phi}(\V{x})=\tensor{\Phi(\V{x})}{^i_j^k}\,
		\V{g}_i(\V{x})\tp\V{g}^j(\V{x})\tp\V{g}_k(\V{x})
	\in\Tensors{3} \comma
\end{equation}
其中的 $\V{g}_i(\V{x}),\,\V{g}^j(\V{x}),\,\V{g}_k(\V{x})$
都是\emphB{局部}基,而张量分量则定义为\footnote{%
	请注意,下式 $\T{\Phi}$ 之后的第一个圆括号表示\emphB{位于}
	$\V{x}$ 处;而后面的方括号则表示\emphB{作用在}这几个向量上。}
\begin{equation}
	\tensor{\Phi(\V{x})}{^i_j^k}
	\defeq \T{\Phi}(\V{x})
		\qty[\V{g}_i(\V{x}),\,\V{g}^j(\V{x}),\,\V{g}_k(\V{x})]
	\in\realR \fullstop
\end{equation}
类似地,当点沿着 $x^\mu$-线运动到
$\V{X}\qty(\V{x}+\lambda\,\V{e}_\mu)$ 处时,有
\begin{equation}
	\T{\Phi}\qty(\V{x}+\lambda\,\V{e}_\mu)
	=\tensor{\Phi\qty(\V{x}+\lambda\,\V{e}_\mu)}{^i_j^k}\,
		\V{g}_i \qty(\V{x}+\lambda\,\V{e}_\mu)
		\tp\V{g}^j \qty(\V{x}+\lambda\,\V{e}_\mu)
		\tp\V{g}_k \qty(\V{x}+\lambda\,\V{e}_\mu) \fullstop
\end{equation}

现在研究 $\lambda\to 0 \in\realR$ 时的极限
\begin{equation}
	\lim_{\lambda\to 0}
	\frac{\T{\Phi}\qty(\V{x}+\lambda\,\V{e}_\mu)-\T{\Phi}(\V{x})}
		{\lambda}
	\eqcolon \pdv{\T{\Phi}}{x^\mu} (\V{x})
	\in\Tensors{3} \fullstop
\end{equation}
%	\backmatter
%	{
%		\small
%		\bibliography{Reference}
%	}
%	\printindex[pkg]
%	\printindex[cmd]
\end{document}