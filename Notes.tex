%TODO 等的使用
%%TODO 内容问题
%%HACK 粗鄙技巧
%%CODE 代码改进

%TODO——注意事项

\documentclass[oneside]{book}

%TODO——宏包
%%方便定义命令
%\usepackage{suffix}

%%页面尺寸
\usepackage{geometry}
	\geometry{
		a4paper,
		left = 2.54 cm, right = 2.54 cm, top = 3.18 cm, bottom = 3.18 cm,
		headheight = 3 cm
	}

%%设置标题
\usepackage{titlesec}

%%交叉引用,超链接等
\usepackage[hyperindex]{hyperref}
	\hypersetup{
		%  PDF 书签
		bookmarksopen = true,
		bookmarksopenlevel = 1,
		bookmarksnumbered = true,
		%  PDF 标题作者
%		pdftitle = {},
%		pdfauthor = {},
		%脚注
%		hyperfootnotes = false,
		%目录  只引用页码
		linktoc = page,
		%超链接颜色
		colorlinks,
		linkcolor = {red!60!black},
		citecolor = {green!50!black},
		urlcolor = {blue!70!black}
	}

%%常规字体选择
\usepackage[no-math]{fontspec}
	\setmainfont[
		Extension = .otf,
		BoldFont = xits-bold,
		ItalicFont = xits-italic,
		BoldItalicFont = xits-bolditalic,
		SlantedFont = xits-italic
	]{xits-regular}
%	\setsansfont[
%		Extension = .otf,
%		BoldFont = texgyreheros-bold,
%		ItalicFont = texgyreheros-italic,
%		SlantedFont = texgyreheros-italic
%	]{texgyreheros-regular}
%	\setmonofont[
%		Extension = .otf,
%		BoldFont = texgyrecursor-bold,
%		ItalicFont = texgyrecursor-italic,
%		SlantedFont = texgyrecursor-italic,
%		Ligatures = NoCommon
%	]{texgyrecursor-regular}
%	\newfontfamily{\SourceSans}{Source Sans Pro}
%	\newfontfamily{\Songti}{方正书宋_GBK}

%%AMS 数学支持
\usepackage{amsmath}
\usepackage{amssymb}

%%Pi 符号
\usepackage{pifont}
	% 正确 √
	\newcommand{\cmark}{\ding{51}}
	% 错误 ×
	\newcommand{\xmark}{\ding{55}}

%%调用 Unicode OpenType 数学字体
\usepackage{unicode-math}
	\setmathfont[
		math-style = ISO,
		bold-style = ISO
	]{xits-math.otf}
%加粗使用 \symbf{}
%直立希腊字母:\uppi 等

%%中文文字处理
\usepackage[UTF8, heading = true]{ctex}
	\pagestyle{plain}
\usepackage{xeCJK}
	\setCJKmainfont[
		BoldFont = 方正黑体_GBK,
		ItalicFont = 方正楷体_GBK,
		Mapping = fullwidth-stop
	]{方正书宋_GBK}
	\setCJKsansfont[
		BoldFont = 方正黑体_GBK,
		ItalicFont = 方正黑体_GBK,
		Mapping = fullwidth-stop
	]{方正黑体_GBK}
%	\setCJKsansfont[
%		BoldFont = 思源黑体 CN Regular,
%		ItalicFont = 思源黑体 CN Regular,
%		Mapping = fullwidth-stop
%	]{思源黑体 CN Regular}
	\setCJKmonofont[
		BoldFont = 方正仿宋_GBK,
		ItalicFont = 方正楷体_GBK,
		Mapping = fullwidth-stop
	]{方正仿宋_GBK}
	\setCJKfamilyfont{宋体}{方正书宋_GBK}
	\setCJKfamilyfont{楷体}{方正楷体_GBK}
	\setCJKfamilyfont{黑体}{方正黑体_GBK}
	\setCJKfamilyfont{仿宋}{方正仿宋_GBK}

%%脚注增强版
\usepackage[stable, perpage, bottom]{footmisc}
	%需要调用 pifont 宏包
	%衬线加圈阳文数字:\ding{172}~\ding{181} (1~10)
	%无衬线加圈阳文数字:\ding{192}~\ding{201} (1~10)
	\renewcommand{\thefootnote}{\ding{\numexpr191+\value{footnote} } }
	%TODO:20160705 脚注不用上标
	%HACK:20160709 见 http://tex.stackexchange.com/questions/19844/how-to-set-superscript-footnote-mark-in-the-text-body-but-normalsized-in-the-foo
	\makeatletter
	\newlength{\fnBreite}
	\renewcommand{\@makefntext}[1]{%
		\settowidth{\fnBreite}{\footnotesize\@thefnmark.i}%
		\protect\footnotesize\upshape%
		\setlength{\@tempdima}{\columnwidth}%
		\addtolength{\@tempdima}{-\fnBreite}%
		\makebox[\fnBreite][l]{\@thefnmark\phantom{  }}%
		\parbox[t]{\@tempdima}%
		{\everypar{\hspace*{1em}}\hspace*{-1em}\upshape#1}%
	}
	\makeatother

%%颜色
\usepackage[svgnames]{xcolor}

%%图形
%\usepackage{graphicx}

%%绘图
%\usepackage{tikz}
%	\usepgflibrary{arrows.meta}
%	\tikzset{>=Stealth}

%%数学工具
\usepackage{mathtools}

%%强调公式
\usepackage[ntheorem]{empheq}

%%张量
\usepackage{tensor}

%%物理、数学符号
\usepackage{physics}

%%单位
%\usepackage{siunitx}

%%定制列表环境
%\usepackage{enumitem}
%	%定义带缩进的左对齐格式
%	%前一个参数 2.1 em 确定标签的位置
%	%后一个参数 1.2 em 确定标签与文字的距离
%	%HACK:20160709 可能与字体、字号、标签内容有关
%	\SetLabelAlign{leftalignwithindent}{\hspace{2.1 em} \makebox[1.2 em][l]{#1}}

%%定理类环境
\usepackage[thmmarks, amsmath]{ntheorem}
	\theoremstyle{nonumberplain} %带编号
	\theoremheaderfont{\bfseries}
	\theorembodyfont{\normalfont}
	\theoremsymbol{\mbox{$\Box$}}
		\newtheorem{myProof}{证明:}

%%索引

%TODO——环境
%%定制列表(编号)
%\newenvironment{myEnumerate}
%	{\begin{enumerate} [
%%		label=\bullet, %标签样式:圆点
%		align = leftalignwithindent, %对齐(见上)
%		listparindent = 2 em, %条目段落缩进
%		leftmargin = 0 pt, %文字左边距
%		topsep = 0 pt,
%		itemsep = 0 pt,
%		parsep = 0 pt
%	]}
%	{\end{enumerate}}
%子公式
\newenvironment{mySubEq}
	{\subequations \renewcommand{\theequation}%
		{\theparentequation-\alph{equation}}}%
	{\endsubequations%
		\ignorespacesafterend}
%大括号公式
\newenvironment{braceEq}[1][align]
	{\mySubEq%
		\setkeys{EmphEqEnv}{#1}%
		\setkeys{EmphEqOpt}{left = \empheqlbrace}%
		\EmphEqMainEnv}%
	{\endEmphEqMainEnv \endmySubEq}

%TODO——命令
%% 空行
\newcommand{\blankline}{\mbox{}\par\mbox{}}
%% 强调
\newcommand{\emphA}[1]{{\bfseries #1}}
\newcommand{\emphB}[1]{{\itshape #1}}
%% 数集黑板粗体
\newcommand{\setbb}[1]{\symbb{#1}}
\newcommand{\realR}{\setbb{R}}
%% 哥特体
%\newcommand{\gothic}[1]{\symfrak{#1}}
%\newcommand{\domainB}{\gothic{B}} %邻域
%\newcommand{\domainD}{\gothic{D}} %定义域

%% 向量
\newcommand{\V}[1]{\symbf{#1}}
%% 张量(Tensor)
\newcommand{\Tens}[1]{\symbf{#1}}
%% 矩阵(Matrix)
\newcommand{\Mat}[1]{\symbf{#1}}

%% Kronecker Delta
\newcommand{\kroneckerDelta}[2]{\delta^{#1}_{#2}}
%% 内积括号形式(Inner Product Braket)
\newcommand{\ipb}[3]{\qty(#1,\,#2)_{#3}}

%% 张量积(Tensor Product)
\newcommand{\tp}{\otimes}
%% 定义等号
%\newcommand{\defeq}{\coloneq}
\newcommand{\defeq}{\triangleq}

%% m 维实空间
\newcommand{\Rm}{\realR^m}
%% 以 m 维 Euclid 空间为底空间的 r 阶张量全体
\newcommand{\Tensors}[1]{\symcal{J}^{#1}\qty(\Rm)}

%% 公式用标点、文字
\newcommand{\comma}{\text{,}}
\newcommand{\fullstop}{\text{.}}
\newcommand{\semicomma}{\text{;}}
%\newcommand{\const}{\text{const.}}

%TODO——标题页
\title{
	\vspace{-4 cm} \color{Sienna} \Huge 张量分析
}
\author{
	\CJKfamily{楷体} \color{DarkRed} \Large 复旦大学\phantom{空格}谢锡麟
}
\date{
	\CJKfamily{楷体} \color{Goldenrod} \Large \today
}


\begin{document}
%	\frontmatter
%	\maketitle
%	
%%	\tableofcontents
%	
%	\mainmatter
	\chapter{张量的定义及表示}
		\section{对偶基,度量}
\subsection{对偶基}
	$\Rm$ 空间中的基可分为两类:指标写在\emphB{下面}的基
	\begin{equation*}
		\qty{\V{g}_i}^m_{i=1} \subset \Rm
	\end{equation*}
	称为\emphA{协变基},指标写在\emphB{上面}的基
	\begin{equation*}
		\qty{\V{g}^i}^m_{i=1} \subset \Rm
	\end{equation*}
	称为\emphA{逆变基}。
	它们满足\emphA{对偶关系}:
	\begin{equation}
		\qty(\V{g}^i,\,\V{g}_j)_{\Rm}=\KroneckerDelta{i}{j}
		=\left\{\begin{aligned}
			1,\quad i&=j \semicomma\\
			0,\quad i&\neq j \fullstop
		\end{aligned}\right.
		\label{eq:对偶关系}
	\end{equation}
	这里的 $\KroneckerDelta{i}{j}$ 是 \emphA{Kronecker δ 函数}。
	
\subsection{度量}
	下面引入\emphA{度量}的概念。其定义为
	\begin{braceEq}
		g_{ij} &\defeq \qty(\V{g}_i,\,\V{g}_j)_{\Rm} \comma 
		\label{eq:度量的定义_协变} \\
		g^{ij} &\defeq \qty(\V{g}^i,\,\V{g}^j)_{\Rm} \fullstop
		\label{eq:度量的定义_逆变}
	\end{braceEq}
	
	下面证明
	\begin{equation}
		g_{ik}\,g^{kj} = \KroneckerDelta{j}{i} \fullstop
		\label{eq:度量之积}
	\end{equation}
	它也可以写成矩阵的形式:
	\begin{equation}
		\qty[g_{ik}]\qty[g^{kj}]
		=\qty[\KroneckerDelta{j}{i}]=\Mat{I}_m \comma
	\end{equation}
	其中的 $\Mat{I}_m$ 是 $m$ 阶单位阵。
	\begin{myProof}
		\begin{equation}
			g_{ik}\,g^{kj}
			=\qty(\V{g}_i,\,\V{g}_k)_{\Rm} \, g^{kj}
			=\qty(\V{g}_i,\,g^{kj}\V{g}_k)_{\Rm}
		\end{equation}
		后文将说明 $g^{kj}\V{g}_k=\V{g}^j$,因此可得
		\begin{equation}
			g_{ik}\,g^{kj}=\qty(\V{g}_i,\,\V{g}^j)_{\Rm}
			=\KroneckerDelta{j}{i} \fullstop
		\end{equation}
		
		要注意的是,这里的指标 $k$ 是\emphB{哑标}。
		根据\emphA{Einstein 求和约定},
		重复指标并且一上一下时,就表示对它求和。后文除非特殊说明,也均是如此。
	\end{myProof}
	
	现在澄清\emphA{基向量转换关系}。第 $i$ 个协变基向量 $\V{g}_i$ 既然是向量,
	就必然可以用协变基或逆变基来表示。
	根据对偶关系式~\eqref{eq:对偶关系} 和度量的定义
	式~\eqref{eq:度量的定义_协变}、\eqref{eq:度量的定义_逆变},可知
	\begin{braceEq}
		\V{g}_i &=\qty(\V{g}_i,\,\V{g}_k)_{\Rm} \,\V{g}^k
		=g_{ik}\,\V{g}^k \comma \label{eq:基向量转换关系1} \\
		\V{g}_i &=\qty(\V{g}_i,\,\V{g}^k)_{\Rm} \,\V{g}_k
		=\KroneckerDelta{k}{i}\V{g}_k \label{eq:基向量转换关系2}
	\end{braceEq}
	以及
	\begin{braceEq}
		\V{g}^i &=\qty(\V{g}^i,\,\V{g}_k)_{\Rm} \,\V{g}^k
		=\KroneckerDelta{i}{k}\,\V{g}^k \comma \label{eq:基向量转换关系3} \\
		\V{g}^i &=\qty(\V{g}^i,\,\V{g}^k)_{\Rm} \,\V{g}_k
		=g^{ik}\V{g}_k \fullstop \label{eq:基向量转换关系4}
	\end{braceEq}
	这四个式子中,式~\eqref{eq:基向量转换关系2} 和
	\eqref{eq:基向量转换关系3} 是平凡的,
	而式~\eqref{eq:基向量转换关系1} 和 \eqref{eq:基向量转换关系4}
	则通过\emphB{度量}建立起了协变基与逆变基之间的关系。
	这就称为基向量转换关系,也可以叫做“指标升降游戏”。
	
\subsection{向量的分量}
	对于任意的向量 $\V{\xi} \in \Rm$,它可以用协变基表示:
	\begin{equation}
		\V{\xi}=\ipb{\V{\xi}}{\V{g}^k}\,\V{g}_k
		=\xi^k \V{g}_k \comma
	\end{equation}
	也可以用逆变基表示:
	\begin{equation}
		\V{\xi}=\ipb{\V{\xi}}{\V{g}_k}\,\V{g}^k
		=\xi_k \V{g}^k \comma
	\end{equation}
	式中,$\xi^k$ 是 $\V{\xi}$ 与第 $k$ 个\emphB{逆变基}做内积的结果,
	称为 $\V{\xi}$ 的第 $k$ 个\emphA{逆变分量};
	而 $\xi_k$ 是 $\V{\xi}$ 与第 $k$ 个\emphB{协变基}做内积的结果,
	称为 $\V{\xi}$ 的第 $k$ 个\emphA{协变分量}。
	
	以后凡是指标在下的(下标),均称为\emphB{协变}某某;
	指标在上的(上标),称为\emphB{逆变}某某。
	
\section{张量的表示}
\subsection{张量的表示与简单张量}\label{subsec:张量的表示与简单张量}
	所谓\emphA{张量},即\emphA{多重线性函数}。
	
	首先用三阶张量举个例子。考虑任意的 $\T{\Phi}\in\Tensors{3}$,
	其中的 $\Tensors{3}$ 表示以 $\Rm$ 为底空间的三阶张量全体。
	所谓三阶(或三重)线性函数,指“吃掉”三个向量之后变成数,
	并且“吃法”具有线性性。
	
	一般地,$r$ 阶张量的定义如下:
	\begin{equation}
		\mdef{\T{\Phi}}{\underbrace{\Rm\times\Rm\times\cdots\times\Rm}_
			{\text{$r$ 个 $\Rm$}}
			\ni\qty{\V{u}_1,\,\V{u}_2,\,\cdots,\,\V{u}_r}}
		{\T{\Phi}\qty(\V{u}_1,\,\V{u}_2,\,\cdots,\,\V{u}_r)} \comma
	\end{equation}
	式中的 $\T{\Phi}$ 满足
	\begin{align}
		\forall\,\alpha,\,\beta\in\realR,\,
		&\mathrel{\phantom{=}}\T{\Phi}\qty(\V{u}_1,\,\cdots,\,
			\alpha\tilde{\V{u}}_i+\beta\hat{\V{u}}_i,\,\cdots,\,\V{u}_r)
			\notag \\
		&=\alpha\,\T{\Phi}\qty(\V{u}_1,\,\cdots,\,
			\tilde{\V{u}}_i,\,\cdots,\,\V{u_r})
		+\beta\,\T{\Phi}\qty(\V{u}_1,\,\cdots,\,
			\hat{\V{u}}_i,\,\cdots,\,\V{u_r}) \comma
	\end{align}
	即所谓“\emphB{对第 $i$ 个变元的线性性}”。

	在张量空间 $\Tensors{r}$ 上,我们引入线性结构:
	\begin{align}
		\forall\,\alpha,\,\beta\in\realR,\,
		\T{\Phi},\,\T{\Psi}\in\Tensors{r},
		&\mathrel{\phantom{=}}\qty(\alpha\,\T{\Phi}+\beta\,\T{\Psi})
		\qty(\V{u}_1,\,\V{u}_2,\,\cdots,\,\V{u}_r) \notag \\
		&\mathrel{\defeq}
			\alpha\,\T{\Phi}\qty(\V{u}_1,\,\V{u}_2,\,\cdots,\,\V{u}_r)
			+\beta\,\T{\Psi}\qty(\V{u}_1,\,\V{u}_2,\,\cdots,\,\V{u}_r)
		\comma
	\end{align}
	于是
	\begin{equation}
		\alpha\,\T{\Phi}+\beta\,\T{\Psi} \in \Tensors{r} \fullstop
	\end{equation}
	
	下面我们要获得 $\T{\Phi}$ 的表示。
	根据之前任意向量用协变基或逆变基的表示,有
	\begin{align}
		\forall\,\V{u},\,\V{v},\,\V{w}\in\Rm,
		&\mathrel{\phantom{=}}
		\T{\Phi}\qty(\V{u},\,\V{v},\,\V{w}) \notag\\
		&=\T{\Phi}\qty(u^i\V{g}_i,\,v_j\V{g}^j,\,w^k\V{g}_k) \notag
		\intertext{考虑到 $\T{\Phi}$ 对第一变元的线性性,可得}
		&=u^i\,\T{\Phi}\qty(\V{g}_i,\,v_j\V{g}^j,\,w^k\V{g}_k) \notag
		\intertext{同理,}
		&=u^i v_j w^k\,\T{\Phi}\qty(\V{g}_i,\,\V{g}^j,\,\V{g}_k)
		\fullstop
		\label{eq:张量的表示1}
	\end{align}
	注意这里自然需要满足 Einstein 求和约定。
	
	上式中的 $\T{\Phi}\qty(\V{g}_i,\,\V{g}^j,\,\V{g}_k)$ 是一个数。
	它是张量 $\T{\Phi}$ “吃掉”三个基向量的结果。
	至于 $u^i v_j w^k$ 部分,三项分别是 $\V{u}$ 的第 $i$ 个逆变分量、
	$\V{v}$ 的第 $j$ 个协变分量和 $\V{w}$ 的第 $k$ 个逆变分量。
	根据向量分量的定义,可知
	\begin{equation}
		u^i v_j w^k
		= \ipb{\V{u}}{\V{g}^i}
		\cdotp \ipb{\V{v}}{\V{g}_j}
		\cdotp \ipb{\V{w}}{\V{g}^k} \fullstop
		\label{eq:张量的表示2}
	\end{equation}
	
	\blankline
	
	暂时中断一下思路,先给出\emphA{简单张量}的定义。
	\begin{equation}
		\forall\,\V{u},\,\V{v},\,\V{w}\in\Rm,\quad
		\V{\xi}\tp\V{\eta}\tp\V{\zeta}\qty(\V{u},\,\V{v},\,\V{w})
		\defeq \ipb{\V{\xi}}{\V{u}}
		\cdotp \ipb{\V{\eta}}{\V{v}}
		\cdotp \ipb{\V{\zeta}}{\V{w}} \in\realR \comma
	\end{equation}
	式中 $\V{\xi},\,\V{\eta},\,\V{\zeta}\in\Rm$,
	而暂时把 $\V{\xi}\tp\V{\eta}\tp\V{\zeta}$ 理解为一种记号。
	简单张量作为一个映照,组成它的三个向量分别与它们“吃掉”的第一、二、三个变元
	做内积并相乘,结果为一个实数。
	
	考虑到内积的线性性,便有(以第二个变元为例)
	\begin{align}
		\V{\xi}\tp\V{\eta}\tp\V{\zeta}
		\qty(\V{u},\,\alpha\tilde{\V{v}}+\beta\hat{\V{v}},\,\V{w})
		&\defeq \ipb{\V{\xi}}{\V{u}}
		\cdotp \ipb{\V{\eta}}{\alpha\tilde{\V{v}}+\beta\hat{\V{v}}}
		\cdotp \ipb{\V{\zeta}}{\V{w}} \in\realR \notag
		\intertext{注意到
			$\ipb{\V{\eta}}{\alpha\tilde{\V{v}}+\beta\hat{\V{v}}}
				=\alpha\ipb{\V{\eta}}{\tilde{\V{v}}}
				+\beta\ipb{\V{\eta}}{\hat{\V{v}}}$,
			同时再次利用简单张量的定义,可得}
		&= \alpha \V{\xi}\tp\V{\eta}\tp\V{\zeta}
			\qty(\V{u},\,\tilde{\V{v}},\,\V{w})
			+\beta \V{\xi}\tp\V{\eta}\tp\V{\zeta}
			\qty(\V{u},\,\hat{\V{v}},\,\V{w}) \fullstop
	\end{align}
	类似地,对第一变元和第三变元,同样具有线性性。因此,可以知道
	\begin{equation}
		\V{\xi}\tp\V{\eta}\tp\V{\zeta}
		\in\Tensors{3} \fullstop
	\end{equation}
	可见,“简单张量”的名字是名副其实的,它的确是一个特殊的张量。
	
	回过头来看 \eqref{eq:张量的表示2}~式。很明显,它可以用简单张量来表示。
	要注意,由于内积的对称性,可以有两种\footnote{%
		这里只考虑把 $\V{u}$、$\V{v}$、$\V{w}$%
		和 $\V{g}^i$、$\V{g}_j$、$\V{g}^k$ 分别放在一起的情况。}表示方法:
	\begin{gather}
		\V{g}^i\tp\V{g}_j\tp\V{g}^k
		\qty(\V{u},\,\V{v},\,\V{w})
		\intertext{或者}
		\V{u}\tp\V{v}\tp\V{w}
		\qty(\V{g}^i,\,\V{g}_j,\,\V{g}^k) \comma
	\end{gather}
	我们这里取上面一种。代入式~\eqref{eq:张量的表示1},得
	\begin{align}
		&\mathrel{\phantom{=}}
			\T{\Phi}\qty(\V{u},\,\V{v},\,\V{w}) \notag\\
		&=\T{\Phi}\qty(\V{g}_i,\,\V{g}^j,\,\V{g}_k)
			\cdotp\V{g}^i\tp\V{g}_j\tp\V{g}^k
			\qty(\V{u},\,\V{v},\,\V{w}) \notag
		\intertext{由于
			$\T{\Phi}\qty(\V{g}_i,\,\V{g}^j,\,\V{g}_k) \in\Rm$,因此}
		&=\qty[\T{\Phi}\qty(\V{g}_i,\,\V{g}^j,\,\V{g}_k)
			\V{g}^i\tp\V{g}_j\tp\V{g}^k]
			\qty(\V{u},\,\V{v},\,\V{w}) \fullstop
	\end{align}
	方括号里的部分,就是根据 Einstein 求和约定,
	用 $\T{\Phi}\qty(\V{g}_i,\,\V{g}^j,\,\V{g}_k)$
	对 $\V{g}^i\tp\V{g}_j\tp\V{g}^k$ 进行线性组合。
	
	由于 $\V{u}$、$\V{v}$、$\V{w}$ 选取的任意性,可以引入如下记号:
	\begin{equation}
		\T{\Phi}
		=\T{\Phi}\qty(\V{g}_i,\,\V{g}^j,\,\V{g}_k) \,
			\V{g}^i\tp\V{g}_j\tp\V{g}^k
		\eqcolon \tensor{\Phi}{_i^j_k} \,
			\V{g}^i\tp\V{g}_j\tp\V{g}^k \comma
	\end{equation}
	即
	\begin{equation}
		\tensor{\Phi}{_i^j_k}
		\coloneq \T{\Phi}\qty(\V{g}_i,\,\V{g}^j,\,\V{g}_k) \comma
	\end{equation}
	这称为张量的\emphA{分量}。
	它说明一个张量可以用\emphB{张量分量}和基向量组成的\emphB{简单张量}来表示。
	
	指标 $i$、$j$、$k$ 的上下是任意的。这里,
	它有赖于式~\eqref{eq:张量的表示1} 中基向量的选取。
	实际上,对于这里的三阶张量,指标的上下一共有 8 种可能。
	指标全部在下面的,称为\emphA{协变分量}:
	\begin{equation}
		\tensor{\Phi}{^i^j^k} \coloneq
		\T{\Phi}\qty(\V{g}^i,\,\V{g}^j,\,\V{g}^k) \semicomma
	\end{equation}
	指标全部在上面的,称为\emphA{逆变分量}:
	\begin{equation}
		\tensor{\Phi}{_i_j_k} \coloneq
		\T{\Phi}\qty(\V{g}_i,\,\V{g}_j,\,\V{g}_k) \semicomma
	\end{equation}
	其余 6 种,称为\emphA{混合分量}。
	对于一个 $r$ 阶张量,显然共有 $2^r$ 种分量表示,
	其中协变分量与逆变分量各一种,混合分量 $2^r-2$ 种。
	
\subsection{张量分量之间的关系}
	我们已经知道,
	对于任意一个向量 $\V{\xi}\in\Rm$,它可以用协变基或逆变基表示:
	\begin{equation}
		\V{\xi}=\left\{\begin{aligned}
			\xi^i\V{g}_i \comma \\
			\xi_i\V{g}^i \fullstop
		\end{aligned}\right.
	\end{equation}
	式中,协变分量与逆变分量满足\emphB{坐标转换关系}:
	\begin{braceEq}
		\xi^i &=\ipb{\V{\xi}}{\V{g}^i}
		=\ipb{\V{\xi}}{g^{ik}\V{g}_k}
		=g^{ik}\ipb{\V{\xi}}{\V{g}_k}
		=g^{ik}\xi_k \comma \\
		\xi_i &=\ipb{\V{\xi}}{\V{g}_i}
		=\ipb{\V{\xi}}{g_{ik}\V{g}^k}
		=g_{ik}\ipb{\V{\xi}}{\V{g}^k}
		=g_{ik}\xi^k \fullstop
	\end{braceEq}
	每一式的第二个等号都用到了\emphB{基向量转换关系},
	见式~\eqref{eq:基向量转换关系1} 和 \eqref{eq:基向量转换关系4}。
	
	现在再来考虑张量的分量。仍以上文中的张量 $\tensor{\Phi}{_i^j_k}
		\coloneq \T{\Phi}\qty(\V{g}_i,\,\V{g}^j,\,\V{g}_k)$ 为例,
	我们想要知道它与张量 $\tensor{\Phi}{^p_q^r} \coloneq
		\T{\Phi}\qty(\V{g}^p,\,\V{g}_q,\,\V{g}^r)$ 之间的关系。
	利用基向量转换关系,可有
	\begin{align}
		\tensor{\Phi}{_i^j_k}
		&\coloneq\T{\Phi}\qty(\V{g}_i,\,\V{g}^j,\,\V{g}_k) \notag \\
		&=\T{\Phi}
			\qty(g_{ip}\V{g}^p,\,g^{jq}\V{g}_q,\,g_{kr}\V{g}^r) \notag
		\intertext{又利用张量的线性性,得}
		&=g_{ip}g^{jq}g_{kr}
			\T{\Phi}\qty(\V{g}^p,\,\V{g}_q,\,\V{g}^r) \notag \\
		&=g_{ip}g^{jq}g_{kr} \tensor{\Phi}{^p_q^r} \fullstop
	\end{align}
	可见,张量的分量与向量的分量类似,其指标升降可通过\emphB{度量}来实现。
	用同样的手法,还可以得到诸如
	$\tensor{\Phi}{^i^j^k}=g^{jp}\tensor{\Phi}{^i_p^k}$、
	$\tensor{\Phi}{^i_j^k}=g_{jp}g^{kq}\tensor{\Phi}{^i^p_k}$
	这样的关系式。
	
\subsection{相对不同基的张量分量之间的关系}
	$\Rm$ 空间中,除了 $\qty{\V{g}_i}^m_{i=1}$ 和相应的
	对偶基 $\qty{\V{g}^i}^m_{i=1}$ 之外,当然还可以有其他的基,
	比如带括号的 $\qty{\V{g}_{(i)}}^m_{i=1}$ 以及对应的
	对偶基 $\qty{\V{g}^{(i)}}^m_{i=1}$。
	前者对应形如 $\tensor{\Phi}{^i_j^k}
		\coloneq \T{\Phi}\qty(\V{g}^i,\,\V{g}_j,\,\V{g}^k)$ 的张量,
	后者则对应带括号的张量,如 $\tensor{\Phi}{^{(p)}_{(q)}^{(r)}} \coloneq
		\T{\Phi}\qty(\V{g}^{(p)},\,\V{g}_{(q)},\,\V{g}^{(r)})$。
	下面我们来探讨这两个张量的关系。
	
	首先来建立基之间的关系。带括号的第 $i$ 个基向量
	$\V{g}_{(i)}$,作为 $\Rm$ 空间中的一个向量,自然可以用另一组基来表示:
	\begin{equation}
		\V{g}_{(i)}=\left\{\begin{aligned}
			\ipb{\V{g}_{(i)}}{\V{g}_k}\,\V{g}^k \comma \\
			\ipb{\V{g}_{(i)}}{\V{g}^k}\,\V{g}_k \fullstop
		\end{aligned}\right.
	\end{equation}
	同理,自然还有它的对偶基:
	\begin{equation}
		\V{g}^{(i)}=\left\{\begin{aligned}
			\ipb{\V{g}^{(i)}}{\V{g}_k}\,\V{g}^k \comma \\
			\ipb{\V{g}^{(i)}}{\V{g}^k}\,\V{g}_k \fullstop
		\end{aligned}\right.
	\end{equation}
	引入记号 $c^k_{(i)} \coloneq \ipb{\V{g}_{(i)}}{\V{g}^k}$
	和 $c^{(i)}_k \coloneq \ipb{\V{g}^{(i)}}{\V{g}_k}$,那么有
	\begin{braceEq}
		\V{g}_{(i)} &= c^k_{(i)}\V{g}_k \comma \\
		\V{g}^{(i)} &= c^{(i)}_k\V{g}^k \fullstop
	\end{braceEq}
	
	容易看出,这两个系数具有如下性质:
	\begin{equation}
		c^{(i)}_k c^k_{(j)} = \KroneckerDelta{i}{j} \fullstop
	\end{equation}
	写成矩阵形式\footnote{%
		通常我们约定上面的标号作为行号,下面的标号作为列号。},为
	\begin{equation}
		\qty[c^{(i)}_k]\qty[c^k_{(j)}]
		=\qty[\KroneckerDelta{j}{i}]=\Mat{I}_m \fullstop
	\end{equation}
	换句话说,两个系数矩阵是互逆的。
	\begin{myProof}
		\begin{align}
			c^{(i)}_k c^k_{(j)}
			&=\ipb{\V{g}^{(i)}}{\V{g}_k}\,c^k_{(j)} \notag
			\intertext{利用内积的线性性,有}
			&=\ipb{\V{g}^{(i)}}{c^k_{(j)} \V{g}_k} \notag
			\intertext{根据 $c^k_{(j)}$ 的定义,得到}
			&=\ipb{\V{g}^{(i)}}{\V{g}_{(j)}} \fullstop
		\end{align}
		带括号的基同样满足对偶关系 \eqref{eq:对偶关系}~式,于是得证。
	\end{myProof}
	
	上面我们用不带括号的基表示了带括号的基。反之也是可以的:
	\begin{braceEq}
		\V{g}_i &= \ipb{\V{g}_i}{\V{g}^{(k)}}{\Rm}\,\V{g}_{(k)}
			=c^{(k)}_i\V{g}_{(k)} \comma \\
		\V{g}^i &= \ipb{\V{g}^i}{\V{g}_{(k)}}{\Rm}\,\V{g}^{(k)}
			=c^i_{(k)}\V{g}^{(k)} \fullstop
	\end{braceEq}
	这样一来,就建立起了不同基之间的转换关系。
	
	现在我们回到张量。根据张量分量的定义,
	\begin{align}
		\tensor{\Phi}{^i_j^k}
		&\coloneq \T{\Phi}\qty(\V{g}^i,\,\V{g}_j,\,\V{g}^k) \notag
		\intertext{利用之前推导的不同基向量之间的转换关系,得}
		&=\T{\Phi}\qty(
			c^i_{(p)}\V{g}^{(p)},\,c^{(q)}_j\V{g}_{(q)},\,
			c^k_{(r)}\V{g}^{(r)}) \notag
		\intertext{由张量的线性性,提出系数:}
		&=c^i_{(p)} c^{(q)}_j c^k_{(r)} \,
			\T{\Phi}\qty(\V{g}^{(p)},\,\V{g}_{(q)},\V{g}^{(r)}) \notag\\
		&=c^i_{(p)} c^{(q)}_j c^k_{(r)} \,
			\tensor{\Phi}{^{(p)}_{(q)}^{(r)}} \fullstop
	\end{align}
	完全类似,还可以有
	\begin{equation}
		\tensor{\Phi}{^{(i)}_{(j)}^{(k)}}
		=c^{(i)}_p c^g_{(j)} c^{(k)}_r \tensor{\Phi}{^p_q^r} \fullstop
	\end{equation}
	
	\blankline
	
	总结一下这两小节得到的结果。
	对于同一组基下的张量分量,其指标升降通过\emphB{度量}来实现;
	对于不同基下的张量分量,其指标转换则通过不同基之间的转换系数来完成。
	
	\chapter{张量的代数运算}
		\section{张量积}
	\emphA{张量积}也叫\emphA{张量并},用符号“$\tp$”表示。
	在 \ref{subsec:张量的表示与简单张量}~小节给出简单张量的定义时,
	实际上就用到了张量积。张量积的定义为:
	\begin{align}
		\forall\,\T{\Phi}\in\Tensors{p},\,\T{\Psi}\in\Tensors{q},
		&\mathrel{\phantom{=}} \T{\Phi}\tp\T{\Psi}
			\in\Tensors{p+q} \notag \\
		&=\qty(\Phi^{i_1 \cdots i_p} \,
				\V{g}_{i_1}\tp\cdots\tp\V{g}_{i_p})
			\tp \qty(\Psi_{j_1 \cdots j_q} \,
				\V{g}^{j_1}\tp\cdots\tp\V{g}^{j_q}) \notag \\
		&\defeq \Phi^{i_1 \cdots i_p} \,
			\Psi_{j_1 \cdots j_q}\,
			\qty(\V{g}_{i_1}\tp\cdots\tp\V{g}_{i_p})
			\tp \qty(\V{g}^{j_1}\tp\cdots
				\tp\V{g}^{j_q}_{\phantom{i_p}}) \fullstop
	\end{align}
	由该定义可以知道,关于简单张量 $\qty(\V{g}_{i_1}\tp\cdots
		\tp\V{g}_{i_p}) \tp \qty(\V{g}^{j_1}\tp\cdots
		\tp\V{g}^{j_q}_{\phantom{i_p}})$,相应的张量分量为
	\begin{equation}
		\tensor{\qty\big(\Phi\tp\Psi)}
			{^{i_1 \cdots i_p}_{j_1 \cdots j_q}} \fullstop
	\end{equation}
	
\section{\texorpdfstring{$e$ 点积}{e 点积}}
	张量的 \emphA{$e$ 点积}可以用符号“$\edp$”表示。
	从这个符号可以看出 $e$ 点积的作用:前 $e$ 个指标缩并,后面的点乘。
	
	对于任意的 $\T{\Phi}\in\Tensors{p},\,
		\T{\Psi}\in\Tensors{q},\,
		e\leqslant\min\qty{p,\,q}\in\natN$,$e$ 点积是这样定义的:
	\begin{align}
		&\mathrel{\phantom{=}} \T{\Phi}\edp\T{\Psi} \notag \\
		&=\qty(\Phi^{i_1 \cdots i_{p-e} i_{p-e+1} \cdots i_p} \,
			\V{g}_{i_1}\tp\cdots\tp\V{g}_{i_{p-e}}
			\tp\hl{\V{g}_{i_{p-e+1}}\tp\cdots\tp\V{g}_{i_p}}
			) \notag \\
		&\mathrel{\phantom{=}}\quad\edp
			\qty(\Psi^{j_1 \cdots j_e j_{e+1} \cdots j_q} \,
			\hl{\V{g}_{j_1}\tp\cdots\tp\V{g}_{j_e}}
			\tp\V{g}_{j_{e+1}}\tp\cdots\tp\V{g}_{j_q}) \notag
		\intertext{把高亮的部分做内积,得到\emphB{度量}:}
		&\defeq\Phi^{i_1 \cdots i_{p-e} i_{p-e+1} \cdots i_p} \,
			\Psi^{j_1 \cdots j_e j_{e+1} \cdots j_q} \notag \\
		&\mathrel{\phantom{=}}\quad\cdotp
			g_{i_{p-e+1} j_1} \cdots g_{i_p j_e} \,
			\qty(\V{g}_{i_1}\tp\cdots\tp\V{g}_{i_{p-e}})
			\tp\qty(\V{g}_{j_{e+1}}\tp\cdots\tp\V{g}_{j_q}) \notag
		\intertext{玩一下“指标升降游戏”(注意有两种结合方式:
			与 $\Phi$ 或 $\Psi$),可得}
		&=\left\{\begin{lgathered}
				\tensor{\Phi}{^{i_1 \cdots i_{p-e}}_{\hl{j_1 \cdots j_e}}} \,
				\Psi^{\hl{j_1 \cdots j_e} j_{e+1} \cdots j_q} \\
				\Phi^{i_1 \cdots i_{p-e} \hl{i_{p-e+1} \cdots i_p}} \,
				\tensor{\Psi}{_{\hl{i_{p-e+1} \cdots i_p}}^{j_{e+1}
					\cdots j_q}}
			\end{lgathered}\right\}
			\qty(\V{g}_{i_1}\tp\cdots\tp\V{g}_{i_{p-e}})
			\tp\qty(\V{g}_{j_{e+1}}\tp\cdots\tp\V{g}_{j_q}) \fullstop
	\end{align}
	最后一步的大括号中,高亮的 $j_1 \cdots j_e$
	和 $i_{p-e+1} \cdots i_p$ 都是哑标,可以通过求和求掉。因此有
	\begin{equation}
		\T{\Phi}\edp\T{\Psi} \in \Tensors{p+q-2e} \fullstop
	\end{equation}
	换句话说,$e$ 点积的作用就是将指标\emphB{哑标化}。
	
	作为一个特殊的应用,接下来我们介绍\emphA{全点积},用符号“\fdp”表示。
	对于任意的 $\T{\Phi},\,\T{\Psi}\in\Tensors{p}$,有
	\begin{align}
		&\mathrel{\phantom{=}} \T{\Phi}\fdp\T{\Psi}
			\defeq \T{\Phi}\edp[p]\T{\Psi} \notag \\
		&=\qty(\Phi^{i_1 \cdots i_p}\,\V{g}_{i_1}\tp\cdots\tp\V{g}_{i_p})
			\edp[p]
			\qty(\Psi^{j_1 \cdots j_p}\,\V{g}_{j_1}\tp\cdots\tp\V{g}_{j_p})
			\notag \\
		&=\Phi^{i_1 \cdots i_p} \, \Psi^{j_1 \cdots j_p} \,
			g_{i_1 j_1} \cdots g_{i_p j_p} \notag \\
		&=\left\{\begin{lgathered}
				\Phi_{j_1 \cdots j_p} \, \Psi^{j_1 \cdots j_p} \\
				\Phi^{i_1 \cdots i_p} \, \Psi_{i_1 \cdots i_p}
			\end{lgathered}\right.
			\in\realR \fullstop
	\end{align}
	可见,全点积将\emphB{全部}指标哑标化。
	
	张量自身和自身的全点积,定义为它的\emphA{范数}:
	\begin{equation}
		\T{\Phi}\fdp\T{\Phi}
		=\Phi^{i_1 \cdots i_p} \, \Phi_{i_1 \cdots i_p}
		\eqcolon \qty|\T{\Phi}|^2_{\Tensors{p}} \fullstop
	\end{equation}
	
\section{叉乘}
	张量的\emphA{叉乘}要求底空间为 $\realR^3$。
	对于任意的 $\T{\Phi}\in\Tensors[\realR^3]{p},\,
	\T{\Psi}\in\Tensors[\realR^3]{q}$,叉乘的定义如下:
	\begin{align}
		&\mathrel{\phantom{=}} \T{\Phi}\cp\T{\Psi} \notag \\
		&=\qty(\Phi^{i_1 \cdots i_{p-1} i_p} \,
				\V{g}_{i_1}\tp\cdots\tp\V{g}_{i_{p-1}}\tp\V{g}_{i_p})
			\cp \qty(\Psi_{j_1 j_2 \cdots j_q} \,
				\V{g}^{j_1}\tp\V{g}^{j_2}\cdots\tp\V{g}^{j_q}) \notag \\
		&\defeq \Phi^{i_1 \cdots i_p} \, \Psi_{j_1 \cdots j_p} \,
			\V{g}_{i_1}\tp\cdots\tp\V{g}_{i_{p-1}}
			\tp\qty(\V{g}_{i_p}\cp\V{g}^{j_1})
			\tp\V{g}^{j_2}\cdots\tp\V{g}^{j_q}
			\in\Tensors[\realR^3]{p+q-1} \fullstop
	\end{align}
	注意到,此时简单张量的维数已经降了一阶。
	
	利用\emphA{Levi-Civita 记号},可以进一步展开上式。
	\begin{align}
		\V{g}_{i_p}\cp\V{g}^{j_1}
		=\LeviCivita{_{i_p}^{j_1}_s}\,\V{g}^s \comma
	\end{align}
	式中的
	\begin{equation}
		\LeviCivita{_{i_p}^{j_1}_s}
		=\det[\V{g}_{i_p},\,\V{g}^{j_1},\,\V{g}_s] \fullstop
	\end{equation}
	于是
	\begin{equation}
		\T{\Phi}\cp\T{\Psi} \,
		=\LeviCivita{_{i_p}^{j_1}_s}\,
			\Phi^{i_1 \cdots i_p} \, \Psi_{j_1 \cdots j_p}
			\V{g}_{i_1}\tp\cdots\tp\V{g}_{i_{p-1}} \tp\V{g}^s
			\tp\V{g}^{j_2}\cdots\tp\V{g}^{j_q} \fullstop
	\end{equation}
	
	下面我们再来类比地定义一种混合积“$\edp[\cp]$”。
	对于任意的 $\T{\Phi},\,\T{\Psi}\in\Tensors{3}$,定义
	\begin{align}
		\T{\Phi}\edp[\cp]\T{\Psi}
		&=\qty(\Phi^{ijk}\,\V{g}_i\tp\V{g}_j\tp\V{g}_k)
			\edp[\cp]\qty(\Psi_{pqr}\,\V{g}^p\tp\V{g}^q\tp\V{g}^r)\notag \\
		&\defeq \Phi^{ijk}\,\Psi_{pqr} \,
			\KroneckerDelta{q}{j} \,
			\V{g}_i\tp\qty(\V{g}_k\cp\V{g}^p)\tp\V{g}^r \notag
		\intertext{缩并掉 Kronecker δ,
			同时利用 Levi-Civita 记号展开叉乘项,可有}
		&=\LeviCivita{_k^p_s}\,\Phi^{ijk}\,\Psi_{pjr}\,
			\V{g}_i\tp\V{g}^s\tp\V{g}^r \comma
	\end{align}
	式中的
	\begin{equation}
		\LeviCivita{_k^p_s}=\det[\V{g}_k,\,\V{g}^p,\,\V{g}_s] \fullstop
	\end{equation}
	
	对于这种混合积,并没有一般的约定。不同的研究者往往会采用不同的写法及表示。
	
\section{置换(一)}
	本节主要介绍\emphA{置换运算}的定义及相关概念,
	这将使我们暂时离开张量运算的主线。
	
	置换运算实际上是一种交换位置或者改变次序的运算。
	之后我们还将引入针对张量的\emph{置换算子},它是外积运算和外微分运算的基础。
	这些运算是现代张量分析与微分几何的支柱。
	
\subsection{置换的定义}
	我们从一个例子开始。下面是一个 $2 \times 7$ 的“矩阵”:
	\begin{equation}
		\Perm{\sigma}=\mqty[
			\circNum{1} & \circNum{2} & \circNum{3} & \circNum{4} &
				\circNum{5} & \circNum{6} & \circNum{7} \\
			\circNum{7} & \circNum{4} & \circNum{5} & \circNum{1} &
				\circNum{6} & \circNum{2} & \circNum{3}
		] \fullstop
		\label{eq:置换序号定义}
	\end{equation}
	矩阵里面的每一个数字表示一个位置。可以想象成 7 把椅子,
	先是按第一行的顺序依次排列,再按照第二行的顺序打乱,重新排列。
	于是这就成为一个\emphA{7 阶置换}。这个定义等价于
	\begin{mySubEq}
		\begin{gather}
			\Perm{\sigma}=\mqty*(
				4 & 9 & 2 & 7 & 5 & 8 & 3 \\
				3 & 7 & 5 & 4 & 8 & 9 & 2
			) \comma \label{eq:置换元素表示_数字}
			\intertext{自然也等价于}
			\Perm{\sigma}=\mqty*(
				\spadesuit & \heartsuit & \diamondsuit & \clubsuit &
					\varspadesuit & \varheartsuit & \vardiamondsuit \\
				\vardiamondsuit & \clubsuit & \varspadesuit & \spadesuit &
					\varheartsuit & \heartsuit & \diamondsuit
			) \comma \label{eq:置换元素表示_符号}
		\end{gather}
	\end{mySubEq}
	当然,换用任何元素也都是可以的。
	
	通常我们用方括号表示置换的\emphA{序号定义},即标号的排列轮换;
	用圆括号表示\emphA{元素定义},即标号对应元素的轮换。
	
\subsection{置换的符号}
	接着来定义置换的\emphA{符号} $\sgn\Perm{\sigma}$。
	这里我们把每次交换两个数字称为一次“操作”。
	如果经过\emphB{偶数次}“操作”,可以把经置换后的序列恢复为原来的顺序,
	那么该置换的符号 $\sgn\Perm{\sigma} = 1$;
	而如果经过\emphB{奇数次}“操作”才可以复原,则 $\sgn\Perm{\sigma}=-1$。
	若用一个式子表示,则为
	\begin{equation}
		\sgn\Perm{\sigma} = (-1)^n \comma
	\end{equation}
	其中的 $n$ 是恢复原本顺序所需“操作”的次数.
	
	下面我们以式~\eqref{eq:置换序号定义} 所定义的 $\Perm{\sigma}$ 为例,
	演示求置换符号的过程。这里的关键是通过两两交换,
	按如下步骤把式~\eqref{eq:置换元素表示_符号} 的第二行变换成第一行:
	\begin{gather*}
		\mqty{
			\hl{\vardiamondsuit} & \hlw{\clubsuit} & \hlw{\varspadesuit} &
				\hl{\spadesuit} & \hlw{\varheartsuit} & \hlw{\heartsuit} &
				\hlw{\diamondsuit}
		} \\
		\mqty{ & & & \Downarrow & & & } \\
		\mqty{
			\hl[pink]{\spadesuit} & \hl{\clubsuit} & \hlw{\varspadesuit} &
				\hl[pink]{\vardiamondsuit} & \hlw{\varheartsuit} &
				\hl{\heartsuit} & \hlw{\diamondsuit}
		} \\
		\mqty{ & & & \Downarrow & & & } \\
		\mqty{
			\hlw{\spadesuit} & \hl[pink]{\heartsuit} & \hl{\varspadesuit} &
				\hlw{\vardiamondsuit} & \hlw{\varheartsuit} &
				\hl[pink]{\clubsuit} & \hl{\diamondsuit}
		} \\
		\mqty{ & & & \Downarrow & & & } \\
		\mqty{
			\hlw{\spadesuit} & \hlw{\heartsuit} & \hl[pink]{\diamondsuit} &
				\hl{\vardiamondsuit} & \hlw{\varheartsuit} & \hl{\clubsuit} &
				\hl[pink]{\varspadesuit}
		} \\
		\mqty{ & & & \Downarrow & & & } \\
		\mqty{
			\hlw{\spadesuit} & \hlw{\heartsuit} & \hlw{\diamondsuit} &
				\hl[pink]{\clubsuit} & \hl{\varheartsuit} &
				\hl[pink]{\vardiamondsuit} & \hl{\varspadesuit}
		} \\
		\mqty{ & & & \Downarrow & & & } \\
		\phantom{\mspace{10mu}}\mqty{
			\hlw{\spadesuit} & \hlw{\heartsuit} & \hlw{\diamondsuit} &
				\hlw{\clubsuit} & \hl[pink]{\varspadesuit} &
				\hl{\vardiamondsuit} &
				\hl{\hl[pink]{\varheartsuit}}
		} \\
		\mqty{ & & & \Downarrow & & & } \\
		\mqty{
			\hlw{\spadesuit} & \hlw{\heartsuit} & \hlw{\diamondsuit} &
				\hlw{\clubsuit} & \hlw{\varspadesuit} &
				\hl[pink]{\varheartsuit} & \hl[pink]{\vardiamondsuit}
		}
	\end{gather*}
	一共进行了 6 次两两交换,因此 $\sgn\Perm{\sigma}=1$。
	
\subsection{置换的复合}
	再定义一个置换
	\begin{equation}
		\Perm{\tau}=\mqty[
			1 & 2 & 3 & 4 & 5 & 6 & 7 \\
			5 & 1 & 7 & 3 & 6 & 4 & 2
		] \fullstop
	\end{equation}
	注意这里用了方括号,因此它是一个\emphB{序号定义}。
	方便起见,以后的序号我们都只用不带圈的普通数字表示。
	考虑之前定义的置换
	\begin{equation}
		\Perm{\sigma}=\mqty[
			1 & 2 & 3 & 4 & 5 & 6 & 7 \\
			7 & 4 & 5 & 1 & 6 & 2 & 3
		] \comma
	\end{equation}
	则 $\Perm{\tau}$ 与 $\Perm{\sigma}$ 的复合
	\begin{equation}
		\Perm{\tau}\comp\Perm{\sigma}=
		\qty(\begin{array}{@{}ccccccc@{}}
			\dicei & \diceii & \diceiii & \diceiv & \dicev &
				\dicevi & \circledtwodots \\
			\circledtwodots & \diceiv & \dicev & \dicei & \dicevi &
				\diceii & \diceiii \\
			\hdashline
			\dicevi & \circledtwodots & \diceiii & \dicev & \diceii &
				\dicei & \diceiv
		\end{array})
		\quad\mqty{\\ \leftarrow\Perm{\sigma} \\ \leftarrow\Perm{\tau}}
	\end{equation}
	与函数、线性变换等的复合类似,这里也用小圆圈“$\comp$”表示置换的复合。
	
	假设经过置换 $\Perm{\sigma}$、$\Perm{\tau}$ 作用后得到的序列,
	分别需要 $p$ 次和 $q$ 次两两交换才能复原为原来的序列。
	那么很显然,经过复合置换 $\Perm{\tau}\comp\Perm{\sigma}$ 作用后的序列,
	经过 $q+p$ 次两两交换也一定可以复原。因此,复合置换的符号
	\begin{equation}
		\sgn\qty(\Perm{\tau}\comp\Perm{\sigma})
		=(-1)^{q+p}=(-1)^q \cdotp (-1)^p
		=\sgn\Perm{\tau}\cdotp\sgn\Perm{\sigma} \fullstop
	\end{equation}
	
\subsection{逆置换}
	逆置换 $\Perm{\sigma}^{-1}$ 的定义为
	\begin{equation}
		\Perm{\sigma}^{-1}\comp\Perm{\sigma} = \Id \comma
	\end{equation}
	其中的“$\Id$”是\emphA{恒等映照}。
	
	仍然使用式~\eqref{eq:置换元素表示_符号}:
	\begin{equation}
		\Perm{\sigma}=\mqty*(
			\spadesuit & \heartsuit & \diamondsuit & \clubsuit &
				\varspadesuit & \varheartsuit & \vardiamondsuit \\
			\vardiamondsuit & \clubsuit & \varspadesuit & \spadesuit &
				\varheartsuit & \heartsuit & \diamondsuit
		) \comma
	\end{equation}
	那么自然有
	\begin{equation}
		\Perm{\sigma}^{-1}=\mqty*(
			\vardiamondsuit & \clubsuit & \varspadesuit & \spadesuit &
				\varheartsuit & \heartsuit & \diamondsuit \\
			\spadesuit & \heartsuit & \diamondsuit & \clubsuit &
				\varspadesuit & \varheartsuit & \vardiamondsuit
		) \fullstop
	\end{equation}
	显然,我们有 $\Perm{\sigma}^{-1}\comp\Perm{\sigma} = \Id$。
	
	回忆一下逆矩阵的定义。矩阵 $\Mat{A}$ 的逆 $\Mat{A}^{-1}$ 既要满足
	$\Mat{A}^{-1}\Mat{A}=\Mat{I}$,又要满足
	$\Mat{A}\Mat{A}^{-1}=\Mat{I}$。对于置换也是如此,
	因此我们需要检查 $\Perm{\sigma}\comp\Perm{\sigma}^{-1}$:\footnote{%
		该式中的数字角标用来澄清原始序号。}
	\begin{equation}
		\Perm{\sigma}\comp\Perm{\sigma}^{-1}=
		\qty(\begin{array}{@{}ccccccc@{}}
			\vardiamondsuit & \clubsuit & \varspadesuit & \spadesuit &
				\varheartsuit & \heartsuit & \diamondsuit \\
			\spadesuit_1 & \heartsuit_2 & \diamondsuit_3 & \clubsuit_4 &
				\varspadesuit_5 & \varheartsuit_6 & \vardiamondsuit_7 \\
			\hdashline
			\vardiamondsuit_7 & \clubsuit_4 & \varspadesuit_5 &
				\spadesuit_1 & \varheartsuit_6 &
				\heartsuit_2 & \diamondsuit_3
		\end{array})
		\quad\mqty{
			\\ \leftarrow\Perm{\sigma}^{-1} \\
			\leftarrow\Perm{\sigma}\phantom{^{-1}}
		}
	\end{equation}
	可见的确有 $\Perm{\sigma}\comp\Perm{\sigma}^{-1}=\Id$。
	
	另外,由于恒等映照 $\Id$ 作用后序列不发生变化,
	复原所需的交换次数为 0,因此
	\begin{equation}
		\sgn\Id=(-1)^0=1 \fullstop
	\end{equation}
	而根据定义,
	\begin{equation}
		\Id=\Perm{\sigma}^{-1}\comp\Perm{\sigma} \comma
	\end{equation}
	故有
	\begin{equation}
		\sgn\Perm{\sigma} \cdotp \sgn\Perm{\sigma}^{-1} = 1 \fullstop
	\end{equation}
	由此,可以推知
	\begin{equation}
		\sgn\Perm{\sigma}=\sgn\Perm{\sigma}^{-1} \comma
	\end{equation}
	即置换与它的逆具有\emphB{相同}的符号。
	
\section{置换(二)}
	本节将介绍置换运算的基本性质。
	
\subsection{置换的穷尽}
	先要做一点铺垫。设有序数组
	\begin{equation*}
		\qty{i_1,\,i_2,\,\cdots,\,i_r}
	\end{equation*}
	经置换 $\Perm{\sigma}$ 作用后成为
	\begin{equation*}
		\qty{\Perm{\sigma}(i_1),\,\Perm{\sigma}(i_2),\,
			\cdots,\,\Perm{\sigma}(i_r)} \comma
	\end{equation*}
	则根据之前的元素定义(圆括号),可以把 $\Perm{\sigma}$ 记为
	\begin{equation}
		\Perm{\sigma}=\mqty*(
			i_1 & i_2 & \cdots & i_r \\
			\Perm{\sigma}(i_1) & \Perm{\sigma}(i_2) &
				\cdots & \Perm{\sigma}(i_r)
		)\fullstop
	\end{equation}
	每次置换都将得到一个有序数组。把它们组合到一起,就可以得到集合
	\begin{equation}
		\set[\bigg]
			{\qty(\Perm{\sigma}(i_1),\,\Perm{\sigma}(i_2),\,
				\cdots,\,\Perm{\sigma}(i_r))}
			{\forall\,\Perm{\sigma}\in\Permutations{r}} \fullstop
	\end{equation}
	其中的 $\Permutations{r}$ 表示 $r$ 阶置换的全体。
	根据排列组合原理,$r$ 阶置换的总数等于 $r$ 个元素的\emphB{全排列数}。
	即该集合共有 $r!$ 个元素。
	
	下面我们要证明
	\begin{mySubEq}
		\begin{align}
			&\mathrel{\phantom{=}}\set[\bigg]
				{\qty(\Perm{\sigma}(i_1),\,\Perm{\sigma}(i_2),\,
					\cdots,\,\Perm{\sigma}(i_r))}
				{\forall\,\Perm{\sigma}\in\Permutations{r}} \notag \\
			%
			&=\set[\bigg]
				{\qty(
					\Perm{\tau}\comp\Perm{\sigma}(i_1),\,
					\Perm{\tau}\comp\Perm{\sigma}(i_2),\,\cdots,\,
					\Perm{\tau}\comp\Perm{\sigma}(i_r) )}
				{\forall\,\Perm{\sigma},\,\Perm{\tau}\in\Permutations{r}}
				\label{eq:置换的穷尽_复合1} \\
			&=\set[\bigg]
				{\qty(
					\Perm{\sigma}\comp\Perm{\tau}(i_1),\,
					\Perm{\sigma}\comp\Perm{\tau}(i_2),\,\cdots,\,
					\Perm{\sigma}\comp\Perm{\tau}(i_r) )}
				{\forall\,\Perm{\sigma},\,\Perm{\tau}\in\Permutations{r}}
				\label{eq:置换的穷尽_复合2} \\
			&=\set[\bigg]
				{\qty(
					\Perm{\sigma}^{-1}(i_1),\,
					\Perm{\sigma}^{-1}(i_2),\,\cdots,\,
					\Perm{\sigma}^{-1}(i_r))}
				{\forall\,\Perm{\sigma}\in\Permutations{r}} 
				\label{eq:置换的穷尽_逆} \fullstop
		\end{align}
	\end{mySubEq}
	\colorbox{pink}{这说明置换构成了置换群。?}
	
	\begin{myProof}
		证明的思路是说明集合互相包含。
		
		对于式~\eqref{eq:置换的穷尽_复合1},
		右边的 $\Perm{\tau}\comp\Perm{\sigma}$ 也是一个 $r$ 阶置换,
		自然符合左边集合的定义,因此 $\text{右边}\subset\text{左边}$。
		由于这一步是相当显然的,以下的几个证明我们将略去该步。
		另一方面,左边的 $\Perm{\sigma}$ 可以表示成
		\begin{equation}
			\Perm{\sigma}
			=\Id\comp\Perm{\sigma}
			=\qty(\Perm{\tau}\comp\Perm{\tau}^{-1}) \comp\Perm{\sigma}
			=\Perm{\tau}\comp \qty(\Perm{\tau}^{-1}\comp\Perm{\sigma})
			\comma
		\end{equation}
		这就是右边集合的定义,因此 $\text{左边}\subset\text{右边}$。
		故可证得等式成立。
		
		对于式~\eqref{eq:置换的穷尽_复合2},我们有
		\begin{equation}
			\Perm{\sigma}
			=\Perm{\sigma}\comp\Id
			=\Perm{\sigma}\comp \qty(\Perm{\tau}^{-1}\comp\Perm{\tau})
			=\qty(\Perm{\sigma}\comp\Perm{\tau}^{-1}) \comp\Perm{\tau}
			\comma
		\end{equation}
		它符合了右边集合的定义,因此 $\text{左边}\subset\text{右边}$。
		于是等式成立。
		
		对于式~\eqref{eq:置换的穷尽_逆},我们有
		\begin{equation}
			\Perm{\sigma}=\qty(\Perm{\sigma}^{-1})^{-1} \comma
		\end{equation}
		它符合了右边集合的定义,因此 $\text{左边}\subset\text{右边}$。
		于是等式成立。
	\end{myProof}
	
\subsection{数组元素的乘积} \label{subsec:数组元素的乘积}
	设有序数组 $\qty{i_1,\,i_2,\,\cdots,\,i_r}$、
	$\qty{j_1,\,j_2,\,\cdots,\,j_r}$ 和 $\qty{k_1,\,k_2,\,\cdots,\,k_r}$
	经 $r$ 阶置换 $\Perm{\sigma}$ 作用后分别成为
	$\qty{\Perm{\sigma}(i_1),\,\Perm{\sigma}(i_2),\,
		\cdots,\,\Perm{\sigma}(i_r)}$、
	$\qty{\Perm{\sigma}(j_1),\,\Perm{\sigma}(j_2),\,
		\cdots,\,\Perm{\sigma}(j_r)}$ 和
	$\qty{\Perm{\sigma}(k_1),\,\Perm{\sigma}(k_2),\,
		\cdots,\,\Perm{\sigma}(k_r)}$,也就是说
	\begin{equation}
		\Perm{\sigma}=\mqty*(
			i_1 & i_2 & \cdots & i_r \\
			\Perm{\sigma}(i_1) & \Perm{\sigma}(i_2) &
				\cdots & \Perm{\sigma}(i_r) )
		=\mqty*(
			j_1 & j_2 & \cdots & j_r \\
			\Perm{\sigma}(j_1) & \Perm{\sigma}(j_2) &
				\cdots & \Perm{\sigma}(j_r) )
		=\mqty*(
			k_1 & k_2 & \cdots & k_r \\
			\Perm{\sigma}(k_1) & \Perm{\sigma}(k_2) &
				\cdots & \Perm{\sigma}(k_r) ) \fullstop
	\end{equation}
	我们有如下结论:
	\begin{equation}
		\forall\,\Perm{\sigma}\in\Permutations{r}\, ,\quad
		A_{i_1 j_1 k_1} A_{i_2 j_2 k_2} \cdots
			A_{i_r j_r k_r}
		=A_{\Perm{\sigma}(i_1)\,\Perm{\sigma}(j_1)\,\Perm{\sigma}(k_1)}
			A_{\Perm{\sigma}(i_2)\,\Perm{\sigma}(j_2)\,\Perm{\sigma}(k_2)}
			\cdots
			A_{\Perm{\sigma}(i_r)\,\Perm{\sigma}(j_r)\,\Perm{\sigma}(k_r)}
			\comma
	\end{equation}
	式中的 $A_{ijk}$ 表示三维数组 $\Mat{A}$ 的一个元素,其指标为 $ijk$。
	
	下面通过一个例子来说明这一条性质。还是用式~\eqref{eq:置换元素表示_数字} 和
	\eqref{eq:置换元素表示_符号} 所定义的置换 $\Perm{\sigma}$:
	\begin{equation}
		\Perm{\sigma}=\mqty*(
			4 & 9 & 2 & 7 & 5 & 8 & 3 \\
			3 & 7 & 5 & 4 & 8 & 9 & 2 )
		=\mqty*(
			\spadesuit & \heartsuit & \diamondsuit & \clubsuit &
				\varspadesuit & \varheartsuit & \vardiamondsuit \\
			\vardiamondsuit & \clubsuit & \varspadesuit & \spadesuit &
				\varheartsuit & \heartsuit & \diamondsuit ) \fullstop
				\label{eq:置换元素表示_数组元素的乘积举例}
	\end{equation}
	随意写出一个数组元素乘积:
	\begin{equation}
		A_{379}A_{264}A_{157}A_{483}A_{698}
		A_{\diamondsuit\clubsuit\heartsuit}
		A_{\vardiamondsuit\varspadesuit\varheartsuit} \fullstop
		\label{eq:数组元素乘积举例}
	\end{equation}
	三组下标分别为
	\begin{equation}
		\left\{\begin{lgathered}
			3,\,2,\,1,\,4,\,6,\,\diamondsuit,\,\vardiamondsuit;\\
			7,\,6,\,5,\,8,\,9,\,\clubsuit,\,\varspadesuit;\\
			9,\,4,\,7,\,3,\,8,\,\heartsuit,\,\varheartsuit.\\
		\end{lgathered}
		\right.
	\end{equation}
	考虑 $\Perm{\sigma}$ 的\emphB{序号定义}式~\eqref{eq:置换序号定义}:
	\begin{equation}
		\Perm{\sigma}=\mqty[
			1 & 2 & 3 & 4 & 5 & 6 & 7 \\
			7 & 4 & 5 & 1 & 6 & 2 & 3
		] \fullstop
	\end{equation}
	所谓序号只是位置的抽象表示,而不代表任何真实的元素。
	请记住:置换始终是\emphB{位置}的变换,而非\emphB{元素}的变换,
	不要被式~\eqref{eq:置换元素表示_数组元素的乘积举例} 给迷惑了。
	把 $\Perm{\sigma}$ 作用在这三组下标上,可得
	\begin{equation}
		\left\{\begin{lgathered}
			\vardiamondsuit,\,4,\,6,\,3,\,\diamondsuit,\,2,\,1;\\
			\varspadesuit,\,8,\,9,\,7,\,\clubsuit,\,6,\,5;\\
			\varheartsuit,\,3,\,8,\,9,\,\heartsuit,\,4,\,7.\\
		\end{lgathered}
		\right.
	\end{equation}
	于是之前的数组元素乘积就变成了
	\begin{equation}
		A_{\vardiamondsuit\varspadesuit\varheartsuit}
		A_{483}A_{698}A_{379}
		A_{\diamondsuit\clubsuit\heartsuit}
		A_{264}A_{157} \fullstop
	\end{equation}
	比对一下各元素,可见与式~\eqref{eq:数组元素乘积举例} 的确是完全一样的。
	
\subsection{哑标的穷尽} \label{subsec:哑标的穷尽}
	考虑如下集合:
	\begin{equation}
		\set[\bigg]
		{\qty(i_1,\,i_2,\,\cdots,\,i_r)}
		{\qty{i_1,\,i_2,\,\cdots,\,i_r}
			\text{\ 可取\ } 1,\,2,\,\cdots,\,m}
		\fullstop
	\end{equation}
	每个 $i_k$ 都有 $m$ 种取法,而 $i_k$ 又有 $r$ 个,
	因此该集合一共有 $m^r$ 元素。我们有
	\begin{mySubEq}
		\begin{align}
			\forall\,\Perm{\sigma}\in\Permutations{r}\, ,
			&\mathrel{\phantom{=}}\set[\bigg]
			{\qty(i_1,\,i_2,\,\cdots,\,i_r)}
			{\qty{i_1,\,i_2,\,\cdots,\,i_r}
				\text{\ 可取\ } 1,\,2,\,\cdots,\,m} \notag \\
			&=\set[\bigg]
			{\qty(\Perm{\sigma}(i_1),\,\Perm{\sigma}(i_2),\,
				\cdots,\,\Perm{\sigma}(i_r) )}
			{\qty{i_1,\,i_2,\,\cdots,\,i_r}
				\text{\ 可取\ } 1,\,2,\,\cdots,\,m}
			\label{eq:哑标的穷尽_置换} \\
			&=\set[\bigg]
			{\qty(\Perm{\sigma}^{-1}(i_1),\,\Perm{\sigma}^{-1}(i_2),\,
				\cdots,\,\Perm{\sigma}^{-1}(i_r) )}
			{\qty{i_1,\,i_2,\,\cdots,\,i_r}
				\text{\ 可取\ } 1,\,2,\,\cdots,\,m} \fullstop
			\label{eq:哑标的穷尽_逆置换}
		\end{align}
	\end{mySubEq}
	这里,$i_k$ 起的就是\emphB{哑标}的作用。
	
	\begin{myProof}
		无论怎样置换,$\Perm{\sigma}(i_k)$ 都是 $1,\,2,\,\cdots,\,m$ 中的数。
		因此,对于 $\forall\,\Perm{\sigma}\in\Permutations{r}$,
		\begin{equation}
			\qty(\Perm{\sigma}(i_1),\,\Perm{\sigma}(i_2),\,
				\cdots,\,\Perm{\sigma}(i_r) )
			\in\set[\bigg]
				{\qty(i_1,\,i_2,\,\cdots,\,i_r)}
				{\qty{i_1,\,i_2,\,\cdots,\,i_r}
					\text{\ 可取\ } 1,\,2,\,\cdots,\,m} \comma
		\end{equation}
		即
		\begin{align}
			&\mathrel{\phantom{\subset}}\set[\bigg]
			{\qty(\Perm{\sigma}(i_1),\,\Perm{\sigma}(i_2),\,
				\cdots,\,\Perm{\sigma}(i_r) )}
			{\qty{i_1,\,i_2,\,\cdots,\,i_r}
				\text{\ 可取\ } 1,\,2,\,\cdots,\,m} \notag \\
			&\subset\set[\bigg]
			{\qty(i_1,\,i_2,\,\cdots,\,i_r)}
			{\qty{i_1,\,i_2,\,\cdots,\,i_r}
				\text{\ 可取\ } 1,\,2,\,\cdots,\,m} \fullstop
		\end{align}
		另一方面,由于 $\Id=\Perm{\sigma}^{-1}\comp\Perm{\sigma}$,即
		\begin{equation}
			\qty(i_1,\,i_2,\,\cdots,\,i_r)
			=\qty(\Perm{\sigma}^{-1}\comp\Perm{\sigma}(i_1),\,
				\Perm{\sigma}^{-1}\comp\Perm{\sigma}(i_2),\,
				\cdots,\,\Perm{\sigma}^{-1}\comp\Perm{\sigma}(i_r) ) \comma
		\end{equation}
		而进行一次逆置换仍然使得元素不离开原有的范围,也就是说
		\begin{equation}
			\qty(i_1,\,i_2,\,\cdots,\,i_r)
			\in\set[\bigg]
				{\qty(\Perm{\sigma}(i_1),\,\Perm{\sigma}(i_2),\,
					\cdots,\,\Perm{\sigma}(i_r) )}
				{\qty{i_1,\,i_2,\,\cdots,\,i_r}
					\text{\ 可取\ } 1,\,2,\,\cdots,\,m} \comma
		\end{equation}
		即
		\begin{align}
			&\mathrel{\phantom{\subset}}\set[\bigg]
			{\qty(i_1,\,i_2,\,\cdots,\,i_r)}
			{\qty{i_1,\,i_2,\,\cdots,\,i_r}
				\text{\ 可取\ } 1,\,2,\,\cdots,\,m} \notag \\
			&\subset\set[\bigg]
			{\qty(\Perm{\sigma}(i_1),\,\Perm{\sigma}(i_2),\,
				\cdots,\,\Perm{\sigma}(i_r) )}
			{\qty{i_1,\,i_2,\,\cdots,\,i_r}
				\text{\ 可取\ } 1,\,2,\,\cdots,\,m} \fullstop
		\end{align}
		两个集合互相包含,也就证得了式~\eqref{eq:哑标的穷尽_置换}。
		
		用相同的方法也可证得关于逆置换的 \eqref{eq:哑标的穷尽_逆置换}~式,
		此处从略。
	\end{myProof}
	
\section{置换(三)}
	本节将给出
	
\section{置换(四)}
	本节将回归张量运算的主线,引入\emphA{置换算子}。
	
\subsection{置换算子;对称张量与反对称张量}
	对于任意的置换 $\Perm{\sigma}\in\Permutations{r}$,定义\emphA{置换算子}
	\begin{equation}
		\mdef{\opPerm}
			{\Tensors{r}\ni\T{\Phi}}
			{\opPerm(\T{\Phi})\in\Tensors{r}} \comma
	\end{equation}
	式中
	\begin{equation}
		\opPerm(\T{\Phi})\qty(\V{u}_1,\,\V{u}_2,\,\cdots,\,\V{u}_r)
		\defeq\T{\Phi}\qty(\V{u}_{\Perm{\sigma}(1)},\,
			\V{u}_{\Perm{\sigma}(2)},\,\cdots,\,
			\V{u}_{\Perm{\sigma}(r)})
		\in\realR \fullstop
	\end{equation}
	这里的“$\cdots\in\realR$”是由于张量的定义:\emphB{多重线性函数}。
	
	如果我们的置换
	\begin{equation}
		\Perm{\sigma}=\mqty*(
			i_1 & i_2 & \cdots & i_r \\
			\Perm{\sigma}(i_1) & \Perm{\sigma}(i_1) & \cdots
				& \Perm{\sigma}(i_1)
		) \comma
	\end{equation}
	那么对应的置换算子将满足
	\begin{equation}
		\opPerm(\T{\Phi})\qty(\V{u}_{i_1},\,\V{u}_{i_2},\,
			\cdots,\,\V{u}_{i_r})
		\defeq\T{\Phi}\qty(\V{u}_{\Perm{\sigma}(i_1)},\,
			\V{u}_{\Perm{\sigma}(i_2)},\,\cdots,\,
			\V{u}_{\Perm{\sigma}(i_r)}) \fullstop
	\end{equation}
	
	\blankline
	
	有了置换算子,我们就可以来定义\emphA{对称张量}和\emphA{反对称张量}。
	对称张量的全体记为 $\Sym$,反对称张量的全体记为 $\Skw$。
	如果以 $\Rm$ 为底空间,
	又分别可以记为 $\SymTensors{r}$ 和 $\SkwTensors{r}$。
	
	对于任意的 $\T{\Phi}\in\Tensors{r}$,如果
	\begin{equation}
		\opPerm(\T{\Phi})=\T{\Phi} \comma
	\end{equation}
	则称 $\T{\Phi}$ 为\emphB{对称张量},
	即 $\T{\Phi}\in\Sym \text{\ 或\ } \SymTensors{r}$;如果
	\begin{equation}
		\opPerm(\T{\Phi})=\sgn\Perm{\sigma}\cdotp\T{\Phi} \comma
	\end{equation}
	则称 $\T{\Phi}$ 为\emphB{反对称张量},
	即 $\T{\Phi}\in\Skw \text{\ 或\ } \SkwTensors{r}$。
	
\subsection{置换算子的表示}
	根据上文给出的定义,我们有
	\begin{equation}
		\opPerm(\T{\Phi})\qty(\V{u}_{i_1},\,\cdots,\,\V{u}_{i_r})
		\defeq\T{\Phi}\qty(\V{u}_{\Perm{\sigma}(i_1)},\,\cdots,\,
			\V{u}_{\Perm{\sigma}(i_r)}) \fullstop
	\end{equation}
	首先回忆一下 \ref{subsec:张量的表示与简单张量}~小节中张量的表示:
	选一组基(协变、逆变均可),然后把张量用这组基表示。于是
	\begin{align}
		&\mathrel{\phantom{=}}\opPerm(\T{\Phi})
			\qty(\V{u}_{i_1},\,\cdots,\,\V{u}_{i_r})
		=\T{\Phi}\qty(\V{u}_{\Perm{\sigma}(i_1)},\,\cdots,\,
			\V{u}_{\Perm{\sigma}(i_r)}) \notag
		\intertext{把向量用协变基表示:}
		&=\T{\Phi}\qty(u^{i_1}_{\Perm{\sigma}(i_1)}\,\V{g}_{i_1},\,
			\cdots,\,u^{i_r}_{\Perm{\sigma}(i_r)}\,\V{g}_{i_r}) \notag
		\intertext{根据张量的线性性,提出系数:}
		&=\T{\Phi}\qty(\V{g}_{i_1},\,\cdots,\,\V{g}_{i_r}) \cdotp
			\qty(u^{i_1}_{\Perm{\sigma}(i_1)} \cdots
				u^{i_r}_{\Perm{\sigma}(i_r)}) \notag
		\intertext{前半部分可以用张量分量表示;
			而后半部分是一组逆变分量,可以写成内积的形式}
		&=\tensor{\Phi}{_{i_1 \cdots i_p}}
			\qty[\ipb{\V{u}_{\Perm{\sigma}(i_1)}}{\V{g}^{i_1}}\cdots
				\ipb{\V{u}_{\Perm{\sigma}(i_r)}}{\V{g}^{i_r}}]
		\addtocounter{equation}{1}
		\tag{\theequation*}
		\label{eq:置换算子的表示_中间步骤}
		\intertext{注意到方括号中的其实是简单张量的定义,这就有}
		&=\tensor{\Phi}{_{i_1 \cdots i_p}}
			\V{g}^{i_1}\tp\cdots\tp\V{g}^{i_r}
			\qty(\V{u}_{\Perm{\sigma}(i_1)},\,\cdots,\,
				\V{u}_{\Perm{\sigma}(i_r)}) \fullstop
		\addtocounter{equation}{-1}
	\end{align}
	最后一步仍然没能回到 $\qty(\V{u}_{i_1},\,\cdots,\,\V{u}_{i_r})$,
	因此以上推导只是简单地展开了 $\T{\Phi}$,并没有获得实质性的结果。
	
	然而,只要稍作改动,情况就会大不相同。
	考虑一下 \ref{subsec:数组元素的乘积}~小节中置换运算%
	有关\emphB{数组元素乘积}的性质:
	\begin{equation}
		\forall\,\Perm{\tau}\in\Permutations{r},\quad
		A_{i_1 j_1} \cdots A_{i_r j_r}
		=A_{\Perm{\tau}(i_1) \Perm{\tau}(j_1)} \cdots
			A_{\Perm{\tau}(i_r) \Perm{\tau}(j_r)} \comma
		\label{eq:置换算子的表示推导_置换性质}
	\end{equation}
	式中
	\begin{equation}
		\Perm{\tau}
		=\mqty*(
			i_1 & \cdots & i_r \\
			\Perm{\tau}(i_1) & \cdots & \Perm{\tau}(i_r) )
		=\mqty*(
			j_1 & \cdots & j_r \\
			\Perm{\tau}(j_1) & \cdots & \Perm{\tau}(j_r) ) \fullstop
	\end{equation}
	由此可以看出,式~\eqref{eq:置换算子的表示_中间步骤} 方括号中的部分
	其实是由 $\Perm{\sigma}(i_k)$ 和 $i_k$ 两套指标确定的一组数:
	\begin{equation}
		A_{\Perm{\sigma}(i_k)\,i_k}
		=\ipb{\V{u}_{\Perm{\sigma}(i_k)}}{\V{g}^{i_k}} \semicomma
	\end{equation}
	另一方面,显然有 $\Perm{\sigma}^{-1}\in\Permutations{r}$。于是
	\begin{align}
		&\mathrel{\phantom{=}}\opPerm(\T{\Phi})
			\qty(\V{u}_{i_1},\,\cdots,\,\V{u}_{i_r}) \notag \\
		&=\tensor{\Phi}{_{i_1 \cdots i_r}}
			\qty[\ipb{\V{u}_{\Perm{\sigma}(i_1)}}{\V{g}^{i_1}}\cdots
				\ipb{\V{u}_{\Perm{\sigma}(i_r)}}{\V{g}^{i_r}}] \notag
		\intertext{应用置换的性质 \eqref{eq:置换算子的表示推导_置换性质}~式:}
		&=\tensor{\Phi}{_{i_1 \cdots i_r}}
			\qty[
				\ipb{\V{u}_{\Perm{\sigma}^{-1}\comp\Perm{\sigma}(i_1)}}
					{\V{g}^{\Perm{\sigma}^{-1}(i_1)}} \cdots
				\ipb{\V{u}_{\Perm{\sigma}^{-1}\comp\Perm{\sigma}(i_r)}}
					{\V{g}^{\Perm{\sigma}^{-1}(i_r)}}
			] \notag \\
		&=\tensor{\Phi}{_{i_1 \cdots i_r}}
			\qty[
				\ipb{\V{u}_{i_1}}{\V{g}^{\Perm{\sigma}^{-1}(i_1)}} \cdots
				\ipb{\V{u}_{i_2}}{\V{g}^{\Perm{\sigma}^{-1}(i_r)}}
			] \notag
		\intertext{同样,用简单张量表示,可得}
		&=\tensor{\Phi}{_{i_1 \cdots i_r}}
			\V{g}^{\Perm{\sigma}^{-1}(i_1)}\tp\cdots
				\tp\V{g}^{\Perm{\sigma}^{-1}(i_r)}
			\qty(\V{u}_{i_1},\,\cdots,\,\V{u}_{i_r}) \fullstop
	\end{align}
	这样,我们就得到了置换算子的一种表示:
	\begin{align}
		\opPerm(\T{\Phi})
		&=\opPerm\qty(\tensor{\Phi}{_{i_1 \cdots i_r}}
			\V{g}^{i_1}\tp\cdots\tp\V{g}^{i_r}) \notag \\
		&=\tensor{\Phi}{_{i_1 \cdots i_r}}
			\V{g}^{\Perm{\sigma}^{-1}(i_1)}\tp\cdots
				\tp\V{g}^{\Perm{\sigma}^{-1}(i_r)} \fullstop
		\label{eq:置换算子的表示_逆置换在简单张量上}
	\end{align}
	
	在式~\eqref{eq:置换算子的表示_逆置换在简单张量上} 中,
	$i_1,\,\cdots,\,i_r$ 都是哑标,要被求和求掉。
	张量 $\T{\Phi}$ 的底空间是 $\Rm$,所以每个 $i_k$ 都有 $m$ 个取值。
	考虑一下 \ref{subsec:哑标的穷尽}~小节中置换运算%
	有关\emphB{哑标穷尽}的性质,有
	\begin{align}
		\forall\,\Perm{\sigma}\in\Permutations{r}\, ,
		&\mathrel{\phantom{=}}\set[\bigg]
		{\qty(i_1,\,i_2,\,\cdots,\,i_r)}
		{\qty{i_1,\,i_2,\,\cdots,\,i_r}
			\text{\ 可取\ } 1,\,2,\,\cdots,\,m} \notag \\
		&=\set[\bigg]
		{\qty(\Perm{\sigma}(i_1),\,\Perm{\sigma}(i_2),\,
			\cdots,\,\Perm{\sigma}(i_r) )}
		{\qty{i_1,\,i_2,\,\cdots,\,i_r}
			\text{\ 可取\ } 1,\,2,\,\cdots,\,m} \fullstop
	\end{align}
	因此,我们可以把式~\eqref{eq:置换算子的表示_逆置换在简单张量上} 中的指标
	$i_k$ 换成 $\Perm{\sigma}(i_k)$:
	\begin{align}
		\opPerm(\T{\Phi})
		&=\tensor{\Phi}{_{i_1 \cdots i_r}}
			\V{g}^{\Perm{\sigma}^{-1}(i_1)}\tp\cdots
				\tp\V{g}^{\Perm{\sigma}^{-1}(i_r)} \notag \\
		&=\tensor{\Phi}{_{\Perm{\sigma}(i_1) \cdots \Perm{\sigma}(i_r)}}
			\V{g}^{\Perm{\sigma}^{-1}\comp\Perm{\sigma}(i_1)}\tp\cdots
				\tp\V{g}^{\Perm{\sigma}^{-1}\comp
					\Perm{\sigma}(i_r)} \notag \\
		&=\tensor{\Phi}{_{\Perm{\sigma}(i_1) \cdots \Perm{\sigma}(i_r)}}
			\V{g}^{i_1}\tp\cdots\tp\V{g}^{i_r} \fullstop
	\end{align}
	这是置换算子的另一种表示。
	
	综上,要获得置换算子的表示,
	若是对\emphB{张量分量}进行操作,就直接使用对分量指标使用置换;
	若是对\emphB{简单张量}进行操作,则要对其指标使用逆置换:\footnote{%
		这里稍有改动,用了张量的逆变分量,不过实质都是一样的。%
		使用协变分量还是逆变分量,这个嘛,悉听尊便。}
	\begin{mySubEq}
		\begin{align}
			\opPerm(\T{\Phi})
			&=\opPerm\qty(\tensor{\Phi}{^{i_1 \cdots i_r}}
				\V{g}_{i_1}\tp\cdots\tp\V{g}_{i_r}) \notag \\
			&=\tensor{\Phi}{^{\Perm{\sigma}(i_1)\cdots\Perm{\sigma}(i_r)}}
				\V{g}_{i_1}\tp\cdots\tp\V{g}_{i_r} \\
			&=\tensor{\Phi}{^{i_1 \cdots i_r}}
				\V{g}_{\Perm{\sigma}^{-1}(i_1)}\tp\cdots
					\tp\V{g}_{\Perm{\sigma}^{-1}(i_r)} \fullstop
		\end{align}
	\end{mySubEq}

%	\backmatter
%	{
%		\small
%		\bibliography{Reference}
%	}
%	\printindex[pkg]
%	\printindex[cmd]
\end{document}