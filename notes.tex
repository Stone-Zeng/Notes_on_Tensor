%TODO 等的使用
%%TODO 内容问题
%%HACK 粗鄙技巧
%%CODE 代码改进
%%BUG  如题

\documentclass{book}

%\usepackage{syntonly}
%\syntaxonly

\newcommand*{\DEBUG}{}%
\ifdefined\DEBUG
  \relax
\else
  \newcommand*{\RELEASE}{}%
\fi

%TODO——宏查找
% 见 http://tex.stackexchange.com/a/8523
%\usepackage{filehook,currfile}
%\newwrite\finder
%\immediate\openout\finder=\jobname.find
%
%\def\searchmacro#1{
%  \AtBeginOfFiles{%
%    \ifdefined#1
%    \expandafter\def\csname \currfilename:found\endcsname{}%
%    \fi}
%  \AtEndOfFiles{%
%    \ifdefined#1
%    \unless\ifcsname \currfilename:found\endcsname
%    \immediate\write\finder{found in '\currfilename'}%
%    \fi\fi}}
%
%\searchmacro\T

%TODO——宏包
%% 方便定义命令(星号版本)
\usepackage{suffix}

%% 页面尺寸
\usepackage{geometry}
  \geometry{
    a4paper,
    left = 2.54 cm, right = 2.54 cm, top = 3.18 cm, bottom = 3.18 cm,
    headheight = 3 cm,
    marginparwidth = 1.8 cm
  }

\ifdefined\RELEASE
  %% 页面格式
  \usepackage{fancyhdr}
    % 清除所有页面格式
    \fancyhf{}
    % 页眉(见 `ctex' 下的重定义)
    \fancyhead[EL]{\nouppercase{\CJKfamily{楷体} \leftmark}}
    \fancyhead[OR]{\nouppercase{\CJKfamily{楷体} \rightmark}}
    % 页脚
    \fancyfoot[C]{\thepage}
    % 关闭横线显示
    \renewcommand{\headrulewidth}{0 pt}

  %% 空白页不显示页眉页脚
  \usepackage{emptypage}
\fi

%% 设置标题
%\usepackage{titlesec}

%% 图形
\ifdefined\DEBUG
  \usepackage[draft]{graphicx}
\else
  \usepackage{graphicx}
\fi

%% 常规字体选择
% 与 `graphicx' 冲突
%\PassOptionsToPackage{draft}{graphicx}
\usepackage[no-math]{fontspec}
  \setmainfont{XITS}%[SlantedFont = XITS Italic]
  \setsansfont{TeX Gyre Heros}%[SlantedFont = TeX Gyre Heros Italic]
  \setmonofont{TeX Gyre Cursor}[Ligatures = NoCommon]

%% AMS 数学支持
\usepackage{amsmath}
  % 允许多行公式中间分页
  \allowdisplaybreaks
  % 调整多行公式间距
  \setlength{\jot}{6 pt}
  % 调整公式前后间距
% \AtBeginDocument{
%   \abovedisplayshortskip = \abovedisplayskip
%   \belowdisplayshortskip = \belowdisplayskip
% }
\usepackage{amssymb}

%% 数学工具
% 必须在 `unicode-math' 之前
\usepackage{mathtools}

%% Pi 符号
\usepackage{pifont}
  % 正确 √
  \newcommand{\cmark}{\ding{51}}
  % 错误 ×
  \newcommand{\xmark}{\ding{55}}

%% 中文文字处理
%BUG:20170225 似乎必须在 `unicode-math' 之前
\usepackage[UTF8, heading = true, fontset = none]{ctex}
  \ctexset{
    section/format+ = {\normalfont\sffamily},
    subsection/format+ = {\normalfont\sffamily}
  }
  \ifdefined\DEBUG
    \pagestyle{plain}
  \else
    \ctexset{
      % 目录名称
      contentsname = {目 \quad 录},
      % 目录格式
      subsection/tocline = {\CJKfamily{楷体} \CTEXnumberline{#1}#2}
    }
    \pagestyle{fancy}
    % 重定义右页眉格式
    \renewcommand{\sectionmark}[1]{\markright{\thesection\quad #1}}
  \fi

%% XeTeX 下 CJK 文字处理
%\usepackage{xeCJK}
  \setCJKmainfont{FZShuSong-Z01}%
    [BoldFont = FZHei-B01, ItalicFont = FZKai-Z03]
  \setCJKsansfont{FZHei-B01}%
    [BoldFont = FZHei-B01, ItalicFont = FZHei-B01]
  \setCJKmonofont{FZFangSong-Z02}%
    [BoldFont = FZFangSong-Z02, ItalicFont = FZKai-Z03]
  \setCJKfamilyfont{宋体}{FZShuSong-Z01}
  \setCJKfamilyfont{楷体}{FZKai-Z03}
  \setCJKfamilyfont{黑体}{FZHei-B01}
  \setCJKfamilyfont{仿宋}{FZFangSong-Z02}

%% LuaLaTeX 下的句号处理
\catcode`\。 = \active
\newcommand{。}{.}

%% 调用 Unicode OpenType 数学字体
\usepackage{unicode-math}
  \setmathfont{xits-math.otf}[math-style = ISO, bold-style = ISO]
% 加粗使用 \symbf{}
% 直立希腊字母:\uppi 等

%% 脚注增强版
\usepackage[stable, perpage, bottom]{footmisc}
  % 需要调用 pifont 宏包
  % 衬线加圈阳文数字:\ding{172}~\ding{181} (1~10)
  % 无衬线加圈阳文数字:\ding{192}~\ding{201} (1~10)
  \renewcommand{\thefootnote}{\ding{\numexpr191+\value{footnote} } }
  %TODO:20160705 脚注不用上标
  %HACK:20160709 见 http://tex.stackexchange.com/q/19844
  \makeatletter
    \newlength{\fnBreite}
    \renewcommand{\@makefntext}[1]{%
      \settowidth{\fnBreite}{\footnotesize\@thefnmark.i}%
      \protect\footnotesize\upshape%
      \setlength{\@tempdima}{\columnwidth}%
      \addtolength{\@tempdima}{-\fnBreite}%
      \makebox[\fnBreite][l]{\@thefnmark\phantom{  }}%
      \parbox[t]{\@tempdima}%
      {\everypar{\hspace*{1em}}\hspace*{-1em}\upshape#1}%
    }
  \makeatother

%% 颜色
\usepackage[svgnames]{xcolor}

%% 表格
\usepackage{tabularx}

%% 表格虚线
% 需要 `tabularx' 宏包
\usepackage{arydshln}

%% 绘图
%\usepackage{tikz}
% \usepgflibrary{arrows.meta}
% \tikzset{>=Stealth}

%% 强调公式
\usepackage[ntheorem]{empheq}

%% 张量
\usepackage{tensor}

%% 物理、数学符号
\usepackage{physics}

%% 单位
%\usepackage{siunitx}

%% 定制列表环境
\usepackage{enumitem}
  % 定义带缩进的左对齐格式
  % 前一个参数 0 em 确定标签的位置
  % 后一个参数 1.45 em 确定标签与文字的距离
  %HACK:20160924 与字体、字号、标签内容有关
  \SetLabelAlign{leftalignwithindent}%
    {\hspace{0 em} \makebox[1.45 em][l]{#1}}

%% 定理类环境
\usepackage[thmmarks, amsmath]{ntheorem}
  \theoremstyle{nonumberplain} %带编号
  \theoremheaderfont{\bfseries}
  \theorembodyfont{\normalfont}
  \theoremsymbol{\mbox{$\Box$}}
    \newtheorem{myProof}{证明:}

%% 目录内容控制(draft 下禁用)
%% 必须在 `idxlayout' 之前
\ifdefined\RELEASE
  \usepackage[nottoc]{tocbibind}
\fi

%% 索引(draft 下禁用)
\ifdefined\RELEASE
  \usepackage{makeidx}
  \makeindex
  \newcommand{\idx}[1]{\index{#1}}
  %HACK:20170114 缺字,改用浪纹连接符代替
  \newcommand{\idxOmit}{\linkTilde}
\else
  \newcommand{\idx}[1]{}
  \newcommand{\idxOmit}{}
\fi

%% 索引格式(draft 下禁用)
\ifdefined\RELEASE
  \usepackage[
      hangindent = 6 em,
      subindent = 2 em,
      subsubindent = 4 em,
      % 加入目录
      totoc = true
    ]{idxlayout}
\fi

%% 标记引用
\ifdefined\DEBUG
  \usepackage[draft, color]{showkeys}
  \definecolor{refkey}{named}{pink}
  \definecolor{labelkey}{named}{lime}
\fi

%% 错误标注
\usepackage[draft]{fixme}
  \fxsetup{author = }
  \fxusetheme{color}
  \newcommand{\myPROBLEM}[2][2016-09-01]{\fxnote{#1\\#2}}
  \WithSuffix\newcommand\myPROBLEM*[2][2016-09-01]{\fxwarning{#1\\#2}}
  \renewcommand{\englishlistfixmename}{问题列表}
% \definecolor{fxnote}{named}{red}

%% 交叉引用,超链接等
\usepackage[hyperindex]{hyperref}
  \hypersetup{
    %% PDF 标题作者
    pdftitle = {谢锡麟《张量分析与微分几何基础》讲稿},
    pdfauthor = {曾祥东},
    %% PDF 书签
    bookmarksopen = true,
    bookmarksopenlevel = 1,
    bookmarksnumbered = true,
    %% 脚注
%     hyperfootnotes = false,
    %% 目录  只引用页码
    linktoc = page,
    %% 超链接边框
    pdfborder = 0 0 0,
    %% 超链接颜色
    colorlinks,
    linkcolor = {red!60!black},
    citecolor = {green!50!black},
    urlcolor = {blue!70!black}
  }
% 脚注引用
% 直接用 \ref
%\makeatletter
%\newcommand{\footnoteref}[1]{\protected@xdef%
% \@thefnmark{\ref{#1}}\@footnotemark}
%\makeatother

%TODO——环境
%% 定制列表(编号)
\newenvironment{myEnum}
  {\enumerate[
    align = leftalignwithindent,
    % 条目第一段缩进
    itemindent = 2 em,
    % 条目第二段缩进
    listparindent = 2 em,
    % 文字左边距
    leftmargin = 0 pt,
    topsep = 0 pt,
    itemsep = 0 pt,
    parsep = 0 pt
  ]}
  {\endenumerate}
%% 子公式
\newenvironment{mySubEq}
  {\subequations \renewcommand{\theequation}%
    {\theparentequation-\alph{equation}}}%
  {\endsubequations%
    \ignorespacesafterend}
%% 大括号公式
\newenvironment{braceEq}[1][align]
  {\mySubEq%
    \setkeys{EmphEqEnv}{#1}%
    \setkeys{EmphEqOpt}{left=\empheqlbrace}%
    \EmphEqMainEnv}%
  {\endEmphEqMainEnv \endmySubEq}
%% 大括号公式2(分情况讨论)
\newenvironment{braceEq*}[2][align]
  {\mySubEq%
    \setkeys{EmphEqEnv}{#1}%
    \setkeys{EmphEqOpt}{left={{\displaystyle #2}\empheqlbrace}}%
    \EmphEqMainEnv}%
  {\endEmphEqMainEnv \endmySubEq}

%TODO——命令
%% 浪纹连接号、代字符
% Unicode FF5E ~ FULLWIDTH TILDE
\newcommand{\linkTilde}{\symbol{65374}}
% Unicode 2053 ⁓ SWUNG DASH
% see \idxOmit
\newcommand{\swungDash}{\symbol{8275}}
%% 空行
%\newcommand{\blankline}{\mbox{}\par\mbox{}}
\newcommand{\blankline}{\mbox{}}
%% 强调
\newcommand{\emphA}[1]{{\bfseries #1}}
\newcommand{\emphB}[1]{{\itshape #1}}
%% 高亮(Highlight)
% 设置带框盒子边距
\setlength{\fboxsep}{0.5 pt}
\newcommand{\hl}[2][yellow!60!white]{\mathchoice%
  {\colorbox{#1}{$\displaystyle #2$}}%
  {\colorbox{#1}{$\textstyle #2$}}%
  {\colorbox{#1}{$\scriptstyle #2$}}%
  {\colorbox{#1}{$\scriptscriptstyle #2$}}}
\newcommand{\hlw}[1]{\mathchoice%
  {\colorbox{white}{$\displaystyle #1$}}%
  {\colorbox{white}{$\textstyle #1$}}%
  {\colorbox{white}{$\scriptstyle #1$}}%
  {\colorbox{white}{$\scriptscriptstyle #1$}}}
%\newcommand{\myPROBLEM}[2][\today]{\colorbox{red}{#1\quad#2}}

%TODO——数学命令
%--------数学字体--------%
%% 数集 黑板粗体
\let\SET=\symbb
%% 空间 花体
\let\SPACE=\symcal
%% 特殊空间 无衬线直立
\let\SPACEX=\symsfup
%% 域(邻域,定义域) 哥特体
\let\DOMAIN=\symfrak
%% 符号 倾斜粗体
\let\SYMBOL=\symbf
%% 一般算子 无衬线直立
\let\OPERATOR=\symsfup
%% 特殊算子 无衬线直立粗体
\let\OPERATORX=\symbfsfup

%--------数集--------%
\newcommand{\realR}{\SET{R}}
\newcommand{\comC}{\SET{C}}
\newcommand{\intN}{\SET{N}}
%% m 维实空间
\newcommand{\Rm}{\realR^m}
\WithSuffix\newcommand\Rm*{\realR^{m+1}}
%% 自然数集(正整数集)
\newcommand{\natN}{\intN^{*}}
%% Minkowski 空间
\newcommand{\minkM}{\SET{M}^4}

%--------空间--------%
%% 向量空间
\newcommand{\vecV}{\SPACE{V}}
%% 数域
\newcommand{\numF}{\SPACE{F}}
%% 以 m 维实空间为底空间的(r)阶张量全体
\newcommand{\Tensors}[2][\Rm]{\SPACE{T}^{#2}\qty(#1)}
%% (r)阶置换全体
\newcommand{\Permutations}[1]{\SPACE{P}_{#1}}
%% 以 m 维实空间为底空间的(r)阶对称张量全体
\newcommand{\SymTensors}[2][\Rm]{\SPACE{S}^{#2}\qty(#1)}
%% 以 m 维实空间为底空间的(r)阶反对称张量全体
\newcommand{\SkwTensors}[2][\Rm]{\Lambda^{#2}\qty(#1)}
%% 线性变换全体
\newcommand{\LinearT}[2]{\SPACE{L}\qty(#1,\,#2)}

%--------特殊空间--------%
%% 正交矩阵全体(Orthogonal Matrices)
\newcommand{\Orth}{\SPACEX{Orth}}
%% 对称张量全体(Symmetric Tensors)
\newcommand{\Sym}{\SPACEX{Sym}}
%% 反对称张量全体(Anti-symmetric Tensors)
\newcommand{\Skw}{\SPACEX{Skw}}

%--------区域--------%
%% 邻域(Neighborhood)
\newcommand{\domB}[2]{\DOMAIN{B}_{#1}\qty(#2)}
%% 定义域(Domain)
\newcommand{\domD}[1]{\DOMAIN{D}_{#1}}

%--------量(Quantity)--------%
%% 向量(Vector)
\newcommand{\V}[1]{\SYMBOL{#1}}
%% 张量(Tensor)
\renewcommand{\T}[1]{\SYMBOL{#1}}
%% 张量分量(Tensor Components)
%\let\tc=\tensor
\newcommand{\tc}[2]{\tensor{#1}{#2}}
%% 矩阵(Matrix)
\newcommand{\Mat}[1]{\symbfup{#1}}
%% 置换(Permutation)
\newcommand{\Perm}[1]{\SYMBOL{#1}}

%--------特殊量--------%
%% Kronecker Delta
\newcommand{\KroneckerDelta}[2]{\delta^{#1}_{#2}}
%% Kronecker Delta (全下标形式)
\WithSuffix\newcommand\KroneckerDelta*[1]{\delta_{#1}}
%% Levi-Civita 记号
\newcommand{\LeviCivita}[1]{\tc{\epsilon}{#1}}
%% Eddington 张量
\newcommand{\EdTensor}{\T{\epsilon}}
%% 第一类 Christoffel 符号
\newcommand{\ChrA}[3]{\Gamma_{#1#2,\,#3}}
%% 第二类 Christoffel 符号
\newcommand{\ChrB}[3]{\tc{\Gamma}{^{#3}_{#1#2}}}
%% 单位正交基下的 Christoffel 符号
\newcommand{\ChrU}[3]{\Gamma\midscript{\!\orthIdx{#1#2#3}}}
%% Gauss 曲率
\newcommand{\KG}{K_\text{G}}

%--------算子--------%
%% 置换算子(Permutation Operator)
\newcommand{\opPerm}[1][\Perm{\sigma}]{\OPERATOR{I}_{#1}}
%% 对称化算子(Symmetrizer)
\newcommand{\opSym}{\symcal{S}}
%% 反对称化算子(Antisymmetrizer)
\newcommand{\opSkw}{\symcal{A}}
%% Jacobi 矩阵(Differential Operator)
\DeclareMathOperator{\JacobiD}{\OPERATOR{D}\!}
%% 梯度算子(Gradient Operator)
\newcommand{\opGrad}{\T{\nabla}}
%% Laplace 算子(Laplace Operator)
\newcommand{\opLap}{\T{\nabla}^2}
%\newcommand{\opLap}{\symbfup{\Delta}}
%% 切空间(Tangent Space)
\newcommand{\Tspace}[2]{\OPERATOR{T}_{\!#2}{#1}}
%% 余切空间(Cotangent Space)

%--------特殊算子--------%
%% 恒等映照(Identity)
\newcommand{\Id}{\OPERATORX{Id}}

%--------数学符号--------%
%% 映射(X: x → X)(Math Map)
\newcommand{\mmap}[3]{{#1}:\,{#2}\mapsto{#3}}
%% 带描述集合
\DeclarePairedDelimiterX\set[2]{\{}{\}}{#1 \;\delimsize\vert\; #2}
%% 中标
% 见 http://tex.stackexchange.com/q/250961
\makeatletter
\newcommand{\mid@script}[2]{\vcenter{\hbox{$\m@th#1#2$}}}
\DeclareRobustCommand{\midscript}[1]{\mathchoice%
  {\mid@script\scriptstyle{#1}}%
  {\mid@script\scriptstyle{#1}}%
  {\mid@script\scriptscriptstyle{#1}}%
  {\mid@script\scriptscriptstyle{#1}}%
}
\makeatother
%% 单位正交基指标(Orthonormal Index)
\newcommand{\orthIdx}[1]{\langle{#1}\rangle}
%% 范数
% 已在 `physics' 中定义
\let\PHYSICSNORM=\norm
\renewcommand{\norm}[2][\Rm]{\abs{#2}_{#1}}
%% 线性分式,已有命令 \flatfrac
%\WithSuffix\newcommand\frac*[2]{\left.#1\middle/#2\right.}
%% 内积(括号形式)(Inner Product Braket)
\newcommand{\ipb}[3][\Rm]{\left\langle#2,\,#3\right\rangle_{#1}}
%% 张量积(Tensor Product)
\newcommand{\tp}{\otimes}
%% e-点积(e Dot Product)
\newcommand{\edp}[1][e]{\mathbin{\dbinom{#1}{\mathord{\cdot}}}}
%% 全点积(Full Dot Product)
\newcommand{\fdp}{\odot}
%% 点乘([Vector] Dot Product)
% 使用 \cdot 表示普通乘法,使用 \vdp 表示点乘
%% \dp 是 LaTeX 内部定义
\newcommand{\vdp}{\vysmblkcircle}
%% \cdot 作为 ordinary 符号
\newcommand{\cdotord}{\mathord{\cdot}}
%% 叉乘(Cross Product)
% 已在 `physics' 中定义
\let\PHYSICSCP=\cp
\renewcommand{\cp}{\vectimes}
%% 协变导数(Covariant Derivative)
\newcommand{\coD}[2]{\nabla_{\! #1}\,{#2}}
%% 单位正交基下的协变导数
\WithSuffix\newcommand\coD*[2]{%
  \nabla\midscript{\!\orthIdx{#1}}\,{#2}}
%% 逆变导数(Contravariant Derivative)
\newcommand{\ctrD}[2]{\nabla^{#1}\,{#2}}
%% C^p 微分同胚(Diffeomorphism)
\newcommand{\DiffMorp}[1][p]{\symcal{C}^{#1}}
%% 直至 p 阶偏导数连续的函数(Continuous Function)
\newcommand{\cf}[3][p]{\symcal{C}^{#1}\qty(#2;\,#3)}
%% 大 O 记号
\newcommand{\bO}[1]{\mathop{\mscrO\!}\qty(#1)}
%% 小 O 记号
\newcommand{\sO}[1]{\mathop{\mscro\!}\qty(#1)}
%% 小 O 记号(带上下标)
\WithSuffix\newcommand\sO*[2]{\mathop{
  \tc{\mscro}{#1} \!}\qty(#2)}
%% 向量形式小 O 记号
\newcommand{\sOv}[1]{\mathop{\mbfscro\!}\qty(#1)}
%% 向量形式小 O 记号(带上下标)
\WithSuffix\newcommand\sOv*[2]{\mathop{
  \tc{\mbfscro}{#1} \!}\qty(#2)}
%% 复合(Composite)
\newcommand{\comp}{\circ}
%% 定义等号(Definition Equal)
\newcommand{\defeq}{\triangleq}
%% 转置(Transpose)
\newcommand{\trans}{^{\mkern-1.5mu\symsfup{T}}}
%% 取逆并转置(Inverse and Transpose)
\newcommand{\invTrans}{^{-\mkern-1.5mu\symsfup{T}}}
%% 增量(Increment)
\newcommand{\incr}{\increment}

%--------函数名称--------%
%% 符号函数(Sign)
\DeclareMathOperator{\sgn}{sgn}
%% 线性张成(Linear Span)
% `\span' 冲突,见 http://tex.stackexchange.com/q/33264
\DeclareMathOperator{\vspan}{span}

%--------公式杂项--------%
%% 带圈数字(Circle digit)
\newcommand{\circNum}[1]{%
  \ifnum #1=0%
    \mathord{\symbol{9450}}%
  \else%
    \mathord{\symbol{\numexpr9311+#1}}%
  \fi}
%% align 空格
\newcommand{\alspace}{\mathrel{\phantom{=}}}
%% 标点、文字
\newcommand{\comma}{\text{,}}
\newcommand{\fullstop}{\text{.}}
\newcommand{\semicolon}{\text{;}}
\newcommand{\const}{\text{const.}}
\newcommand{\res}{\text{res.}}
%\renewcommand{\intertext}[1]{\shortintertext{\textit{#1}}}

%TODO——标题页
\ifdefined\RELEASE
  \title{现代张量分析讲义}
  \author{复旦大学 谢锡麟} % 此处为全角空格
  \date{\today}
\fi

\makeatletter
  \newcommand{\HUGE}{\@setfontsize\Huge{45}{54}}
\makeatother

\newcommand{\myTitle}{
\begin{titlepage}
  \begin{center}
    \vspace*{2 cm}
    {\HUGE\bfseries 现代张量分析讲义}
    
    \vspace{2 cm}
    {\LARGE 复旦大学 \quad 谢锡麟}
    
    \vspace{15 cm}
    {\LARGE\CJKfamily{楷体} \today}
  \end{center}
\end{titlepage}}

%TODO——选择编译
\ifdefined\DEBUG
  \includeonly{chapters/preliminaries}
  \renewcommand{\part}[1]{}
  %\renewcommand{\chapter}[1]{}
\fi

\begin{document}
\ifdefined\RELEASE
  \frontmatter
    \myTitle
    \tableofcontents

  \mainmatter
\fi

  \part{张量代数}
    % 预备知识
    \chapter{预备知识}

    % 张量的定义及表示
    \chapter{张量的定义及表示}
\section{对偶基,度量}
\subsection{对偶基} \label{subsec:对偶基}
设 $\qty{\V{g}_i}^m_{i=1}$ 是 $\Rm$
空间中的一组\emphA{基(向量)}\idx{基}\idx{基向量|see{基}},
即极大线性无关向量组。此时,$\Rm$ 中将\emphB{唯一}存在另一组基
$\qty{\V{g}^i}^m_{i=1}$,
二者满足\emphA{对偶关系}\idx{对偶!对偶关系}:
\begin{equation}
  \ipb{\V{g}_i}{\V{g}^j}=\KroneckerDelta{j}{i}
  =\left\{\begin{aligned}
    1,\quad i&=j \semicolon\\
    0,\quad i&\neq j \fullstop
  \end{aligned}\right.
  \label{eq:对偶关系}
\end{equation}
式中,$\ipb{\V{g}_i}{\V{g}^j}$ 的 $\KroneckerDelta{j}{i}$ 是 
\emphA{Kronecker δ 函数}\idx{Kronecker δ 函数}。

\begin{myProof}
根据内积的定义,
\begin{equation}
  \ipb{\V{g}_i}{\V{g}^j}=\qty(\V{g}^j)\trans\V{g}_i
  =\KroneckerDelta{j}{i} \comma
\end{equation}
其中的 $i$、$j$ 可取 $1,\,2,\,\cdots,\,m$。
写成矩阵形式,为\footnote{
  除非特殊说明,本文中的所有向量均取\emphB{列向量}。}
\begin{equation}
  \mqty[\qty(\V{g}^1)\trans \\ \vdots \\ \qty(\V{g}^m)\trans]
  \mqty[\V{g}_1,\,\cdots,\,\V{g}_m] = \Mat{I}_m \fullstop
\end{equation}
左边的第一个矩阵拆分成了 $m$ 行,每行是一个 $m$ 维行向量;
第二个矩阵则拆分成了 $m$ 列,每行是一个 $m$ 维列向量。
根据分块矩阵的乘法,所得结果对角元为 $\KroneckerDelta{i}{i}$,
非对角元则为 $\KroneckerDelta{i}{j}$(其中 $i\neq j$),即单位阵。

把第一个矩阵的转置挪到外面,可有
\begin{equation}
  \mqty[\V{g}^1,\,\cdots,\,\V{g}^m]\trans
  \mqty[\V{g}_1,\,\cdots,\,\V{g}_m] = \Mat{I}_m \fullstop
\end{equation}
$\qty{\V{g}_i}^m_{i=1}$ 作为基,必然满足
$\qty[\V{g}_1,\,\cdots,\,\V{g}_m]$ 非奇异。因此
\begin{equation}
  \mqty[\V{g}^1,\,\cdots,\,\V{g}^m]\trans
  =\mqty[\V{g}_1,\,\cdots,\,\V{g}_m]^{-1} \comma
\end{equation}
即
\begin{equation}
  \mqty[\V{g}^1,\,\cdots,\,\V{g}^m]
  =\mqty[\V{g}_1,\,\cdots,\,\V{g}_m]\invTrans \fullstop
  \label{eq:逆变基用协变基表示}
\end{equation}
逆矩阵(及其转置)是存在且唯一的,这就证明了对偶基的存在性和唯一性。
\end{myProof}

满足对偶关系的基称为\emphA{对偶基}
\idx{对偶!对偶基}\idx{基!对偶基}。
我们把指标写在\emphB{下面}的基 $\qty{\V{g}_i}^m_{i=1}$
称为\emphA{协变基}\idx{协变!协变基}\idx{基!协变基};
指标写在\emphB{上面}的基 $\qty{\V{g}^i}^m_{i=1}$
称为\emphA{逆变基}\idx{逆变!逆变基}\idx{基!逆变基}。
式~\eqref{eq:逆变基用协变基表示} 明确指出了逆变基与协变基的关系。

\blankline

下面我们引入\emphA{单位正交基}\idx{单位正交基}
(或\emphA{标准正交基}\idx{标准正交基|see{单位正交基}}),
它是指模长为一且两两正交的一组基。

如果矩阵 $\Mat{A}$ 满足
\begin{equation}
  \Mat{A}\trans\Mat{A}=\Mat{A}\Mat{A}\trans=\Mat{I} \comma
\end{equation}
即
\begin{equation}
  \Mat{A}\trans=\Mat{A}^{-1} \text{\ 或\ }
  \Mat{A}=\Mat{A}\invTrans \comma
\end{equation}
则称 $\Mat{A}$ 为\emphA{正交矩阵}\idx{正交矩阵}。
正交矩阵的全体记作 $\Orth$。
由线性代数中的结论可知,正交矩阵 $\Mat{A}$
的每一行和每一列均是单位正交基;
反之,若矩阵 $\Mat{A}$ 的每一行和每一列均是单位正交基,
则 $\Mat{A}$ 也是正交矩阵。

设 $\Rm$ 空间中的协变基 $\qty{\V{g}_i}^m_{i=1}$
是一组单位正交基,则由它构成的矩阵
\begin{equation}
  \mqty[\V{g}_1,\,\cdots,\,\V{g}_m]\in\Orth \comma
\end{equation}
因此
\begin{equation}
  \mqty[\V{g}_1,\,\cdots,\,\V{g}_m]
  =\mqty[\V{g}_1,\,\cdots,\,\V{g}_m]\invTrans \fullstop
\end{equation}
另一方面,根据对偶关系的矩阵表示 \eqref{eq:逆变基用协变基表示}~式,
我们有
\begin{equation}
  \mqty[\V{g}^1,\,\cdots,\,\V{g}^m]
  =\mqty[\V{g}_1,\,\cdots,\,\V{g}_m]\invTrans \comma
\end{equation}
于是
\begin{equation}
  \mqty[\V{g}_1,\,\cdots,\,\V{g}_m]
  =\mqty[\V{g}^1,\,\cdots,\,\V{g}^m] \comma
\end{equation}
即协变基与逆变基\emphB{完全相同}。
反之,如果协变基与逆变基相同,那么有
\begin{equation}
  \mqty[\V{g}_1,\,\cdots,\,\V{g}_m]\trans
  =\mqty[\V{g}^1,\,\cdots,\,\V{g}^m]\trans
  =\mqty[\V{g}_1,\,\cdots,\,\V{g}_m]^{-1} \comma
\end{equation}
因而
\begin{equation}
  \mqty[\V{g}_1,\,\cdots,\,\V{g}_m]\in\Orth \fullstop
\end{equation}

这样,我们便获得了一个重要结论:
当 $\qty{\V{g}_i}^m_{i=1}\subset\Rm$ 是单位正交基,
即 $\ipb{\V{g}_i}{\V{g}_j}=\KroneckerDelta*{ij}$ 时,
可有 $\V{g}^i=\V{g}_i$。反之亦然。

\subsection{度量} \label{subsec:度量}
下面引入\emphA{度量}\idx{度量}的概念。其定义为
\begin{braceEq}
  g_{ij} &\defeq \ipb{\V{g}_i}{\V{g}_j} \comma 
  \label{eq:度量的定义_协变} \\
  g^{ij} &\defeq \ipb{\V{g}^i}{\V{g}^j} \fullstop
  \label{eq:度量的定义_逆变}
\end{braceEq}
这两种度量满足
\begin{equation}
  g_{ik}\,g^{kj} = \KroneckerDelta{j}{i} \fullstop
  \label{eq:度量之积}
\end{equation}
也可以写成矩阵形式:
\begin{equation}
  \mqty[g_{ik}]\mqty[g^{kj}]
  =\mqty[\KroneckerDelta{j}{i}]=\Mat{I}_m \comma
  \label{eq:度量之积_矩阵形式}
\end{equation}
其中的 $\Mat{I}_m$ 是 $m$ 阶单位阵。该式的证明将在稍后给出。

由于内积具有交换律,因而度量的两个指标显然可以交换:
\begin{equation}
  g_{ij}=g_{ji}, \quad g^{ij}=g^{ji} \fullstop
\end{equation}

利用度量,可以获得\emphA{基向量转换关系}\idx{基向量!基向量转换关系}。
第 $i$ 个协变基向量 $\V{g}_i$ 既然是向量,
就必然可以用协变基或逆变基来表示\footnote{
  所谓用某组基来“表示”一个向量,就是把它朝各个基的方向做投影,然后再求和。}。
根据对偶关系式~\eqref{eq:对偶关系} 和度量的定义
式~\eqref{eq:度量的定义_协变}、\eqref{eq:度量的定义_逆变},可知
\begin{braceEq}
  \V{g}_i &=\qty(\V{g}_i,\,\V{g}_k)_{\Rm} \,\V{g}^k
  =g_{ik}\,\V{g}^k \comma \label{eq:基向量转换关系1} \\
  \V{g}_i &=\qty(\V{g}_i,\,\V{g}^k)_{\Rm} \,\V{g}_k
  =\KroneckerDelta{k}{i}\V{g}_k \label{eq:基向量转换关系2}
\end{braceEq}
以及
\begin{braceEq}
  \V{g}^i &=\qty(\V{g}^i,\,\V{g}_k)_{\Rm} \,\V{g}^k
  =\KroneckerDelta{i}{k}\,\V{g}^k \comma \label{eq:基向量转换关系3} \\
  \V{g}^i &=\qty(\V{g}^i,\,\V{g}^k)_{\Rm} \,\V{g}_k
  =g^{ik}\V{g}_k \fullstop \label{eq:基向量转换关系4}
\end{braceEq}
这四个式子中,式~\eqref{eq:基向量转换关系2} 和
\eqref{eq:基向量转换关系3} 是平凡的,
而式~\eqref{eq:基向量转换关系1} 和 \eqref{eq:基向量转换关系4}
则通过\emphB{度量}建立起了协变基与逆变基之间的关系。
这就称为\emphB{基向量转换关系},也可以叫做“指标升降游戏”。
\idx{指标升降游戏}

需要说明的是,根据\emphA{Einstein 求和约定}\idx{Einstein 求和约定},
重复指标(即\emphB{哑标}\idx{哑标},这里是 $k$)且一上一下时,
已经暗含了求和。后文除非特殊说明,也都是如此。

\blankline

现在我们来证明式~\eqref{eq:度量之积}:
\begin{equation}
  g_{ik}\,g^{kj} = \KroneckerDelta{j}{i} \fullstop
\end{equation}

\begin{myProof}
\begin{equation}
  g_{ik}\,g^{kj}
  =\ipb{\V{g}_i}{\V{g}_k} \, g^{kj}
  =\ipb{\V{g}_i}{g^{kj}\V{g}_k}
\end{equation}
根据式~\eqref{eq:基向量转换关系4},有
\begin{equation}
  g^{kj}\V{g}_k=g^{jk}\V{g}_k=\V{g}^j
\end{equation}
因此可得
\begin{equation}
  g_{ik}\,g^{kj}=\ipb{\V{g}_i}{\V{g}^j}
  =\KroneckerDelta{j}{i} \fullstop
\end{equation}

\end{myProof}

\subsection{向量的分量}
对于任意的向量 $\V{\xi} \in \Rm$,它可以用协变基表示:
\begin{mySubEq}
  \begin{align}
    \V{\xi}&=\ipb{\V{\xi}}{\V{g}^k}\,\V{g}_k
    =\xi^k \V{g}_k \comma
    \intertext{也可以用逆变基表示:}
    \V{\xi}&=\ipb{\V{\xi}}{\V{g}_k}\,\V{g}^k
    =\xi_k \V{g}^k \fullstop
  \end{align}
\end{mySubEq}
式中,$\xi^k$ 是 $\V{\xi}$ 与第 $k$ 个\emphB{逆变基}做内积的结果,
称为 $\V{\xi}$ 的第 $k$ 个\emphA{逆变分量}
\idx{逆变!向量的逆变分量};
而 $\xi_k$ 是 $\V{\xi}$ 与第 $k$ 个\emphB{协变基}做内积的结果,
称为 $\V{\xi}$ 的第 $k$ 个\emphA{协变分量}
\idx{协变!向量的协变分量}。

以后凡是指标在下的(下标),均称为\emphB{协变}某某;
指标在上的(上标),称为\emphB{逆变}某某。

\section{张量的表示}
\subsection{张量的表示与简单张量} \label{subsec:张量的表示与简单张量}
所谓\emphA{张量}\idx{张量},
即\emphA{多重线性函数}\idx{多重线性函数|see{张量}}。

首先用三阶张量举个例子。考虑任意的 $\T{\Phi}\in\Tensors{3}$,
其中的 $\Tensors{3}$ 表示以 $\Rm$ 为底空间的三阶张量全体。
所谓三阶(或三重)线性函数,指“吃掉”三个向量之后变成\emphB{实数},
并且“吃法”具有线性性。

一般地,$r$ 阶张量的定义如下:
\begin{equation}
  \mmap{\T{\Phi}}
    {\underbrace{\Rm\times\Rm\times\cdots\times\Rm}_
      {\text{$r$ 个 $\Rm$}}
      \ni\qty{\V{u}_1,\,\V{u}_2,\,\cdots,\,\V{u}_r}}
    {\T{\Phi}\qty(\V{u}_1,\,\V{u}_2,\,\cdots,\,\V{u}_r)}
      \in\realR \comma
\end{equation}
式中的 $\T{\Phi}$ 满足
\begin{align}
  \forall\,\alpha,\,\beta\in\realR,\,
  &\alspace\T{\Phi}\qty(\V{u}_1,\,\cdots,\,
    \alpha\tilde{\V{u}}_i+\beta\hat{\V{u}}_i,\,\cdots,\,\V{u}_r)
    \notag \\
  &=\alpha\,\T{\Phi}\qty(\V{u}_1,\,\cdots,\,
    \tilde{\V{u}}_i,\,\cdots,\,\V{u_r})
  +\beta\,\T{\Phi}\qty(\V{u}_1,\,\cdots,\,
    \hat{\V{u}}_i,\,\cdots,\,\V{u_r}) \comma
\end{align}
即所谓“\emphB{对第 $i$ 个变元的线性性}”。这里的 $i$ 可取
$1,\,2,\,\cdots,\,r$。

在张量空间 $\Tensors{r}$ 上,我们引入线性结构:
\begin{align}
  \forall\,\alpha,\,\beta\in\realR
  \text{\ 和\ } \T{\Phi},\,\T{\Psi}\in\Tensors{r},
  &\alspace\qty(\alpha\,\T{\Phi}+\beta\,\T{\Psi})
  \qty(\V{u}_1,\,\V{u}_2,\,\cdots,\,\V{u}_r) \notag \\
  &\mathrel{\defeq}
    \alpha\,\T{\Phi}\qty(\V{u}_1,\,\V{u}_2,\,\cdots,\,\V{u}_r)
    +\beta\,\T{\Psi}\qty(\V{u}_1,\,\V{u}_2,\,\cdots,\,\V{u}_r)
  \comma
\end{align}
于是
\begin{equation}
  \alpha\,\T{\Phi}+\beta\,\T{\Psi} \in \Tensors{r} \fullstop
\end{equation}

下面我们要获得 $\T{\Phi}$ 的表示。
根据之前任意向量用协变基或逆变基的表示,有
\begin{align}
  \forall\,\V{u},\,\V{v},\,\V{w}\in\Rm,
  &\alspace
  \T{\Phi}\qty(\V{u},\,\V{v},\,\V{w}) \notag\\
  &=\T{\Phi}\qty(u^i\V{g}_i,\,v_j\V{g}^j,\,w^k\V{g}_k) \notag
  \intertext{考虑到 $\T{\Phi}$ 对第一变元的线性性,可得}
  &=u^i\,\T{\Phi}\qty(\V{g}_i,\,v_j\V{g}^j,\,w^k\V{g}_k) \notag
  \intertext{同理,}
  &=u^i v_j w^k\,\T{\Phi}\qty(\V{g}_i,\,\V{g}^j,\,\V{g}_k)
  \fullstop
  \label{eq:张量的表示1}
\end{align}
注意这里自然需要满足 Einstein 求和约定。

上式中的 $\T{\Phi}\qty(\V{g}_i,\,\V{g}^j,\,\V{g}_k)$ 是一个数。
它是张量 $\T{\Phi}$ “吃掉”三个基向量的结果。
至于 $u^i v_j w^k$ 部分,三项分别是 $\V{u}$ 的第 $i$ 个逆变分量、
$\V{v}$ 的第 $j$ 个协变分量和 $\V{w}$ 的第 $k$ 个逆变分量。
根据向量分量的定义,可知
\begin{equation}
  u^i v_j w^k
  = \ipb{\V{u}}{\V{g}^i}
  \cdot \ipb{\V{v}}{\V{g}_j}
  \cdot \ipb{\V{w}}{\V{g}^k} \fullstop
  \label{eq:张量的表示2}
\end{equation}

\blankline

暂时中断一下思路,
先给出\emphA{简单张量}\idx{简单张量}\idx{张量!简单张量}的定义。
\begin{equation}
  \forall\,\V{u},\,\V{v},\,\V{w}\in\Rm,\quad
  \V{\xi}\tp\V{\eta}\tp\V{\zeta}\qty(\V{u},\,\V{v},\,\V{w})
  \defeq \ipb{\V{\xi}}{\V{u}}
  \cdot \ipb{\V{\eta}}{\V{v}}
  \cdot \ipb{\V{\zeta}}{\V{w}} \in\realR \comma
\end{equation}
式中的 $\V{\xi},\,\V{\eta},\,\V{\zeta}\in\Rm$。
“$\tp$”的定义将在 \ref{sec:张量积}~节中给出,
现在可以暂时把 $\V{\xi}\tp\V{\eta}\tp\V{\zeta}$ 理解为一种记号。
简单张量作为一个映照,组成它的三个向量分别与
它们“吃掉”的第一、二、三个变元做内积并相乘,结果为一个实数。

考虑到内积的线性性,便有(以第二个变元为例)
\begin{align}
  \V{\xi}\tp\V{\eta}\tp\V{\zeta}
  \qty(\V{u},\,\alpha\tilde{\V{v}}+\beta\hat{\V{v}},\,\V{w})
  &\defeq \ipb{\V{\xi}}{\V{u}}
  \cdot \ipb{\V{\eta}}{\alpha\tilde{\V{v}}+\beta\hat{\V{v}}}
  \cdot \ipb{\V{\zeta}}{\V{w}} \in\realR \notag
  \intertext{注意到
    $\ipb{\V{\eta}}{\alpha\tilde{\V{v}}+\beta\hat{\V{v}}}
      =\alpha\ipb{\V{\eta}}{\tilde{\V{v}}}
      +\beta\ipb{\V{\eta}}{\hat{\V{v}}}$,
    同时再次利用简单张量的定义,可得}
  &= \alpha \V{\xi}\tp\V{\eta}\tp\V{\zeta}
    \qty(\V{u},\,\tilde{\V{v}},\,\V{w})
    +\beta \V{\xi}\tp\V{\eta}\tp\V{\zeta}
    \qty(\V{u},\,\hat{\V{v}},\,\V{w}) \fullstop
\end{align}
类似地,对第一变元和第三变元,同样具有线性性。因此,可以知道
\begin{equation}
  \V{\xi}\tp\V{\eta}\tp\V{\zeta}
  \in\Tensors{3} \fullstop
\end{equation}
可见,“简单张量”的名字是名副其实的,它的确是一个特殊的张量。

回过头来看 \eqref{eq:张量的表示2}~式。很明显,它可以用简单张量来表示。
要注意,由于内积的对称性,可以有两种\footnote{
  这里只考虑把 $\V{u}$、$\V{v}$、$\V{w}$%
  和 $\V{g}^i$、$\V{g}_j$、$\V{g}^k$ 分别放在一起的情况。}表示方法:
\begin{mySubEq}
  \begin{align}
    &\V{g}^i\tp\V{g}_j\tp\V{g}^k
    \qty(\V{u},\,\V{v},\,\V{w})
    \intertext{或者}
    &\V{u}\tp\V{v}\tp\V{w}
    \qty(\V{g}^i,\,\V{g}_j,\,\V{g}^k) \comma
  \end{align}
\end{mySubEq}
我们这里取上面一种。代入式~\eqref{eq:张量的表示1},得
\begin{align}
  \T{\Phi}\qty(\V{u},\,\V{v},\,\V{w})
  &=\T{\Phi}\qty(\V{g}_i,\,\V{g}^j,\,\V{g}_k)
    \cdot\V{g}^i\tp\V{g}_j\tp\V{g}^k
    \qty(\V{u},\,\V{v},\,\V{w}) \notag
  \intertext{由于
    $\T{\Phi}\qty(\V{g}_i,\,\V{g}^j,\,\V{g}_k) \in\Rm$,因此}
  &=\qty[\T{\Phi}\qty(\V{g}_i,\,\V{g}^j,\,\V{g}_k)
    \V{g}^i\tp\V{g}_j\tp\V{g}^k]
    \qty(\V{u},\,\V{v},\,\V{w}) \fullstop
\end{align}
方括号里的部分,就是根据 Einstein 求和约定,
用 $\T{\Phi}\qty(\V{g}_i,\,\V{g}^j,\,\V{g}_k)$
对 $\V{g}^i\tp\V{g}_j\tp\V{g}^k$ 进行线性组合。

由于 $\V{u}$、$\V{v}$、$\V{w}$ 选取的任意性,可以引入如下记号:
\begin{equation}
  \T{\Phi}
  =\T{\Phi}\qty(\V{g}_i,\,\V{g}^j,\,\V{g}_k) \,
    \V{g}^i\tp\V{g}_j\tp\V{g}^k
  \eqcolon \tc{\Phi}{_i^j_k} \,
    \V{g}^i\tp\V{g}_j\tp\V{g}^k \comma
\end{equation}
即
\begin{equation}
  \tc{\Phi}{_i^j_k}
  \coloneq \T{\Phi}\qty(\V{g}_i,\,\V{g}^j,\,\V{g}_k) \comma
\end{equation}
这称为张量的\emphA{分量}\idx{张量!张量的分量}。
它说明一个张量可以用\emphB{张量分量}和基向量组成的%
\emphB{简单张量}来表示。

指标 $i$、$j$、$k$ 的上下是任意的。这里,
它有赖于式~\eqref{eq:张量的表示1} 中基向量的选取。
实际上,对于这里的三阶张量,指标的上下一共有 8 种可能。
指标全部在下面的,称为\emphA{协变分量}
\idx{协变!张量的协变分量}\idx{张量!张量的分量!协变分量}:
\begin{equation}
  \tc{\Phi}{_i_j_k} \coloneq
  \T{\Phi}\qty(\V{g}_i,\,\V{g}_j,\,\V{g}_k) \semicolon
\end{equation}
指标全部在上面的,称为\emphA{逆变分量}
\idx{逆变!张量的逆变分量}\idx{张量!张量的分量!逆变分量}:
\begin{equation}
  \tc{\Phi}{^i^j^k} \coloneq
  \T{\Phi}\qty(\V{g}^i,\,\V{g}^j,\,\V{g}^k) \semicolon
\end{equation}
其余 6 种,称为\emphA{混合分量}
\idx{张量!张量的分量!混合分量}。
对于一个 $r$ 阶张量,显然共有 $2^r$ 种分量表示,
其中协变分量与逆变分量各一种,混合分量 $2^r-2$ 种。

\subsection{张量分量之间的关系} \label{subsec:张量分量之间的关系}
我们已经知道,对于任意一个向量 $\V{\xi}\in\Rm$,
它可以用协变基或逆变基表示:
\begin{equation}
  \V{\xi}=\left\{\begin{aligned}
    \xi^i\V{g}_i \comma \\
    \xi_i\V{g}^i \fullstop
  \end{aligned}\right.
\end{equation}
式中,协变分量与逆变分量满足\emphB{坐标转换关系}\idx{坐标转换关系}:
\begin{braceEq}
  \xi^i &=\ipb{\V{\xi}}{\V{g}^i}
  =\ipb{\V{\xi}}{g^{ik}\V{g}_k}
  =g^{ik}\ipb{\V{\xi}}{\V{g}_k}
  =g^{ik}\xi_k \comma \\
  \xi_i &=\ipb{\V{\xi}}{\V{g}_i}
  =\ipb{\V{\xi}}{g_{ik}\V{g}^k}
  =g_{ik}\ipb{\V{\xi}}{\V{g}^k}
  =g_{ik}\xi^k \fullstop
\end{braceEq}
每一式的第二个等号都用到了\emphB{基向量转换关系}
\idx{基向量!基向量转换关系},
见式~\eqref{eq:基向量转换关系1} 和 \eqref{eq:基向量转换关系4}。

现在再来考虑张量的分量。仍以上文中的张量 $\tc{\Phi}{_i^j_k}
  \coloneq \T{\Phi}\qty(\V{g}_i,\,\V{g}^j,\,\V{g}_k)$ 为例,
我们想要知道它与张量 $\tc{\Phi}{^p_q^r} \coloneq
  \T{\Phi}\qty(\V{g}^p,\,\V{g}_q,\,\V{g}^r)$ 之间的关系。
利用基向量转换关系,可有
\begin{align}
  \tc{\Phi}{_i^j_k}
  &\coloneq\T{\Phi}\qty(\V{g}_i,\,\V{g}^j,\,\V{g}_k) \notag \\
  &=\T{\Phi}
    \qty(g_{ip}\V{g}^p,\,g^{jq}\V{g}_q,\,g_{kr}\V{g}^r) \notag
  \intertext{又利用张量的线性性,得}
  &=g_{ip}g^{jq}g_{kr}
    \T{\Phi}\qty(\V{g}^p,\,\V{g}_q,\,\V{g}^r) \notag \\
  &=g_{ip}g^{jq}g_{kr} \tc{\Phi}{^p_q^r} \fullstop
\end{align}
可见,张量的分量与向量的分量类似,其指标升降可通过\emphB{度量}来实现。
用同样的手法,还可以得到诸如
$\tc{\Phi}{^i^j^k}=g^{jp}\tc{\Phi}{^i_p^k}$、
$\tc{\Phi}{^i_j^k}=g_{jp}g^{kq}\tc{\Phi}{^i^p_k}$
这样的关系式。\idx{张量!指标升降}

\subsection{相对不同基的张量分量之间的关系}
\label{subsec:相对不同基的张量分量之间的关系}
$\Rm$ 空间中,除了 $\qty{\V{g}_i}^m_{i=1}$ 和相应的
对偶基 $\qty{\V{g}^i}^m_{i=1}$ 之外,当然还可以有其他的基,
比如带括号的 $\qty{\V{g}_{(i)}}^m_{i=1}$ 以及对应的
对偶基 $\qty{\V{g}^{(i)}}^m_{i=1}$。
前者对应形如 $\tc{\Phi}{^i_j^k}
  \coloneq \T{\Phi}\qty(\V{g}^i,\,\V{g}_j,\,\V{g}^k)$ 的张量,
后者则对应带括号的张量,如 $\tc{\Phi}{^{(p)}_{(q)}^{(r)}}
  \coloneq \T{\Phi}\qty(\V{g}^{(p)},\,\V{g}_{(q)},\,\V{g}^{(r)})$。
下面我们来探讨这两个张量的关系。

首先来建立基之间的关系。带括号的第 $i$ 个基向量
$\V{g}_{(i)}$,作为 $\Rm$ 空间中的一个向量,
自然可以用另一组基来表示:
\begin{equation}
  \V{g}_{(i)}=\left\{\begin{aligned}
    \ipb{\V{g}_{(i)}}{\V{g}_k}\,\V{g}^k \comma \\
    \ipb{\V{g}_{(i)}}{\V{g}^k}\,\V{g}_k \fullstop
  \end{aligned}\right.
\end{equation}
同理,自然还有它的对偶基:
\begin{equation}
  \V{g}^{(i)}=\left\{\begin{aligned}
    \ipb{\V{g}^{(i)}}{\V{g}_k}\,\V{g}^k \comma \\
    \ipb{\V{g}^{(i)}}{\V{g}^k}\,\V{g}_k \fullstop
  \end{aligned}\right.
\end{equation}
引入记号 $c^k_{(i)} \coloneq \ipb{\V{g}_{(i)}}{\V{g}^k}$
和 $c^{(i)}_k \coloneq \ipb{\V{g}^{(i)}}{\V{g}_k}$,那么有
\begin{braceEq}
  \V{g}_{(i)} &= c^k_{(i)}\V{g}_k \comma \\
  \V{g}^{(i)} &= c^{(i)}_k\V{g}^k \fullstop
\end{braceEq}

容易看出,这两个系数具有如下性质:
\begin{equation}
  c^{(i)}_k c^k_{(j)} = \KroneckerDelta{i}{j} \fullstop
  \label{eq:坐标转换系数的乘积}
\end{equation}
写成矩阵形式\footnote{
  通常我们约定上面的标号作为行号,下面的标号作为列号。},为
\begin{equation}
  \qty[c^{(i)}_k]\qty[c^k_{(j)}]
  =\qty[\KroneckerDelta{j}{i}]=\Mat{I}_m \fullstop
\end{equation}
换句话说,两个系数矩阵是互逆的。
\begin{myProof}
\begin{align}
  c^{(i)}_k c^k_{(j)}
  &=\ipb{\V{g}^{(i)}}{\V{g}_k}\,c^k_{(j)} \notag
  \intertext{利用内积的线性性,有}
  &=\ipb{\V{g}^{(i)}}{c^k_{(j)} \V{g}_k} \notag
  \intertext{根据 $c^k_{(j)}$ 的定义,得到}
  &=\ipb{\V{g}^{(i)}}{\V{g}_{(j)}} \fullstop
\end{align}
带括号的基同样满足对偶关系 \eqref{eq:对偶关系}~式,于是得证。
\end{myProof}

上面我们用不带括号的基表示了带括号的基。反之也是可以的:
\begin{braceEq}
  \V{g}_i &= \ipb{\V{g}_i}{\V{g}^{(k)}}{\Rm}\,\V{g}_{(k)}
    =c^{(k)}_i\V{g}_{(k)} \comma \\
  \V{g}^i &= \ipb{\V{g}^i}{\V{g}_{(k)}}{\Rm}\,\V{g}^{(k)}
    =c^i_{(k)}\V{g}^{(k)} \fullstop
\end{braceEq}
这样一来,就建立起了不同基之间的转换关系。

现在我们回到张量。根据张量分量的定义,
\begin{align}
  \tc{\Phi}{^i_j^k}
  &\coloneq \T{\Phi}\qty(\V{g}^i,\,\V{g}_j,\,\V{g}^k) \notag
  \intertext{利用之前推导的不同基向量之间的转换关系,得}
  &=\T{\Phi}\qty(
    c^i_{(p)}\V{g}^{(p)},\,c^{(q)}_j\V{g}_{(q)},\,
    c^k_{(r)}\V{g}^{(r)}) \notag
  \intertext{由张量的线性性,提出系数:}
  &=c^i_{(p)} c^{(q)}_j c^k_{(r)} \,
    \T{\Phi}\qty(\V{g}^{(p)},\,\V{g}_{(q)},\V{g}^{(r)}) \notag\\
  &=c^i_{(p)} c^{(q)}_j c^k_{(r)} \,
    \tc{\Phi}{^{(p)}_{(q)}^{(r)}} \fullstop
\end{align}
完全类似,还可以有
\begin{equation}
  \tc{\Phi}{^{(i)}_{(j)}^{(k)}}
  =c^{(i)}_p c^g_{(j)} c^{(k)}_r \tc{\Phi}{^p_q^r} \fullstop
\end{equation}

\blankline

总结一下这两小节得到的结果。
对于同一组基下的张量分量,其指标升降通过\emphB{度量}来实现;
对于不同基下的张量分量,其指标转换则通过不同基之间的转换系数来完成。
\idx{张量!指标转换}

    % 张量的运算性质
    \chapter{张量的运算性质}
\section{张量积} \label{sec:张量积}
\emphA{张量积}也叫\emphA{张量并},用符号“$\tp$”表示。
在 \ref{subsec:张量的表示与简单张量}~小节给出简单张量的定义时,
实际上就用到了张量积。张量积的定义为:
\begin{align}
  \forall\,\T{\Phi}\in\Tensors{p},\,\T{\Psi}\in\Tensors{q},
  &\alspace \T{\Phi}\tp\T{\Psi}
    \in\Tensors{p+q} \notag \\
  &=\qty(\Phi^{i_1 \cdots i_p} \,
      \V{g}_{i_1}\tp\cdots\tp\V{g}_{i_p})
    \tp \qty(\Psi_{j_1 \cdots j_q} \,
      \V{g}^{j_1}\tp\cdots\tp\V{g}^{j_q}) \notag \\
  &\defeq \Phi^{i_1 \cdots i_p} \,
    \Psi_{j_1 \cdots j_q}\,
    \qty(\V{g}_{i_1}\tp\cdots\tp\V{g}_{i_p})
    \tp \qty(\V{g}^{j_1}\tp\cdots
      \tp\V{g}^{j_q}_{\phantom{i_p}}) \fullstop
\end{align}
由该定义可以知道,关于简单张量 $\qty(\V{g}_{i_1}\tp\cdots
  \tp\V{g}_{i_p}) \tp \qty(\V{g}^{j_1}\tp\cdots
  \tp\V{g}^{j_q}_{\phantom{i_p}})$,相应的张量分量为
\begin{equation}
  \tc{\qty\big(\Phi\tp\Psi)}
    {^{i_1 \cdots i_p}_{j_1 \cdots j_q}} \fullstop
\end{equation}

\section{\texorpdfstring{$e$ 点积}{e 点积}}
张量的 \emphA{$e$ 点积}可以用符号“$\edp$”表示。
从这个符号可以看出 $e$ 点积的作用:前 $e$ 个指标缩并,后面的点乘。

对于任意的 $\T{\Phi}\in\Tensors{p},\,
  \T{\Psi}\in\Tensors{q},\,
  e\leqslant\min\qty{p,\,q}\in\natN$,$e$ 点积是这样定义的:
\begin{align}
  &\alspace \T{\Phi}\edp\T{\Psi} \notag \\
  &=\qty(\Phi^{i_1 \cdots i_{p-e} i_{p-e+1} \cdots i_p} \,
    \V{g}_{i_1}\tp\cdots\tp\V{g}_{i_{p-e}}
    \tp\hl{\V{g}_{i_{p-e+1}}\tp\cdots\tp\V{g}_{i_p}}
    ) \notag \\
  &\alspace\quad\edp
    \qty(\Psi^{j_1 \cdots j_e j_{e+1} \cdots j_q} \,
    \hl{\V{g}_{j_1}\tp\cdots\tp\V{g}_{j_e}}
    \tp\V{g}_{j_{e+1}}\tp\cdots\tp\V{g}_{j_q}) \notag
  \intertext{把高亮的部分做内积,得到\emphB{度量}:}
  &\defeq\Phi^{i_1 \cdots i_{p-e} i_{p-e+1} \cdots i_p} \,
    \Psi^{j_1 \cdots j_e j_{e+1} \cdots j_q} \notag \\
  &\alspace\quad\cdot
    g_{i_{p-e+1} j_1} \cdots g_{i_p j_e} \,
    \qty(\V{g}_{i_1}\tp\cdots\tp\V{g}_{i_{p-e}})
    \tp\qty(\V{g}_{j_{e+1}}\tp\cdots\tp\V{g}_{j_q}) \notag
  \intertext{玩一下“指标升降游戏”(注意有两种结合方式:
    与 $\Phi$ 或 $\Psi$),可得}
  &=\left\{\begin{lgathered}
      \tc{\Phi}{^{i_1 \cdots i_{p-e}}_{\hl{j_1 \cdots j_e}}} \,
      \Psi^{\hl{j_1 \cdots j_e} j_{e+1} \cdots j_q} \\
      \Phi^{i_1 \cdots i_{p-e} \hl{i_{p-e+1} \cdots i_p}} \,
      \tc{\Psi}{_{\hl{i_{p-e+1} \cdots i_p}}^{j_{e+1}
        \cdots j_q}}
    \end{lgathered}\right\}
    \qty(\V{g}_{i_1}\tp\cdots\tp\V{g}_{i_{p-e}})
    \tp\qty(\V{g}_{j_{e+1}}\tp\cdots\tp\V{g}_{j_q}) \fullstop
\end{align}
最后一步的花括号中,高亮的 $j_1 \cdots j_e$
和 $i_{p-e+1} \cdots i_p$ 都是哑标,可以通过求和求掉。因此有
\begin{equation}
  \T{\Phi}\edp\T{\Psi} \in \Tensors{p+q-2e} \fullstop
\end{equation}
换句话说,$e$ 点积的作用就是将指标\emphB{哑标化}。

作为一个特例,接下来我们介绍\emphA{全点积},用符号“\fdp”表示。
对于任意的 $\T{\Phi},\,\T{\Psi}\in\Tensors{p}$,有
\begin{align}
  &\alspace \T{\Phi}\fdp\T{\Psi}
    \defeq \T{\Phi}\edp[p]\T{\Psi} \notag \\
  &=\qty(\Phi^{i_1 \cdots i_p}\,\V{g}_{i_1}\tp\cdots\tp\V{g}_{i_p})
    \edp[p]
    \qty(\Psi^{j_1 \cdots j_p}\,\V{g}_{j_1}\tp\cdots\tp\V{g}_{j_p})
    \notag \\
  &=\Phi^{i_1 \cdots i_p} \, \Psi^{j_1 \cdots j_p} \,
    g_{i_1 j_1} \cdots g_{i_p j_p} \notag \\
  &=\left\{\begin{lgathered}
      \Phi_{j_1 \cdots j_p} \, \Psi^{j_1 \cdots j_p} \\
      \Phi^{i_1 \cdots i_p} \, \Psi_{i_1 \cdots i_p}
    \end{lgathered}\right.
    \in\realR \fullstop
\end{align}
可见,全点积将\emphB{全部}指标哑标化。

张量自身和自身的全点积,定义为它的\emphA{范数}:
\begin{equation}
  \T{\Phi}\fdp\T{\Phi}
  =\Phi^{i_1 \cdots i_p} \, \Phi_{i_1 \cdots i_p}
  \eqcolon \norm[\Tensors{p}]{\T{\Phi}}^2 \fullstop
\end{equation}

\section{叉乘} \label{sec:叉乘}
张量的\emphA{叉乘}要求底空间为 $\realR^3$。
对于任意的 $\T{\Phi}\in\Tensors[\realR^3]{p},\,
\T{\Psi}\in\Tensors[\realR^3]{q}$,叉乘的定义如下:
\begin{align}
  &\alspace \T{\Phi}\cp\T{\Psi} \notag \\
  &=\qty(\Phi^{i_1 \cdots i_{p-1} i_p} \,
      \V{g}_{i_1}\tp\cdots\tp\V{g}_{i_{p-1}}\tp\V{g}_{i_p})
    \cp \qty(\Psi_{j_1 j_2 \cdots j_q} \,
      \V{g}^{j_1}\tp\V{g}^{j_2}\cdots\tp\V{g}^{j_q}) \notag \\
  &\defeq \Phi^{i_1 \cdots i_p} \, \Psi_{j_1 \cdots j_p} \,
    \V{g}_{i_1}\tp\cdots\tp\V{g}_{i_{p-1}}
    \tp\qty(\V{g}_{i_p}\cp\V{g}^{j_1})
    \tp\V{g}^{j_2}\cdots\tp\V{g}^{j_q}
    \in\Tensors[\realR^3]{p+q-1} \fullstop
\end{align}
注意到,此时简单张量的维数已经降了一阶。

利用\emphA{Levi-Civita 记号},可以进一步展开上式。
\begin{align}
  \V{g}_{i_p}\cp\V{g}^{j_1}
  =\LeviCivita{_{i_p}^{j_1}_s}\,\V{g}^s \comma
\end{align}
式中的
\begin{equation}
  \LeviCivita{_{i_p}^{j_1}_s}
  =\det\!\mqty[\V{g}_{i_p},\,\V{g}^{j_1},\,\V{g}_s] \fullstop
  \label{eq:Levi-Civita记号的定义}
\end{equation}
于是
\begin{equation}
  \T{\Phi}\cp\T{\Psi} \,
  =\LeviCivita{_{i_p}^{j_1}_s}\,
    \Phi^{i_1 \cdots i_p} \, \Psi_{j_1 \cdots j_p}
    \V{g}_{i_1}\tp\cdots\tp\V{g}_{i_{p-1}} \tp\V{g}^s
    \tp\V{g}^{j_2}\cdots\tp\V{g}^{j_q} \fullstop
\end{equation}

下面我们再来类比地定义一种混合积“$\edp[\cp]$”。
对于任意的 $\T{\Phi},\,\T{\Psi}\in\Tensors{3}$,定义
\begin{align}
  \T{\Phi}\edp[\cp]\T{\Psi}
  &=\qty(\Phi^{ijk}\,\V{g}_i\tp\V{g}_j\tp\V{g}_k)
    \edp[\cp]\qty(\Psi_{pqr}\,\V{g}^p\tp\V{g}^q\tp\V{g}^r)\notag \\
  &\defeq \Phi^{ijk}\,\Psi_{pqr} \,
    \KroneckerDelta{q}{j} \,
    \V{g}_i\tp\qty(\V{g}_k\cp\V{g}^p)\tp\V{g}^r \notag
  \intertext{缩并掉 Kronecker δ,
    同时利用 Levi-Civita 记号展开叉乘项,可有}
  &=\LeviCivita{_k^p_s}\,\Phi^{ijk}\,\Psi_{pjr}\,
    \V{g}_i\tp\V{g}^s\tp\V{g}^r \comma
\end{align}
式中的
\begin{equation}
  \LeviCivita{_k^p_s}=\det\!\mqty[\V{g}_k,\,\V{g}^p,\,\V{g}_s] \fullstop
\end{equation}

对于这种混合积,并没有一般的约定。不同的研究者往往会采用不同的写法及表示。

\section{置换(一)}
本节主要介绍\emphA{置换运算}的定义及相关概念,
这将使我们暂时离开张量运算的主线。

置换运算实际上是一种交换位置或者改变次序的运算。
之后我们还将引入针对张量的\emph{置换算子},它是外积运算和外微分运算的基础。
这些运算是现代张量分析与微分几何的支柱。

\subsection{置换的定义}
我们从一个例子开始。下面是一个 $2 \times 7$ 的“矩阵”:
\begin{equation}
  \Perm{\sigma}=\mqty[
    \circNum{1} & \circNum{2} & \circNum{3} & \circNum{4} &
      \circNum{5} & \circNum{6} & \circNum{7} \\
    \circNum{7} & \circNum{4} & \circNum{5} & \circNum{1} &
      \circNum{6} & \circNum{2} & \circNum{3}
  ] \fullstop
  \label{eq:置换序号定义}
\end{equation}
矩阵里面的每一个数字表示一个位置。可以想象成 7 把椅子,
先是按第一行的顺序依次排列,再按照第二行的顺序打乱,重新排列。
于是这就成为一个\emphA{7 阶置换}。这个定义等价于
\begin{mySubEq}
  \begin{gather}
    \Perm{\sigma}=\mqty*(
      4 & 9 & 2 & 7 & 5 & 8 & 3 \\
      3 & 7 & 5 & 4 & 8 & 9 & 2
    ) \comma \label{eq:置换元素表示_数字}
    \intertext{自然也等价于}
    \Perm{\sigma}=\mqty*(
      \spadesuit & \heartsuit & \diamondsuit & \clubsuit &
        \varspadesuit & \varheartsuit & \vardiamondsuit \\
      \vardiamondsuit & \clubsuit & \varspadesuit & \spadesuit &
        \varheartsuit & \heartsuit & \diamondsuit
    ) \comma \label{eq:置换元素表示_符号}
  \end{gather}
\end{mySubEq}
当然,换用任何元素也都是可以的。

通常我们用方括号表示置换的\emphA{序号定义},即标号的排列轮换;
用圆括号表示\emphA{元素定义},即标号对应元素的轮换。

\subsection{置换的符号}
接着来定义置换的\emphA{符号} $\sgn\Perm{\sigma}$。
这里我们把每次交换两个数字称为一次“操作”。
如果经过\emphB{偶数次}“操作”,可以把经置换后的序列恢复为原来的顺序,
那么该置换的符号 $\sgn\Perm{\sigma} = 1$;
而如果经过\emphB{奇数次}“操作”才可以复原,则 $\sgn\Perm{\sigma}=-1$。
若用一个式子表示,则为
\begin{equation}
  \sgn\Perm{\sigma} = (-1)^n \comma
\end{equation}
其中的 $n$ 是恢复原本顺序所需“操作”的次数.

下面我们以式~\eqref{eq:置换序号定义} 所定义的 $\Perm{\sigma}$ 为例,
演示求置换符号的过程。这里的关键是通过两两交换,
按如下步骤把式~\eqref{eq:置换元素表示_符号} 的第二行变换成第一行:
{\setlength{\jot}{0 pt} \setlength{\fboxsep}{2 pt}
\begin{gather*}
  \mqty{
    \hl{\vardiamondsuit} & \hlw{\clubsuit} & \hlw{\varspadesuit} &
      \hl{\spadesuit} & \hlw{\varheartsuit} & \hlw{\heartsuit} &
      \hlw{\diamondsuit}
  } \\
  \mqty{ & & & \Downarrow & & & } \\
  \mqty{
    \hl[pink]{\spadesuit} & \hl{\clubsuit} & \hlw{\varspadesuit} &
      \hl[pink]{\vardiamondsuit} & \hlw{\varheartsuit} &
      \hl{\heartsuit} & \hlw{\diamondsuit}
  } \\
  \mqty{ & & & \Downarrow & & & } \\
  \mqty{
    \hlw{\spadesuit} & \hl[pink]{\heartsuit} & \hl{\varspadesuit} &
      \hlw{\vardiamondsuit} & \hlw{\varheartsuit} &
      \hl[pink]{\clubsuit} & \hl{\diamondsuit}
  } \\
  \mqty{ & & & \Downarrow & & & } \\
  \mqty{
    \hlw{\spadesuit} & \hlw{\heartsuit} & \hl[pink]{\diamondsuit} &
      \hl{\vardiamondsuit} & \hlw{\varheartsuit} & \hl{\clubsuit} &
      \hl[pink]{\varspadesuit}
  } \\
  \mqty{ & & & \Downarrow & & & } \\
  \mqty{
    \hlw{\spadesuit} & \hlw{\heartsuit} & \hlw{\diamondsuit} &
      \hl[pink]{\clubsuit} & \hl{\varheartsuit} &
      \hl[pink]{\vardiamondsuit} & \hl{\varspadesuit}
  } \\
  \mqty{ & & & \Downarrow & & & } \\
  \phantom{\mspace{10mu}}\mqty{
    \hlw{\spadesuit} & \hlw{\heartsuit} & \hlw{\diamondsuit} &
      \hlw{\clubsuit} & \hl[pink]{\varspadesuit} &
      \hl{\vardiamondsuit} &
      \hl{\hl[pink]{\varheartsuit}}
  } \\
  \mqty{ & & & \Downarrow & & & } \\
  \mqty{
    \hlw{\spadesuit} & \hlw{\heartsuit} & \hlw{\diamondsuit} &
      \hlw{\clubsuit} & \hlw{\varspadesuit} &
      \hl[pink]{\varheartsuit} & \hl[pink]{\vardiamondsuit}
  }
\end{gather*} }
一共进行了 6 次两两交换,因此 $\sgn\Perm{\sigma}=1$。

\subsection{置换的复合}
再定义一个置换
\begin{equation}
  \Perm{\tau}=\mqty[
    1 & 2 & 3 & 4 & 5 & 6 & 7 \\
    5 & 1 & 7 & 3 & 6 & 4 & 2
  ] \fullstop
\end{equation}
注意这里用了方括号,因此它是一个\emphB{序号定义}。
方便起见,以后的序号我们都只用不带圈的普通数字表示。
考虑之前定义的置换
\begin{equation}
  \Perm{\sigma}=\mqty[
    1 & 2 & 3 & 4 & 5 & 6 & 7 \\
    7 & 4 & 5 & 1 & 6 & 2 & 3
  ] \comma
\end{equation}
则 $\Perm{\tau}$ 与 $\Perm{\sigma}$ 的复合
\begin{equation}
  \Perm{\tau}\comp\Perm{\sigma}=
  \qty(\begin{array}{@{}ccccccc@{}}
    \dicei & \diceii & \diceiii & \diceiv & \dicev &
      \dicevi & \circledtwodots \\
    \circledtwodots & \diceiv & \dicev & \dicei & \dicevi &
      \diceii & \diceiii \\
    \hdashline
    \dicevi & \circledtwodots & \diceiii & \dicev & \diceii &
      \dicei & \diceiv
  \end{array})
  \quad\mqty{\\ \leftarrow\Perm{\sigma} \\ \leftarrow\Perm{\tau}}
\end{equation}
与函数、线性变换等的复合类似,这里也用小圆圈“$\comp$”表示置换的复合。

假设经过置换 $\Perm{\sigma}$、$\Perm{\tau}$ 作用后得到的序列,
分别需要 $p$ 次和 $q$ 次两两交换才能复原为原来的序列。
那么很显然,经过复合置换 $\Perm{\tau}\comp\Perm{\sigma}$ 作用后的序列,
经过 $q+p$ 次两两交换也一定可以复原。因此,复合置换的符号
\begin{equation}
  \sgn\qty(\Perm{\tau}\comp\Perm{\sigma})
  =(-1)^{q+p}=(-1)^q \cdot (-1)^p
  =\sgn\Perm{\tau}\cdot\sgn\Perm{\sigma} \fullstop
  \label{eq:置换复合的符号}
\end{equation}

\subsection{逆置换}
逆置换 $\Perm{\sigma}^{-1}$ 的定义为
\begin{equation}
  \Perm{\sigma}^{-1}\comp\Perm{\sigma} = \Id \comma
\end{equation}
其中的“$\Id$”是\emphA{恒等映照}。

仍然使用式~\eqref{eq:置换元素表示_符号}:
\begin{equation}
  \Perm{\sigma}=\mqty*(
    \spadesuit & \heartsuit & \diamondsuit & \clubsuit &
      \varspadesuit & \varheartsuit & \vardiamondsuit \\
    \vardiamondsuit & \clubsuit & \varspadesuit & \spadesuit &
      \varheartsuit & \heartsuit & \diamondsuit
  ) \comma
\end{equation}
那么自然有
\begin{equation}
  \Perm{\sigma}^{-1}=\mqty*(
    \vardiamondsuit & \clubsuit & \varspadesuit & \spadesuit &
      \varheartsuit & \heartsuit & \diamondsuit \\
    \spadesuit & \heartsuit & \diamondsuit & \clubsuit &
      \varspadesuit & \varheartsuit & \vardiamondsuit
  ) \fullstop
\end{equation}
显然,我们有 $\Perm{\sigma}^{-1}\comp\Perm{\sigma} = \Id$。

回忆一下逆矩阵的定义。矩阵 $\Mat{A}$ 的逆 $\Mat{A}^{-1}$ 既要满足
$\Mat{A}^{-1}\Mat{A}=\Mat{I}$,又要满足
$\Mat{A}\Mat{A}^{-1}=\Mat{I}$。对于置换也是如此,
因此我们需要检查 $\Perm{\sigma}\comp\Perm{\sigma}^{-1}$:\footnote{
  该式中的数字角标用来澄清原始序号。}
\begin{equation}
  \Perm{\sigma}\comp\Perm{\sigma}^{-1}=
  \qty(\begin{array}{@{}ccccccc@{}}
    \vardiamondsuit & \clubsuit & \varspadesuit & \spadesuit &
      \varheartsuit & \heartsuit & \diamondsuit \\
    \spadesuit_1 & \heartsuit_2 & \diamondsuit_3 & \clubsuit_4 &
      \varspadesuit_5 & \varheartsuit_6 & \vardiamondsuit_7 \\
    \hdashline
    \vardiamondsuit_7 & \clubsuit_4 & \varspadesuit_5 &
      \spadesuit_1 & \varheartsuit_6 &
      \heartsuit_2 & \diamondsuit_3
  \end{array})
  \quad\mqty{
    \\ \leftarrow\Perm{\sigma}^{-1} \\
    \leftarrow\Perm{\sigma}\phantom{^{-1}}
  }
\end{equation}
可见的确有 $\Perm{\sigma}\comp\Perm{\sigma}^{-1}=\Id$。

另外,由于恒等映照 $\Id$ 作用后序列不发生变化,
复原所需的交换次数为 0,因此
\begin{equation}
  \sgn\Id=(-1)^0=1 \fullstop
\end{equation}
而根据定义,
\begin{equation}
  \Id=\Perm{\sigma}^{-1}\comp\Perm{\sigma} \comma
\end{equation}
故有
\begin{equation}
  \sgn\Perm{\sigma} \cdot \sgn\Perm{\sigma}^{-1} = 1 \fullstop
\end{equation}
由此,可以推知
\begin{equation}
  \sgn\Perm{\sigma}=\sgn\Perm{\sigma}^{-1} \comma
  \label{eq:逆置换的符号}
\end{equation}
即置换与它的逆具有\emphB{相同}的符号。

\section{置换(二)}
本节将介绍置换运算的基本性质。

\subsection{置换的穷尽} \label{subsec:置换的穷尽}
先要做一点铺垫。设有序数组
\begin{equation*}
  \qty{i_1,\,i_2,\,\cdots,\,i_r}
\end{equation*}
经置换 $\Perm{\sigma}$ 作用后成为
\begin{equation*}
  \qty{\Perm{\sigma}(i_1),\,\Perm{\sigma}(i_2),\,
    \cdots,\,\Perm{\sigma}(i_r)} \comma
\end{equation*}
则根据之前的元素定义(圆括号),可以把 $\Perm{\sigma}$ 记为
\begin{equation}
  \Perm{\sigma}=\mqty*(
    i_1 & i_2 & \cdots & i_r \\
    \Perm{\sigma}(i_1) & \Perm{\sigma}(i_2) &
      \cdots & \Perm{\sigma}(i_r)
  )\fullstop
\end{equation}
每次置换都将得到一个有序数组。把它们组合到一起,就可以得到集合
\begin{equation}
  \set[\bigg]
    {\qty(\Perm{\sigma}(i_1),\,\Perm{\sigma}(i_2),\,
      \cdots,\,\Perm{\sigma}(i_r))}
    {\forall\,\Perm{\sigma}\in\Permutations{r}} \fullstop
\end{equation}
其中的 $\Permutations{r}$ 表示 $r$ 阶置换的全体。
根据排列组合原理,$r$ 阶置换的总数等于 $r$ 个元素的\emphB{全排列数}。
即该集合共有 $r!$ 个元素。

下面我们要证明
\begin{mySubEq}
  \begin{align}
    &\alspace\set[\bigg]
      {\qty(\Perm{\sigma}(i_1),\,\Perm{\sigma}(i_2),\,
        \cdots,\,\Perm{\sigma}(i_r))}
      {\forall\,\Perm{\sigma}\in\Permutations{r}} \notag \\
    %
    &=\set[\bigg]
      {\qty(
        \Perm{\tau}\comp\Perm{\sigma}(i_1),\,
        \Perm{\tau}\comp\Perm{\sigma}(i_2),\,\cdots,\,
        \Perm{\tau}\comp\Perm{\sigma}(i_r) )}
      {\forall\,\Perm{\sigma},\,\Perm{\tau}\in\Permutations{r}}
      \label{eq:置换的穷尽_复合1} \\
    &=\set[\bigg]
      {\qty(
        \Perm{\sigma}\comp\Perm{\tau}(i_1),\,
        \Perm{\sigma}\comp\Perm{\tau}(i_2),\,\cdots,\,
        \Perm{\sigma}\comp\Perm{\tau}(i_r) )}
      {\forall\,\Perm{\sigma},\,\Perm{\tau}\in\Permutations{r}}
      \label{eq:置换的穷尽_复合2} \\
    &=\set[\bigg]
      {\qty(
        \Perm{\sigma}^{-1}(i_1),\,
        \Perm{\sigma}^{-1}(i_2),\,\cdots,\,
        \Perm{\sigma}^{-1}(i_r))}
      {\forall\,\Perm{\sigma}\in\Permutations{r}} 
      \label{eq:置换的穷尽_逆} \fullstop
  \end{align}
\end{mySubEq}

所谓“穷尽”,就是将 $\Permutations{r}$ 中的所有置换 $\Perm{\sigma}$
全部枚举出来。关于 $\Perm{\sigma}$ 的求和就是一个例子。
以上这条性质说明,置换 $\Perm{\sigma}$ 如果作为一个广义上的“哑标”,
那么穷尽的结果与用 $\Perm{\tau}\comp\Perm{\sigma}$、
$\Perm{\sigma}\comp\Perm{\tau}$ 或 $\Perm{\sigma}^{-1}$
代替该“哑标”的结果是一样的。

\myPROBLEM{这说明置换构成了置换群。?}

\begin{myProof}
证明的思路是说明集合互相包含。

对于式~\eqref{eq:置换的穷尽_复合1},
右边的 $\Perm{\tau}\comp\Perm{\sigma}$ 也是一个 $r$ 阶置换,
自然符合左边集合的定义,因此 $\text{右边}\subset\text{左边}$。
由于这一步是相当显然的,以下的几个证明我们将略去该步。
另一方面,左边的 $\Perm{\sigma}$ 可以表示成
\begin{equation}
  \Perm{\sigma}
  =\Id\comp\Perm{\sigma}
  =\qty(\Perm{\tau}\comp\Perm{\tau}^{-1}) \comp\Perm{\sigma}
  =\Perm{\tau}\comp \qty(\Perm{\tau}^{-1}\comp\Perm{\sigma})
  \comma
\end{equation}
这就是右边集合的定义,因此 $\text{左边}\subset\text{右边}$。
故可证得等式成立。

对于式~\eqref{eq:置换的穷尽_复合2},我们有
\begin{equation}
  \Perm{\sigma}
  =\Perm{\sigma}\comp\Id
  =\Perm{\sigma}\comp \qty(\Perm{\tau}^{-1}\comp\Perm{\tau})
  =\qty(\Perm{\sigma}\comp\Perm{\tau}^{-1}) \comp\Perm{\tau}
  \comma
\end{equation}
它符合了右边集合的定义,因此 $\text{左边}\subset\text{右边}$。
于是等式成立。

对于式~\eqref{eq:置换的穷尽_逆},我们有
\begin{equation}
  \Perm{\sigma}=\qty(\Perm{\sigma}^{-1})^{-1} \comma
\end{equation}
它符合了右边集合的定义,因此 $\text{左边}\subset\text{右边}$。
于是等式成立。
\end{myProof}

\subsection{数组元素的乘积} \label{subsec:数组元素的乘积}
设有序数组 $\qty{i_1,\,i_2,\,\cdots,\,i_r}$、
$\qty{j_1,\,j_2,\,\cdots,\,j_r}$ 和 $\qty{k_1,\,k_2,\,\cdots,\,k_r}$
经 $r$ 阶置换 $\Perm{\sigma}$ 作用后分别成为
$\qty{\Perm{\sigma}(i_1),\,\Perm{\sigma}(i_2),\,
  \cdots,\,\Perm{\sigma}(i_r)}$、
$\qty{\Perm{\sigma}(j_1),\,\Perm{\sigma}(j_2),\,
  \cdots,\,\Perm{\sigma}(j_r)}$ 和
$\qty{\Perm{\sigma}(k_1),\,\Perm{\sigma}(k_2),\,
  \cdots,\,\Perm{\sigma}(k_r)}$,也就是说
\begin{equation}
  \Perm{\sigma}=\mqty*(
    i_1 & i_2 & \cdots & i_r \\
    \Perm{\sigma}(i_1) & \Perm{\sigma}(i_2) &
      \cdots & \Perm{\sigma}(i_r) )
  =\mqty*(
    j_1 & j_2 & \cdots & j_r \\
    \Perm{\sigma}(j_1) & \Perm{\sigma}(j_2) &
      \cdots & \Perm{\sigma}(j_r) )
  =\mqty*(
    k_1 & k_2 & \cdots & k_r \\
    \Perm{\sigma}(k_1) & \Perm{\sigma}(k_2) &
      \cdots & \Perm{\sigma}(k_r) ) \fullstop
\end{equation}
我们有如下结论:
\begin{equation}
  \forall\,\Perm{\sigma}\in\Permutations{r}\, ,\quad
  A_{i_1 j_1 k_1} A_{i_2 j_2 k_2} \cdots
    A_{i_r j_r k_r}
  =A_{\Perm{\sigma}(i_1)\,\Perm{\sigma}(j_1)\,\Perm{\sigma}(k_1)}
    A_{\Perm{\sigma}(i_2)\,\Perm{\sigma}(j_2)\,\Perm{\sigma}(k_2)}
    \cdots
    A_{\Perm{\sigma}(i_r)\,\Perm{\sigma}(j_r)\,\Perm{\sigma}(k_r)}
    \comma
\end{equation}
式中的 $A_{ijk}$ 表示三维数组 $\Mat{A}$ 的一个元素,其指标为 $ijk$。

下面通过一个例子来说明这一条性质。还是用式~\eqref{eq:置换元素表示_数字} 和
\eqref{eq:置换元素表示_符号} 所定义的置换 $\Perm{\sigma}$:
\begin{equation}
  \Perm{\sigma}=\mqty*(
    4 & 9 & 2 & 7 & 5 & 8 & 3 \\
    3 & 7 & 5 & 4 & 8 & 9 & 2 )
  =\mqty*(
    \spadesuit & \heartsuit & \diamondsuit & \clubsuit &
      \varspadesuit & \varheartsuit & \vardiamondsuit \\
    \vardiamondsuit & \clubsuit & \varspadesuit & \spadesuit &
      \varheartsuit & \heartsuit & \diamondsuit ) \fullstop
      \label{eq:置换元素表示_数组元素的乘积举例}
\end{equation}
随意写出一个数组元素乘积:
\begin{equation}
  A_{379}A_{264}A_{157}A_{483}A_{698}
  A_{\diamondsuit\clubsuit\heartsuit}
  A_{\vardiamondsuit\varspadesuit\varheartsuit} \fullstop
  \label{eq:数组元素乘积举例}
\end{equation}
三组下标分别为
\begin{equation}
  \left\{\begin{lgathered}
    3,\,2,\,1,\,4,\,6,\,\diamondsuit,\,\vardiamondsuit;\\
    7,\,6,\,5,\,8,\,9,\,\clubsuit,\,\varspadesuit;\\
    9,\,4,\,7,\,3,\,8,\,\heartsuit,\,\varheartsuit.\\
  \end{lgathered}
  \right.
\end{equation}
考虑 $\Perm{\sigma}$ 的\emphB{序号定义}式~\eqref{eq:置换序号定义}:
\begin{equation}
  \Perm{\sigma}=\mqty[
    1 & 2 & 3 & 4 & 5 & 6 & 7 \\
    7 & 4 & 5 & 1 & 6 & 2 & 3
  ] \fullstop
\end{equation}
所谓序号只是位置的抽象表示,而不代表任何真实的元素。
请记住:置换始终是\emphB{位置}的变换,而非\emphB{元素}的变换,
不要被式~\eqref{eq:置换元素表示_数组元素的乘积举例} 给迷惑了。
把 $\Perm{\sigma}$ 作用在这三组下标上,可得
\begin{equation}
  \left\{\begin{lgathered}
    \vardiamondsuit,\,4,\,6,\,3,\,\diamondsuit,\,2,\,1;\\
    \varspadesuit,\,8,\,9,\,7,\,\clubsuit,\,6,\,5;\\
    \varheartsuit,\,3,\,8,\,9,\,\heartsuit,\,4,\,7.\\
  \end{lgathered}
  \right.
\end{equation}
于是之前的数组元素乘积就变成了
\begin{equation}
  A_{\vardiamondsuit\varspadesuit\varheartsuit}
  A_{483}A_{698}A_{379}
  A_{\diamondsuit\clubsuit\heartsuit}
  A_{264}A_{157} \fullstop
\end{equation}
比对一下各元素,可见与式~\eqref{eq:数组元素乘积举例} 的确是完全一样的。

\subsection{哑标的穷尽} \label{subsec:哑标的穷尽}
考虑如下集合:
\begin{equation}
  \set[\bigg]
  {\qty(i_1,\,i_2,\,\cdots,\,i_r)}
  {\qty{i_1,\,i_2,\,\cdots,\,i_r}
    \text{\ 可取\ } 1,\,2,\,\cdots,\,m}
  \fullstop
\end{equation}
每个 $i_k$ 都有 $m$ 种取法,而 $i_k$ 又有 $r$ 个,
因此该集合一共有 $m^r$ 元素。我们有
\begin{mySubEq}
  \begin{align}
    \forall\,\Perm{\sigma}\in\Permutations{r}\, ,
    &\alspace\set[\bigg]
    {\qty(i_1,\,i_2,\,\cdots,\,i_r)}
    {\qty{i_1,\,i_2,\,\cdots,\,i_r}
      \text{\ 可取\ } 1,\,2,\,\cdots,\,m} \notag \\
    &=\set[\bigg]
    {\qty(\Perm{\sigma}(i_1),\,\Perm{\sigma}(i_2),\,
      \cdots,\,\Perm{\sigma}(i_r) )}
    {\qty{i_1,\,i_2,\,\cdots,\,i_r}
      \text{\ 可取\ } 1,\,2,\,\cdots,\,m}
    \label{eq:哑标的穷尽_置换} \\
    &=\set[\bigg]
    {\qty(\Perm{\sigma}^{-1}(i_1),\,\Perm{\sigma}^{-1}(i_2),\,
      \cdots,\,\Perm{\sigma}^{-1}(i_r) )}
    {\qty{i_1,\,i_2,\,\cdots,\,i_r}
      \text{\ 可取\ } 1,\,2,\,\cdots,\,m} \fullstop
    \label{eq:哑标的穷尽_逆置换}
  \end{align}
\end{mySubEq}
这里,$i_k$ 起的就是\emphB{哑标}的作用。

\begin{myProof}
无论怎样置换,$\Perm{\sigma}(i_k)$ 都是 $1,\,2,\,\cdots,\,m$ 中的数。
因此,对于 $\forall\,\Perm{\sigma}\in\Permutations{r}$,
\begin{equation}
  \qty(\Perm{\sigma}(i_1),\,\Perm{\sigma}(i_2),\,
    \cdots,\,\Perm{\sigma}(i_r) )
  \in\set[\bigg]
    {\qty(i_1,\,i_2,\,\cdots,\,i_r)}
    {\qty{i_1,\,i_2,\,\cdots,\,i_r}
      \text{\ 可取\ } 1,\,2,\,\cdots,\,m} \comma
\end{equation}
即
\begin{align}
  &\mathrel{\phantom{\subset}}\set[\bigg]
  {\qty(\Perm{\sigma}(i_1),\,\Perm{\sigma}(i_2),\,
    \cdots,\,\Perm{\sigma}(i_r) )}
  {\qty{i_1,\,i_2,\,\cdots,\,i_r}
    \text{\ 可取\ } 1,\,2,\,\cdots,\,m} \notag \\
  &\subset\set[\bigg]
  {\qty(i_1,\,i_2,\,\cdots,\,i_r)}
  {\qty{i_1,\,i_2,\,\cdots,\,i_r}
    \text{\ 可取\ } 1,\,2,\,\cdots,\,m} \fullstop
\end{align}
另一方面,由于 $\Id=\Perm{\sigma}^{-1}\comp\Perm{\sigma}$,即
\begin{equation}
  \qty(i_1,\,i_2,\,\cdots,\,i_r)
  =\qty(\Perm{\sigma}^{-1}\comp\Perm{\sigma}(i_1),\,
    \Perm{\sigma}^{-1}\comp\Perm{\sigma}(i_2),\,
    \cdots,\,\Perm{\sigma}^{-1}\comp\Perm{\sigma}(i_r) ) \comma
\end{equation}
而进行一次逆置换仍然使得元素不离开原有的范围,也就是说
\begin{equation}
  \qty(i_1,\,i_2,\,\cdots,\,i_r)
  \in\set[\bigg]
    {\qty(\Perm{\sigma}(i_1),\,\Perm{\sigma}(i_2),\,
      \cdots,\,\Perm{\sigma}(i_r) )}
    {\qty{i_1,\,i_2,\,\cdots,\,i_r}
      \text{\ 可取\ } 1,\,2,\,\cdots,\,m} \comma
\end{equation}
即
\begin{align}
  &\mathrel{\phantom{\subset}}\set[\bigg]
  {\qty(i_1,\,i_2,\,\cdots,\,i_r)}
  {\qty{i_1,\,i_2,\,\cdots,\,i_r}
    \text{\ 可取\ } 1,\,2,\,\cdots,\,m} \notag \\
  &\subset\set[\bigg]
  {\qty(\Perm{\sigma}(i_1),\,\Perm{\sigma}(i_2),\,
    \cdots,\,\Perm{\sigma}(i_r) )}
  {\qty{i_1,\,i_2,\,\cdots,\,i_r}
    \text{\ 可取\ } 1,\,2,\,\cdots,\,m} \fullstop
\end{align}
两个集合互相包含,也就证得了式~\eqref{eq:哑标的穷尽_置换}。

用相同的方法也可证得关于逆置换的 \eqref{eq:哑标的穷尽_逆置换}~式,
此处从略。
\end{myProof}

\section{置换(三)}
本节将给出置换运算在线性代数中的一些应用。

\subsection{行列式}

\section{置换(四)}
本节将重回张量运算的主线,引入\emphA{置换算子}。

\subsection{置换算子;对称张量与反对称张量}
对于任意的置换 $\Perm{\sigma}\in\Permutations{r}$,定义\emphA{置换算子}
\begin{equation}
  \mmap{\opPerm}
    {\Tensors{r}\ni\T{\Phi}}
    {\opPerm(\T{\Phi})\in\Tensors{r}} \comma
\end{equation}
式中
\begin{equation}
  \opPerm(\T{\Phi})\qty(\V{u}_1,\,\V{u}_2,\,\cdots,\,\V{u}_r)
  \defeq\T{\Phi}\qty(\V{u}_{\Perm{\sigma}(1)},\,
    \V{u}_{\Perm{\sigma}(2)},\,\cdots,\,
    \V{u}_{\Perm{\sigma}(r)})
  \in\realR \fullstop
\end{equation}
这里的“$\cdots\in\realR$”是根据张量的定义:\emphB{多重线性函数}。

如果我们的置换
\begin{equation}
  \Perm{\sigma}=\mqty*(
    i_1 & i_2 & \cdots & i_r \\
    \Perm{\sigma}(i_1) & \Perm{\sigma}(i_1) & \cdots
      & \Perm{\sigma}(i_1)
  ) \comma
\end{equation}
那么对应的置换算子将满足
\begin{equation}
  \opPerm(\T{\Phi})\qty(\V{u}_{i_1},\,\V{u}_{i_2},\,
    \cdots,\,\V{u}_{i_r})
  \defeq\T{\Phi}\qty(\V{u}_{\Perm{\sigma}(i_1)},\,
    \V{u}_{\Perm{\sigma}(i_2)},\,\cdots,\,
    \V{u}_{\Perm{\sigma}(i_r)}) \fullstop
\end{equation}

根据张量的线性性,容易知道置换算子也具有线性性:
\begin{align}
  \forall\,\T{\Phi},\,\T{\Psi}\in\Tensors{r}
    \text{\ 以及\ } \alpha,\,\beta\in\realR,\quad
  \opPerm(\alpha\,\T{\Phi}+\beta\,\T{\Psi})
  =\alpha\,\opPerm(\T{\Phi})
    +\beta\,\alpha\opPerm(\T{\Psi}) \fullstop
  \label{eq:置换算子的线性性}
\end{align}

\begin{myProof}
\begin{align}
  \opPerm(\alpha\,\T{\Phi}+\beta\,\T{\Psi})
    \qty(\V{u}_1,\,\cdots,\,\V{u}_r)
  &=(\alpha\,\T{\Phi}+\beta\,\T{\Psi})
    \qty(\V{u}_{\Perm{\sigma}(1)},\,\cdots,\,
      \V{u}_{\Perm{\sigma}(r)}) \notag \\
  &=\alpha\,\T{\Phi}\qty(\V{u}_{\Perm{\sigma}(1)},\,\cdots,\,
      \V{u}_{\Perm{\sigma}(r)})
    +\beta\,\T{\Psi}\qty(\V{u}_{\Perm{\sigma}(1)},\,\cdots,\,
      \V{u}_{\Perm{\sigma}(r)}) \notag \\
  &=\alpha\opPerm(\T{\Phi})
      \qty(\V{u}_1,\,\cdots,\,\V{u}_r)
    +\beta\opPerm(\T{\Psi})
      \qty(\V{u}_1,\,\cdots,\,\V{u}_r) \notag \\
  &=\qty[\alpha\opPerm(\T{\Phi})
      +\beta\opPerm(\T{\Psi})]
    \qty(\V{u}_1,\,\cdots,\,\V{u}_r) \fullstop
\end{align}
\end{myProof}
两个置换算子复合的结果也是很显然的:
\begin{equation}
  \forall\,\Perm{\sigma},\,\Perm{\tau}\in\Permutations{r},\quad
  \opPerm\comp\opPerm[\Perm{\tau}]
  =\opPerm[\Perm{\sigma}\comp\Perm{\tau}] \fullstop
  \label{eq:置换算子的复合}
\end{equation}

\begin{myProof}
\begin{align}
  \opPerm\comp\opPerm[\Perm{\tau}](\T{\Phi})
    \qty(\V{u}_{i_1},\,\cdots,\,\V{u}_{i_r})
  &=\opPerm(\T{\Phi})
      \qty(\V{u}_{\Perm{\tau}(1)},\,
      \cdots,\,\V{u}_{\Perm{\tau}(r)}) \notag \\
  &=\T{\Phi}\qty(\V{u}_{\Perm{\sigma}\comp\Perm{\tau}(1)},\,
    \cdots,\,\V{u}_{\Perm{\sigma}\comp\Perm{\tau}(r)}) \notag \\
  &=\opPerm[\Perm{\sigma}\comp\Perm{\tau}](\T{\Phi})
    \qty(\V{u}_{i_1},\,\cdots,\,\V{u}_{i_r}) \fullstop
\end{align}
\end{myProof}

\blankline

有了置换算子,我们就可以来定义\emphA{对称张量}和\emphA{反对称张量}。
对称张量的全体记为 $\Sym$,反对称张量的全体记为 $\Skw$。
如果以 $\Rm$ 为底空间,
又分别可以记为 $\SymTensors{r}$ 和 $\SkwTensors{r}$。

对于任意的 $\T{\Phi}\in\Tensors{r}$,如果
\begin{equation}
  \opPerm(\T{\Phi})=\T{\Phi} \comma
  \label{eq:对称张量的定义}
\end{equation}
则称 $\T{\Phi}$ 为\emphB{对称张量},
即 $\T{\Phi}\in\Sym \text{\ 或\ } \SymTensors{r}$;如果
\begin{equation}
  \opPerm(\T{\Phi})=\sgn\Perm{\sigma}\cdot\T{\Phi} \comma
  \label{eq:反对称张量的定义}
\end{equation}
则称 $\T{\Phi}$ 为\emphB{反对称张量},
即 $\T{\Phi}\in\Skw \text{\ 或\ } \SkwTensors{r}$。

有些书中采用分量形式来定义(反)对称张量。这与此处的定义是等价的:
\begin{mySubEq}
  \begin{align}
    \opPerm(\T{\Phi})=\T{\Phi} &\iff
      \Phi_{\Perm{\sigma}(i_1)\cdots\Perm{\sigma}(i_p)}
        =\Phi_{i_1 \cdots i_p} \comma\\
    \opPerm(\T{\Phi})=\sgn\Perm{\sigma}\cdot\T{\Phi} &\iff
      \Phi_{\Perm{\sigma}(i_1)\cdots\Perm{\sigma}(i_p)}
        =\sgn\Perm{\sigma}\cdot\Phi_{i_1 \cdots i_p}
        \fullstop
  \end{align}
\end{mySubEq}
反对称张量与我们熟知的\emphB{行列式}有些类似:
交换两列(对于张量就是两个分量),符号相反。
全部分量两两交换一遍,前面的系数自然是置换的符号。
而如果无论怎么交换分量(当然需要全部两两交换一遍),符号都不变,
那这样的张量就是对称张量。

\myPROBLEM{一个二阶张量的协变(或逆变)分量,可以用一个矩阵表示。}如果这个张量是一个反对称张量,交换任意两个分量要添加负号;对于矩阵而言,这就意味着交换两行(或两列)……

\subsection{置换算子的表示}
根据上文给出的定义,我们有
\begin{equation}
  \opPerm(\T{\Phi})\qty(\V{u}_{i_1},\,\cdots,\,\V{u}_{i_r})
  \defeq\T{\Phi}\qty(\V{u}_{\Perm{\sigma}(i_1)},\,\cdots,\,
    \V{u}_{\Perm{\sigma}(i_r)}) \fullstop
\end{equation}
首先回忆一下 \ref{subsec:张量的表示与简单张量}~小节中张量的表示:
选一组基(协变、逆变均可),然后把张量用这组基表示。于是
\begin{align}
  &\alspace\opPerm(\T{\Phi})
    \qty(\V{u}_{i_1},\,\cdots,\,\V{u}_{i_r})
  =\T{\Phi}\qty(\V{u}_{\Perm{\sigma}(i_1)},\,\cdots,\,
    \V{u}_{\Perm{\sigma}(i_r)}) \notag
  \intertext{把向量用协变基表示:}
  &=\T{\Phi}\qty(u^{i_1}_{\Perm{\sigma}(i_1)}\,\V{g}_{i_1},\,
    \cdots,\,u^{i_r}_{\Perm{\sigma}(i_r)}\,\V{g}_{i_r}) \notag
  \intertext{根据张量的线性性,提出系数:}
  &=\T{\Phi}\qty(\V{g}_{i_1},\,\cdots,\,\V{g}_{i_r}) \cdot
    \qty(u^{i_1}_{\Perm{\sigma}(i_1)} \cdots
      u^{i_r}_{\Perm{\sigma}(i_r)}) \notag
  \intertext{前半部分可以用张量分量表示;
    而后半部分是一组逆变分量,可以写成内积的形式}
  &=\Phi_{i_1 \cdots i_p}
    \qty[\ipb{\V{u}_{\Perm{\sigma}(i_1)}}{\V{g}^{i_1}}\cdots
      \ipb{\V{u}_{\Perm{\sigma}(i_r)}}{\V{g}^{i_r}}]
  \addtocounter{equation}{1}
  \tag{\theequation*}
  \label{eq:置换算子的表示_中间步骤}
  \intertext{注意到方括号中的其实是简单张量的定义,这就有}
  &=\Phi_{i_1 \cdots i_p}
    \V{g}^{i_1}\tp\cdots\tp\V{g}^{i_r}
    \qty(\V{u}_{\Perm{\sigma}(i_1)},\,\cdots,\,
      \V{u}_{\Perm{\sigma}(i_r)}) \fullstop
  \addtocounter{equation}{-1}
\end{align}
最后一步仍然没能回到 $\qty(\V{u}_{i_1},\,\cdots,\,\V{u}_{i_r})$,
因此以上推导只是简单地展开了 $\T{\Phi}$,并没有获得实质性的结果。

然而,只要稍作改动,情况就会大不相同。
考虑一下 \ref{subsec:数组元素的乘积}~小节中置换运算%
有关\emphB{数组元素乘积}的性质:
\begin{equation}
  \forall\,\Perm{\tau}\in\Permutations{r},\quad
  A_{i_1 j_1} \cdots A_{i_r j_r}
  =A_{\Perm{\tau}(i_1) \Perm{\tau}(j_1)} \cdots
    A_{\Perm{\tau}(i_r) \Perm{\tau}(j_r)} \comma
  \label{eq:置换算子的表示_推导_置换性质}
\end{equation}
式中
\begin{equation}
  \Perm{\tau}
  =\mqty*(
    i_1 & \cdots & i_r \\
    \Perm{\tau}(i_1) & \cdots & \Perm{\tau}(i_r) )
  =\mqty*(
    j_1 & \cdots & j_r \\
    \Perm{\tau}(j_1) & \cdots & \Perm{\tau}(j_r) ) \fullstop
\end{equation}
由此可以看出,式~\eqref{eq:置换算子的表示_中间步骤} 方括号中的部分
其实是由 $\Perm{\sigma}(i_k)$ 和 $i_k$ 两套指标确定的一组数:
\begin{equation}
  A_{\Perm{\sigma}(i_k)\,i_k}
  =\ipb{\V{u}_{\Perm{\sigma}(i_k)}}{\V{g}^{i_k}} \semicolon
\end{equation}
另一方面,显然有 $\Perm{\sigma}^{-1}\in\Permutations{r}$。于是
\begin{align}
  &\alspace\opPerm(\T{\Phi})
    \qty(\V{u}_{i_1},\,\cdots,\,\V{u}_{i_r}) \notag \\
  &=\Phi_{i_1 \cdots i_r}
    \qty[\ipb{\V{u}_{\Perm{\sigma}(i_1)}}{\V{g}^{i_1}}\cdots
      \ipb{\V{u}_{\Perm{\sigma}(i_r)}}{\V{g}^{i_r}}] \notag
  \intertext{应用置换的性质 \eqref{eq:置换算子的表示_推导_置换性质}~式:}
  &=\Phi_{i_1 \cdots i_r}
    \qty[
      \ipb{\V{u}_{\Perm{\sigma}^{-1}\comp\Perm{\sigma}(i_1)}}
        {\V{g}^{\Perm{\sigma}^{-1}(i_1)}} \cdots
      \ipb{\V{u}_{\Perm{\sigma}^{-1}\comp\Perm{\sigma}(i_r)}}
        {\V{g}^{\Perm{\sigma}^{-1}(i_r)}}
    ] \notag \\
  &=\Phi_{i_1 \cdots i_r}
    \qty[
      \ipb{\V{u}_{i_1}}{\V{g}^{\Perm{\sigma}^{-1}(i_1)}} \cdots
      \ipb{\V{u}_{i_2}}{\V{g}^{\Perm{\sigma}^{-1}(i_r)}}
    ] \notag
  \intertext{同样,用简单张量表示,可得}
  &=\Phi_{i_1 \cdots i_r}
    \V{g}^{\Perm{\sigma}^{-1}(i_1)}\tp\cdots
      \tp\V{g}^{\Perm{\sigma}^{-1}(i_r)}
    \qty(\V{u}_{i_1},\,\cdots,\,\V{u}_{i_r}) \fullstop
\end{align}
这样,我们就得到了置换算子的一种表示:
\begin{align}
  \opPerm(\T{\Phi})
  &=\opPerm\qty(\Phi_{i_1 \cdots i_r}
    \V{g}^{i_1}\tp\cdots\tp\V{g}^{i_r}) \notag \\
  &=\Phi_{i_1 \cdots i_r}
    \V{g}^{\Perm{\sigma}^{-1}(i_1)}\tp\cdots
      \tp\V{g}^{\Perm{\sigma}^{-1}(i_r)} \fullstop
  \label{eq:置换算子的表示_逆置换在简单张量上}
\end{align}

在式~\eqref{eq:置换算子的表示_逆置换在简单张量上} 中,
$i_1,\,\cdots,\,i_r$ 都是哑标,要被求和求掉。
张量 $\T{\Phi}$ 的底空间是 $\Rm$,所以每个 $i_k$ 都有 $m$ 个取值。
考虑一下 \ref{subsec:哑标的穷尽}~小节中置换运算%
有关\emphB{哑标穷尽}的性质,有
\begin{align}
  \forall\,\Perm{\sigma}\in\Permutations{r}\, ,
  &\alspace\set[\bigg]
  {\qty(i_1,\,i_2,\,\cdots,\,i_r)}
  {\qty{i_1,\,i_2,\,\cdots,\,i_r}
    \text{\ 可取\ } 1,\,2,\,\cdots,\,m} \notag \\
  &=\set[\bigg]
  {\qty(\Perm{\sigma}(i_1),\,\Perm{\sigma}(i_2),\,
    \cdots,\,\Perm{\sigma}(i_r) )}
  {\qty{i_1,\,i_2,\,\cdots,\,i_r}
    \text{\ 可取\ } 1,\,2,\,\cdots,\,m} \fullstop
\end{align}
因此,我们可以把式~\eqref{eq:置换算子的表示_逆置换在简单张量上} 中的指标
$i_k$ 换成 $\Perm{\sigma}(i_k)$:
\begin{align}
  \opPerm(\T{\Phi})
  &=\Phi_{i_1 \cdots i_r}
    \V{g}^{\Perm{\sigma}^{-1}(i_1)}\tp\cdots
      \tp\V{g}^{\Perm{\sigma}^{-1}(i_r)} \notag \\
  &=\Phi_{\Perm{\sigma}(i_1) \cdots \Perm{\sigma}(i_r)}
    \V{g}^{\Perm{\sigma}^{-1}\comp\Perm{\sigma}(i_1)}\tp\cdots
      \tp\V{g}^{\Perm{\sigma}^{-1}\comp
        \Perm{\sigma}(i_r)} \notag \\
  &=\Phi_{\Perm{\sigma}(i_1) \cdots \Perm{\sigma}(i_r)}
    \V{g}^{i_1}\tp\cdots\tp\V{g}^{i_r} \fullstop
\end{align}
这是置换算子的另一种表示。

综上,要获得置换算子的表示,
若是对\emphB{张量分量}进行操作,就直接使用对分量指标使用置换;
若是对\emphB{简单张量}进行操作,则要对其指标使用逆置换:\footnote{
  这里稍有改动,用了张量的逆变分量,不过实质都是一样的。%
  使用协变分量还是逆变分量,这个嘛,悉听尊便。}
\begin{mySubEq}
  \begin{align}
    \opPerm(\T{\Phi})
    &=\opPerm\qty(\Phi^{i_1 \cdots i_r}
      \V{g}_{i_1}\tp\cdots\tp\V{g}_{i_r}) \notag \\
    &=\Phi^{\Perm{\sigma}(i_1)\cdots\Perm{\sigma}(i_r)}
      \V{g}_{i_1}\tp\cdots\tp\V{g}_{i_r}
    \label{eq:置换算子的表示_总结_对张量分量} \\
    &=\Phi^{i_1 \cdots i_r}
      \V{g}_{\Perm{\sigma}^{-1}(i_1)}\tp\cdots
        \tp\V{g}_{\Perm{\sigma}^{-1}(i_r)} \fullstop
    \label{eq:置换算子的表示_总结_对简单张量} 
  \end{align}
\end{mySubEq}

\section{对称化算子与反对称化算子}
\subsection{定义}
\emphA{对称化算子 $\opSym$} 和\emphA{反对称化算子 $\opSkw$} 的定义分别为
\begin{align}
  \opSym(\T{\Phi})
    &\defeq \frac{1}{r!} \sum_{\Perm{\sigma}\in\Permutations{r}}
      \opPerm(\T{\Phi})
  \intertext{和}
  \opSkw(\T{\Phi})
    &\defeq \frac{1}{r!} \sum_{\Perm{\sigma}\in\Permutations{r}}
      \sgn\Perm{\sigma}\cdot\opPerm(\T{\Phi}) \comma
\end{align}
式中,$\T{\Phi}\in\Tensors{r}$。
根据置换算子的线性性,很容易知道对称化算子与反对称化算子也具有线性性。

对于任意的 $\T{\Phi}\in\Tensors{r}$,我们有
\begin{braceEq}
  \opSym(\T{\Phi}) &\in \Sym \comma \\
  \opSkw(\T{\Phi}) &\in \Skw \fullstop
\end{braceEq}
这说明任意一个张量,对它作用对称化算子之后,将变为对称张量;
反之,作用反对称算子之后,将变为反对称张量。\footnote{
  换一个角度,(反)对称张量实际上可以用(反)对称化算子来定义。}

\begin{myProof}
要判断 $\opSym(\T{\Phi})$ 是不是对称张量,
首先需要在其上作用一个置换算子 $\opPerm[\Perm{\tau}]$:
\begin{align}
  \opPerm[\Perm{\tau}]\qty\big[\opSym(\T{\Phi})]
  &=\opPerm[\Perm{\tau}]\qty[\frac{1}{r!}
      \sum_{\Perm{\sigma}\in\Permutations{r}}
      \opPerm(\T{\Phi})] \notag
  \intertext{根据置换算子的线性性 \eqref{eq:置换算子的线性性}~式,可有}
  &=\frac{1}{r!} \sum_{\Perm{\sigma}\in\Permutations{r}}
    \opPerm[\Perm{\tau}]\comp\opPerm(\T{\Phi}) \notag
  \intertext{再用一下 \eqref{eq:置换算子的复合}~式,得到}
  &=\frac{1}{r!} \sum_{\Perm{\sigma}\in\Permutations{r}}
    \opPerm[\Perm{\tau}\comp\Perm{\sigma}](\T{\Phi}) \notag
  \intertext{这里求和的作用就是把置换 $\Perm{\sigma}$ 穷尽了。
    根据 \ref{subsec:置换的穷尽}~小节中的内容,
    再在 $\Perm{\sigma}$ 上复合一个置换 $\Perm{\tau}$,
    结果将保持不变:}
  &=\frac{1}{r!} \sum_{\Perm{\sigma}\in\Permutations{r}}
    \opPerm(\T{\Phi})
  =\opSym(\T{\Phi}) \fullstop
\end{align}
对照一下对称张量的定义 \eqref{eq:对称张量的定义}~式,可见的确有
$\opSym(\T{\Phi})\in\Sym$。

类似地,
\begin{align}
  \opPerm[\Perm{\tau}]\qty\big[\opSkw(\T{\Phi})]
  &=\opPerm[\Perm{\tau}]\qty[\frac{1}{r!}
      \sum_{\Perm{\sigma}\in\Permutations{r}}
      \sgn\Perm{\sigma}\cdot\opPerm(\T{\Phi})] \notag \\
  &=\frac{1}{r!} \sum_{\Perm{\sigma}\in\Permutations{r}}
    \sgn\Perm{\sigma}\cdot
    \qty\bigg[\opPerm[\Perm{\tau}]\comp\opPerm(\T{\Phi})]
    \notag \\
  &=\frac{1}{r!} \sum_{\Perm{\sigma}\in\Permutations{r}}
    \sgn\Perm{\sigma}\cdot
    \opPerm[\Perm{\tau}\comp\Perm{\sigma}](\T{\Phi}) \notag
  \intertext{根据式~\eqref{eq:置换复合的符号},
    $\sgn\Perm{\tau}\cdot\sgn\Perm{\sigma}
    =\sgn(\Perm{\tau}\comp\Perm{\sigma})$,于是}
  &=\frac{1}{r!} \sum_{\Perm{\sigma}\in\Permutations{r}}
    \frac{\sgn(\Perm{\tau}\comp\Perm{\sigma})}{\sgn\Perm{\tau}}
    \cdot\opPerm[\Perm{\tau}\comp\Perm{\sigma}](\T{\Phi}) \notag
  \intertext{注意到始终成立
    $\sgn\Perm{\tau}\cdot\sgn\Perm{\tau}=1$
    (因为 $\sgn\Perm{\tau}=\pm 1$),又有}
  &=\frac{\sgn\Perm{\tau}}{r!}
    \sum_{\Perm{\sigma}\in\Permutations{r}}
    \sgn(\Perm{\tau}\comp\Perm{\sigma})
    \cdot\opPerm[\Perm{\tau}\comp\Perm{\sigma}](\T{\Phi}) \notag
  \intertext{利用置换的穷尽,
    $\Perm{\tau}\comp\Perm{\sigma}$ 与 $\Perm{\sigma}$ 相比,
    结果将保持不变:}
  &=\sgn\tau\cdot\qty[\frac{1}{r!}
    \sum_{\Perm{\sigma}\in\Permutations{r}}
    \sgn\Perm{\sigma}\cdot\opPerm(\T{\Phi})]
  =\sgn\tau\cdot\opSkw(\T{\Phi}) \fullstop
\end{align}
与反对称张量的定义 \eqref{eq:反对称张量的定义}~式相比,可见的确有
$\opSym(\T{\Phi})\in\Skw$。

这里的操作直接对张量本身进行,没有采用涉及到张量“自变量”(向量)的繁琐计算,
因而显得更加干净利落。
\end{myProof}

\subsection{反对称化算子的性质}
上文已经定义了\emphA{反对称化算子 $\opSkw$}:
\begin{equation}
  \forall\,\T{\Phi}\in\Tensors{r},\quad
  \opSkw(\T{\Phi})\defeq
  \frac{1}{r!} \sum_{\Perm{\sigma}\in\Permutations{r}}
  \sgn\Perm{\sigma}\cdot\opPerm(\T{\Phi})
  \in\Skw \text{\ 或\ } \SkwTensors{r} \fullstop
\end{equation}
即任意一个 $r$ 阶张量,作用反对称化算子后就变成了 $r$ 阶反对称张量。
$r$ 阶反对称张量也称为 \emphA{$r$-form} (\emphA{$r$-形式})。

下面列出反对称化算子的几条性质。

\begin{myEnum}
\item 反对称化算子若重复作用,仅相当于一次作用:
\begin{equation}
  \opSkw^2 \coloneq \opSkw\comp\opSkw = \opSkw \fullstop
\end{equation}
根据数学归纳法,显然有
\begin{equation}
  \forall\,p\in\natN,\quad
  \opSkw^p \coloneq \underbrace{\opSkw\comp\cdots\comp\opSkw}_
    {\text{$p$ 个 $\opSkw$}} = \opSkw \fullstop
\end{equation}

\begin{myProof}
\begin{align}
  \opSkw^2 &= \opSkw\qty\big[\opSkw(\T{\Phi})] \notag \\
  &\defeq \opSkw\qty[\frac{1}{r!}
    \sum_{\Perm{\sigma}\in\Permutations{r}}
    \sgn\Perm{\sigma}\cdot\opPerm(\T{\Phi})] \notag \\
  &\defeq \frac{1}{r!}
    \sum_{\Perm{\tau}\in\Permutations{r}}
    \sgn\Perm{\tau}\cdot\opPerm[\Perm{\tau}]\qty[\frac{1}{r!}
      \sum_{\Perm{\sigma}\in\Permutations{r}}
      \sgn\Perm{\sigma}\cdot\opPerm(\T{\Phi})] \notag
  \intertext{根据线性性,可有}
  &=\frac{1}{(r!)^2} \sum_{\Perm{\tau}\in\Permutations{r}}
    \sum_{\Perm{\sigma}\in\Permutations{r}}
    \sgn\Perm{\tau}\sgn\Perm{\sigma}
    \cdot\opPerm[\Perm{\tau}]\comp\opPerm(\T{\Phi}) \notag
  \intertext{根据式~\eqref{eq:置换复合的符号}
    和式~\eqref{eq:置换算子的复合},有}
  &=\frac{1}{(r!)^2} \sum_{\Perm{\tau}\in\Permutations{r}}
    \sum_{\Perm{\sigma}\in\Permutations{r}}
    \sgn\qty(\Perm{\tau}\comp\Perm{\sigma}) \cdot
    \opPerm[\Perm{\tau}\comp\Perm{\sigma}](\T{\Phi}) \notag \\
  &=\frac{1}{(r!)^2} \sum_{\Perm{\tau}\in\Permutations{r}}
    \qty[\sum_{\Perm{\sigma}\in\Permutations{r}}
      \sgn\qty(\Perm{\tau}\comp\Perm{\sigma}) \cdot
      \opPerm[\Perm{\tau}\comp\Perm{\sigma}](\T{\Phi})] \notag
  \intertext{注意到方括号中的部分穷尽了置换 $\Perm{\sigma}$,
    因此可以用 $\Perm{\sigma}$ 取代“指标”
    $\Perm{\tau}\comp\Perm{\sigma}$:}
  &=\frac{1}{(r!)^2} \sum_{\Perm{\tau}\in\Permutations{r}}
    \qty[\sum_{\Perm{\sigma}\in\Permutations{r}}
      \sgn\Perm{\sigma}\cdot\opPerm(\T{\Phi})] \notag
  \intertext{回到定义,有}
  &=\frac{1}{r!} \sum_{\Perm{\tau}\in\Permutations{r}}
    \opSkw(\T{\Phi})
  =\frac{1}{r!} \cdot r! \opSkw(\T{\Phi})
  =\opSkw(\T{\Phi}) \fullstop
\end{align}
\end{myProof}

\blankline

\item 对任意两个张量 $\T{\Phi}\in\Tensors{p}$ 和
$\T{\Psi}\in\Tensors{q}$ 的并施加反对称化算子,可以得到如下结果:
\begin{mySubEq}
  \begin{align}
    \opSkw\qty\big(\T{\Phi}\tp\T{\Psi})
    &=\opSkw\qty\big[\opSkw(\T{\Phi})\tp\T{\Psi}]
    \label{eq:对张量并施加反对称化算子_1} \\
    &=\opSkw\qty\big[\T{\Phi}\tp\opSkw(\T{\Psi})]
    \label{eq:对张量并施加反对称化算子_2} \\
    &=\opSkw\qty\big[\opSkw(\T{\Phi})\tp\opSkw(\T{\Psi})]
    \fullstop
    \label{eq:对张量并施加反对称化算子_3}
  \end{align}
\end{mySubEq}

\begin{myProof}
这里只给出式~\eqref{eq:对张量并施加反对称化算子_2} 的证明。
另外两式的证明是类似的。
\begin{align}
  \opSkw\qty\big[\T{\Phi}\tp\opSkw(\T{\Psi})]
  &=\opSkw\,\qty[\T{\Phi}\tp
    \qty\Bigg(\frac{1}{q!}
      \sum_{\Perm{\tau}\in\Permutations{q}}\sgn\Perm{\tau}
      \cdot\opPerm[\Perm{\tau}] (\T{\Psi}) ) ] \notag
  \intertext{根据张量积的线性性提出系数:}
  &=\opSkw\,\qty[\frac{1}{q!}
    \sum_{\Perm{\tau}\in\Permutations{q}} \sgn\Perm{\tau}
    \cdot\T{\Phi}\tp\opPerm[\Perm{\tau}](\T{\Psi})] \notag
  \intertext{利用置换的穷尽,
    可以把 $\Perm{\tau}$ 换作 $\Perm{\tau}^{-1}$:}
  &=\opSkw\,\qty[\frac{1}{q!}
    \sum_{\Perm{\tau}\in\Permutations{q}}
    \sgn\Perm{\tau}^{-1} \cdot\T{\Phi}\tp
    \opPerm[\Perm{\tau}^{-1}](\T{\Psi})] \notag
  \intertext{注意到
    $\sgn\Perm{\tau}=\sgn\Perm{\tau}^{-1}$,于是}
  &=\opSkw\,\qty[\frac{1}{q!}
    \sum_{\Perm{\tau}\in\Permutations{q}}
    \sgn\Perm{\tau} \cdot\T{\Phi}\tp
    \opPerm[\Perm{\tau}^{-1}](\T{\Psi})] \comma
  \label{eq:对张量并施加反对称化算子_证明_Part1}
\end{align}
式中,
\begin{equation}
  \opPerm[\Perm{\tau}^{-1}](\T{\Psi})
  =\Psi^{j_1 \cdots j_q} \,
    \V{g}_{\Perm{\tau}(j_1)}\tp\cdots
    \tp\V{g}_{\Perm{\tau}(j_q)} \fullstop
\end{equation}
于是有
\begin{equation}
  \T{\Phi}\tp\opPerm[\Perm{\tau}^{-1}](\T{\Psi})
  =\Phi^{i_1 \cdots i_p}\,\Psi^{j_1 \cdots j_q}\,
    \qty(\V{g}_{i_1}\tp\cdots\tp\V{g}_{i_p}) \tp
    \qty(\V{g}_{\Perm{\tau}(j_1)}\tp\cdots
    \tp\V{g}_{\Perm{\tau}(j_q)}) \fullstop
\end{equation}

置换 $\Perm{\tau}\in\Permutations{q}$ 的元素定义为
\begin{equation}
  \Perm{\tau}=\mqty*(
    j_1 & \cdots & j_q \\
    \Perm{\tau}(j_1) & \cdots & \Perm{\tau}(j_q)
  ) \fullstop
\end{equation}
引入它的“延拓”(或曰“增广”)置换
$\Perm{\hat{\tau}}\in\Permutations{p+q}$,其定义为
\begin{equation}
  \Perm{\hat{\tau}}=\mqty*(
    i_1 & \cdots & i_p & j_1 & \cdots & j_q \\
    i_1 & \cdots & i_p &
      \Perm{\tau}(j_1) & \cdots & \Perm{\tau}(j_q)
  ) \fullstop
\end{equation}
这样一来,就有
\begin{align}
  \T{\Phi}\tp\opPerm[\Perm{\tau}^{-1}](\T{\Psi})
  &=\Phi^{i_1 \cdots i_p}\,\Psi^{j_1 \cdots j_q}\,
    \qty(\V{g}_{i_1}\tp\cdots\tp\V{g}_{i_p}) \tp
    \qty(\V{g}_{\Perm{\tau}(j_1)}\tp\cdots
    \tp\V{g}_{\Perm{\tau}(j_q)}) \notag \\
  &=\Phi^{i_1 \cdots i_p}\,\Psi^{j_1 \cdots j_q}\,
    \qty(\V{g}_{\Perm{\hat{\tau}}(i_1)}\tp\cdots
    \tp\V{g}_{\Perm{\hat{\tau}}(i_p)}) \tp
    \qty(\V{g}_{\Perm{\hat{\tau}}(j_1)}\tp\cdots
    \tp\V{g}_{\Perm{\hat{\tau}}(j_q)}) \notag \\
  &=\opPerm[\Perm{\hat{\tau}}^{-1}]
    \qty\big(\T{\Phi}\tp\T{\Psi}) \fullstop
\end{align}
另一方面,参与轮换的元素只有后面的 $q$ 个,
因此 $\Perm{\hat{\tau}}$ 起到的作用
实际上等同于 $\Perm{\tau}$(当然两者作用范围不同)。所以可知
\begin{equation}
  \sgn\Perm{\hat{\tau}} = \sgn\Perm{\tau} \fullstop
\end{equation}
把以上这两点代入%
式~\eqref{eq:对张量并施加反对称化算子_证明_Part1} 的推导,有
\begin{align}
  \opSkw\qty\big[\T{\Phi}\tp\opSkw(\T{\Psi})]
  &=\opSkw\,\qty[\frac{1}{q!}
    \sum_{\Perm{\tau}\in\Permutations{q}}
    \sgn\Perm{\tau} \cdot\T{\Phi}\tp
    \opPerm[\Perm{\tau}^{-1}](\T{\Psi})] \notag \\
  &=\opSkw\,\qty[\frac{1}{q!}
    \sum_{\Perm{\tau}\in\Permutations{q}}
    \sgn\Perm{\hat{\tau}} \cdot
    \opPerm[\Perm{\hat{\tau}}^{-1}]
    \qty\big(\T{\Phi}\tp\T{\Psi}) ] \notag
  \intertext{再用一次线性性,可得}
  &=\frac{1}{q!} \sum_{\Perm{\tau}\in\Permutations{q}}
    \sgn\Perm{\hat{\tau}} \cdot
    \opSkw\qty\Big[\opPerm[\Perm{\hat{\tau}}^{-1}]
      \qty\big(\T{\Phi}\tp\T{\Psi})] \fullstop
  \label{eq:对张量并施加反对称化算子_证明_Part2}
\end{align}

$\T{\Phi}\tp\T{\Psi}$ 是一个 $p+q$ 阶张量,
它作用置换算子后阶数当然保持不变。
根据反对称化算子的定义,可有
\begin{align}
  \opSkw\qty\Big[\opPerm[\Perm{\hat{\sigma}}^{-1}]
    \qty\big(\T{\Phi}\tp\T{\Psi})]
  &=\frac{1}{(p+q)!}
    \sum_{\Perm{\hat{\sigma}}\in\Permutations{p+q}}
    \sgn\Perm{\hat{\sigma}} \cdot
    \qty\Big[\opPerm[\Perm{\hat{\sigma}}]
      \comp \opPerm[\Perm{\hat{\tau}}^{-1}]
      \qty\big(\T{\Phi}\tp\T{\Psi})] \notag \\
  &=\frac{1}{(p+q)!}
    \sum_{\Perm{\hat{\sigma}}\in\Permutations{p+q}}
    \sgn\Perm{\hat{\sigma}} \cdot
    \qty\Big[\opPerm[\Perm{\hat{\sigma}}
        \comp\Perm{\hat{\tau}}^{-1}]
      \qty\big(\T{\Phi}\tp\T{\Psi})] \comma
  \label{eq:对张量并施加反对称化算子_证明_Part3}
\end{align}
式中的 $\Perm{\hat{\sigma}}$ 和之前定义的
$\Perm{\hat{\tau}}$ 含义相同,只是为了确保哑标不重复,
我们采用了不同的字母来表示。
该式~\eqref{eq:对张量并施加反对称化算子_证明_Part3} 中的
$\sgn\Perm{\hat{\sigma}}$ 可以写成
\begin{equation}
  \sgn\Perm{\hat{\sigma}}
  =\frac{\sgn\qty(
      \Perm{\hat{\sigma}}\comp\Perm{\hat{\tau}}^{-1})}
    {\sgn\Perm{\hat{\tau}}^{-1}}
  =\frac{\sgn\qty(
      \Perm{\hat{\sigma}}\comp\Perm{\hat{\tau}}^{-1})}
    {\sgn\Perm{\hat{\tau}}} \fullstop
  \label{eq:对张量并施加反对称化算子_证明_Part4}
\end{equation}
第二个等号是根据式~\eqref{eq:逆置换的符号}。

注意到 \eqref{eq:对张量并施加反对称化算子_证明_Part2}~式中
也有一个 $\sgn\Perm{\hat{\tau}}$,因此
\begin{align}
  \opSkw\qty\big[\T{\Phi}\tp\opSkw(\T{\Psi})]
  &=\frac{1}{q!} \sum_{\Perm{\tau}\in\Permutations{q}}
    \sgn\Perm{\hat{\tau}} \cdot
    \opSkw\qty\Big[\opPerm[\Perm{\hat{\tau}}^{-1}]
      \qty\big(\T{\Phi}\tp\T{\Psi})] \notag \\
  &=\frac{1}{q!} \sum_{\Perm{\tau}\in\Permutations{q}}
    \sgn\Perm{\hat{\tau}} \cdot
    \qty[\frac{1}{(p+q)!}
      \sum_{\Perm{\hat{\sigma}}\in\Permutations{p+q}}
      \sgn\Perm{\hat{\sigma}} \cdot
      \qty\Big(\opPerm[\Perm{\hat{\sigma}}
          \comp\Perm{\hat{\tau}}^{-1}]
        \qty\big(\T{\Phi}\tp\T{\Psi}))] \notag
  \intertext{用式~\eqref{eq:对张量并施加反对称化算子_证明_Part4}
    合并掉 $\sgn\Perm{\hat{\tau}}$ 和
    $\sgn\Perm{\hat{\sigma}}$:}
  &=\frac{1}{q!} \sum_{\Perm{\tau}\in\Permutations{q}}
    \frac{1}{(p+q)!}
    \sum_{\Perm{\hat{\sigma}}\in\Permutations{p+q}}
    \sgn\qty(\Perm{\hat{\sigma}}\comp\Perm{\hat{\tau}}^{-1})
    \cdot\qty\Big[\opPerm[\Perm{\hat{\sigma}}
        \comp\Perm{\hat{\tau}}^{-1}]
        \qty\big(\T{\Phi}\tp\T{\Psi})] \notag
  \intertext{再次利用置换穷尽的性质改变“哑标”置换:}
  &=\frac{1}{q!} \sum_{\Perm{\tau}\in\Permutations{q}}
    \frac{1}{(p+q)!}
    \sum_{\Perm{\hat{\sigma}}\in\Permutations{p+q}}
    \sgn\Perm{\hat{\sigma}}
    \cdot\qty\Big[\opPerm[\Perm{\hat{\sigma}}]
        \qty\big(\T{\Phi}\tp\T{\Psi})] \notag
  \intertext{终于拨开云雾见青天,看到了似曾相识的定义:}
  &=\frac{1}{q!} \sum_{\Perm{\tau}\in\Permutations{q}}
    \opSkw\qty\big(\T{\Phi}\tp\T{\Psi}) \notag \\
  &=\frac{1}{q!} \cdot q!
    \opSkw\qty\big(\T{\Phi}\tp\T{\Psi})
  =\opSkw\qty\big(\T{\Phi}\tp\T{\Psi}) \fullstop
\end{align}
\end{myProof}

\blankline

\item 反对称化算子具有所谓\emphA{反导性}:
\begin{equation}
  \forall\,\T{\Phi}\in\Tensors{p},\,\T{\Psi}\in\Tensors{q},\quad
  \opSkw\qty\big(\T{\Phi}\tp\T{\Psi})
  =(-1)^{pq}\cdot\opSkw\qty\big(\T{\Psi}\tp\T{\Phi}) \fullstop
\end{equation}

\begin{myProof}
首先单独把反对称算子展开。它所作用的张量为 $p+q$ 阶,因而相应的置换
$\Perm{\sigma}\in\Permutations{p+q}$:
\begin{align}
  \opSkw&=\frac{1}{(p+q)!} \sum_{\Perm{\sigma}\in\Permutations{p+q}}
    \sgn\Perm{\sigma}\cdot\opPerm \notag \\
  &=\frac{1}{(p+q)!} \sum_{\Perm{\sigma}\in\Permutations{p+q}}
    \sgn\qty(\Perm{\sigma}\comp\Perm{\tau}^{-1})
    \cdot\opPerm[\Perm{\sigma}\comp\Perm{\tau}^{-1}] \fullstop
\end{align}
第二的等号与之前一样,利用了置换的穷尽。
这里的 $\Perm{\tau}$ 是 $\Permutations{p+q}$ 中一个任意的置换。
利用置换符号的性质,有
\begin{equation}
  \sgn\qty(\Perm{\sigma}\comp\Perm{\tau}^{-1})
  =\sgn\Perm{\sigma} \cdot \sgn\Perm{\tau}^{-1}
  =\sgn\Perm{\sigma} \cdot \sgn\Perm{\tau} \fullstop
\end{equation}
因此
\begin{align}
  \opSkw\qty\big(\T{\Phi}\tp\T{\Psi})
  &=\frac{1}{(p+q)!} \sum_{\Perm{\sigma}\in\Permutations{p+q}}
    \sgn\qty(\Perm{\sigma}\comp\Perm{\tau}^{-1})
    \cdot\opPerm[\Perm{\sigma}\comp\Perm{\tau}^{-1}]
      \qty\big(\T{\Phi}\tp\T{\Psi}) \notag \\
  &=\frac{\sgn\Perm{\tau}}{(p+q)!}
    \sum_{\Perm{\sigma}\in\Permutations{p+q}}
    \sgn\Perm{\sigma}
    \cdot\opPerm[\Perm{\sigma}\comp\Perm{\tau}^{-1}]
      \qty\big(\T{\Phi}\tp\T{\Psi}) \fullstop
  \label{eq:反对称化算子的反导性_证明_Part1}
\end{align}

把张量展开成分量形式,可以有
\begin{align}
  \opPerm[\Perm{\sigma}\comp\Perm{\tau}^{-1}]
    \qty\big(\T{\Phi}\tp\T{\Psi})
  &=\opPerm[\Perm{\sigma}\comp\Perm{\tau}^{-1}]
    \qty[\Phi^{i_1 \cdots i_p}\,\Psi^{j_1 \cdots j_q}\,
      \qty(\V{g}_{i_1}\tp\cdots\tp\V{g}_{i_p}) \tp
      \qty(\V{g}_{j_1}\tp\cdots\tp\V{g}_{j_q})] \notag
  \intertext{根据式~\eqref{eq:置换算子的复合},可得}
  &=\opPerm\comp\opPerm[\Perm{\tau}^{-1}]
    \qty[\Phi^{i_1 \cdots i_p}\,\Psi^{j_1 \cdots j_q}\,
      \qty(\V{g}_{i_1}\tp\cdots\tp\V{g}_{i_p}) \tp
      \qty(\V{g}_{j_1}\tp\cdots\tp\V{g}_{j_q})] \notag
  \intertext{利用置换算子的表示
    \eqref{eq:置换算子的表示_总结_对简单张量}~一式,对简单张量进行操作:}
  &=\opPerm\qty[
      \Phi^{i_1 \cdots i_p}\,\Psi^{j_1 \cdots j_q}\,
      \qty(\V{g}_{\Perm{\tau}(i_1)}\tp\cdots
        \tp\V{g}_{\Perm{\tau}(i_p)}) \tp
      \qty(\V{g}_{\Perm{\tau}(j_1)}\tp\cdots
        \tp\V{g}_{\Perm{\tau}(j_q)})] \fullstop
\end{align}

根据 $\Perm{\tau}$ 的任意性,不妨取\footnote{
  矩阵中的 $i_p$ 和 $j_q$ 等未必是对齐的,这里的写法只是为了表示方便。}
\begin{equation}
  \Perm{\tau}=\mqty*(
    i_1 & \cdots & i_p & j_1 & \cdots & j_q \\
    j_1 & \cdots & j_q & i_1 & \cdots & i_p ) \fullstop
\end{equation}
这种取法恰好可以使指标为 $i$ 和 $j$ 的向量交换一下位置。于是
\begin{align}
  \opPerm[\Perm{\sigma}\comp\Perm{\tau}^{-1}]
    \qty\big(\T{\Phi}\tp\T{\Psi})
  &=\opPerm\qty[
      \Phi^{i_1 \cdots i_p}\,\Psi^{j_1 \cdots j_q}\,
      \qty(\V{g}_{\Perm{\tau}(j_1)}\tp\cdots
        \tp\V{g}_{\Perm{\tau}(j_q)}) \tp
      \qty(\V{g}_{\Perm{\tau}(i_1)}\tp\cdots
        \tp\V{g}_{\Perm{\tau}(i_p)})] \notag \\
  &=\opPerm\qty[
      \Phi^{i_1 \cdots i_p}\,\Psi^{j_1 \cdots j_q}\,
      \qty(\V{g}_{i_1}\tp\cdots\tp\V{g}_{i_p}) \tp
      \qty(\V{g}_{j_1}\tp\cdots\tp\V{g}_{j_q})] \notag
  \intertext{张量分量作为\emphB{数},交换律自然无需多言:}
  &=\opPerm\qty[
      \Psi^{j_1 \cdots j_q}\,\Phi^{i_1 \cdots i_p}\,
      \qty(\V{g}_{i_1}\tp\cdots\tp\V{g}_{i_p}) \tp
      \qty(\V{g}_{j_1}\tp\cdots\tp\V{g}_{j_q})] \notag \\
  &=\opPerm \qty\big(\T{\Psi}\tp\T{\Phi}) \fullstop
\end{align}
下面再考虑一下 $\Perm{\tau}$ 的符号:$j_1$ 先和 $i_p$ 交换,
再和 $i_{p-1}$ 交换,以此类推,直到移动至 $i_1$ 的位置,一共交换了 $p$ 次。
而 $j_2,\,\cdots,\,j_q$ 也是同理,各需进行 $p$ 次交换。
所以总共是 $p \cdot q$ 次两两交换。因此,
\begin{equation}
  \sgn\Perm{\tau}=(-1)^{pq} \fullstop
\end{equation}

回到式~\eqref{eq:反对称化算子的反导性_证明_Part1} 的推导,有
\begin{align}
  \opSkw\qty\big(\T{\Phi}\tp\T{\Psi})
  &=\frac{\sgn\Perm{\tau}}{(p+q)!}
    \sum_{\Perm{\sigma}\in\Permutations{p+q}}
    \sgn\Perm{\sigma}
    \cdot\opPerm[\Perm{\sigma}\comp\Perm{\tau}^{-1}]
      \qty\big(\T{\Phi}\tp\T{\Psi}) \notag \\
  &=\frac{(-1)^{pq}}{(p+q)!}
    \sum_{\Perm{\sigma}\in\Permutations{p+q}}
    \sgn\Perm{\sigma}
    \cdot\opPerm\qty\big(\T{\Psi}\tp\T{\Phi}) \notag \\
  &=(-1)^{pq}\cdot\opSkw\qty\big(\T{\Psi}\tp\T{\Phi}) \fullstop
\end{align}
\end{myProof}

\end{myEnum}

    % 相对论简介
    \chapter{相对论简介}
\section{Lorentz 变换}
狭义相对论研究的空间是 \emphA{Minkowski 空间}(或称
\emphA{Minkowski 时空})$\minkM$。它是一个四维空间,第一个维度是时间,
之后的三个维度则是一般的 Euclid 空间。也就是说,Minkowski 空间中的
向量可以表示为
\begin{equation}
  \V{x} = \qty(x_0,\,x_1,\,x_2,\,x_3)
  = \qty(ct,\,x,\,y,\,z)\in\minkM \comma
\end{equation}
其中的 $c$ 是真空中的光速,它是一个常数。在自然单位制下,光速 $c=1$,
此时时间与空间具有相同的量纲。

根据惯例,我们用\emphB{希腊字母}指代任意一个指标($0,\,1,\,2,\,3$),
而用\emphB{拉丁字母}指代空间指标($1,\,2,\,3$)。

Minkowski 空间中也可定义协变基 $\qty{\V{g}_\mu}_{\mu=0}^3$ 与逆变基
$\qty{\V{g}^\mu_{\vphantom{\mu}}}_{\mu=0}^3$。向量 $\V{x}\in\minkM$
可以用它们展开:
\begin{equation}
  \V{x} = x^\mu\,\V{g}_\mu = x_\mu\,\V{g}^\mu \fullstop
\end{equation}
由基向量的内积可获得度量(物理上也称为\emphA{度规}):
\begin{equation}
  \eta_{\mu\nu} = \ipb[\minkM]{\V{g}_\mu}{\V{g}_\nu} \fullstop
\end{equation}
写成矩阵形式,为\footnote{
  在不同的著作中,度量的符号可以有不同的取法。此处为
  $(\mathord{+}\mathord{-}\mathord{-}\mathord{-})$,为粒子物理学家
  所偏爱;另一种是 $(\mathord{-}\mathord{+}\mathord{+}\mathord{+})$,
  在 $c\to\infty$ 的极限下自然退化到 Euclid 空间,数学家更喜欢用。
  当然,这两种符号约定并没有本质区别。}
\begin{equation}
  \qty[\eta_{\mu\nu}] = \mqty[\dmat{c^2, -1, -1, -1}] \fullstop
\end{equation}
根据度量满足的一般关系 \eqref{eq:度量之积}~式
\begin{equation}
  g_{ik}\,g^{kj} = \KroneckerDelta{j}{i} \comma
\end{equation}
可以知道
\begin{equation}
  \qty[\eta^{\mu\nu}] = \mqty[\dmat{c^{-2}, -1, -1, -1}] \fullstop
\end{equation}

%\begin{itemize}
%  \item 在本文的惯例中,
%  \item 在自然单位制下,光速 $c=1$,此时可将度量写为
%    \begin{equation*}
%      \qty[\eta_{\mu\nu}] = \mqty[\dmat{1, -1, -1, -1}] \fullstop
%    \end{equation*}
%\end{itemize}

%%\section{Lorentz 与 Poincaré 对称性}
%%
%%\begin{problem}{1.1}
%%\question{证明 Lorentz 变换满足 $\mat{\Lambda}\trans \mat{g} \mat{\Lambda} =\mat{g}$,
%%  并证明它们构成一个群。}
%%根据定义,变换后的向量
%%\[ {x'}^\mu = \Lambda\ids{^\mu_\nu} x^\nu \]
%%满足
%%\[ \qty(x')^2 = (x)^2 \]
%%因此
%%\[
%%  \qty(x')^2 = g_{\mu\nu} {x'}^\mu {x'}^\nu
%%  = g_{\mu\nu} \qty(\Lambda\ids{^\mu_\tau} x^\tau) \qty(\Lambda\ids{^\nu_\sigma} x^\sigma)
%%  = (x)^2 = g_{\tau\sigma} x^\tau x^\sigma
%%\]
%%由于 $\vb{x}$ 的任意性,可知
%%\[ g_{\mu\nu} \Lambda\ids{^\mu_\tau} \Lambda\ids{^\nu_\sigma} = g_{\tau\sigma} \]
%%为把该式写成矩阵形式,需要保证各指标依次排列,即
%%\[ \qty(\Lambda\trans)\ids{_\tau^\mu} g_{\mu\nu} \Lambda\ids{^\nu_\sigma} = g_{\tau\sigma} \]
%%这里用到了 $\qty(\Lambda\trans)\ids{_\nu^\mu} = \Lambda\ids{^\mu_\nu}$。此时可将上式写成
%%\[ \mat{\Lambda}\trans\mat{g}\mat{\Lambda} =\mat{g} \]
%%
%%群满足
%%\end{problem}


  \part{体积上的张量场场论}
    % 微分同胚(曲线坐标系)
    \chapter{微分同胚(曲线坐标系)} \label{chap:微分同胚}
\section{微分同胚}
\subsection{双射}
设 $f$ 是集合 $A$ 到 $B$ 的映照。
如果 $A$ 中不同的元素有不同的像,
则称 $f$ 为\emphA{单射}(也叫“一对一”);
如果 $B$ 中每个元素都是 $A$ 中元素的像,则称 $f$ 为\emphA{满射};
如果 $f$ 既是单射又是满射,
则称 $f$ 为\emphA{双射}(也叫“一一对应”)。
三种情况的示意见图~\ref{fig:单射满射双射}。

\begin{figure}[h]
  \centering
  \includegraphics{images/three-mappings.png}
  \caption{单射、满射与双射}
  \label{fig:单射满射双射}
\end{figure}

设开集 $\domD{\V{X}},\,\domD{\V{x}}\subset\Rm$,
它们之间存在\emphB{双射},即\emphB{一一对应}关系:
\begin{equation}
  \mmap{\V{X}(\V{x})}
    {\domD{\V{x}}\ni\V{x}=\mqty[x^1 \\ \vdots \\ x^m]}
    {\V{X}(\V{x})=\mqty[X^1 \\ \vdots \\ X^m](\V{x})
      \in\domD{\V{X}}} \fullstop
\end{equation}
由于该映照实现了 $\domD{\V{x}}$ 到 $\domD{\V{X}}$ 之间的双射,
因此它存在逆映照:
\begin{equation}
  \mmap{\V{x}(\V{X})}
    {\domD{\V{X}}\ni\V{X}=\mqty[X^1 \\ \vdots \\ X^m]}
    {\V{x}(\V{X})=\mqty[x^1 \\ \vdots \\ x^m](\V{X})
      \in\domD{\V{x}}} \fullstop
\end{equation}

我们把 $\domD{\V{X}}$ 称为\emphA{物理域},它是实际物理事件
发生的区域;$\domD{\V{x}}$ 则称为\emphA{参数域}。
由于物理域通常较为复杂,因此我们常把参数域取为规整的形状,
以便之后的处理。

设物理量 $f(\V{X})$ 定义在物理域
$\domD{\V{X}}\subset\Rm$ 上\footnote{
  实际的物理事件当然只会发生在三维 Euclid 空间中
  (只就“空间”而言),但在数学上也可以推广到 $m$ 维。
},则 $f$ 就定义了一个\emphA{场}:
\begin{equation}
  \mmap{f}
    {\domD{\V{X}}\ni\V{X}}
    {f(\V{X})} \fullstop
\end{equation}
所谓的“场”,就是自变量用\emphB{位置}刻画的映照。
它可以是\emphA{标量场},如温度、压强、密度等,
此时 $f(\V{X})\in\realR$;也可以是\emphA{向量场},
如速度、加速度、力等,此时 $f(\V{X})\in\Rm$;
对于更深入的物理、力学研究,往往还需引入\emphA{张量场},
此时 $f(\V{X})\in\Tensors{r}$。

$\V{X}$ 存在于物理域 $\domD{\V{X}}$ 中,我们称它为\emphA{物理坐标}。
由于上文已经定义了 $\domD{\V{x}}$ 到 $\domD{\V{X}}$ 之间的双射
(不是 $f$!),因此 $\domD{\V{x}}$ 中就有\emphB{唯一的}
$\V{x}$ 与 $\V{X}$ 相对应,
它称为\emphA{参数坐标}(也叫\emphA{曲线坐标})。
又因为物理域 $\domD{\V{X}}$ 上已经定义了场 $f(\V{X})$,
参数域中必然\emphB{唯一}存在场 $\tilde{f}(\V{x})$ 与之对应:
\begin{equation}
  \mmap{\tilde{f}}
    {\domD{\V{x}}\ni\V{x}}
    {\tilde{f}(\V{x})=f\comp\V{X}(\V{x})
      =f\qty\big(\V{X}(\V{x}))} \fullstop
\end{equation}
$\V{x}$ 与 $\V{X}$ 是完全等价的,
因而 $\tilde{f}$ 与 $f$ 也是完全等价的,所以同样有
\begin{equation}
  f(\V{X})=\tilde{f}\qty\big(\V{x}(\V{X})) \fullstop
\end{equation}

物理域中的场要满足\emphB{守恒定律},如质量守恒、动量守恒、
能量守恒等。从数学上看,这些守恒定律就是 $f(\V{X})$
需要满足的一系列\emphB{偏微分方程}。将场变换到参数域后,
它仍要满足这些方程。但我们已经设法将参数域取得较为规整,
故在其上进行数值求解就会相当方便。

\subsection{参数域方程} \label{subsec:参数域方程}
上文已经提到,物理域中的场 $f(\V{X})$ 需满足守恒定律,
这等价于一系列偏微分方程(PDE)。
在物理学和力学中,用到的 PDE 通常是\emphB{二阶}的,它们可以写成
\begin{equation}
  \forall\,\V{X}\in\domD{\V{X}},\quad
  \sum_{\alpha=1}^{m} A_\alpha(\V{X}) \pdv{f}{X^\alpha} (\V{X})
  +\sum_{\alpha=1}^{m}\sum_{\beta=1}^{m}
    B_{\alpha\beta}(\V{X}) \pdv{f}{X^\beta}{X^\alpha} (\V{X}) = 0
\end{equation}
的形式。我们的目标是把该\emphB{物理域}方程转化为\emphB{参数域}方程,
即关于 $\tilde{f}(\V{x})$ 的 PDE。
多元微积分中已经提供了解决方案:\emphA{链式求导法则}。

考虑到
\begin{equation}
  f(\V{X})=\tilde{f}\qty\big(\V{x}(\V{X}))
  =\tilde{f}\qty(x^1(\V{X}),\,\cdots,\,x^m(\V{X})) \comma
\end{equation}
于是有
\begin{equation}
  \pdv{f}{X^\alpha} (\V{X})
  =\sum_{s=1}^{m} \pdv{\tilde{f}}{\displaystyle x^s}
    \qty\big(\V{x}(\V{X})) \cdot
    \pdv{x^s}{X^\alpha} (\V{X}) \fullstop
  \label{eq:参数域方程_一阶偏导项}
\end{equation}
这里用到的链式法则,由\emphB{复合映照可微性定理}驱动,
它要求 $\tilde{f}$ 关于 $\V{x}$ 可微,同时 $\V{x}$ 关于
$\V{X}$ 可微。

\myPROBLEM{对于更高阶的项,往往需要更强的条件。}一般地,我们要求
\begin{braceEq}
  \V{X}(\V{x})&\in\cf{\domD{\V{x}}}{\Rm} \semicolon \\
  \V{x}(\V{X})&\in\cf{\domD{\V{X}}}{\Rm} \fullstop
\end{braceEq}
这里的 $\DiffMorp$ 指\emphB{直至 $p$ 阶偏导数存在且连续}的
映照全体;$p=1$ 时,它就等价于可微。至于 $p$ 的具体取值,
则由 PDE 的阶数所决定。

通常情况下,已知条件所给定的往往都是
$\domD{\V{x}}$ 到 $\domD{\V{X}}$ 的映照
\begin{equation}
  \mmap{\V{X}(\V{x})}
    {\domD{\V{x}}\ni\V{x}=\mqty[x^1 \\ \vdots \\ x^m]}
    {\V{X}(\V{x})=\mqty[X^1 \\ \vdots \\ X^m](\V{x})
      \in\domD{\V{X}}} \comma
\end{equation}
用它不好直接得到式~\eqref{eq:参数域方程_一阶偏导项} 中的
$\pdv*{x^s}{X^\alpha}$ 项,但获得它的“倒数”
$\pdv*{X^\alpha}{x^s}$ 却很容易,只需利用 \emphA{Jacobi 矩阵}:
\begin{equation}
  \JacobiD{\V{X}}(\V{x})
  \defeq\mqty[
    \displaystyle\pdv{X^1}{\displaystyle x^1} & \cdots &
      \displaystyle\pdv{X^1}{\displaystyle x^m} \\[1ex]
    \vdots & \ddots & \vdots \\[0.5ex]
    \displaystyle\pdv{X^m}{\displaystyle x^1} & \cdots &
      \displaystyle\pdv{X^m}{\displaystyle x^m}
    ]\, (\V{x}) \in\realR^{m \times m} \comma
  \label{eq:参数域方程_Jacobi矩阵}
\end{equation}
它是一个方阵。

有了 Jacobi 矩阵,施加一些手法就可以得到所需要的
$\pdv*{x^s}{X^\alpha}$ 项。考虑到
\begin{equation}
  \forall\,\V{X}\in\domD{\V{X}},\quad
  \V{X}\qty\big(\V{x}(\V{X}))=\V{X} \comma
\end{equation}
并且其中的 $\V{X}(\V{x})$ 和 $\V{x}(\V{X})$ 均可微,可以得到
\begin{equation}
  \JacobiD{\V{X}}\qty\big(\V{x}(\V{X}))
  \cdot \JacobiD{\V{x}}(\V{X})
  =\Mat{I}_m \comma
\end{equation}
其中的 $\Mat{I}_m$ 是单位阵。因此
\begin{equation}
  \JacobiD{\V{x}}(\V{X})
  \defeq\mqty[
    \displaystyle\pdv{x^1}{\displaystyle X^1} & \cdots &
      \displaystyle\pdv{x^1}{\displaystyle X^m} \\[1ex]
    \vdots & \ddots & \vdots \\[0.5ex]
    \displaystyle\pdv{x^m}{\displaystyle X^1} & \cdots &
      \displaystyle\pdv{x^m}{\displaystyle X^m} ]\, (\V{X})
  =(\JacobiD{\V{X}})^{-1}(\V{x})
  =\mqty[
    \displaystyle\pdv{X^1}{\displaystyle x^1} & \cdots &
      \displaystyle\pdv{X^1}{\displaystyle x^m} \\[1ex]
    \vdots & \ddots & \vdots \\[0.5ex]
    \displaystyle\pdv{X^m}{\displaystyle x^1} & \cdots &
      \displaystyle\pdv{X^m}{\displaystyle x^m}
    ]^{-1} (\V{x}) \fullstop
\end{equation}
用代数的方法总可以求出
\begin{equation}
  \varphi^s_\alpha\coloneq\pdv{x^s}{X^\alpha}(\V{X}) \comma
  \label{eq:Jacobi矩阵的元素}
\end{equation}
它是通过求逆运算确定的函数,
即位于矩阵 $\JacobiD{\V{x}}$ 第 $s$ 行第 $\alpha$ 列的元素。这样就有
\begin{equation}
  \pdv{f}{X^\alpha} (\V{X})
  =\sum_{s=1}^{m} \pdv{\tilde{f}}{\displaystyle x^s}
    \qty\big(\V{x}(\V{X})) \cdot
    \varphi^s_\alpha \qty\big(\V{x}(\V{X})) \fullstop
\end{equation}

接下来处理二阶偏导数。由上式,
\begin{align}
  \pdv{f}{X^\beta}{X^\alpha} (\V{X})
  &=\sum_{s=1}^{m} \left[\qty\Bigg(\sum_{k=1}^{m}
      \pdv{\tilde{f}}{\displaystyle x^k}{\displaystyle x^s}
      \qty\big(\V{x}(\V{X})) \cdot \pdv{x^s}{X^\beta} (\V{X}) )
    \cdot \varphi^s_\alpha \qty\big(\V{x}(\V{X}))
    \right. \notag \\*
  &\alspace\left. \phantom{\sum_{s=1}^{m}}+
    \pdv{\tilde{f}}{\displaystyle x^s}
    \qty\big(\V{x}(\V{X})) \cdot \qty\Bigg(\sum_{k=1}^{m}
      \pdv{\displaystyle \varphi^s_\alpha}{\displaystyle x^k}
      \qty\big(\V{x}(\V{X}))
      \cdot \pdv{\displaystyle x^k}{X^\beta} (\V{X}) )
    \right] \notag
  \intertext{继续利用式~\eqref{eq:Jacobi矩阵的元素},有}
  &=\sum_{s=1}^{m} \left[\qty\Bigg(\sum_{k=1}^{m}
      \pdv{\tilde{f}}{\displaystyle x^k}{\displaystyle x^s}
      \qty\big(\V{x}(\V{X})) \cdot
      \varphi^s_\beta \qty\big(\V{x}(\V{X})) )
    \cdot \varphi^s_\alpha \qty\big(\V{x}(\V{X}))
    \right. \notag \\*
  &\alspace\left. \phantom{\sum_{s=1}^{m}}+
    \pdv{\tilde{f}}{\displaystyle x^s}
    \qty\big(\V{x}(\V{X})) \cdot \qty\Bigg(\sum_{k=1}^{m}
      \pdv{\displaystyle \varphi^s_\alpha}{\displaystyle x^k}
      \qty\big(\V{x}(\V{X}))
      \cdot \varphi^k_\beta \qty\big(\V{x}(\V{X})) )
    \right] \fullstop
\end{align}
这样,就把一阶和二阶偏导数项全部用关于 $\V{x}$ 的函数\footnote{
  当然它仍然是 $\V{X}$ 的\emphB{隐}函数:
  $\V{x}=\V{x}(\V{X})$。}表达了出来。
换句话说,我们已经把\emphB{物理域}中 $f$ 关于 $\V{X}$ 的 PDE,
转化成了\emphB{参数域}中 $\tilde{f}$ 关于 $\V{x}$ 的 PDE。
这就是上文要实现的目标。

\subsection{微分同胚的定义}
上文已经指出了 $\domD{\V{x}}$ 到 $\domD{\V{X}}$ 的映照
$\V{X}(\V{x})$ 所需满足的一些条件。这里再次罗列如下:

\begin{myEnum}
\item $\domD{\V{X}},\,\domD{\V{x}}\subset\Rm$
均为\emphA{开集}\footnote{
  用形象化的语言来说,如果在区域中的任意一点都可以吹出一个球,
  并能使球上的每个点都落在区域内,那么这个区域就是\emphA{开集}。
  这是\emphB{复合映照可微性定理}的一个要求。};

\item 存在 $\domD{\V{x}}$ 同 $\domD{\V{X}}$ 之间的\emphA{双射}
$\V{X}(\V{x})$,即存在\emphA{一一对应}关系;

\item $\V{X}(\V{x})$ 和它的逆映照 $\V{x}(\V{X})$
满足一定的\emphA{正则性}要求。
\end{myEnum}

\myPROBLEM{对第3点要稍作说明。}

如果满足这三点,则称 $\V{X}(\V{x})$ 为 $\domD{\V{x}}$ 与
$\domD{\V{X}}$ 之间的 \emphA{$\DiffMorp$-微分同胚},
记为 $\V{X}(\V{x})\in\cf{\domD{\V{x}}}{\domD{\V{X}}}$。
把物理域中的一个部分对应到参数域上的一个部分,
需要的仅仅是\emphB{双射}这一条件;而要使得物理域中所满足的
PDE 能够转换到参数域上,就需要“过去”和“回来”
都满足 $p$ 阶偏导数连续的条件(即\emphB{正则性}要求)。

有了微分同胚,物理域中的位置就可用参数域中的位置等价地进行刻画。
因此我们也把微分同胚称为\emphA{曲线坐标系}。

\section{向量值映照的可微性} \label{sec:向量值映照的可微性}
\subsection{可微性的定义}
设 $\V{x}_0$ 是参数域 $\domD{\V{x}}$ 中的一个内点。
在映照 $\V{X}(\V{x})$ 的作用下,
它对应到物理域 $\domD{\V{X}}$ 中的点 $\V{X}\qty(\V{x}_0)$。
参数域是一个\emphB{开集}。根据开集的定义,
必然存在一个实数 $\lambda>0$,使得以 $\V{x}_0$ 为球心、
$\lambda$ 为半径的球能够完全落在定义域 $\domD{\V{x}}$ 内,即
\begin{equation}
  \domB{\lambda}{\V{x}_0}\subset\domD{\V{x}} \comma
\end{equation}
其中的 $\domB{\lambda}{\V{x}_0}$ 表示 $\V{x}_0$ 的 $\lambda$ 邻域。

如果 $\exists\,\JacobiD{\V{X}}\qty(\V{x}_0)\in\LinearT{\Rm}{\Rm}$
\footnote{正如之前已经定义的,$\JacobiD{\V{X}}$
  已经用来表示 Jacobi 矩阵。这里还是请先暂时将它视为一种记号,
  其具体形式将在下一小节给出。},满足
\begin{equation}
  \forall\,\V{x}_0+\V{h}\in\domB{\lambda}{\V{x}_0},\quad
  \V{X}\qty(\V{x}_0+\V{h})-\V{X}\qty(\V{x}_0)
  =\JacobiD{\V{X}}\qty(\V{x}_0)(\V{h})+\sO{\norm{\V{h}}}
  \in\Rm \comma
  \label{eq:向量值映照可微性的定义}
\end{equation}
则称向量值映照 $\V{X}(\V{x})$ 在 $\V{x}_0$ 点\emphA{可微}。
其中,$\LinearT{\Rm}{\Rm}$ 表示从 $\Rm$ 到 $\Rm$
的\emphA{线性变换}全体。

根据这个定义,所谓\emphB{可微性},指由自变量变化所引起的因变量变化,
可以用一个\emphB{线性变换}近似,而误差为\emphB{一阶}无穷小量。
自变量可见到因变量空间最简单的映照形式就是线性映照(线性变换),
因而具有可微性的向量值映照具有至关重要的作用。

\subsection{Jacobi 矩阵}
下面我们研究 $\JacobiD{\V{X}}\qty(\V{x}_0)\in\LinearT{\Rm}{\Rm}$
的表达形式。由于 $\V{h}\in\Rm$,所以
\begin{equation}
  \V{h}=\mqty[h^1 \\ \vdots \\ h^m]
  =h^1\V{e}_1+\cdots+h^i\V{e}_i+\cdots+h^m\V{e}_m \fullstop
\end{equation}
另一方面,$\JacobiD{\V{X}}\qty(\V{x}_0)\in\LinearT{\Rm}{\Rm}$
具有\emphB{线性性}:
\begin{equation}
  \forall\,\alpha,\,\beta \in\realR
    \text{\ 和\ } \tilde{\V{h}},\,\hat{\V{h}}\in\Rm,\quad
  \JacobiD{\V{X}}\qty(\V{x}_0)
    \qty(\alpha\tilde{\V{h}}+\beta\hat{\V{h}})
  =\alpha \JacobiD{\V{X}}\qty(\V{x}_0)\qty(\tilde{h})
    +\beta \JacobiD{\V{X}}\qty(\V{x}_0)\qty(\hat{h}) \fullstop
\end{equation}
这样就有
\begin{align}
  \JacobiD{\V{X}}\qty(\V{x}_0)\qty(\V{h})
  &=\JacobiD{\V{X}}\qty(\V{x}_0)
    \qty(h^1\V{e}_1+\cdots+h^i\V{e}_i+\cdots+h^m\V{e}_m) \notag \\
  &=h^1\JacobiD{\V{X}}\qty(\V{x}_0)\qty(\V{e}_1) + \cdots
    +h^i\JacobiD{\V{X}}\qty(\V{x}_0)\qty(\V{e}_i) + \cdots
    +h^m\JacobiD{\V{X}}\qty(\V{x}_0)\qty(\V{e}_m)
    \label{eq:推导Jacobi矩阵表达形式_Part1}
  \intertext{注意到 $h^i\in\realR$ 以及
    $\JacobiD{\V{X}}\qty(\V{x}_0)\qty(\V{e}_i)\in\Rm$,
    因而该式可以用矩阵形式表述:}
  &=\mqty[\JacobiD{\V{X}}\qty(\V{x}_0)\qty(\V{e}_1),\,\cdots,\,
      \JacobiD{\V{X}}\qty(\V{x}_0)\qty(\V{e}_m)]
    \mqty[h^1 \\ \vdots \\ h^m] \fullstop
\end{align}
最后一步要用到\emphB{分块矩阵}的思想:左侧的矩阵为 1“行” $m$ 列,
每一“行”是一个 $m$ 维列向量;右侧的矩阵(向量)则为 $m$ 行 1 列。
两者相乘,得到 1“行” 1 列的矩阵(当然实际为 $m$ 行),
即之前的 \eqref{eq:推导Jacobi矩阵表达形式_Part1}~式。
在线性代数中,$m \times m$ 的矩阵
$\mqty[\JacobiD{\V{X}}\qty(\V{x}_0)\qty(\V{e}_1) & \cdots
& \JacobiD{\V{X}}\qty(\V{x}_0)\qty(\V{e}_m)]$
通常称为\emphA{变换矩阵}(也叫\emphA{过渡矩阵})。

接下来要搞清楚变换矩阵的具体形式。取
\begin{equation}
  \V{h}=\mqty[0,\,\cdots,\,\lambda,\,\cdots,\,0]\trans
  =\lambda\,\V{e}_i\in\Rm \comma
\end{equation}
即除了 $\V{h}$ 的第 $i$ 个元素为 $\lambda$ 外,其余元素均为 0
($\lambda \neq 0$)。因而有 $\norm{\V{h}}=\lambda$。
代入可微性的定义 \eqref{eq:向量值映照可微性的定义}~式,可得
\begin{align}
  &\alspace\V{X}\qty(\V{x}_0+\V{h})-\V{X}\qty(\V{x}_0)
  =\V{X}\qty(\V{x}_0+\lambda\,\V{e}_i)
    -\V{X}\qty(\V{x}_0) \notag \\
  &=\mqty[\JacobiD{\V{X}}\qty(\V{x}_0)\qty(\V{e}_1),\,\cdots,\,
      \JacobiD{\V{X}}\qty(\V{x}_0)\qty(\V{e}_i),\,\cdots,\,
      \JacobiD{\V{X}}\qty(\V{x}_0)\qty(\V{e}_m)]
    \mqty[0,\,\cdots,\,\lambda,\,\cdots,\,0]\trans
    +\sO{\lambda} \notag \\
  &=\lambda\cdot\JacobiD{\V{X}}\qty(\V{x}_0)\qty(\V{e}_i)
    +\sO{\lambda} \fullstop
\end{align}
由于 $\lambda$ 是非零实数,故可以在等式两边同时除以
$\lambda$ 并取极限:
\begin{equation}
  \lim_{\lambda\to 0}
  \frac{\V{X}\qty(\V{x}_0+\lambda\,\V{e}_i)
    -\V{X}\qty(\V{x}_0)}{\lambda}
  =\JacobiD{\V{X}}\qty(\V{x}_0)\qty(\V{e}_i) \comma
\end{equation}
这里的 $\sO{\lambda}$ 根据其定义自然趋于 0。
该式左侧极限中的分子部分,是自变量 $\V{x}$ 第 $i$ 个分量的变化
所引起因变量的变化;而分母,则是自变量第 $i$ 个分量的变化大小。
我们引入下面的记号:
\begin{equation}
  \pdv{\V{X}}{x^i}\qty(\V{x}_0)
  \coloneq \lim_{\lambda\to 0}
  \frac{\V{X}\qty(\V{x}_0+\lambda\,\V{e}_i)
    -\V{X}\qty(\V{x}_0)}{\lambda} \in\Rm \comma
\end{equation}
它表示因变量 $\V{X}\in\Rm$ 作为一个\emphB{整体},
相对于自变量 $\V{x}\in\Rm$ 第 $i$ 个\emphB{分量} $x^i\in\realR$
的“变化率”,即 $\V{X}$ 关于 $x^i$(在 $\V{x}_0$ 处)%
的\emphA{偏导数}。由于我们没有定义向量的除法,
因此自变量作为\emphB{整体}所引起因变量的变化,是没有意义的。
利用偏导数的定义,可有
\begin{align}
  &\alspace\mqty[\JacobiD{\V{X}}\qty(\V{x}_0)\qty(\V{e}_1),\,\cdots,
    \,\JacobiD{\V{X}}\qty(\V{x}_0)\qty(\V{e}_i),\,\cdots,\,
    \JacobiD{\V{X}}\qty(\V{x}_0)\qty(\V{e}_m)] \notag \\
  &=\mqty[\displaystyle \pdv{\V{X}}{x^1}\qty(\V{x}_0),\,\cdots,\,
    \displaystyle \pdv{\V{X}}{x^i}\qty(\V{x}_0),\,\cdots,\,
    \displaystyle \pdv{\V{X}}{x^m}\qty(\V{x}_0)]
    \in\realR^{m \times m} \fullstop
\end{align}

\blankline

下面给出 $\pdv*{\V{X}}{x^i}\qty(\V{x}_0)$ 的计算式。根据定义,有
\begin{align}
  \pdv{\V{X}}{x^i}\qty(\V{x}_0)
  &\coloneq \lim_{\lambda\to 0}
    \frac{\V{X}\qty(\V{x}_0+\lambda\,\V{e}_i)
    -\V{X}\qty(\V{x}_0)}{\lambda} \in\Rm \notag \\
  &=\lim_{\lambda\to 0} \frac{1}{\lambda}\cdot
    \left(\vphantom{\mqty{0\\[0.8ex]0}}\right.\!
      \mqty[X^1 \\ \vdots \\ X^m] \qty(\V{x}_0+\lambda\,\V{e}_i)
      -\mqty[X^1 \\ \vdots \\ X^m] \qty(\V{x}_0)
    \!\!\left.\vphantom{\mqty{0\\[0.8ex]0}}\right) \notag \\
  &=\lim_{\lambda\to 0}\,
    \mqty[
      \dfrac{X^1\qty(\V{x}_0+\lambda\,\V{e}_i)-X^1\qty(\V{x}_0)}
        {\lambda} \\[1ex] \vdots \\[0.5ex]
      \dfrac{X^m\qty(\V{x}_0+\lambda\,\V{e}_i)-X^m\qty(\V{x}_0)}
        {\lambda} ] \fullstop
\end{align}
向量极限存在的\emphB{充要条件}是各分量极限均存在,即存在
\begin{equation}
  \pdv{X^\alpha}{x^i}\qty(\V{x}_0) \coloneq
  \lim_{\lambda\to 0}
  \frac{X^\alpha\qty(\V{x}_0+\lambda\,\V{e}_i)-X^\alpha\qty(\V{x}_0)}
    {\lambda\vphantom{X^1\qty(\V{x}_0)}}\in\realR \comma
\end{equation}
其中的 $\alpha=1,\,\cdots,\,m$。
这其实就是我们熟知的\emphB{多元函数}偏导数的定义。
用它来表示\emphB{向量值映照}的偏导数,可有
\begin{equation}
  \pdv{\V{X}}{x^i}\qty(\V{x}_0)
  =\mqty[\displaystyle \pdv{X^1}{x^i}\qty(\V{x}_0) \\[1ex]
    \vdots \\[0.7ex] \displaystyle \pdv{X^m}{x^i}\qty(\V{x}_0)]
  =\sum_{\alpha=1}^{m} \pdv{X^\alpha}{x^i}\qty(\V{x}_0) \,
    \V{e}_\alpha \fullstop
\end{equation}

向量值映照 $\V{X}$ 关于 $x^i$ 的偏导数,
从代数的角度来看,是 Jacobi 矩阵的第 $i$ 列;
从几何的角度来看,则是物理域中 $x^i$ 线的切向量;
从计算的角度来看,又是(该映照)每个分量偏导数的组合。

\blankline

现在我们重新回到 Jacobi 矩阵。情况已经十分明了:
只需把之前获得的各列并起来,就可以得到完整的 Jacobi 矩阵。于是
\begin{align}
  \JacobiD{\V{X}}\qty(\V{x}_0)\qty(\V{h})
  &=\mqty[\displaystyle \pdv{\V{X}}{x^1},\,\cdots,\,\pdv{\V{X}}{x^m}]
    \qty(\V{x}_0)\qty(\V{h}) \notag \\
  &=\,\mqty[
    \displaystyle\pdv{X^1}{\displaystyle x^1} & \cdots &
      \displaystyle\pdv{X^1}{\displaystyle x^m} \\[1ex]
    \vdots & \ddots & \vdots \\[0.5ex]
    \displaystyle\pdv{X^m}{\displaystyle x^1} & \cdots &
      \displaystyle\pdv{X^m}{\displaystyle x^m}
    ]\, \qty(\V{x}_0) \cdot
    \mqty[h^1 \\ \vdots \\ h^m] \fullstop
\end{align}
这与 \ref{subsec:参数域方程}~小节中
\eqref{eq:参数域方程_Jacobi矩阵}~式给出的定义是完全一致的。

\subsection{偏导数的几何意义} \label{subsec:偏导数的几何意义}
这一小节中,我们要回过头来,澄清向量值映照偏导数的几何意义。

如图~\ref{fig:偏导数的几何意义},$\V{X}(\V{x})$ 是定义域空间
$\domD{\V{x}}\subset\Rm$ 到值域空间 $\domD{\V{X}}\subset\Rm$
的向量值映照。在定义域空间 $\domD{\V{x}}$ 中,
过点 $\V{x}_0$ 作一条平行于 $x^i$ 轴的直线,称为 \emphB{$x^i$-线}。
$x^i$ 轴定义了向量 $\V{e}_i$,因而 $x^i$-线上的任意一点均可表示为
$\V{x}_0+\lambda\,\V{e}_i$,其中 $\lambda\in\realR$。

\begin{figure}[h]
  \centering
  \includegraphics{images/vector-value-mapping.png}
  \caption{向量值映照偏导数的几何意义}
  \label{fig:偏导数的几何意义}
\end{figure}

在 $\V{X}(\V{x})$ 的作用下,点 $\V{x}_0$ 被映照到
$\V{X}\qty(\V{x}_0)$,而 $\V{x}_0+\lambda\,\V{e}_i$ 则被映照到了
$\V{X}\qty(\V{x}_0+\lambda\,\V{e}_i)$。这样一来,
$x^i$-线也就被映照到了值域空间 $\domD{\V{X}}$ 中,成为一条曲线。

根据前面的定义,当 $\lambda\to 0$ 时,
\begin{equation}
  \frac{\V{X}\qty(\V{x}_0+\lambda\,\V{e}_i) - \V{X}\qty(\V{x}_0)}
  {\lambda} \to \pdv{\V{X}}{x^i}\qty(\V{x}_0) \fullstop
\end{equation}
对应到图~\ref{fig:偏导数的几何意义} 中,就是 $x^i$-线
(值域空间中)在 $\V{X}\qty(\V{x}_0)$ 处的\emphA{切向量}。

完全类似,在定义域空间 $\domD{\V{x}}$ 中,过点 $\V{x}_0$
作出 \emphB{$x^j$-线}(自然是平行于 $x^j$ 轴),
其上的点可以表示为 $\V{x}_0+\lambda\,\V{e}_j$。
映射到值域空间 $\domD{\V{X}}$ 上,
则成为 $\V{X}\qty(\V{x}_0+\lambda\,\V{e}_j)$。很显然,
\begin{equation}
  \pdv{\V{X}}{x^j}\qty(\V{x}_0)
  =\frac{\V{X}\qty(\V{x}_0+\lambda\,\V{e}_j) - \V{X}\qty(\V{x}_0)}
  {\lambda}
\end{equation}
就是 $x^j$-线在 $\V{X}\qty(\V{x}_0)$ 处的切向量。在定义域空间中,
$x^i$-线作为直线共有 $m$ 条,它们之间互相垂直。作用到值域空间后,
这样的 $x^i$-线尽管变为了曲线,但仍为 $m$ 条。相应的切向量,
自然也有 $m$ 个。

\section{局部基} \label{sec:局部基}
这里的讨论基于曲线坐标系(即微分同胚)
$\V{X}(\V{x})\in\cf{\domD{\V{x}}}{\domD{\V{X}}}$。

\subsection{局部协变基} \label{subsec:局部协变基}
我们已经知道,$\V{X}(\V{x})$ 的 Jacobi 矩阵可以表示为
\begin{equation}
  \JacobiD{\V{X}}(\V{x})
  =\mqty[\displaystyle \pdv{\V{X}}{x^1},\,\cdots,\,
    \displaystyle \pdv{\V{X}}{x^i},\,\cdots,\,
    \displaystyle \pdv{\V{X}}{x^m}] (\V{x})
    \in\realR^{m \times m} \comma
\end{equation}
式中的
\begin{equation}
  \pdv{\V{X}}{x^i} (\V{x})
  =\lim_{\lambda\to 0}
    \frac{\V{X}\qty(\V{x}+\lambda\,\V{e}_i) - \V{X}(\V{x})}
    {\lambda} \fullstop
  \label{eq:局部基_偏导数}
\end{equation}
在参数域 $\domD{\V{x}}$ 中作出 $x^i$-线。映照到物理域后,
它变成一条曲线,我们仍称之为 $x^i$-线。
\ref{subsec:偏导数的几何意义}~小节已经说明,
\eqref{eq:局部基_偏导数}~式表示物理域中 $x^i$-线的\emphB{切向量}。
在张量分析中,我们通常把它记作 $\V{g}_i(\V{x})$。

由于微分同胚要求是\emphB{双射},因而 Jacobi 矩阵
\begin{equation}
  \JacobiD{\V{X}}(\V{x})
  =\mqty[\V{g}_1,\,\cdots,\,\V{g}_i,\,\cdots,\,\V{g}_m](\V{x})
  \in\realR^{m \times m}
\end{equation}
必须是\emphB{非奇异}的。这等价于
\begin{equation}
  \qty{\V{g}_i(\V{x})=\pdv{\V{X}}{x^i} (\V{x})}^m_{i=1}
  \subset\Rm
\end{equation}
\emphB{线性无关}。由此,它们可以构成 $\Rm$ 上的一组\emphA{基}。

用任意的 $\V{x}\in\domD{\V{x}}$ 均可构建一组基。
但选取不同的 $\V{x}$,将会使所得基的取向有所不同。
因而这种基称为\emphA{局部协变基}。和之前一样,
我们用“协变”表示指标在下方。

\subsection{局部逆变基;对偶关系}
有了局部协变基 $\qty{\V{g}_i(\V{x})}^m_{i=1}$,根据
\ref{subsec:对偶基}~小节中的讨论,
必然唯一存在与之对应的\emphA{局部逆变基}
$\qty{\V{g}^i(\V{x})}^m_{i=1}$,满足
\begin{equation}
  \mqty[\V{g}^1(\V{x}),\,\cdots,\,\V{g}^m(\V{x})]\trans
    \mqty[\V{g}_1(\V{x}),\,\cdots,\,\V{g}_m(\V{x})]
  =\mqty[\qty(\V{g}^1)\trans \\ \vdots \\ \qty(\V{g}^m)\trans]\,
    (\V{x}) \cdot \JacobiD{\V{X}}(\V{x})
  =\Mat{I}_m \fullstop
\end{equation}

下面我们来寻找逆变基 $\qty{\V{g}^i(\V{x})}^m_{i=1}$ 的具体表示。
考虑到\footnote{
  这里的几步推导在 \ref{subsec:参数域方程}~小节中也有所涉及。}
\begin{equation}
  \V{X}\qty\big(\V{x}(\V{X}))=\V{X}\in\Rm \comma
\end{equation}
并利用复合映照可微性定理,可知
\begin{equation}
  \JacobiD{\V{X}}\qty\big(\V{x}(\V{X}))
  \cdot \JacobiD{\V{x}}(\V{X})
  =\Mat{I}_m \comma
  \label{eq:局部逆变基_两个Jacobi矩阵互逆}
\end{equation}
即有
\begin{equation}
  \JacobiD{\V{x}}(\V{X})
  =(\JacobiD{\V{X}})^{-1}\qty\big(\V{x}(\V{X})) \fullstop
\end{equation}
于是
\begin{equation}
  \mqty[\qty(\V{g}^1)\trans \\ \vdots \\ \qty(\V{g}^m)\trans]\,
    (\V{x})
  =(\JacobiD{\V{X}})^{-1}(\V{x})
  =\JacobiD{\V{x}}(\V{X})
  =\mqty[
    \displaystyle\pdv{x^1}{\displaystyle X^1} & \cdots &
      \displaystyle\pdv{x^1}{\displaystyle X^m} \\[1ex]
    \vdots & \ddots & \vdots \\[0.5ex]
    \displaystyle\pdv{x^m}{\displaystyle X^1} & \cdots &
      \displaystyle\pdv{x^m}{\displaystyle X^m} ]\,(\V{X}) \fullstop
\end{equation}
这样我们就得到了局部逆变基的具体表示(注意转置):
\begin{equation}
  \V{g}^i(\V{x})
  =\mqty[\displaystyle \pdv{x^i}{X^1} \\[1ex]
    \vdots \\[0.5ex] \displaystyle \pdv{x^i}{X^m}]\,(\V{X})
  =\sum_{\alpha=1}^{m} \pdv{x^i}{X^\alpha} (\V{X})\,\V{e}_\alpha
  \fullstop
\end{equation}
定义标量场 $f(\V{x})$ 的\emphA{梯度}为
\begin{equation}
  \nabla f(\V{x}) \defeq
  \sum_{\alpha=1}^{m} \pdv{f}{x^\alpha} (\V{x})\,\V{e}_\alpha \comma
\end{equation}
则局部逆变基又可以表示成
\begin{equation}
  \V{g}^i(\V{x})=\nabla x^i(\V{X}) \fullstop
\end{equation}
此处的梯度实际上就是我们熟知的三维情况在 $m$ 维下的推广。

\blankline

在 \ref{subsec:偏导数的几何意义}~小节中已经指出,
局部协变基的几何意义是 $x^i$-线的\emphB{切向量}。
现在,我们来讨论局部逆变基的几何意义。

\begin{figure}[h]
  \centering
  \includegraphics{images/local-basis.png}
  \caption{局部逆变基的几何意义}
  \label{fig:局部逆变基的几何意义}
\end{figure}

如图~\ref{fig:局部逆变基的几何意义} 所示,在参数空间中,
过点 $\V{x}$ 作垂直于 $x^i$ 轴的平面,记为 $x^i$-面。
在 $x^i$-面上,自然有 $x^i=\const$ \ 映照到物理空间后,
$x^i$-面变为一个曲面,其上仍有 $x^i(\V{X})=\const$,
即它是一个\emphB{等值面}。
等值面的梯度方向显然与该曲面的\emphB{法向}相同。因此,局部逆变基
$\V{g}^i(\V{x})$ 的几何意义就是 $x^i$-面的\emphA{法向量}。

现在来验证一下\emphA{对偶关系}。
\begin{align}
  \ipb{\V{g}_i(\V{x})}{\V{g}^j(\V{x})}
  &=\ipb{\pdv{\V{X}}{x^i} (\V{x})}{\nabla x^j(\V{X})} \notag \\
  &=\ipb{\sum_{\alpha=1}^{m} \pdv{X^\alpha}{x^i} (\V{x})\,
      \V{e}_\alpha}
    {\sum_{\beta=1}^{m} \pdv{x^j}{X^\beta} (\V{X})\,
      \V{e}_\beta} \notag
  \intertext{利用内积的线性性,有}
  &=\sum_{\alpha=1}^{m} \sum_{\beta=1}^{m}
    \pdv{X^\alpha}{x^i} (\V{x}) \pdv{x^j}{X^\beta} (\V{X})
    \cdot \ipb{\V{e}_\alpha}{\V{e}_\beta} \notag \\
  &=\sum_{\alpha=1}^{m} \sum_{\beta=1}^{m}
    \pdv{X^\alpha}{x^i} (\V{x}) \pdv{x^j}{X^\beta} (\V{X})
    \cdot\KroneckerDelta*{\alpha\beta} \notag
  \intertext{合并掉指标 $\beta$,可得}
  &=\sum_{\alpha=1}^{m}
    \pdv{X^\alpha}{x^i} (\V{x})
    \pdv{x^j}{X^\alpha} (\V{X}) \notag \\
  &=\sum_{\alpha=1}^{m}
    \pdv{x^j}{X^\alpha} (\V{X})
    \pdv{X^\alpha}{x^i} (\V{x}) \fullstop
\end{align}
最后一步求和号中的第一项位于 Jacobi 矩阵 $\JacobiD{\V{x}}(\V{X})$
的第 $j$ 行第 $\alpha$ 列,而第二项位于 $\JacobiD{\V{X}}(\V{x})$
的第 $\alpha$ 行第 $i$ 列,因此关于 $\alpha$
的求和结果便是乘积矩阵的第 $j$ 行第 $i$ 列。
根据式~\eqref{eq:局部逆变基_两个Jacobi矩阵互逆},
这两个 Jacobi 矩阵的乘积为单位阵,所以有
\begin{equation}
  \ipb{\V{g}_i(\V{x})}{\V{g}^j(\V{x})}
  =\KroneckerDelta{j}{i} \fullstop
\end{equation}

\blankline

总结一下我们得到的结果。对于体积形态的连续介质,存在着
\begin{braceEq}
  \text{局部协变基:}\quad \qty{\V{g}_i(\V{x})
    \defeq \pdv{\V{X}}{x^i} (\V{x})}^m_{i=1} \comma \notag \\
  \text{局部逆变基:}\quad \qty{\V{g}^i(\V{x})
    \defeq \nabla x^i(\V{X})}^m_{i=1} \comma \notag
\end{braceEq}
它们满足\emphB{对偶关系}
\begin{equation}
  \ipb{\V{g}_i(\V{x})}{\V{g}^j(\V{x})}
  =\KroneckerDelta{j}{i} \fullstop
  \label{eq:局部基_对偶关系}
\end{equation}
这样,在研究连续介质中的一个点时,我们就有三种基可以使用:
局部协变基、局部逆变基,
当然还有\emphB{典则基} $\qty{\V{e}_i}^m_{i=1}$。

\section{标架运动方程}
\subsection{向量在局部基下的表示}
对于 $\Rm$ 空间中的任意一个向量 $\V{b}$,
它可以用\emphA{典则基}表示:
\begin{equation}
  \V{b}=\sum_{\alpha=1}^{m} b_\alpha \V{e}_\alpha
  =b_\alpha \V{e}_\alpha \fullstop
\end{equation}
第二步省略掉了求和号,这是根据\emphA{Einstein 求和约定}:
指标出现两次,则表示对它求和。\footnote{
  在 \ref{subsec:度量}~小节中,还要求重复指标一上一下。
  典则基不分协变、逆变,标号均在下方,可以视为一个特例。}
根据之前一小节的结论,$\V{b}$ 还可以用局部协变基和局部逆变基来表示:
\begin{align}
  \V{b} = b^i\V{g}_i(\V{x}) = b_j\V{g}^j(\V{x}) \comma
\end{align}
式中,
\begin{mySubEq}
  \begin{align}
    b^i&=\ipb{\V{b}}{\V{g}^i(\V{x})} \label{eq:活动标架_逆变分量表示}
    \intertext{和}
    b_j&=\ipb{\V{b}}{\V{g}_j(\V{x})} \label{eq:活动标架_协变分量表示}
  \end{align}
\end{mySubEq}
分别称为向量 $b$ 的\emphA{逆变分量}和\emphA{协变分量}。
注意,这里同样用到了 Einstein 求和约定。

将 $\V{b}=b^i\V{g}_i(\V{x})$ 的两边分别与
$\V{g}^j(\V{x})$ 作内积,可有
\begin{align}
  \ipb{\V{b}}{\V{g}^j(\V{x})}
  &=\ipb{b^i\V{g}_i(\V{x})}{\V{g}^j(\V{x})} \notag
  \intertext{利用内积的线性性,提出系数:}
  &=b^i\ipb{\V{g}_i(\V{x})}{\V{g}^j(\V{x})} \notag
  \intertext{利用对偶关系 \eqref{eq:局部基_对偶关系}~式,可有}
  &=b^i\KroneckerDelta{j}{i}=b^j \comma
\end{align}
这就得到了逆变分量的表示式~\eqref{eq:活动标架_逆变分量表示}。
同理,将 $\V{b}=b_j\V{g}^j(\V{x})$ 的两边分别与
$\V{g}_i(\V{x})$ 作内积,就有
\begin{equation}
  \ipb{\V{b}}{\V{g}_i(\V{x})}
  =\ipb{b_j\V{g}^j(\V{x})}{\V{g}_i(\V{x})}
  =b_j\ipb{\V{g}^j(\V{x})}{\V{g}_i(\V{x})}
  =b_j\KroneckerDelta{j}{i}=b_i \comma
\end{equation}
这便是协变分量的表示 \eqref{eq:活动标架_协变分量表示}~式。

\subsection{局部基的偏导数}
所谓局部基(或曰“活动标架”),顾名思义,它在不同的点上往往是不同的。
根据之前的定义,我们有
\begin{mySubEq}
  \begin{align}
  &\mmap{\V{g}_i(\V{x})}{\domD{\V{x}}\ni\V{x}}{\V{g}_i(\V{x})
    =\mqty[\displaystyle \pdv{X^1}{x^i} \\[1ex] \vdots \\[0.7ex]
      \displaystyle \pdv{X^m}{x^i}]\,(\V{x})\in\Rm} \comma
    \label{eq:局部协变基定义} \\
  &\mmap{\V{g}^i(\V{x})}{\domD{\V{x}}\ni\V{x}}{\V{g}^i(\V{x})
    =\mqty[\displaystyle \pdv{x^i}{X^1} \\[1ex] \vdots \\[0.5ex]
      \displaystyle \pdv{x^i}{X^m}]\,\qty\big(\V{X}(\V{x}))\in\Rm}
    \fullstop
    \label{eq:局部逆变基定义}
  \end{align}
\end{mySubEq}
从\emphB{映照}的角度来看,局部基定义了新的向量值映照,
其定义域仍为参数域,而值域则为 $\Rm$ 空间。这样一来,
我们在 \ref{sec:向量值映照的可微性}~节中所引入的操作均可完全
类似地应用在局部基上。例如,我们可以来求局部基的 Jacobi 矩阵:
\begin{braceEq}
  \JacobiD{\V{g}_i}(\V{x})
  &=\mqty[\displaystyle \pdv{\V{g}_i}{x^1},\,\cdots,\,
    \pdv{\V{g}_i}{x^m}]\,(\V{x}) \in\realR^{m \times m} \comma \\
  \JacobiD{\V{g}^i}(\V{x})
  &=\mqty[\displaystyle \pdv{\V{g}^i}{x^1},\,\cdots,\,
    \pdv{\V{g}^i}{x^m}]\,(\V{x}) \in\realR^{m \times m} \fullstop
\end{braceEq}
Jacobi 矩阵中的每一列都是局部基作为整体相对自变量第 $j$
个分量的变化率,即\emphB{偏导数}:
\begin{braceEq}
  \pdv{\V{g}_i}{x^j} (\V{x}) &\defeq \lim_{\lambda\to 0}
    \frac{\V{g}_i\qty(\V{x}+\lambda\,\V{e}_j)-\V{g}_i(\V{x})}
      {\lambda} \in\Rm \comma \\
  \pdv{\V{g}^i}{x^j} (\V{x}) &\defeq \lim_{\lambda\to 0}
    \frac{\V{g}^i\qty(\V{x}+\lambda\,\V{e}_j)-\V{g}^i(\V{x})}
      {\lambda} \in\Rm \fullstop
\end{braceEq}

下面澄清局部基偏导数的几何意义。
如图~\ref{fig:局部基偏导数的几何意义} 所示,在参数空间中,
过点 $\V{x}$ 作出 $x^j$-线,并在其上取点 $\V{x}+\lambda\,\V{e}_j$。
分别过点 $\V{x}$ 和 $\V{x}+\lambda\,\V{e}_j$ 作出 $x^i$-线,于是
$\pdv*{\V{g}_i}{x^j} (\V{x})$ 就表示 $\V{g}_i(\V{x})$
(即 $x^i$-线的切向量)沿 $x^j$-线的变化率。
同理,过点 $\V{x}$ 和 $\V{x}+\lambda\,\V{e}_j$ 作出 $x^i$-面,
则 $\pdv*{\V{g}^i}{x^j} (\V{x})$ 就表示 $\V{g}^i(\V{x})$
(即 $x^i$-面的法向量)沿 $x^j$-线的变化率。

\begin{figure}[h]
  \centering
  \includegraphics{images/local-basis-pdv-1.png}
  \includegraphics{images/local-basis-pdv-2.png}
  \caption{局部基偏导数的几何意义}
  \label{fig:局部基偏导数的几何意义}
\end{figure}

\subsection{Christoffel 符号}
\label{subsec:Christoffel符号}
考察 $\pdv*{\V{g}_i}{x^j} (\V{x})$,
即\emphB{协变基}的偏导数\footnote{
  以下在不引起歧义之处,将省略局部协变基、
  局部逆变基的“局部”二字。为了方便,$\V{g}_i(\V{x})$ 和
  $\V{g}^i(\V{x})$ 中的“$(\V{x})$”有时也会省略。
}。它是 $\Rm$ 空间中的一个向量,因而可以用协变基或逆变基来表示:
\begin{braceEq*}
  {\label{eq:协变基偏导数的协变与逆变表示} \pdv{\V{g}_i}{x^j} (\V{x})=}
  &\ipb{\pdv{\V{g}_i}{x^j}}{\V{g}^k} \V{g}_k \comma \\
  &\ipb{\pdv{\V{g}_i}{x^j}}{\V{g}_k} \V{g}^k \fullstop
\end{braceEq*}
引入\emphA{第一类 Christoffel 符号}
\begin{equation}
  \ChrA{j}{i}{k} \defeq \ipb{\pdv{\V{g}_i}{x^j}}{\V{g}_k}
  \label{eq:第一类Christoffel符号定义}
\end{equation}
和\emphA{第二类 Christoffel 符号}
\begin{equation}
  \ChrB{j}{i}{k} \defeq \ipb{\pdv{\V{g}_i}{x^j}}{\V{g}^k}
  \comma
  \label{eq:第二类Christoffel符号定义}
\end{equation}
则式~\eqref{eq:协变基偏导数的协变与逆变表示} 可以写成
\begin{braceEq*}
  {\pdv{\V{g}_i}{x^j} (\V{x})=}
  &\ChrB{j}{i}{k}\, \V{g}_k \comma \\
  &\ChrA{j}{i}{k}\, \V{g}^k \fullstop
\end{braceEq*}

下面我们来探讨 Christoffel 符号的基本性质——指标 $i$、
$j$ 可以交换:
\begin{braceEq}
  &\ChrB{j}{i}{k}=\ChrB{i}{j}{k} \comma
  \label{eq:第二类Christoffel符号指标交换} \\
  &\ChrA{j}{i}{k}=\ChrA{i}{j}{k} \fullstop
  \label{eq:第一类Christoffel符号指标交换}
\end{braceEq}

\begin{myProof}
根据定义 \eqref{eq:第一类Christoffel符号定义} 和
\eqref{eq:第二类Christoffel符号定义}~式,指标 $i$、$j$
来源于协变基的偏导数 $\pdv*{\V{g}_i}{x^j} (\V{x})$。
只要偏导数中的 $i$、$j$ 可以交换,Christoffel 符号中的指标 $i$、
$j$ 自然也可以。回顾协变基的定义\eqref{eq:局部协变基定义}~式:
\begin{equation}
  \V{g}_i (\V{x})
  \defeq\mqty[\displaystyle\pdv{X^1}{x^i} \\[1ex]
    \vdots \\[0.7ex] \displaystyle\pdv{X^m}{x^i}]\,(\V{x})
  \fullstop
\end{equation}
其偏导数为
\begin{equation}
  \pdv{\V{g}_i}{x^j} (\V{x})
  =\mqty[\displaystyle\pdv{X^1}{x^j}{x^i} \\[1ex]
    \vdots \\[0.7ex] \displaystyle\pdv{X^m}{x^j}{x^i}]\,(\V{x})
  =\mqty[\displaystyle\pdv{X^1}{x^i}{x^j} \\[1ex]
    \vdots \\[0.7ex] \displaystyle\pdv{X^m}{x^i}{x^j}]\,(\V{x})
  =\pdv{\V{g}_j}{x^i} (\V{x}) \fullstop
\end{equation}
注意第二个等号处交换了偏导数的次序,
其条件是\emphB{二阶}偏导数均存在且连续。
只要微分同胚达到了 $\DiffMorp[2]$,就可以满足该要求,
在一般的物理情境这都是成立的。于是我们便完成了证明。
\end{myProof}

现在再来看\emphB{逆变基}的偏导数 $\pdv*{\V{g}^i}{x^j} (\V{x})$。
它也是 $\Rm$ 空间中的向量,因此
\begin{braceEq*}
  {\label{eq:逆变基偏导数的协变与逆变表示} \pdv{\V{g}^i}{x^j} (\V{x})=}
  &\ipb{\pdv{\V{g}^i}{x^j}}{\V{g}^k} \V{g}_k \comma \\
  &\ipb{\pdv{\V{g}^i}{x^j}}{\V{g}_k} \V{g}^k \fullstop
\end{braceEq*}
利用 Christoffel 符号,可以表示出
$\ipb{\pdv*{\V{g}^i}{x^j}}{\V{g}_k}$。根据对偶关系,
\begin{equation}
  \ipb{\V{g}^i}{\V{g}_k} (\V{x})=\KroneckerDelta{i}{k} \fullstop
\end{equation}
两边对 $x^j$ 求偏导,用一下内积的求导公式,同时注意到
$\KroneckerDelta{i}{k}$ 是与 $\V{x}$ 无关的常数,因而
\begin{equation}
  \pdv{x^j} \ipb{\V{g}^i}{\V{g}_k}
  =\ipb{\pdv{\V{g}^i}{x^j}}{\V{g}_k}
    +\ipb{\V{g}^i}{\pdv{\V{g}_k}{x^j}}
  =\pdv{\KroneckerDelta{i}{k}}{x^j}=0 \fullstop
\end{equation}
所以
\begin{equation}
  \ipb{\pdv{\V{g}^i}{x^j}}{\V{g}_k}
  =-\ipb{\V{g}^i}{\pdv{\V{g}_k}{x^j}}
  =-\ipb{\pdv{\V{g}_k}{x^j}}{\V{g}^i}
  =-\ChrB{j}{k}{i} \fullstop
\end{equation}

至于 $\ipb{\pdv*{\V{g}^i}{x^j}}{\V{g}^k}$,将在以后讨论。
\myPROBLEM{你想在什么时候?}

\subsection{指标升降}
首先引入\emphA{度量}:
\begin{braceEq}
  g_{ij} (\V{x})
    &\defeq \ipb{\V{g}_i}{\V{g}_j} (\V{x}) \comma \\
  g^{ij} (\V{x})
    &\defeq \ipb{\V{g}^i}{\V{g}^j} (\V{x}) \fullstop
\end{braceEq}
由此可以获得\emphB{基向量}的指标升降
\begin{braceEq}
  \V{g}_i (\V{x}) &= g_{ij}(\V{x}) \, \V{g}^j(\V{x}) \comma \\
  \V{g}^i (\V{x}) &= g^{ij}(\V{x}) \, \V{g}_j(\V{x}) \fullstop
\end{braceEq}

如前所述,对于任意的 $\V{b}\in\Rm$,它可以表示成
\begin{equation}
  \V{b} = b^i\V{g}_i(\V{x}) = b_j\V{g}^j(\V{x}) \fullstop
\end{equation}
利用度量,同样可以获得\emphB{向量分量}的指标升降
\begin{braceEq}
  b^i&=\ipb{\V{b}}{\V{g}^i}
    =\ipb{\V{b}}{g^{ik}\,\V{g}_k}
    =g^{ik} \, \ipb{\V{b}}{\V{g}_k}
    =g^{ik} b_k \comma \\
  b_j&=\ipb{\V{b}}{\V{g}_j}
    =\ipb{\V{b}}{g_{jk}\,\V{g}^k}
    =g_{jk} \, \ipb{\V{b}}{\V{g}^k}
    =g_{jk} b^k \fullstop
\end{braceEq}

关于度量,再多说一句。由于内积的交换律,显然有
\begin{equation}
  g_{ij}(\V{x})=g_{ji}(\V{x}), \quad g^{ij}=g^{ji} \fullstop
\end{equation}

\subsection{度量的性质;Christoffel 符号的计算}
\label{subsec:度量的性质_Christoffel符号的计算}
首先,我们来澄清度量的两条性质。

\begin{myEnum}
\item 矩阵 $\qty[g_{ik}]$ 与 $\qty[g^{kj}]$ 互逆,即
\begin{equation}
  g_{ik}\,g^{kj} = \KroneckerDelta{j}{i} \fullstop
\end{equation}
证明见 \ref{subsec:度量}~小节(尽管省略了“$(\V{x})$”,
但请不要忘记这里的基是\emphB{局部基})。

\blankline

\item 第一类 Christoffel 符号满足
\begin{equation}
  \ChrA{i}{j}{k}=\frac{1}{2}\,
    \qty(\pdv{g_{jk}}{x^i}+\pdv{g_{ik}}{x^j}-\pdv{g_{ij}}{x^k})
    (\V{x}) \fullstop
  \label{eq:第一类Christoffel符号与度量的关系}
\end{equation}

\begin{myProof}
根据式~\eqref{eq:第一类Christoffel符号定义},
第一类 Christoffel 符号的定义为
\begin{equation}
  \ChrA{i}{j}{k} \defeq \ipb{\pdv{\V{g}_j}{x^i}}{\V{g}_k}
  \fullstop
\end{equation}
考虑度量的定义
\begin{equation}
  g_{ij}(\V{x})\defeq\ipb{\V{g}_i}{\V{g}_j} (\V{x}) \fullstop
\end{equation}
两边对 $x^k$ 求偏导,可得
\begin{align}
  \pdv{g_{ij}}{x^k} (\V{x})
  &=\ipb{\pdv{\V{g}_i}{x^k}}{\V{g}_j} (\V{x})
  +\ipb{\V{g}_i}{\pdv{\V{g}_j}{x^k}} (\V{x}) \notag
  \intertext{利用上面 Christoffel 符号的定义,有}
  &=\ChrA{k}{i}{j}+\ChrA{k}{j}{i} \fullstop
\end{align}
这样就获得了度量偏导数用 Christoffel 符号的表示。
但我们需要的却是 Christoffel 符号用度量偏导数的表示。
下面的工作就是完成这一“调转”。

利用指标轮换
\begin{equation*}
  i \to j, \quad j \to k, \quad k \to i \comma
\end{equation*}
可有
\begin{equation}
  \pdv{g_{jk}}{x^i} (\V{x})
  =\ChrA{i}{j}{k}+\ChrA{i}{k}{j} \fullstop
\end{equation}
再进行一次指标轮换:
\begin{equation}
  \pdv{g_{ki}}{x^j} (\V{x})
  =\ChrA{j}{k}{i}+\ChrA{j}{i}{k} \fullstop
\end{equation}
以上三式联立,就有
\begin{align}
  &\alspace\frac{1}{2}\,\qty(
    \pdv{g_{jk}}{x^i}+\pdv{g_{ki}}{x^j}-\pdv{g_{ij}}{x^k})
    (\V{x}) \notag \\
  &=\frac{1}{2}\,\qty\Big[
    \qty(\ChrA{i}{j}{k}+\ChrA{i}{k}{j})
    +\qty(\ChrA{j}{k}{i}+\ChrA{j}{i}{k})
    -\qty(\ChrA{k}{i}{j}+\ChrA{k}{j}{i})] \notag
  \intertext{利用 \eqref{eq:第一类Christoffel符号指标交换}~式
    所指出的 Christoffel 符号的指标交换性:}
  &=\frac{1}{2}\,\qty[
    \qty(\ChrA{i}{j}{k}+\hl{\ChrA{k}{i}{j}})
    +\qty(\hl[pink]{\ChrA{j}{k}{i}}
      +\ChrA{i}{j}{k})
    -\qty(\hl{\ChrA{k}{i}{j}}
      +\hl[pink]{\ChrA{j}{k}{i}}) ] \notag
  \intertext{高亮部分相互抵消,于是可得}
  &=\ChrA{i}{j}{k} \fullstop
\end{align}
\end{myProof}
\end{myEnum}

\blankline

有了这两条性质,我们就能够很容易地获取 Christoffel 符号的计算方法。

第一步从度量开始。根据 \ref{subsec:局部协变基}~小节,
在曲线坐标系(即微分同胚)
$\V{X}(\V{x})\in\cf{\domD{\V{x}}}{\domD{\V{X}}}$ 中,
Jacobi 矩阵可以用\emphB{协变基}表示为
\begin{equation}
  \JacobiD{\V{X}}(\V{x})
  =\mqty[\V{g}_1,\,\cdots,\,\V{g}_i,\,\cdots,\,\V{g}_m]
  (\V{x}) \fullstop
\end{equation}
因此协变形式的度量(矩阵形式)就可以写成
\begin{equation}
  \mqty[g_{ij}] \defeq \mqty[\ipb{\V{g}_i}{\V{g}_j}]
  =\JacobiD{\V{X}}\trans(\V{x}) \cdot
    \JacobiD{\V{X}}(\V{x}) \fullstop
\end{equation}
两种形式的度量是互逆的,于是 $g^{ij}$ 实际上也已经算出来了。

第二步,将求得的度量代入
式~\eqref{eq:第一类Christoffel符号与度量的关系}:
\begin{equation}
  \ChrA{i}{j}{k}=\frac{1}{2}\,
    \qty(\pdv{g_{jk}}{x^i}+\pdv{g_{ik}}{x^j}-\pdv{g_{ij}}{x^k})
    (\V{x}) \comma
\end{equation}
就得到了第一类 Christoffel 符号。
至于第二类 Christoffel 符号,它可以表示成
\begin{equation}
  \ChrB{i}{j}{k} \defeq \ipb{\pdv{\V{g}_j}{x^i}}{\V{g}^k}
  =\ipb{\pdv{\V{g}_j}{x^i}}{g^{kl}\,\V{g}_l}
  =g^{kl}\,\ipb{\pdv{\V{g}_j}{x^i}}{\V{g}_l}
  =g^{kl}\,\ChrA{i}{j}{l} \fullstop
  \label{eq:第二类Christoffel符号用第一类表示}
\end{equation}
这样一来,它的表示也就明确了。

\section{度量张量与 Eddington 张量}
\subsection{度量张量的定义}
在曲线坐标系(即微分同胚)
$\V{X}(\V{x})\in\cf{\domD{\V{x}}}{\domD{\V{X}}}$ 中,
可以引入\emphA{度量张量}
\begin{equation}
  \T{G}=g_{ij}\,\V{g}^i\tp\V{g}^j \in\Tensors{2} \fullstop
\end{equation}
这是用协变形式表达的。当然也可以切换成其他形式:
\begin{align}
  \T{G}&=g_{ij}\,\V{g}^i\tp\V{g}^j \notag
  \intertext{利用指标升降,有}
  &=g_{ij}\,\qty(g^{ik}\,\V{g}_k)\tp\V{g}^j \notag
  \intertext{再根据线性性提出系数:}
  &=g_{ij}\,g^{ik}\,\V{g}_k\tp\V{g}^j \notag \\
  &=\KroneckerDelta{k}{j}\,\V{g}_k\tp\V{g}^j \fullstop
\end{align}
类似地,还可以得到
\begin{align}
  \T{G}&=\KroneckerDelta{k}{j}\,\V{g}_k\tp\V{g}^j \notag \\
  &=\KroneckerDelta{k}{j}\,
    \V{g}_k\tp\qty(g^{jl}\,\V{g}_l) \notag \\
  &=\KroneckerDelta{k}{j}\,g^{jl}\,\V{g}_k\tp\V{g}_l \notag \\
  &=g^{kl}\,\V{g}_k\tp\V{g}_l \fullstop
\end{align}

综上,度量张量有三种表示:
\begin{braceEq*}{\T{G}=}
  &g_{ij}\,\V{g}^i\tp\V{g}^j \comma \\
  &g^{ij}\,\V{g}_i\tp\V{g}_j \comma \\
  &\KroneckerDelta{i}{j}\,\V{g}_i\tp\V{g}^j \comma
\end{braceEq*}
式中,协变分量 $g_{ij}=\ipb{\V{g}_i}{\V{g}_j}$,
逆变分量 $g^{ij}=\ipb{\V{g}^i}{\V{g}^j}$,
混合分量 $\KroneckerDelta{i}{j}=\ipb{\V{g}^i}{\V{g}_j}$。

\subsection{Eddington 张量的定义}
接下来引入 \emphA{Eddington 张量}
\begin{equation}
  \EdTensor=\LeviCivita{_{ijk}}\,\V{g}^i\tp\V{g}^j\tp\V{g}^k
  \in\Tensors[\realR^3]{3} \comma
\end{equation}
式中的 $\LeviCivita{_{ijk}}=\det[\V{g}_i,\,\V{g}_j,\,\V{g}_k]$。
和之前一样,仍是利用指标升降来获得等价定义:
\begin{align}
  \EdTensor
  &=\LeviCivita{_{ijk}}\,\V{g}^i\tp\V{g}^j\tp\V{g}^k \notag \\
  &=\LeviCivita{_{ijk}}\,\V{g}^i\tp\qty(g^{jl}\,\V{g}_l)
    \tp\V{g}^k \notag \\
  &=\LeviCivita{_{ijk}}\,g^{jl}\, \V{g}^i\tp\V{g}_l\tp\V{g}^k \notag
  \intertext{根据张量分量之间的关系
    (回顾 \ref{subsec:张量分量之间的关系}~小节),我们有}
  &=\LeviCivita{_i^l_k}\,\V{g}^i\tp\V{g}_l\tp\V{g}^k \fullstop
\end{align}
当然,这里的 $\LeviCivita{_i^l_k}$ 只是一个形式。
要将它显式地表达出来,需要利用行列式的线性性:
\begin{align}
  \forall\, \V{\xi},\,\hat{\V{\eta}},\,\tilde{\V{\eta}},\,\V{\zeta}
    \in\realR^3 \text{\ 以及\ }
    \alpha,\,\beta \in\realR,
  &\alspace \det\!\mqty[\V{\xi},\,\alpha\,\hat{\V{\eta}}
    +\beta\,\tilde{\V{\eta}},\,\V{\zeta}] \notag \\
  &=\alpha\det\!\mqty[\V{\xi},\,\hat{\V{\eta}},\,\V{\zeta}]
    +\beta\det\!\mqty[\V{\xi},\,\tilde{\V{\eta}},\,\V{\zeta}]
  \fullstop
\end{align}
由此可知
\begin{align}
  \LeviCivita{_i^l_k}
  &=\LeviCivita{_{ijk}}\,g^{jl} \notag \\
  &=g^{jl}\,\det\!\mqty[\V{g}_i,\,\V{g}_j,\,\V{g}_k] \notag \\
  &=\det\!\mqty[\V{g}_i,\,g^{jl}\,\V{g}_j,\,\V{g}_k] \notag \\
  &=\det\!\mqty[\V{g}_i,\,\V{g}^l,\,\V{g}_k] \fullstop
\end{align}

一般来说,张量在定义时,只需给出其分量的一种形式。
而其他的形式,则都可以通过\emphB{度量}来获得。
说得直白一些,这其实就是一套“指标升降游戏”。

\blankline

顺带一说,在 Descartes 坐标系下,$\realR^3$
空间中的叉乘可以用 Eddington 张量表示为
\begin{braceEq*}{\V{g}_i\cp\V{g}^j=}
  \LeviCivita{_i^{jk}}\,\V{g}_k \comma \\
  \LeviCivita{_i^j_k}\,\V{g}^k \fullstop
\end{braceEq*}

\begin{myProof}
利用对偶关系可以很容易地获得这一结果。$\V{g}_i\cp\V{g}^j$
仍然得到一个$\realR^3$ 空间中的向量,它自然可以用协变基来表示:
\begin{align}
  \V{g}_i\cp\V{g}^j
  &=\ipb{\V{g}_i\cp\V{g}^j}{\V{g}^k}\,\V{g}_k \notag
  \intertext{这里的内积也就是点积。根据向量三重积的知识,可以把
    $\V{A}\cp\V{B}\cdot\V{C}$ 表示成\emphB{行列式}:}
  &=\det\!\mqty[\V{g}_i,\,\V{g}^j,\,\V{g}^k]\,\V{g}_k \notag
  \intertext{根据 Eddington 张量的定义即得到}
  &=\LeviCivita{_i^{jk}}\,\V{g}_k \fullstop
\end{align}
同理,若用逆变基表示,则为
\begin{align}
  \V{g}_i\cp\V{g}^j
  &=\ipb{\V{g}_i\cp\V{g}^j}{\V{g}_k}\,\V{g}^k \notag \\
  &=\det\!\mqty[\V{g}_i,\,\V{g}^j,\,\V{g}_k]\,\V{g}^k \notag \\
  &=\LeviCivita{_i^j_k}\,\V{g}^k \fullstop
\end{align}
\end{myProof}

\subsection{两种度量的关系} \label{subsec:两种度量的关系}
两个 Eddington 张量的分量之积可以用
一个由度量张量分量所组成的行列式来表示:
\begin{equation}
  \LeviCivita{^i_j^k}\,\LeviCivita{_{pq}^r}
  =\mqty|
    \KroneckerDelta{i}{p} & \KroneckerDelta{i}{q} & g^{ir} \\
    g_{jp} & g_{jq} & \KroneckerDelta{r}{j} \\[0.8ex]
    \KroneckerDelta{k}{p} & \KroneckerDelta{k}{q} & g^{kr}
  | \fullstop
\end{equation}
类似矩阵乘法,行列式中第 $m$ 行 $n$ 列的元素,
由第一个 Eddington 张量的第 $m$ 个指标与
第二个 Eddington 张量的第 $n$ 个指标组合而成。
两个指标均在上面,则获得度量张量的\emphB{逆变分量};
两个指标均在下面,则获得\emphB{协变分量};
若是一上一下,则将得到\emphB{混合分量}(即 Kronecker δ)。

这里的 $i$、$j$、$k$ 和 $p$、$q$、$r$ 都不是\emphB{哑标},
无需考虑求和的限制,可以任意选取。
至于它们的上下位置,同样是由实际问题来确定的。

\begin{myProof}
证明思路就是化为矩阵乘法。根据定义,
\begin{align}
  \LeviCivita{^i_j^k}\,\LeviCivita{_{pq}^r}
  &=\det\!\mqty[\V{g}^i,\,\V{g}_j,\,\V{g}^k] \,
    \det\!\mqty[\V{g}_p,\,\V{g}_q,\,\V{g}^r] \notag
  \intertext{考虑行列式的性质
    $\det(\Mat{A}\Mat{B})=\det(\Mat{A})\det(\Mat{B})$ 和
    $\det(\Mat{A}\trans)=\det(\Mat{A})$,则有}
  &=\det\! \qty\Bigg(
    \mqty[\qty(\V{g}^i)\trans \\ \qty(\V{g}_j)\trans \\
      \qty(\V{g}^k)\trans]
    \mqty[\V{g}_p,\,\V{g}_q,\,\V{g}^r] ) \fullstop
\end{align}
这个矩阵可以直接算出。
\end{myProof}

如果 Eddington 张量中存在\emphB{哑标},情况就会有所不同:
\begin{equation}
  \LeviCivita{^i_j^s}\,\LeviCivita{^p_{qs}}
  =\sum_{s=1}^{3} \mqty|
    g^{ip} & \KroneckerDelta{i}{q} & \KroneckerDelta{i}{s} \\
    \KroneckerDelta{p}{j} & g_{jq} & g_{js} \\[0.6ex]
    g^{sp} & \KroneckerDelta{s}{q} & \KroneckerDelta{s}{s}
  | \fullstop
\end{equation}
由于式中的 $k$ 是哑标,因此需要对它求和。
行列式按第一行展开,可得(下面仍将根据 Einstein 约定省略求和号)
\begin{align}
  \alspace\LeviCivita{^i_j^s}\,\LeviCivita{^p_{qs}}
  &=g^{ip} \qty(g_{jq}\,\KroneckerDelta{s}{s}
      -g_{js}\,\KroneckerDelta{s}{q} )
    -\KroneckerDelta{i}{q} \, \qty(
      \KroneckerDelta{p}{j}\,\KroneckerDelta{s}{s}
      -g_{js}\,g^{sp} )
    +\KroneckerDelta{i}{s} \, \qty(
      \KroneckerDelta{p}{j}\,\KroneckerDelta{s}{q}
      -g_{jq}\,g^{sp} ) \notag \\
  &=g^{ip} \qty(3 g_{jq} - g_{jq})
    -\KroneckerDelta{i}{q} \,
      \qty(3 \KroneckerDelta{p}{j} - \KroneckerDelta{p}{j})
    +\qty(\KroneckerDelta{p}{j}\,\KroneckerDelta{i}{q}
      -g_{jq}\,g^{ip}) \notag \\
  &=g^{ip}\,g_{jq}-\KroneckerDelta{p}{j}\,\KroneckerDelta{i}{q}
  \fullstop
\end{align}
这一串稍显复杂的表达式,可以用口诀
“\emphB{前前后后,里里外外}”来记忆。
具体操作如图~\ref{fig:Eddington张量乘积口诀} 所示。

\begin{figure}[h]
  \centering
  \includegraphics[width=8cm]{images/eddington-tensors-product.png}
  \caption{Eddington 张量乘积口诀“前前后后,里里外外”的示意图}
  \label{fig:Eddington张量乘积口诀}
\end{figure}

下面再举两个例子来说明:
\begin{align}
  \LeviCivita{^{ij}_s}\,\LeviCivita{_{pq}^s}
  &=\KroneckerDelta{i}{p}\,\KroneckerDelta{j}{q}
    -\KroneckerDelta{j}{p}\,\KroneckerDelta{i}{q} \semicolon \\
  \LeviCivita{^{ij}_s}\,\LeviCivita{^p_q^s}
  &=g^{ip}\,\KroneckerDelta{j}{q}
    -g^{jp}\,\KroneckerDelta{i}{q} \fullstop
\end{align}
以后将会看到,这是一个相当重要的基本结构。

    % 张量场可微性
    \chapter{张量场可微性}
\section{张量的范数}
\subsection{赋范线性空间}
对于一个\emphA{线性空间} $\SPACE{V}$,它总是定义了\emphA{线性结构}:
\begin{equation}
  \forall\, \V{x},\,\V{y}\in\SPACE{V}
  \text{\ 和\ } \forall\, \alpha,\,\beta\in\realR,\quad
  \alpha\,\V{x}+\beta\,\V{y} \in \SPACE{V} \fullstop
\end{equation}
为了进一步研究的需要,我们还要引入\emphA{范数}的概念。
所谓“范数”,就是对线性空间中任意元素\emphB{大小}的一种刻画。
举个我们熟悉的例子, $m$ 维 Euclid 空间 $\Rm$ 中某个向量的范数,
就定义为该向量在 Descartes 坐标下各分量的平方和的平方根。

一般而言,线性空间 $\SPACE{V}$ 中的范数
$\norm[\SPACE{V}]{\cdotord}$ 是从 $\SPACE{V}$ 到 $\realR$
的一个映照,并且需要满足以下三个条件:

\begin{myEnum}
\item \emphA{非负性}
\begin{equation}
  \forall\,\V{x}\in\SPACE{V},\quad
  \norm[\SPACE{V}]{\V{x}} \geqslant 0
\end{equation}
以及\emphA{非退化性}
\begin{equation}
  \forall\,\V{x}\in\SPACE{V},\quad
  \norm[\SPACE{V}]{\V{x}}=0
  \iff \V{x}=\V{0}\in\SPACE{V} \comma
\end{equation}
这里的 $\V{0}$ 是线性空间 $\SPACE{V}$ 中的\emphA{零元素},
它是唯一存在的。

\blankline

\item 零元是唯一的,线性空间中的元素 $\V{x}$
与从 $\V{0}$ 指向它的向量一一对应。
因此,线性空间中的元素也常被称为“向量”。

考虑线性空间中的数乘运算。从几何上看, $\V{x}$ 乘上 $\lambda$,
就是将 $\V{x}$ 沿着原来的指向进行伸缩。显然有
\begin{equation}
  \forall\,\V{x}\in\SPACE{V}
  \text{\ 和\ } \forall\,\lambda\in\realR,\quad
  \norm[\SPACE{V}]{\lambda\,\V{x}}
  =\abs{\lambda}\cdot\norm[\SPACE{V}]{\V{x}} \comma
\end{equation}
这称为\emphA{正齐次性}。

\myPROBLEM{想要图吗?}

\item 线性空间中的加法满足\emphB{平行四边形法则}。直观地看,就有
\begin{equation}
  \forall\, \V{x},\,\V{y}\in\SPACE{V},\quad
  \norm[\SPACE{V}]{\V{x}+\V{y}} \leqslant
  \norm[\SPACE{V}]{\V{x}}+\norm[\SPACE{V}]{\V{y}} \comma
\end{equation}
这称为\emphA{三角不等式}。
\end{myEnum}

定义了范数的线性空间称为\emphA{赋范线性空间}。

\subsection{张量范数的定义}
考虑 $p$ 阶张量 $\T{\Phi}\in\Tensors{p}$,
它可以用\emphB{逆变分量}或\emphB{协变分量}来表示:
\begin{braceEq*}{\T{\Phi}=}
  \Phi^{i_1 \cdots i_p}\,
    \V{g}_{i_1}\tp\cdots\tp\V{g}_{i_p} \comma \\
  \Phi_{i_1 \cdots i_p}\,
    \V{g}^{i_1}\tp\cdots\tp\V{g}^{i_p} \comma
\end{braceEq*}
其中
\begin{braceEq}
  \Phi^{i_1 \cdots i_p}&=
    \T{\Phi}\qty(\V{g}^{i_1},\,\cdots,\,\V{g}^{i_p}) \fullstop \\
  \Phi_{i_1 \cdots i_p}&=
    \T{\Phi}\qty\big(\V{g}_{i_1},\,\cdots,\,\V{g}_{i_p}) \comma
\end{braceEq}
张量的\emphA{范数}定义为
\begin{equation}
  \norm[\Tensors{p}]{\T{\Phi}}\defeq
  \sqrt{\Phi^{i_1 \cdots i_p} \, \Phi_{i_1 \cdots i_p}}
  \in\realR \fullstop
\end{equation}
$i_1 \cdots i_p$ 可独立取值,每个又有 $m$ 种取法,
所以根号下共有 $m^p$ 项。
注意 $\Phi^{i_1 \cdots i_p}$ 与 $\Phi_{i_1 \cdots i_p}$
未必相等,因而根号下的部分未必是平方和,
这与 Euclid 空间中向量的模是不同的。

复习一下 \ref{subsec:相对不同基的张量分量之间的关系}~小节,
我们可以用另一组(带括号的)基表示张量 $\T{\Phi}$:
\begin{braceEq}
  \Phi^{i_1 \cdots i_p} &=
    c^{i_1}_{(\xi_1)} \cdots c^{i_p}_{(\xi_p)} \,
    \Phi^{(\xi_1)\cdots(\xi_p)} \comma \\
  \Phi_{i_1 \cdots i_p} &=
    c^{(\eta_1)}_{i_1} \cdots c^{(\eta_p)}_{i_p} \,
    \Phi_{(\eta_1)\cdots(\eta_p)} \comma
\end{braceEq}
其中的 $c^i_{(\xi)}=\ipb{\V{g}_{(\xi)}}{\V{g}^i}$,
$c^{(\eta)}_i=\ipb{\V{g}^{(\eta)}}{\V{g}_i}$,它们满足
\begin{equation}
  c^i_{(\xi)}\,c^{(\eta)}_i
  =\KroneckerDelta{(\eta)}{(\xi)} \fullstop
\end{equation}
于是
\begin{align}
  &\alspace\Phi^{i_1 \cdots i_p} \, \Phi_{i_1 \cdots i_p} \notag \\
  &=\qty(c^{i_1}_{(\xi_1)} \cdots c^{i_p}_{(\xi_p)} \,
      \Phi^{(\xi_1)\cdots(\xi_p)})
    \qty(c^{(\eta_1)}_{i_1} \cdots c^{(\eta_p)}_{i_p} \,
      \Phi_{(\eta_1)\cdots(\eta_p)}) \notag \\
  &=\qty(c^{i_1}_{(\xi_1)} c^{(\eta_1)}_{i_1}) \cdots
    \qty(c^{i_p}_{(\xi_p)} c^{(\eta_p)}_{i_p}) \,
    \Phi^{(\xi_1)\cdots(\xi_p)} \Phi_{(\eta_1)\cdots(\eta_p)}
    \notag \\
  &=\KroneckerDelta{(\eta_1)}{(\xi_1)} \cdots
    \KroneckerDelta{(\eta_p)}{(\xi_p)} \,
    \Phi^{(\xi_1)\cdots(\xi_p)} \Phi_{(\eta_1)\cdots(\eta_p)}
    \notag \\
  &=\Phi^{(\xi_1)\cdots(\xi_p)} \Phi_{(\xi_1)\cdots(\xi_p)}
  \fullstop
\end{align}
它是 $\T{\Phi}$ 在另一组基下的逆变分量与协变分量乘积之和。

以上结果说明,张量的范数不依赖于基的选取,
这就好比用不同的秤来称同一个人的体重,都将获得相同的结果。
既然如此,不妨采用\emphB{单位正交基}来表示张量的范数:
\begin{align}
  \norm[\Tensors{p}]{\T{\Phi}}
  &\defeq\sqrt{\Phi^{i_1 \cdots i_p} \, \Phi_{i_1 \cdots i_p}}
  \notag \\
  &=\sqrt{\Phi^{\orthIdx{i_1}\cdots\orthIdx{i_p}} \,
    \Phi_{\orthIdx{i_1}\cdots\orthIdx{i_p}}} \notag \\
  &\eqcolon\sqrt{\sum\nolimits_{i_1,\,\cdots,\,i_p=1}^{m}
    \qty\big(\Phi\midscript{\orthIdx{i_1,\,\cdots,\,i_p}})^2}
  \fullstop
\end{align}
这里的 $\Phi\midscript{\orthIdx{i_1,\,\cdots,\,i_p}}$
表示张量 $\T{\Phi}$ 在单位正交基下的分量,它的指标不区分上下。

有了这样的表示,很容易就可以验证张量范数符合之前的三个要求。
一组数的平方和开根号,必然是\emphB{非负}的。
至于\emphB{非退化性},若范数为零,则所有分量均为零,自然成为零张量;
反之,对于零张量,所有分量为零,范数也为零。
将 $\T{\Phi}$ 乘上 $\lambda$,则有
\begin{align}
  \norm[\Tensors{p}]{\lambda\,\T{\Phi}}
  &=\sqrt{\sum\nolimits_{i_1,\,\cdots,\,i_p=1}^{m}
    \qty\big(\lambda\,
      \Phi\midscript{\orthIdx{i_1,\,\cdots,\,i_p}})^2} \notag \\
  &=\sqrt{\lambda^2 \sum\nolimits_{i_1,\,\cdots,\,i_p=1}^{m}
      \qty\big(\Phi\midscript{\orthIdx{i_1,\,\cdots,\,i_p}})^2}
    \notag \\
  &=\abs{\lambda} \sqrt{\sum\nolimits_{i_1,\,\cdots,\,i_p=1}^{m}
      \qty\big(\Phi\midscript{\orthIdx{i_1,\,\cdots,\,i_p}})^2}
    \notag \\
  &=\abs{\lambda}\cdot\norm[\Tensors{p}]{\T{\Phi}} \comma
\end{align}
于是\emphB{正齐次性}也得以验证。
最后,利用 Cauchy--Schwarz 不等式,可有
\begin{align}
  &\alspace\norm[\Tensors{p}]{\T{\Phi}+\T{\Psi}}^2 \notag \\
  &=\sum \qty\Big(\Phi\midscript{\orthIdx{i_1,\,\cdots,\,i_p}}
      +\Psi\midscript{\orthIdx{i_1,\,\cdots,\,i_p}})^2 \notag \\
  &=\sum \qty[
      \qty\big(\Phi\midscript{\orthIdx{i_1,\,\cdots,\,i_p}})^2
      +2\, \Phi\midscript{\orthIdx{i_1,\,\cdots,\,i_p}} \,
        \Psi\midscript{\orthIdx{i_1,\,\cdots,\,i_p}}
      +\qty\big(\Psi\midscript{\orthIdx{i_1,\,\cdots,\,i_p}})^2]
    \notag \\
  &=\sum \qty\big(\Phi\midscript{\orthIdx{i_1,\,\cdots,\,i_p}})^2
    +2\sum \Phi\midscript{\orthIdx{i_1,\,\cdots,\,i_p}} \,
      \Psi\midscript{\orthIdx{i_1,\,\cdots,\,i_p}}
    +\sum \qty\big(\Psi\midscript{\orthIdx{i_1,\,\cdots,\,i_p}})^2
    \notag \\
  &\leqslant\norm[\Tensors{p}]{\T{\Phi}}^2
    +2 \sqrt{\sum \qty\big(
        \Phi\midscript{\orthIdx{i_1,\,\cdots,\,i_p}})^2}
      \sqrt{\sum \qty\big(
        \Psi\midscript{\orthIdx{i_1,\,\cdots,\,i_p}})^2}
    +\norm[\Tensors{p}]{\T{\Phi}}^2 \notag \\
  &=\norm[\Tensors{p}]{\T{\Phi}}^2
    +2\norm[\Tensors{p}]{\T{\Phi}}\cdot\norm[\Tensors{p}]{\T{\Psi}}
    +\norm[\Tensors{p}]{\T{\Phi}}^2 \notag \\
  &=\qty\Big(\norm[\Tensors{p}]{\T{\Phi}}
    +\norm[\Tensors{p}]{\T{\Phi}})^2 \fullstop
\end{align}
两边开方,即为\emphB{三角不等式}。

\blankline

由此,我们就完整地给出了张量大小的刻画手段。
可以看出,它实际上就是 Euclid 空间中向量模的直接推广。

\subsection{简单张量的范数}
根据 \ref{subsec:张量的表示与简单张量}~小节中的定义,
简单张量是形如 $\V{\xi}\tp\V{\eta}\tp\V{\zeta}$ 的张量,
其中的 $\V{\xi},\,\V{\eta},\,\V{\zeta}\in\Rm$,
它是三个向量的张量积。
简单张量的范数为
\begin{equation}
  \norm[\Tensors{3}]{\V{\xi}\tp\V{\eta}\tp\V{\zeta}}
  =\norm{\V{\xi}}\cdot\norm{\V{\eta}}\cdot\norm{\V{\zeta}}
  \fullstop \label{eq:简单张量的范数}
\end{equation}

\begin{myProof}
$\V{\xi}\tp\V{\eta}\tp\V{\zeta}$ 的逆变分量为
\begin{equation}
  \qty(\V{\xi}\tp\V{\eta}\tp\V{\zeta})^{ijk}
  \defeq \V{\xi}\tp\V{\eta}\tp\V{\zeta}
    \qty(\V{g}^i,\,\V{g}^j,\,\V{g}^k)
  =\xi^i \eta^j \zeta^k \fullstop
\end{equation}
同理,它的协变分量为
\begin{equation}
  \qty(\V{\xi}\tp\V{\eta}\tp\V{\zeta})_{ijk}
  \defeq \V{\xi}\tp\V{\eta}\tp\V{\zeta}
    \qty(\V{g}_i,\,\V{g}_j,\,\V{g}_k)
  =\xi_i \eta_j \zeta_k \fullstop
\end{equation}
二者相乘,有
\begin{align}
  &\alspace\qty(\V{\xi}\tp\V{\eta}\tp\V{\zeta})^{ijk}
    \cdot \qty(\V{\xi}\tp\V{\eta}\tp\V{\zeta})_{ijk} \notag \\
  &=\qty(\xi^i \eta^j \zeta^k)
    \cdot \qty(\xi_i \eta_j \zeta_k) \notag \\
  &=\qty(\xi^i \xi_i) \cdot \qty(\eta^j \eta_j)
    \cdot \qty(\zeta^k \zeta_k) \fullstop
\end{align}
注意到
\begin{align}
  \norm{\xi}^2
  &=\ipb{\xi}{\xi} \notag
  \intertext{分别把二者用协变和逆变分量表示:}
  &=\ipb{\xi^i\,\V{g}_i}{\xi_j\,\V{g}^j} \notag \\
  &=\xi^i \xi_j \ipb{\V{g}_i}{\V{g}^j} \notag \\
  &=\xi^i \xi_j \KroneckerDelta{j}{i}
  =\xi^i \xi_i \comma
\end{align}
于是
\begin{equation}
  \qty(\V{\xi}\tp\V{\eta}\tp\V{\zeta})^{ijk}
    \cdot \qty(\V{\xi}\tp\V{\eta}\tp\V{\zeta})_{ijk}
  =\norm{\xi}^2 \cdot \norm{\V{\eta}}^2 \cdot \norm{\V{\zeta}}^2
  \fullstop
\end{equation}
两边开方,即得 \eqref{eq:简单张量的范数}~式。
\end{myProof}

\section{张量场的偏导数;协变导数}
\label{sec:张量场的偏导数_协变导数}
在区域 $\domD{\V{x}}\subset\Rm$ 上,
若存在一个自变量用\emphB{位置}刻画的映照
\begin{equation}
  \mmap{\T{\Phi}}{\domD{\V{x}}\ni\V{x}}
    {\T{\Phi}(\V{x})\in\Tensors{r}} \comma
\end{equation}
则称张量 $\T{\Phi}(\V{x})$ \footnote{
  类似“$\T{\Phi}(\V{x})$”的记号在前文也表示张量 $\T{\Phi}$
  \emphB{作用}在向量 $\V{x}$ 上(“吃掉”了 $\V{x}$),此时有
  $\T{\Phi}(\V{x})\in\realR$,注意不要混淆。
  符号有限,难免如此,还望诸位体谅。}是定义在 $\domD{\V{x}}$
上的一个\emphA{张量场}。

下面我们以三阶张量为例。设在物理域 $\domD{\V{X}}\subset\Rm$
和参数域 $\domD{\V{x}}\subset\Rm$ 之间已经建立了微分同胚
$\V{X}(\V{x})\in\cf{\domD{\V{x}}}{\domD{\V{X}}}$。
在 $\V{X}(\V{x})$ 处,张量场 $\T{\Phi}(\V{x})$
可以用分量形式表示为
\begin{equation}
  \T{\Phi}(\V{x})=\tc{\Phi}{^i_j^k}(\V{x})\,
    \V{g}_i(\V{x})\tp\V{g}^j(\V{x})\tp\V{g}_k(\V{x})
  \in\Tensors{3} \comma
\end{equation}
其中的 $\V{g}_i(\V{x}),\,\V{g}^j(\V{x}),\,\V{g}_k(\V{x})$
都是\emphB{局部}基,而张量分量则定义为\footnote{
  请注意,下式 $\T{\Phi}$ 之后的第一个圆括号表示\emphB{位于}
  $\V{x}$ 处;而后面的方括号则表示\emphB{作用在}这几个向量上。}
\begin{equation}
  \tc{\Phi}{^i_j^k}(\V{x})
  \defeq \T{\Phi}(\V{x})
    \qty[\V{g}_i(\V{x}),\,\V{g}^j(\V{x}),\,\V{g}_k(\V{x})]
  \in\realR \fullstop
\end{equation}
类似地,当点沿着 $x^\mu$-线运动到
$\V{X}\qty(\V{x}+\lambda\,\V{e}_\mu)$ 处时,有
\begin{equation}
  \T{\Phi}\qty(\V{x}+\lambda\,\V{e}_\mu)
  =\tc{\Phi}{^i_j^k}\qty(\V{x}+\lambda\,\V{e}_\mu)\,
    \V{g}_i \qty(\V{x}+\lambda\,\V{e}_\mu)
    \tp\V{g}^j \qty(\V{x}+\lambda\,\V{e}_\mu)
    \tp\V{g}_k \qty(\V{x}+\lambda\,\V{e}_\mu) \fullstop
  \label{eq:x+lambda_e_mu处的张量场}
\end{equation}

现在研究 $\lambda\to 0 \in\realR$ 时的极限
\begin{equation}
  \lim_{\lambda\to 0} \frac{\T{\Phi}\qty(\V{x}+\lambda\,\V{e}_\mu)
    -\T{\Phi}(\V{x})} {\lambda}
  \eqcolon \pdv{\T{\Phi}}{x^\mu} (\V{x})
  \in\Tensors{3} \fullstop
  \label{eq:张量场偏导数的极限定义}
\end{equation}
与之前的向量值映照类似,该极限表示张量场 $\T{\Phi}(\V{x})$
作为一个\emphB{整体},相对于自变量第 $\mu$ 个分量的变化率,
即 $\T{\Phi}$ 关于 $x^\mu$(在 $\V{x}$ 处)的\emphA{偏导数}。
式中,$\T{\Phi}\qty(\V{x}+\lambda\,\V{e}_\mu)$
已由 \eqref{eq:x+lambda_e_mu处的张量场}~式给出。
注意到张量分量实际上就是一个\emphB{多元函数},于是
\begin{equation}
  \tc{\Phi}{^i_j^k}\qty(\V{x}+\lambda\,\V{e}_\mu)
  =\tc{\Phi}{^i_j^k}(\V{x})
  +\pdv{\tc{\Phi}{^i_j^k}}{x^\mu} (\V{x}) \cdot\lambda
  +\sO*{^i_j^k}{\lambda} \in\realR \fullstop
\end{equation}
另外,三个基向量作为\emphB{向量值映照},同样可以展开:
\begin{braceEq}
  \V{g}_i\qty(\V{x}+\lambda\,\V{e}_\mu)
  &=\V{g}_i(\V{x})+\pdv{\V{g}_i}{x^\mu} (\V{x}) \cdot\lambda
    +\sOv*{_i}{\lambda} \in\Rm \comma \\
  \V{g}^j\qty(\V{x}+\lambda\,\V{e}_\mu)
  &=\V{g}^j(\V{x})+\pdv{\V{g}^j}{x^\mu} (\V{x}) \cdot\lambda
    +\sOv*{^j}{\lambda} \in\Rm \comma \\
  \V{g}_k\qty(\V{x}+\lambda\,\V{e}_\mu)
  &=\V{g}_k(\V{x})+\pdv{\V{g}_k}{x^\mu} (\V{x}) \cdot\lambda
    +\sOv*{_k}{\lambda} \in\Rm \fullstop
\end{braceEq}
如果直接展开,一共有 81 项,显然过于繁杂,不便操作。
我们将按 $\lambda$ 的次数逐次展开。首先看 $\lambda$ 的零次项:
\begin{equation}
  \tc{\Phi}{^i_j^k}(\V{x})\,
  \V{g}_i(\V{x})\tp\V{g}^j(\V{x})\tp\V{g}_k(\V{x}) \comma
\end{equation}
这就是 $\T{\Phi}(\V{x})$。然后是 $\lambda$ 的一次项:
\begin{align}
  \lambda &\cdot\left[\pdv{\tc{\Phi}{^i_j^k}}{x^\mu} (\V{x})\,
    \V{g}_i(\V{x})\tp\V{g}^j(\V{x})\tp\V{g}_k(\V{x})
  +\tc{\Phi}{^i_j^k} (\V{x})\,
    \pdv{\V{g}_i}{x^\mu} (\V{x})\tp\V{g}^j(\V{x})\tp\V{g}_k(\V{x})
  \right. \notag \\*
  &\phantom{M}\left.\phantom{}+\tc{\Phi}{^i_j^k} (\V{x})\,
    \V{g}_i(\V{x})\tp\pdv{\V{g}^j}{x^\mu} (\V{x})\tp\V{g}_k(\V{x})
  +\tc{\Phi}{^i_j^k} (\V{x})\,
    \V{g}_i(\V{x})\tp\V{g}^j(\V{x})\tp\pdv{\V{g}_k}{x^\mu} (\V{x})
  \vphantom{\pdv{\tc{\Phi}{^i_j^k}}{x^\mu}} \right] \fullstop
\end{align}
剩下的至少是 $\lambda$ 的二次项,
我们将其统一写作“$\res$”(余项)。

现在回头看之前的极限 \eqref{eq:张量场偏导数的极限定义}~式。
$\lambda$ 的零次项与 $\T{\Phi}(\V{x})$ 相互抵消,
而一次项就只剩下了系数部分。至于余项 $\res$,则要证明它趋于零。以
\begin{equation}
  \tc{\Phi}{^i_j^k} (\V{x})\,
  \sOv*{_i}{\lambda}\tp\V{g}^j(\V{x})\tp\V{g}_k(\V{x})
\end{equation}
为例,我们需要证明它等于 $\sOv{\lambda}\in\Tensors{3}$,即
\begin{equation}
  \lim_{\lambda\to 0}
  \frac{\norm[\Tensors{3}]{\tc{\Phi}{^i_j^k} (\V{x})\,
      \sOv*{_i}{\lambda}\tp\V{g}^j(\V{x})\tp\V{g}_k(\V{x})}}
    {\lambda}=0\in\realR \fullstop
  \label{eq:余项证明举例}
\end{equation}

\begin{myProof}
这里为了叙述方便,我们将暂时不使用 Einstein 求和约定,
而是把求和号显式地写出来。于是分子部分可以写成
\begin{align}
  &\alspace\norm[\Tensors{3}]{\sum_{i,\,j,\,k=1}^{m}
    \tc{\Phi}{^i_j^k} (\V{x})\,
    \sOv*{_i}{\lambda}\tp\V{g}^j(\V{x})\tp\V{g}_k(\V{x})} \notag
  \intertext{根据范数的\emphB{三角不等式},有}
  &\leqslant\sum_{i,\,j,\,k=1}^{m}
    \norm[\Tensors{3}]{\tc{\Phi}{^i_j^k} (\V{x})\,
      \sOv*{_i}{\lambda}\tp\V{g}^j(\V{x})\tp\V{g}_k(\V{x})} \notag
  \intertext{再利用\emphB{正齐次性},可得}
  &=\sum_{i,\,j,\,k=1}^{m} \abs{\tc{\Phi}{^i_j^k} (\V{x})}
    \cdot \norm[\Tensors{3}]
      {\sOv*{_i}{\V{x}}\tp\V{g}^j(\V{x})\tp\V{g}_k(\V{x})} \notag
  \intertext{代入简单张量的范数,便有}
  &=\sum_{i,\,j,\,k=1}^{m} \abs{\tc{\Phi}{^i_j^k} (\V{x})}
    \cdot \norm{\sOv*{_i}{\lambda}}
    \cdot \norm{\V{g}^j(\V{x})}
    \cdot \norm{\V{g}_k(\V{x})} \fullstop
\end{align}
这几项中只有 $\norm[\Tensors{3}]{\sOv*{_i}{\lambda}}$ 与
$\lambda$ 有关。于是
\begin{align}
  &\alspace\lim_{\lambda\to 0}
  \frac{\norm[\Tensors{3}]{\tc{\Phi}{^i_j^k} (\V{x})\,
      \sOv*{_i}{\lambda}\tp\V{g}^j(\V{x})\tp\V{g}_k(\V{x})}}
    {\lambda} \notag \\
  &=\sum_{i,\,j,\,k=1}^{m} \abs{\tc{\Phi}{^i_j^k} (\V{x})}
    \cdot \norm{\V{g}^j(\V{x})}
    \cdot \norm{\V{g}_k(\V{x})}
    \cdot \lim_{\lambda\to 0}
    \frac{\norm{\sOv*{_i}{\lambda}}}{\lambda} \notag
  \intertext{根据定义,最后的极限为零,因此}
  &=0 \fullstop
\end{align}
\end{myProof}

类似地,其他七十多项也都是 $\lambda$ 的一阶无穷小量。
而有限个无穷小量之和仍为无穷小量,于是 $\res\to 0$。

\blankline

综上,我们有
\begin{align}
  &\alspace \pdv{\T{\Phi}}{x^\mu} (\V{x})  \coloneq
  \lim_{\lambda\to 0} \frac{\T{\Phi}\qty(\V{x}+\lambda\,\V{e}_\mu)
    -\T{\Phi}(\V{x})} {\lambda} \notag \\
  &=\qty(\pdv{\tc{\Phi}{^i_j^k}}{x^\mu} \,
    \V{g}_i\tp\V{g}^j\tp\V{g}_k
  +\tc{\Phi}{^i_j^k} \,
    \pdv{\V{g}_i}{x^\mu} \tp\V{g}^j\tp\V{g}_k
  +\tc{\Phi}{^i_j^k} \,
    \V{g}_i\tp\pdv{\V{g}^j}{x^\mu} \tp\V{g}_k
  +\tc{\Phi}{^i_j^k} \,
    \V{g}_i\tp\V{g}^j\tp\pdv{\V{g}_k}{x^\mu}) \, (\V{x})
  \fullstop \label{eq:张量场偏导数展开式}
\end{align}
式中,$\pdv*{\V{g}_i}{x^\mu} (\V{x})$ 可以用 Christoffel 符号表示:
\begin{equation}
  \pdv{\V{g}_i}{x^\mu} (\V{x})
  =\ChrB{\mu}{i}{s} \V{g}_s (\V{x}) \fullstop
\end{equation}
因此 \eqref{eq:张量场偏导数展开式}~式的第二项
\begin{align}
  &\alspace\tc{\Phi}{^i_j^k} (\V{x}) \,
    \pdv{\V{g}_i}{x^\mu} (\V{x})
    \tp\V{g}^j(\V{x})\tp\V{g}_k(\V{x}) \notag \\
  &=\ChrB{\mu}{i}{s} \tc{\Phi}{^i_j^k}(\V{x}) \,
    \V{g}_s(\V{x})\tp\V{g}^j(\V{x})\tp\V{g}_k(\V{x}) \notag
  \intertext{$i$ 和 $s$ 都是哑标,不妨进行一下交换:}
  &=\ChrB{\mu}{s}{i} \tc{\Phi}{^s_j^k}(\V{x}) \,
    \V{g}_i(\V{x})\tp\V{g}^j(\V{x})\tp\V{g}_k(\V{x}) \fullstop
\end{align}
同样,后面的两项也可进行类似的处理。这样便有
\begin{align}
  &\alspace \pdv{\T{\Phi}}{x^\mu} (\V{x}) \coloneq
  \lim_{\lambda\to 0} \frac{\T{\Phi}\qty(\V{x}+\lambda\,\V{e}_\mu)
    -\T{\Phi}(\V{x})} {\lambda} \notag \\
  &=\qty[\qty\Bigg(\pdv{\tc{\Phi}{^i_j^k}}{x^\mu}
    +\ChrB{\mu}{s}{i} \tc{\Phi}{^s_j^k}
    -\ChrB{\mu}{j}{s} \tc{\Phi}{^i_s^k}
    +\ChrB{\mu}{s}{k} \tc{\Phi}{^i_j^s}) \,
    \V{g}_i\tp\V{g}^j\tp\V{g}_k] (\V{x}) \notag \\
  &\eqcolon \coD{\mu}{\tc{\Phi}{^i_j^k} (\V{x})} \,
    \V{g}_i(\V{x})\tp\V{g}^j(\V{x})\tp\V{g}_k(\V{x}) \comma
\end{align}
式中,我们称 $\coD{\mu}{\tc{\Phi}{^i_j^k}(\V{x})}\in\realR$
为张量分量 $\tc{\Phi}{^i_j^k}(\V{x})$ 相对于 $x^\mu$
的\emphA{协变导数},其定义为:
\begin{equation}
  \coD{\mu}{\tc{\Phi}{^i_j^k} (\V{x})} \defeq
  \pdv{\tc{\Phi}{^i_j^k}}{x^\mu} (\V{x})
  +\ChrB{\mu}{s}{i} \tc{\Phi}{^s_j^k}(\V{x})
  -\ChrB{\mu}{j}{s} \tc{\Phi}{^i_s^k}(\V{x})
  +\ChrB{\mu}{s}{k} \tc{\Phi}{^i_j^s}(\V{x}) \fullstop
  \label{eq:协变导数定义}
\end{equation}

\section{张量场的梯度} \label{sec:张量场的梯度}
\subsection{梯度;可微性}
我们在参数域中的内点 $\V{x}_0$ 处取一个半径为 $\delta$ 的邻域
$\domB{\delta}{\V{x}_0}$。若使自变量变化到 $\V{x}_0+\V{h}$,
则在物理域中,对应的点就将从 $\V{X}\qty(\V{x}_0)$ 变化到
$\V{X}\qty(\V{x}_0+\V{h})$。考察定义在参数域 $\domD{x}$ 上的张量场
$\T{\Phi}(\V{x})$,它的变化为
\begin{align}
  &\alspace\T{\Phi}\qty(\V{x}_0+\V{h})-\T{\Phi}\qty(\V{x}_0)
    \notag \\
  &=\tc{\Phi}{^i_j^k}\qty(\V{x}_0+\V{h}) \,
    \V{g}_i \qty(\V{x}_0+\V{h})
    \tp\V{g}^j \qty(\V{x}_0+\V{h})
    \tp\V{g}_k \qty(\V{x}_0+\V{h}) \notag \\
  &\alspace\phantom{}
    -\tc{\Phi}{^i_j^k}\qty(\V{x}_0) \,
    \V{g}_i \qty(\V{x}_0)
    \tp\V{g}^j \qty(\V{x}_0)
    \tp\V{g}_k \qty(\V{x}_0) \fullstop
  \label{eq:张量场沿任意方向的变化_1}
\end{align}
与之前一样将第一部分逐项展开,有
\begin{braceEq}
  &\tc{\Phi}{^i_j^k}\qty(\V{x}_0+\V{h})
  =\tc{\Phi}{^i_j^k}\qty(\V{x}_0)
  +\pdv{\tc{\Phi}{^i_j^k}}{x^\mu} \qty(\V{x}_0) \cdot h^\mu
  +\sO*{^i_j^k}{\norm{\V{h}}} \in\realR \comma \\
  &\V{g}_i \qty(\V{x}_0+\V{h})
  =\V{g}_i \qty(\V{x}_0)
    +\pdv{\V{g}_i}{x^\mu} \qty(\V{x}_0) \cdot h^\mu
    +\sOv*{_i}{\norm{\V{h}}} \in\Rm \comma \\
  &\V{g}^j \qty(\V{x}_0+\V{h})
  =\V{g}^j \qty(\V{x}_0)
    +\pdv{\V{g}^j}{x^\mu} \qty(\V{x}_0) \cdot h^\mu
    +\sOv*{^j}{\norm{\V{h}}} \in\Rm \comma \\
  &\V{g}_k \qty(\V{x}_0+\V{h})
  =\V{g}_k \qty(\V{x}_0)
    +\pdv{\V{g}_k}{x^\mu} \qty(\V{x}_0) \cdot h^\mu
    +\sOv*{_k}{\norm{\V{h}}} \in\Rm \fullstop
\end{braceEq}
不要忘记对哑标 $\mu$ 进行求和。

不显含 $\V{h}$ 的只有每行的第一项;它们组合起来,
与式~\eqref{eq:张量场沿任意方向的变化_1} 的第二部分相互抵消。
再看 $\V{h}$ 的一次项 :\footnote{
  实际是 $h^\mu$ 的一次项。别忘了求和。}
\begin{equation}
  \qty[
    \pdv{\tc{\Phi}{^i_j^k}}{x^\mu}\,
    \V{g}_i \tp \V{g}^j \tp \V{g}_k
  +\tc{\Phi}{^i_j^k} \qty(
    \pdv{\V{g}_i}{x^\mu}\tp\V{g}^j\tp\V{g}_k
    +\V{g}_i\tp\pdv{\V{g}^j}{x^\mu}\tp\V{g}_k
    +\V{g}_i\tp\V{g}^j\tp\pdv{\V{g}_k}{x^\mu} )]
  \qty(\V{x}_0) \cdot h^\mu \fullstop
\end{equation}
利用 Christoffel 符号又可以把它写成
\begin{align}
  &\alspace \left[\pdv{\tc{\Phi}{^i_j^k}}{x^\mu}\,
      \V{g}_i \tp \V{g}^j \tp \V{g}_k
    +\tc{\Phi}{^i_j^k}\, \qty\Big(\ChrB{\mu}{i}{s}\V{g}_s)
      \tp\V{g}^j\tp\V{g}_k \right.
    \notag \\*
  &\phantom{MM} \left.\phantom{}-\tc{\Phi}{^i_j^k} \,
      \V{g}_i\tp\qty\Big(\ChrB{\mu}{s}{j}\V{g}^s)\tp\V{g}_k
    +\tc{\Phi}{^i_j^k} \,
      \V{g}_i\tp\V{g}^j\tp\qty\Big(\ChrB{\mu}{k}{s}\V{g}_s)
    \vphantom{\pdv{\tc{\Phi}{^i_j^k}}{x^\mu}} \right]
    \qty(\V{x}_0) \cdot h^\mu \notag \\
  %%
  &=\left(\pdv{\tc{\Phi}{^i_j^k}}{x^\mu}\,
      \V{g}_i \tp \V{g}^j \tp \V{g}_k
    +\ChrB{\mu}{s}{i} \tc{\Phi}{^s_j^k} \,
      \V{g}_i\tp\V{g}^j\tp\V{g}_k \right.
    \notag \\*
  &\phantom{MM} \left.\phantom{}
    -\ChrB{\mu}{j}{s} \tc{\Phi}{^i_s^k} \,
      \V{g}_i\tp\V{g}^j\tp\V{g}_k
    +\ChrB{\mu}{s}{k} \tc{\Phi}{^i_j^s} \,
      \V{g}_i\tp\V{g}^j\tp\V{g}_k
    \vphantom{\pdv{\tc{\Phi}{^i_j^k}}{x^\mu}} \right)
    \qty(\V{x}_0) \cdot h^\mu \notag \\
  %%
  &=\qty[\qty\Bigg(\pdv{\tc{\Phi}{^i_j^k}}{x^\mu}
    +\ChrB{\mu}{s}{i} \tc{\Phi}{^s_j^k}
    -\ChrB{\mu}{j}{s} \tc{\Phi}{^i_s^k}
    +\ChrB{\mu}{s}{k} \tc{\Phi}{^i_j^s}) \,
    \V{g}_i\tp\V{g}^j\tp\V{g}_k] \qty(\V{x}_0)
    \cdot h^\mu \fullstop
\end{align}
至于高阶项,它们都等于 $\sOv{\norm{\V{h}}}\in\Tensors{3}$,并且满足
\begin{equation}
  \lim_{\V{h}\to\V{0}\,\in\,\Rm}
    \frac{\norm[\Tensors{3}]{\sOv{\norm{\V{h}}}}}{\norm{\V{h}}}
  =0 \in\realR \fullstop
\end{equation}
证明与 \eqref{eq:余项证明举例}~式类似。

整理一下,我们有
\begin{align}
  &\alspace\T{\Phi}\qty(\V{x}_0+\V{h})-\T{\Phi}\qty(\V{x}_0)
    \notag \\
  &=\qty[\qty\Bigg(\pdv{\tc{\Phi}{^i_j^k}}{x^\mu}
      +\ChrB{\mu}{s}{i} \tc{\Phi}{^s_j^k}
      -\ChrB{\mu}{j}{s} \tc{\Phi}{^i_s^k}
      +\ChrB{\mu}{s}{k} \tc{\Phi}{^i_j^s}) \,
      \V{g}_i\tp\V{g}^j\tp\V{g}_k] \qty(\V{x}_0) \cdot h^\mu
    +\sOv{\norm{\V{h}}} \notag
  \intertext{利用协变导数,有}
  &\eqcolon \qty[\vphantom{\frac{0^0}{0^0}}
    \coD{\mu}{\tc{\Phi}{^i_j^k} \qty(\V{x}_0)} \,
    \V{g}_i \qty(\V{x}_0)
    \tp\V{g}^j \qty(\V{x}_0)
    \tp\V{g}_k \qty(\V{x}_0)] h^\mu
    +\sOv{\norm{\V{h}}} \fullstop
  \label{eq:张量场沿任意方向的变化_2}
\end{align}
至此,从微分学的角度来看,任务已经完成。
但对于张量分析而言,我们还需要再做一点微小的工作。
简单张量部分再并上一个 $\V{g}^\mu$,从而使张量升一阶;
后面则改成 $h^\nu\,\V{g}_\nu\qty(\V{x}_0)$,并利用点乘保持总阶数不变:
\begin{align}
  &\alspace\qty[\vphantom{\frac{0^0}{0^0}}
    \coD{\mu}{\tc{\Phi}{^i_j^k} \qty(\V{x}_0)} \,
    \V{g}_i \qty(\V{x}_0) \tp \V{g}^j \qty(\V{x}_0)
    \tp\V{g}_k \qty(\V{x}_0)] h^\mu \notag \\
  &=\qty[\vphantom{\frac{0^0}{0^0}}
      \coD{\mu}{\tc{\Phi}{^i_j^k} \qty(\V{x}_0)} \,
      \V{g}_i \qty(\V{x}_0) \tp \V{g}^j \qty(\V{x}_0)
      \tp\V{g}_k \qty(\V{x}_0) \tp \V{g}^\mu \qty(\V{x}_0)]
    \cdot \qty[\vphantom{\frac{0}{0}} h^\nu\V{g}_\nu\qty(\V{x}_0)]
  \fullstop
\end{align}
所谓“点乘”,其实就是 \emphB{$e$ 点积}在 $e=1$ 时的情况。实际上,
\begin{equation}
  \V{g}^\mu\qty(\V{x}_0) \cdot
    \qty[\vphantom{\frac{0}{0}} h^\nu\V{g}_\nu\qty(\V{x}_0)]
  =h^\nu\,\KroneckerDelta{\mu}{\nu}
  =h^\mu \fullstop
\end{equation}
这里用到了局部基的\emphB{对偶关系} \eqref{eq:局部基_对偶关系}~式。

此时, 我们获得了一个四阶张量与
$h^\nu\V{g}_\nu\qty(\V{x}_0)$ 的点积。接下来讨论该项的意义。
参数域中 $\V{x}_0$ 发生 $\V{h}=h^i\,\V{e}_i$ 的变化时,
根据向量值映照 $\V{X}(\V{x})$ 的\emphB{可微性},
对应物理域中的变化为
\begin{align}
  \V{X}\qty(\V{x}_0+\V{h})-\V{X}\qty(\V{x}_0)
  &=\JacobiD{\V{X}}\qty(\V{x}_0)(\V{h})
    +\sOv{\norm{\V{h}}} \notag \\
  &\eqcolon\pdv{\V{X}}{x^\nu} \qty(\V{x}_0) \, h^\nu
    +\sOv{\norm{\V{h}}} \notag
  \intertext{代入 \ref{subsec:局部协变基}~小节中%
    \emphB{局部协变基}的定义,可有}
  &\eqcolon h^\nu\V{g}_\nu\qty(\V{x}_0)
    +\sOv{\norm{\V{h}}} \fullstop
\end{align}
代入式~\eqref{eq:张量场沿任意方向的变化_2},有
\begin{align}
  &\alspace\T{\Phi}\qty(\V{x}_0+\V{h})-\T{\Phi}\qty(\V{x}_0)
    \notag \\
  &=\qty[\vphantom{\frac{0^0}{0^0}}
      \coD{\mu}{\tc{\Phi}{^i_j^k} \qty(\V{x}_0)} \,
      \V{g}_i \qty(\V{x}_0) \tp \V{g}^j \qty(\V{x}_0)
      \tp\V{g}_k \qty(\V{x}_0) \tp \V{g}^\mu \qty(\V{x}_0)]
    \notag \\*
  &\alspace\phantom{} \cdot \qty[\vphantom{\frac{0}{0}}
      \V{X}\qty(\V{x}_0+\V{h})-\V{X}\qty(\V{x}_0)
      +\sOv{\norm{\V{h}}}]+\sOv{\norm{\V{h}}} \notag
  \intertext{合并掉一阶无穷小量\footnotemark{},可得}
  &=\qty[\vphantom{\frac{0^0}{0^0}}
      \coD{\mu}{\tc{\Phi}{^i_j^k} \qty(\V{x}_0)} \,
      \V{g}_i \qty(\V{x}_0) \tp \V{g}^j \qty(\V{x}_0)
      \tp\V{g}_k \qty(\V{x}_0) \tp \V{g}^\mu \qty(\V{x}_0)]
    \notag \\*
  &\alspace\phantom{} \cdot \qty[\vphantom{\frac{0}{0}}
      \V{X}\qty(\V{x}_0+\V{h})-\V{X}\qty(\V{x}_0)]
    +\sOv{\norm{\V{h}}} \notag \\
  &=\qty[\pdv{\T{\Phi}}{x^\mu} \qty(\V{x}_0)
    \tp \V{g}^\mu\qty(\V{x}_0)]
    \cdot \qty[\vphantom{\frac{0}{0}}
      \V{X}\qty(\V{x}_0+\V{h})-\V{X}\qty(\V{x}_0)]
    +\sOv{\norm{\V{h}}} \in\Tensors{3} \fullstop
\end{align}
\footnotetext{式中的两个 $\sOv{\norm{\V{h}}}$ 是不同的,
  前者属于 $\Rm$,后者属于 $\Tensors{3}$。}%
引入记号
\begin{equation}
  \T{\Phi} \qty(\V{x}_0)
  \tp\qty[\V{g}^\mu \pdv{x^\mu} \qty(\V{x}_0)]
  \coloneq \pdv{\T{\Phi}}{x^\mu} \qty(\V{x}_0)
    \tp \V{g}^\mu\qty(\V{x}_0) \comma
\end{equation}
再引入\emphA{梯度算子}
\begin{equation}
  \opGrad \qty(\V{x}_0) \coloneq
  \V{g}^\mu \pdv{x^\mu} \qty(\V{x}_0) \comma
\end{equation}
我们得到的结论就可以表述为
\begin{equation}
  \T{\Phi}\qty(\V{x}_0+\V{h})-\T{\Phi}\qty(\V{x}_0)
  =\qty\big(\T{\Phi}\tp\opGrad) \qty(\V{x}_0)
  \cdot \qty[\vphantom{\frac{0}{0}}
    \V{X}\qty(\V{x}_0+\V{h})-\V{X}\qty(\V{x}_0)]
  +\sOv{\norm{\V{h}}} \in\Tensors{3} \comma
\end{equation}
式中的“$\cdot$”表示点乘。以上结果称之为张量场的\emphA{可微性},
它表明,由一点处的位置移动所引起张量场的变化,
可以用该点处张量场的\emphA{梯度}(即 $\T{\Phi}\tp\opGrad$)
点乘物理空间中的位置差别来近似,误差为一阶无穷小量。
以上分析基于三阶张量。但显然,对于 $p$ 阶张量,
将会有完全一致的结果。

\subsection{方向导数}
现在来研究张量场沿 $\V{e}$ 方向的变化率(设 $\norm{\V{e}}=1$)。
取一个与 $\V{e}$ 平行的向量 $\lambda\,\V{e}$。
注意到 $\lambda\,\V{e}$ 其实就是物理空间中的位置变化,
于是根据张量场的可微性,我们有
\begin{equation}
  \T{\Phi}\qty(\V{x}_0+\lambda\,\V{e})-\T{\Phi}\qty(\V{x}_0)
  =\qty\big(\T{\Phi}\tp\opGrad) \qty(\V{x}_0)
  \cdot \qty\big(\lambda\,\V{e}) + \sOv{\lambda} \comma
\end{equation}
该式等价于
\begin{equation}
  \lim_{\lambda\to 0} \frac{\T{\Phi}\qty(\V{x}_0+\lambda\,\V{e})
    -\T{\Phi}\qty(\V{x}_0)}{\lambda}
  =\qty\big(\T{\Phi}\tp\opGrad) \qty(\V{x}_0) 
    \cdot \V{e} \fullstop
\end{equation}
我们把它定义为张量场 $\T{\Phi}(\V{x})$ 沿 $\V{e}$ 方向
的\emphA{方向导数}:
\begin{equation}
  \pdv{\T{\Phi}}{\V{e}} (\V{x}_0)
  \defeq \qty\big(\T{\Phi}\tp\opGrad) \qty(\V{x}_0) 
    \cdot \V{e} \fullstop
\end{equation}

\subsection{左梯度与右梯度} \label{subsec:左梯度与右梯度}
我们已经知道,利用梯度算子
\begin{equation}
  \opGrad \coloneq
  \V{g}^\mu \pdv{x^\mu} \qty(\V{x}_0) \comma
\end{equation}
可以把张量场的梯度表示为
\begin{equation}
  \qty\big(\T{\Phi}\tp\opGrad) \qty(\V{x}_0)
  =\T{\Phi}\qty(\V{x}_0)
    \tp\qty[\V{g}^\mu \pdv{x^\mu} \qty(\V{x}_0)]
  \coloneq \pdv{\T{\Phi}}{x^\mu} \qty(\V{x}_0)
    \tp \V{g}^\mu \qty(\V{x}_0) \fullstop
\end{equation}
“$\opGrad$”在右边,故称之为\emphA{右梯度}(简称\emphA{梯度})。
相应地,自然会有\emphA{左梯度}:
\begin{equation}
  \qty\big(\opGrad\tp\T{\Phi}) \qty(\V{x}_0)
  =\qty[\V{g}^\mu \pdv{x^\mu} \qty(\V{x}_0)]
    \tp \T{\Phi}\qty(\V{x}_0)
  \coloneq \V{g}^\mu \qty(\V{x}_0)
    \tp \pdv{\T{\Phi}}{x^\mu} \qty(\V{x}_0) \fullstop
\end{equation}
张量积不存在交换律,因而这两者是不同的。
注意,梯度运算将使张量的阶数增加一阶。

张量场的可微性可以用\emphB{左梯度}来等价表述:
\begin{equation}
  \T{\Phi}\qty(\V{x}_0+\V{h})-\T{\Phi}\qty(\V{x}_0)
  =\qty[\vphantom{\frac{0}{0}}
    \V{X}\qty(\V{x}_0+\V{h})-\V{X}\qty(\V{x}_0)]
  \cdot \qty\big(\opGrad\tp\T{\Phi}) \qty(\V{x}_0)
  +\sOv{\norm{\V{h}}} \fullstop
\end{equation}
类似地,还有方向导数:
\begin{equation}
  \pdv{\T{\Phi}}{\V{e}} (\V{x}_0)
  \defeq \qty\big(\T{\Phi}\tp\opGrad) \qty(\V{x}_0) \cdot \V{e}
  =\V{e} \cdot \qty\big(\opGrad\tp\T{\Phi}) \qty(\V{x}_0)
  \fullstop
\end{equation}

\section{场论恒等式(一)}
为了给下一节做好铺垫,本节将证明几个重要引理。

\subsection{Ricci 引理}
首先来证明两个结论:
\begin{braceEq}
  \pdv{\T{G}}{x^\mu} (\V{x})
    &=\T{0}\in\Tensors{2} \comma \\
  \pdv{\EdTensor}{x^\mu} (\V{x})
    &=\T{0}\in\Tensors[\realR^3]{3} \comma
\end{braceEq}
其中的 $\T{G}$ 和 $\EdTensor$ 分别是度量张量和 Eddington 张量。

\begin{myProof}
为方便起见,证明中我们将省去“$(\V{x})$”。

先考察度量张量的偏导数:
\begin{align}
  \pdv{\T{G}}{x^\mu}
  &=\pdv{x^\mu} \qty(g_{ij}\,\V{g}^i\tp\V{g}^j) \notag \\
  &=\coD{\mu}{g_{ij}} \, \V{g}^i\tp\V{g}^j \comma
\end{align}
式中,协变导数定义为
\begin{equation}
  \coD{\mu}{g_{ij}} \defeq \pdv{g_{ij}}{x^\mu}
    -\ChrB{\mu}{i}{s} g_{sj}
    -\ChrB{\mu}{j}{s} g_{is} \fullstop
\end{equation}
以下有两种方法证明 $\coD{\mu}{g_{ij}}=0$。

方法一利用度量的定义:
\begin{align}
  \pdv{g_{ij}}{x^\mu}
  &\defeq \pdv{x^\mu} \ipb{\V{g}_i}{\V{g}_j} \notag \\
  &=\ipb{\pdv{\V{g}_i}{x^\mu}}{\V{g}_j}
    +\ipb{\V{g}_i}{\pdv{\V{g}_j}{x^\mu}} \notag
  \intertext{根据 Christoffel 符号的定义
    (见 \ref{subsec:Christoffel符号}~小节),有}
  &=\ChrA{\mu}{i}{j}+\ChrA{\mu}{j}{i} \fullstop
\end{align}
另一方面,回忆 \eqref{eq:第二类Christoffel符号用第一类表示}~式:
\begin{equation}
  \ChrB{i}{j}{k}=\ChrA{i}{j}{l} g^{kl}\ \comma
\end{equation}
可有
\begin{braceEq}
  \ChrB{\mu}{i}{s} g_{sj}
    &=\ChrA{\mu}{i}{k} g^{sk} g_{sj}
    =\ChrA{\mu}{i}{k} \KroneckerDelta{k}{j}
    =\ChrA{\mu}{i}{j} \comma \\
  \ChrB{\mu}{j}{s} g_{is}
    &=\ChrA{\mu}{j}{k} g^{sk} g_{is}
    =\ChrA{\mu}{j}{k} \KroneckerDelta{k}{i}
    =\ChrA{\mu}{j}{i} \fullstop
\end{braceEq}
于是
\begin{equation}
  \coD{\mu}{g_{ij}}
  =\ChrA{\mu}{i}{j}+\ChrA{\mu}{j}{i}
  -\ChrA{\mu}{i}{j}-\ChrA{\mu}{j}{i}
  =0 \fullstop
\end{equation}

方法二则利用第一类 Christoffel 符号的性质
\eqref{eq:第一类Christoffel符号与度量的关系}~式:
\begin{equation}
  \ChrA{i}{j}{k}=\frac{1}{2}\, \qty(
    \pdv{g_{jk}}{x^i}+\pdv{g_{ik}}{x^j}-\pdv{g_{ij}}{x^k})
  \fullstop
\end{equation}
因而
\begin{align}
  &\alspace \ChrB{\mu}{i}{s} g_{sj}
    +\ChrB{\mu}{j}{s} g_{is} \notag \\
  &=\ChrA{\mu}{i}{j}+\ChrA{\mu}{j}{i} \notag \\
  &=\frac{1}{2}\, \qty(\pdv{g_{ij}}{x^\mu}
      +\pdv{g_{\mu j}}{x^i}-\pdv{g_{\mu i}}{x^j})
    +\frac{1}{2}\, \qty(\pdv{g_{ji}}{x^\mu}
      +\pdv{g_{\mu i}}{x^j}-\pdv{g_{\mu j}}{x^i}) \notag \\
  &=\frac{1}{2}\, \qty(\pdv{g_{ij}}{x^\mu}+\pdv{g_{ji}}{x^\mu})
  =\pdv{g_{ij}}{x^\mu} \fullstop
\end{align}
显然,立刻就有
\begin{equation}
  \coD{\mu}{g_{ij}}
  =\pdv{g_{ij}}{x^\mu}-\pdv{g_{ij}}{x^\mu} = 0 \fullstop
\end{equation}

综上,因为 $\coD{\mu} g_{ij}=0\in\realR$,所以
\begin{equation}
  \pdv{\T{G}}{x^\mu} (\V{x})=\T{0}\in\Tensors{2} \fullstop
\end{equation}
如果用其他形式的分量来表述这一结果,我们便有
\begin{equation}
  \coD{\mu}{g_{ij}} = \coD{\mu}{g^{ij}}
  =\coD{\mu}{\KroneckerDelta{i}{j}} = 0 \fullstop
\end{equation}
此结论称为\emphA{Ricci引理}。

\blankline

再来看 Eddington 张量的偏导数:
\begin{align}
  \pdv{\EdTensor}{x^\mu}
  &=\pdv{x^\mu} \qty(\LeviCivita{^i_j^k}\,
    \V{g}_i\tp\V{g}^j\tp\V{g}_k) \notag \\
  &=\coD{\mu}{\LeviCivita{^i_j^k}}\,
    \V{g}_i\tp\V{g}^j\tp\V{g}_k \comma
\end{align}
式中,
\begin{align}
  \coD{\mu}{\LeviCivita{^i_j^k}}
  \defeq \pdv{\LeviCivita{^i_j^k}}{x^\mu}
    +\ChrB{\mu}{s}{i} \LeviCivita{^s_j^k}
    -\ChrB{\mu}{j}{s} \LeviCivita{^i_s^k}
    +\ChrB{\mu}{s}{k} \LeviCivita{^i_j^s} \fullstop
  \label{eq:Eddington张量偏导数推导}
\end{align}
根据定义,$\LeviCivita{^i_j^k}=\det[\V{g}^i,\,\V{g}_j,\,\V{g}^k]$。
因此
\myPROBLEM*[2017-03-18]{det 命令写法不一致},
\begin{align}
  \pdv{\LeviCivita{^i_j^k}}{x^\mu}
  &=\pdv{x^\mu} \qty\bigg(
    \det[\V{g}^i,\,\V{g}_j,\,\V{g}^k]) \notag \\
  &=\det[\pdv{\V{g}^i}{x^\mu},\,\V{g}_j,\,\V{g}^k]
    +\det[\V{g}^i,\,\pdv{\V{g}_j}{x^\mu},\,\V{g}^k]
    +\det[\V{g}^i,\,\V{g}_j,\,\pdv{\V{g}^k}{x^\mu}] \notag
  \intertext{利用标架运动方程,有}
  &=\det[-\ChrB{\mu}{s}{i}\V{g}^s,\,\V{g}_j,\,\V{g}^k]
    +\det[\V{g}^i,\,\ChrB{\mu}{j}{s}\V{g}_s,\,\V{g}^k]
    +\det[\V{g}^i,\,\V{g}_j,\,-\ChrB{\mu}{s}{k}\V{g}^s]
    \notag
  \intertext{再利用行列式的线性性,提出系数:}
  &=-\ChrB{\mu}{s}{i}\det[\V{g}^s,\,\V{g}_j,\,\V{g}^k]
    +\ChrB{\mu}{j}{s}\det[\V{g}^i,\,\V{g}_s,\,\V{g}^k]
    -\ChrB{\mu}{s}{k}\det[\V{g}^i,\,\V{g}_j,\,\V{g}^s]
    \notag
  \intertext{代回 Eddington 张量的定义,可得}
  &=-\ChrB{\mu}{s}{i} \LeviCivita{^s_j^k}
    +\ChrB{\mu}{j}{s} \LeviCivita{^i_s^k}
    -\ChrB{\mu}{s}{k} \LeviCivita{^i_j^s} \fullstop
\end{align}
这与式~\eqref{eq:Eddington张量偏导数推导} 的后三项恰好抵消。
于是便有 $\coD{\mu}{\LeviCivita{^i_j^k}}=0$。进而
\begin{equation}
  \pdv{\EdTensor}{x^\mu} (\V{x})=\T{0}\in\Tensors[\realR^3]{3}
  \fullstop
\end{equation}

和度量张量类似,Eddington 张量其他分量的偏导数,
如 $\coD{\mu}{\LeviCivita{_{ijk}}}$、
$\coD{\mu}{\LeviCivita{^{ijk}}}$ 等,也都等于零。
此结论同样称为\emphA{Ricci 引理}。
\end{myProof}

\subsection{Leibniz 法则}
协变导数满足\emphA{Leibniz 法则}:
\begin{equation}
  \coD{\mu}{\qty\Big(\tc{\Phi}{^i_j^k}\,\tc{\Psi}{_p^q})}
  =\qty\Big(\coD{\mu}{\tc{\Phi}{^i_j^k}})\,\tc{\Psi}{_p^q}
  +\tc{\Phi}{^i_j^k}\,\qty\Big(\coD{\mu}{\tc{\Psi}{_p^q}})
  \fullstop
\end{equation}
式中,张量分量的形式可以是任意的。

\begin{myProof}
显然,$\tc{\Phi}{^i_j^k}\,\tc{\Psi}{_p^q}\in\Tensors{5}$。不妨令
\begin{equation}
  \tc{\Omega}{^i_j^k_p^q}
  =\tc{\Phi}{^i_j^k}\,\tc{\Psi}{_p^q} \fullstop
\end{equation}
则
\begin{align}
  &\alspace \coD{\mu}{\qty\Big(\tc{\Phi}{^i_j^k}\,\tc{\Psi}{_p^q})}
  =\coD{\mu}{\tc{\Omega}{^i_j^k_p^q}} \notag \\
  &\defeq \pdv{\tc{\Omega}{^i_j^k_p^q}}{x^\mu}
    +\ChrB{\mu}{s}{i} \tc{\Omega}{^s_j^k_p^q}
    -\ChrB{\mu}{j}{s} \tc{\Omega}{^i_s^k_p^q}
    +\ChrB{\mu}{s}{k} \tc{\Omega}{^i_j^s_p^q}
    -\ChrB{\mu}{p}{s} \tc{\Omega}{^i_j^k_s^q}
    +\ChrB{\mu}{s}{q} \tc{\Omega}{^i_j^k_p^s} \notag \\
  &=\pdv{x^\mu} \qty\Big(\tc{\Phi}{^i_j^k}\,\tc{\Psi}{_p^q})
    +\qty\Big(\ChrB{\mu}{s}{i} \tc{\Phi}{^s_j^k}
      -\ChrB{\mu}{j}{s} \tc{\Phi}{^i_s^k}
      +\ChrB{\mu}{s}{k} \tc{\Phi}{^i_j^s})\,\tc{\Psi}{_p^q}
    +\tc{\Phi}{^i_j^k}\,\qty\Big(
      -\ChrB{\mu}{p}{s} \tc{\Psi}{_s^q}
      +\ChrB{\mu}{s}{q} \tc{\Psi}{_p^s}) \fullstop
\end{align}
第一项偏导数自然满足乘积法则:
\begin{equation}
  \pdv{x^\mu} \qty\Big(\tc{\Phi}{^i_j^k}\,\tc{\Psi}{_p^q})
  =\pdv{\tc{\Phi}{^i_j^k}}{x^\mu}\,\tc{\Psi}{_p^q}
    +\tc{\Phi}{^i_j^k}\,\pdv{\tc{\Psi}{_p^q}}{x^\mu} \fullstop
\end{equation}
代回前一式,即有
\begin{align}
  \coD{\mu}{\qty\Big(\tc{\Phi}{^i_j^k}\,\tc{\Psi}{_p^q})}
  &=\qty\Bigg(\pdv{\tc{\Phi}{^i_j^k}}{x^\mu}
      +\ChrB{\mu}{s}{i} \tc{\Phi}{^s_j^k}
      -\ChrB{\mu}{j}{s} \tc{\Phi}{^i_s^k}
      +\ChrB{\mu}{s}{k} \tc{\Phi}{^i_j^s}) \,
    \tc{\Psi}{_p^q} \notag \\*
  &\alspace\phantom{} +\tc{\Phi}{^i_j^k} \,
    \qty\Bigg(\pdv{\tc{\Psi}{_p^q}}{x^\mu}
      -\ChrB{\mu}{p}{s} \tc{\Psi}{_s^q}
      +\ChrB{\mu}{s}{q} \tc{\Psi}{_p^s}) \notag \\
  &\defeq \qty\Big(\coD{\mu}{\tc{\Phi}{^i_j^k}})\,\tc{\Psi}{_p^q}
    +\tc{\Phi}{^i_j^k}\,\qty\Big(\coD{\mu}{\tc{\Psi}{_p^q}})
    \fullstop
\end{align}
\end{myProof}

现在来考虑 $\coD{\mu}{\qty(\tc{\Phi}{^i_j^k}\,\tc{\Psi}{_k^q})}$,
注意其中的 $k$ 是哑标。若按照 Leibniz 法则,似乎有
\begin{align}
  \coD{\mu}{\qty\Big(\tc{\Phi}{^i_j^k}\,\tc{\Psi}{_k^q})}
  &=\qty\Big(\coD{\mu}{\tc{\Phi}{^i_j^k}})\,\tc{\Psi}{_k^q}
    +\tc{\Phi}{^i_j^k}\,\qty\Big(\coD{\mu}{\tc{\Psi}{_k^q}})
    \notag \\
  &=\qty\Bigg(\pdv{\tc{\Phi}{^i_j^k}}{x^\mu}
      +\ChrB{\mu}{s}{i} \tc{\Phi}{^s_j^k}
      -\ChrB{\mu}{j}{s} \tc{\Phi}{^i_s^k}
      +\ChrB{\mu}{s}{k} \tc{\Phi}{^i_j^s}) \,
    \tc{\Psi}{_k^q} \notag \\*
  &\alspace\phantom{} +\tc{\Phi}{^i_j^k} \,
    \qty\Bigg(\pdv{\tc{\Psi}{_k^q}}{x^\mu}
      -\ChrB{\mu}{k}{s} \tc{\Psi}{_s^q}
      +\ChrB{\mu}{s}{q} \tc{\Psi}{_k^s}) \notag \\
  &=\pdv{x^\mu} \qty\Big(\tc{\Phi}{^i_j^k}\,\tc{\Psi}{_k^q})
    +\qty\Big(\ChrB{\mu}{s}{i} \tc{\Phi}{^s_j^k}
      -\ChrB{\mu}{j}{s}\tc{\Phi}{^i_s^k})\,\tc{\Psi}{_k^q}
    +\tc{\Phi}{^i_j^k}\,
      \qty\Big(\ChrB{\mu}{s}{q}\tc{\Psi}{_k^s}) \notag \\*
  &\alspace\phantom{}
    +\ChrB{\mu}{s}{k} \tc{\Phi}{^i_j^s}\,\tc{\Psi}{_k^q}
    -\ChrB{\mu}{k}{s} \tc{\Phi}{^i_j^k}\,\tc{\Psi}{_s^q}
  \fullstop \label{eq:带哑标的Leibniz法则}
\end{align}
而根据定义,则
\begin{equation}
  \coD{\mu}{\qty\Big(\tc{\Phi}{^i_j^k}\,\tc{\Psi}{_k^q})}
  \defeq \pdv{x^\mu} \qty\Big(\tc{\Phi}{^i_j^k}\,\tc{\Psi}{_k^q})
    +\qty\Big(\ChrB{\mu}{s}{i} \tc{\Phi}{^s_j^k}
      -\ChrB{\mu}{j}{s} \tc{\Phi}{^i_s^k})\,\tc{\Psi}{_k^q}
    +\tc{\Phi}{^i_j^k}\,
      \qty\Big(\ChrB{\mu}{s}{q} \tc{\Psi}{_k^s})
  \fullstop
\end{equation}
很明显,式~\eqref{eq:带哑标的Leibniz法则} 中多了两项。
不过稍作计算,就可知道
\begin{align}
  &\alspace \ChrB{\mu}{s}{k}
    \tc{\Phi}{^i_j^s}\,\tc{\Psi}{_k^q}
  -\ChrB{\mu}{k}{s}
    \tc{\Phi}{^i_j^k}\,\tc{\Psi}{_s^q} \notag
  \intertext{$k$ 和 $s$ 都是哑标,不妨在第二项中将二者交换:}
  &=\ChrB{\mu}{s}{k} \tc{\Phi}{^i_j^s}\,\tc{\Psi}{_k^q}
    -\ChrB{\mu}{s}{k} \tc{\Phi}{^i_j^s}\,\tc{\Psi}{_k^q}
  =0 \fullstop
\end{align}
可见,Leibniz 法则经受住了考验。

\blankline

把 Ricci 引理和 Leibniz 法则联合起来,便有
\begin{braceEq}
  \coD{\mu}{\qty\Big(g_{ij}\,\tc{\Psi}{^p_q})}
  &=\qty\Big(\coD{\mu}{g_{ij}})\,\tc{\Psi}{^p_q}
    +g_{ij} \qty\Big(\coD{\mu}{\tc{\Psi}{^p_q}})
  =g_{ij}\,\coD{\mu}{\tc{\Psi}{^p_q}} \comma \\
  \coD{\mu}{\qty\Big(\LeviCivita{^i_j^k}\,\tc{\Psi}{^p_q})}
  &=\qty\Big(\coD{\mu}{\LeviCivita{^i_j^k}})\,\tc{\Psi}{^p_q}
    +\LeviCivita{^i_j^k} \qty\Big(\coD{\mu}{\tc{\Psi}{^p_q}})
  =\LeviCivita{^i_j^k}\,\coD{\mu}{\tc{\Psi}{^p_q}} \fullstop
\end{braceEq}
这说明度量张量和 Eddington 张量类似常数,可以提到协变导数的外面。

\subsection{混合协变导数}
与混合偏导数定理类似,协变导数满足
\begin{equation}
  \coD{\nu}{\coD{\mu}{\tc{\Phi}{^i_j^k}}}
  =\coD{\mu}{\coD{\nu}{\tc{\Phi}{^i_j^k}}} \fullstop
  \label{eq:混合协变导数定理}
\end{equation}
不必多说,张量分量依然可以任意选取。
只是需要注意,该定理只在体积上的张量场场论中成立。

\myPROBLEM{体积上张量场场论}

\begin{myProof}
首先计算张量场\emphB{整体}的一阶偏导数:
\begin{equation}
  \pdv{\T{\Phi}}{x^\mu}
  =\pdv{x^\mu} \qty\Big(
    \tc{\Phi}{^i_j^k} \V{g}_i\tp\V{g}^j\tp\V{g}_k)
  =\coD{\mu}{\tc{\Phi}{^i_j^k}} \, \V{g}_i\tp\V{g}^j\tp\V{g}_k
  \in\Tensors{3} \fullstop
\end{equation}
再求一次偏导数,可有
\begin{align}
  \pdv{\T{\Phi}}{x^\nu}{x^\mu}
  \coloneq \pdv{x^\nu} \qty(\pdv{\T{\Phi}}{x^\mu})
  =\pdv{x^\nu} \qty\Big(\coD{\mu}{\tc{\Phi}{^i_j^k}} \,
    \V{g}_i\tp\V{g}^j\tp\V{g}_k) \fullstop
\end{align}
请注意,括号里的张量带有一个\emphB{独立指标} $\mu$。
按照极限分析,有
\begin{align}
  &\alspace \pdv{x^\nu} \qty\Big(\coD{\mu}{\tc{\Phi}{^i_j^k}} \,
    \V{g}_i\tp\V{g}^j\tp\V{g}_k) \notag \\
  &=\pdv{x^\nu}\qty\Big(\coD{\mu}{\tc{\Phi}{^i_j^k}}) \,
    \V{g}_i\tp\V{g}^j\tp\V{g}_k
    +\coD{\mu}{\tc{\Phi}{^i_j^k}} \qty(
      \pdv{\V{g}_i}{x^\nu} \tp\V{g}^j\tp\V{g}_k
      +\V{g}_i\tp\pdv{\V{g}^j}{x^\nu} \tp\V{g}_k
      +\V{g}_i\tp\V{g}^j\tp\pdv{\V{g}_k}{x^\nu}) \notag \\
  &=\pdv{x^\nu}\qty\Big(\coD{\mu}{\tc{\Phi}{^i_j^k}}) \,
    \V{g}_i\tp\V{g}^j\tp\V{g}_k
    +\coD{\mu}{\tc{\Phi}{^i_j^k}} \qty\Big(
      \ChrB{\nu}{i}{s} \V{g}_s\tp\V{g}^j\tp\V{g}_k
      -\ChrB{\nu}{s}{j} \V{g}_i\tp\V{g}^s\tp\V{g}_k
      +\ChrB{\nu}{k}{s} \V{g}_i\tp\V{g}^j\tp\V{g}_s)
    \notag \\
  &=\qty[\pdv{x^\nu} \qty\Big(\coD{\mu}{\tc{\Phi}{^i_j^k}})
      +\ChrB{\nu}{s}{i} \coD{\mu}{\tc{\Phi}{^s_j^k}}
      -\ChrB{\nu}{j}{s} \coD{\mu}{\tc{\Phi}{^i_s^k}}
      +\ChrB{\nu}{s}{k} \coD{\mu}{\tc{\Phi}{^i_j^s}}] \,
    \V{g}_i\tp\V{g}^j\tp\V{g}_k \fullstop
  \label{eq:混合协变导数推导}
\end{align}
但是 $\coD{\mu}{\tc{\Phi}{^i_j^k}}$ 本身带有 4 个指标,因而
\begin{equation}
  \coD{\nu}{\qty\Big(\coD{\mu}{\tc{\Phi}{^i_j^k}})}
  \defeq \pdv{x^\nu} \qty\Big(\coD{\mu}{\tc{\Phi}{^i_j^k}})
    +\ChrB{\nu}{s}{i} \coD{\mu}{\tc{\Phi}{^s_j^k}}
    -\ChrB{\nu}{j}{s} \coD{\mu}{\tc{\Phi}{^i_s^k}}
    +\ChrB{\nu}{s}{k} \coD{\mu}{\tc{\Phi}{^i_j^s}}
    -\ChrB{\nu}{\mu}{s} \coD{s}{\tc{\Phi}{^i_j^k}}
  \fullstop
\end{equation}
代入 \eqref{eq:混合协变导数推导}~式,可得
\begin{equation}
  \pdv{\T{\Phi}}{x^\nu}{x^\mu}
  =\pdv{x^\nu} \qty\Big(\coD{\mu}{\tc{\Phi}{^i_j^k}} \,
    \V{g}_i\tp\V{g}^j\tp\V{g}_k)
  =\qty\Big(\coD{\nu}{\coD{\mu}{\tc{\Phi}{^i_j^k}}}
    +\ChrB{\nu}{\mu}{s} \coD{s}{\tc{\Phi}{^i_j^k}}) \,
    \V{g}_i\tp\V{g}^j\tp\V{g}_k \fullstop
\end{equation}
同理,
\begin{equation}
  \pdv{\T{\Phi}}{x^\mu}{x^\nu}
  =\qty\Big(\coD{\mu}{\coD{\nu}{\tc{\Phi}{^i_j^k}}}
    +\ChrB{\mu}{\nu}{s} \coD{s}{\tc{\Phi}{^i_j^k}}) \,
    \V{g}_i\tp\V{g}^j\tp\V{g}_k \fullstop
\end{equation}

根据 Christoffel 符号的性质,
\begin{equation}
  \ChrB{\nu}{\mu}{s}=\ChrB{\mu}{\nu}{s}
  \semicolon
\end{equation}
而按照\myPROBLEM{一般赋范线性空间上的微分学}, 
当张量场具有足够正则性时, 成立
\begin{equation}
  \pdv{\T{\Phi}}{x^\nu}{x^\mu}=\pdv{\T{\Phi}}{x^\mu}{x^\nu}
  \fullstop
\end{equation}
这样就可得到
\begin{equation}
  \coD{\nu}{\coD{\mu}{\tc{\Phi}{^i_j^k}}}
  =\coD{\mu}{\coD{\nu}{\tc{\Phi}{^i_j^k}}} \fullstop
\end{equation}
\end{myProof}

\section{场论恒等式(二)}
本节将给出微分形式张量场场论中的若干恒等式,以及它们的推演过程。

\subsection{微分算子}
在 \ref{subsec:左梯度与右梯度}~小节中,
我们已经定义了\emphA{左梯度}
\begin{mySubEq}
  \begin{align}
    \qty\big(\opGrad\tp\T{\Phi}) (\V{x})
      &\defeq \qty[\V{g}^\mu\pdv{x^\mu} (\V{x})]\tp\T{\Phi}(\V{x})
      \coloneq \V{g}^\mu(\V{x})\tp\pdv{\T{\Phi}}{x^\mu} (\V{x})
    \intertext{和\emphA{(右)梯度}}
    \qty\big(\T{\Phi}\tp\opGrad) (\V{x})
      &\defeq \T{\Phi}(\V{x})\tp\qty[\V{g}^\mu\pdv{x^\mu} (\V{x})]
      \coloneq \pdv{\T{\Phi}}{x^\mu} (\V{x})\tp\V{g}^\mu(\V{x})
      \fullstop
  \end{align}
\end{mySubEq}
如果 $\T{\Phi}(\V{x})$ 是 $r$ 阶张量,则左右梯度都是 $r+1$ 阶张量。
%
%\blankline

类似地,我们还可以定义\emphA{左散度}
\begin{mySubEq}
  \begin{align}
    \qty\big(\opGrad\cdot\T{\Phi}) (\V{x})
      &\defeq \qty[\V{g}^\mu\pdv{x^\mu} (\V{x})]\cdot\T{\Phi}(\V{x})
      \coloneq \V{g}^\mu(\V{x})\cdot\pdv{\T{\Phi}}{x^\mu} (\V{x})
    \intertext{和\emphA{右散度}}
    \qty\big(\T{\Phi}\cdot\opGrad) (\V{x})
      &\defeq \T{\Phi}(\V{x})\cdot\qty[\V{g}^\mu\pdv{x^\mu} (\V{x})]
      \coloneq \pdv{\T{\Phi}}{x^\mu} (\V{x})\cdot\V{g}^\mu(\V{x})
      \fullstop
  \end{align}
\end{mySubEq}
如果 $\T{\Phi}(\V{x})$ 是 $r$ 阶张量,则左右散度都是 $r-1$ 阶张量。
%
%\blankline

当然,如果底空间是 $\realR^3$,
即 $\T{\Phi}(\V{x})\in\Tensors[\realR^3]{r}$,
还不能忘了定义\emphA{左旋度}
\begin{mySubEq}
  \begin{align}
    \qty\big(\opGrad\cp\T{\Phi}) (\V{x})
      &\defeq \qty[\V{g}^\mu\pdv{x^\mu} (\V{x})]\cp\T{\Phi}(\V{x})
      \coloneq \V{g}^\mu(\V{x})\cp\pdv{\T{\Phi}}{x^\mu} (\V{x})
    \intertext{和\emphA{右旋度}}
    \qty\big(\T{\Phi}\cp\opGrad) (\V{x})
      &\defeq \T{\Phi}(\V{x})\cp\qty[\V{g}^\mu\pdv{x^\mu} (\V{x})]
      \coloneq \pdv{\T{\Phi}}{x^\mu} (\V{x})\cp\V{g}^\mu(\V{x})
      \fullstop
  \end{align}
\end{mySubEq}
旋度不改变张量的阶数,即左右旋度仍属于 $\Tensors[\realR^3]{r}$。

\blankline

左右梯度、散度和梯度都是张量场中常用的微分算子。
向量微积分中的梯度、散度和梯度,其实就是一阶张量的特殊情况。

\subsection{推演举例}
首先是为人熟知的“梯度场无旋,旋度场无源”:
\begin{braceEq*}
  {\forall\,\T{\Phi}\in\Tensors[\realR^3]{r},\quad\text{有\ }}
  &\opGrad \cp \qty\big(\opGrad\tp\T{\Phi})=\T{0}
    \in\Tensors[\realR^3]{r+1} \comma \\
  &\opGrad \cdot \qty\big(\opGrad\cp\T{\Phi})=\T{0}
    \in\Tensors[\realR^3]{r-1} \fullstop
\end{braceEq*}

\begin{myProof}
不失一般性,我们设 $\T{\Phi}$ 是一个三阶张量。
代入上一小节中梯度和旋度的定义,可有
\begin{align}
  \opGrad \cp \qty\big(\opGrad\tp\T{\Phi})
  &=\qty(\V{g}^\nu\pdv{x^\nu})
    \cp \qty(\V{g}^\mu\tp\pdv{\T{\Phi}}{x^\mu}) \notag \\
  &=\qty(\V{g}^\nu\pdv{x^\nu})
    \cp \qty\Big(\coD{\mu}{\tc{\Phi}{^i_j^k}}
      \V{g}^\mu\tp\V{g}_i\tp\V{g}^j\tp\V{g}_k) \notag \\
  &=\V{g}^\nu\cp\pdv{x^\nu}
    \qty\Big(\coD{\mu}{\tc{\Phi}{^i_j^k}}
      \V{g}^\mu\tp\V{g}_i\tp\V{g}^j\tp\V{g}_k) \notag
  \intertext{与 \eqref{eq:混合协变导数推导}~式不同,第二个括号中的 
    $\mu$、$i$、$j$、$k$ 都是\emphB{哑标},
    所以偏导数可以直接用协变导数表示,而不会出现多余的
    Christoffel 符号:}
  &=\V{g}^\nu\cp \qty[ \vphantom{\frac{0}{0}}
      \qty(\coD{\nu}{\coD{\mu}{\tc{\Phi}{^i_j^k}}}) \,
      \V{g}^\mu\tp\V{g}_i\tp\V{g}^j\tp\V{g}_k] \notag
  \intertext{按照叉乘的定义(见 \ref{sec:叉乘}~节),
    $\V{g}^\nu$ 将与构成简单张量的第一个基向量相乘,即}
  &=\qty(\coD{\nu}{\coD{\mu}{\tc{\Phi}{^i_j^k}}})
    \qty(\V{g}^\nu\cp\V{g}^\mu)
    \tp\V{g}_i\tp\V{g}^j\tp\V{g}_k \notag
  \intertext{利用 Levi-Civita 记号展开叉乘项,有}
  &=\LeviCivita{^{\mu\nu s}}
    \qty(\coD{\nu}{\coD{\mu}{\tc{\Phi}{^i_j^k}}}) \,
    \V{g}_s\tp\V{g}_i\tp\V{g}^j\tp\V{g}_k \fullstop
\end{align}
考虑交换哑标 $\mu$、$\nu$,结果必然保持不变。
但是 Eddington 张量关于指标 $\mu\nu$ 反对称\footnote{
  根据式~\eqref{eq:Levi-Civita记号的定义},
  Levi-Civita 记号由行列式定义,而行列式交换两列将改变符号。},
即 $\LeviCivita{^{\mu\nu s}}=-\LeviCivita{^{\nu\mu s}}$;
另一方面,混合协变导数却又满足
$\coD{\nu}{\coD{\mu}{\tc{\Phi}{^i_j^k}}}
=\coD{\mu}{\coD{\nu}{\tc{\Phi}{^i_j^k}}}$,
因此总的结果将变为其相反数。这样,结果只可能为零,即
\begin{equation}
  \opGrad \cp \qty\big(\opGrad\tp\T{\Phi})=\T{0} \fullstop
\end{equation}

同理,也可证明“旋度场无源”:
\begin{align}
  \opGrad \cdot \qty\big(\opGrad\cp\T{\Phi})
  &=\qty(\V{g}^\nu\pdv{x^\nu})
    \cdot \qty(\V{g}^\mu\cp\pdv{\T{\Phi}}{x^\mu}) \notag \\
  &=\qty(\V{g}^\nu\pdv{x^\nu})
    \cdot \qty[ \vphantom{\frac{0}{0}}
      \V{g}^\mu\cp \qty(\coD{\mu}{\tc{\Phi}{^i_j^k}} \,
        \V{g}_i\tp\V{g}^j\tp\V{g}_k)] \notag \\
  &=\V{g}^\nu \cdot \pdv{x^\nu} \qty[ \vphantom{\frac{0}{0}}
      \V{g}^\mu\cp \qty(\coD{\mu}{\tc{\Phi}{^i_j^k}} \,
        \V{g}_i\tp\V{g}^j\tp\V{g}_k)] \notag \\
  &=\V{g}^\nu \cdot \pdv{x^\nu} \qty[ \vphantom{\frac{0}{0}}
      \coD{\mu}{\tc{\Phi}{^i_j^k}} \,
      \qty(\V{g}^\mu\cp\V{g}_i)\tp\V{g}^j\tp\V{g}_k] \notag \\
  &=\V{g}^\nu \cdot \pdv{x^\nu} \qty( \vphantom{\frac{0}{0}}
      \coD{\mu}{\tc{\Phi}{^i_j^k} \LeviCivita{^\mu_i^s}} \,
      \V{g}_s\tp\V{g}^j\tp\V{g}_k) \notag
  \intertext{$\mu$、$s$、$i$、$j$、$k$ 全部是哑标,因此}
  &=\V{g}^\nu \cdot \qty[ \vphantom{\frac{0}{0}}
    \qty(\coD{\nu}{
      \coD{\mu}{\tc{\Phi}{^i_j^k} \LeviCivita{^\mu_i^s}}}) \,
    \V{g}_s\tp\V{g}^j\tp\V{g}_k] \notag
  \intertext{点乘之后出来 Kronecker δ:}
  &=\qty(\coD{\nu}{
      \coD{\mu}{\tc{\Phi}{^i_j^k} \LeviCivita{^\mu_i^s}}}) \,
    \KroneckerDelta{\nu}{s} \, \V{g}^j\tp\V{g}_k \notag \\
  &=\LeviCivita{^\mu_i^\nu} \qty(\coD{\nu}{
      \coD{\mu}{\tc{\Phi}{^i_j^k}}}) \,
    \V{g}^j\tp\V{g}_k \fullstop
\end{align}
交换哑标 $\mu$、$\nu$,仿上,便有
\begin{equation}
  \opGrad \cdot \qty\big(\opGrad\cp\T{\Phi})=\T{0} \fullstop
\end{equation}
\end{myProof}

\blankline

接下来回忆一下\emphB{向量场}旋度的复合
\begin{equation}
  \forall\,\V{A}\in\realR^3,\quad
  \opGrad\cp \qty(\opGrad\cp\V{A})
  =\opGrad\qty\big(\opGrad\cdot\V{A})-\opLap\V{A}
  \in\realR^3 \comma
\end{equation}
式中,$\opLap$ 称为\emphA{Laplace 算子},其定义为
\begin{equation}
  \opLap\V{A} \defeq
  \opGrad\cdot\qty\big(\opGrad\tp\V{A}) \fullstop
\end{equation}
推广到张量场上,我们有如下两式:
\begin{braceEq*}
  {\forall\,\T{\Phi}\in\Tensors[\realR^3]{r},\quad}
  \opGrad\cp\qty\big(\opGrad\cp\T{\Phi})
  &=\opGrad\tp\qty\big(\opGrad\cdot\T{\Phi})
    -\opLap\T{\Phi} \comma \label{eq:张量场旋度复合_1} \\
  \qty\big(\T{\Phi}\cp\opGrad)\cp\opGrad
  &=\qty\big(\T{\Phi}\cdot\opGrad)\tp\opGrad
    -\T{\Phi}\opLap \comma \label{eq:张量场旋度复合_2}
\end{braceEq*}
其中,Laplace 算子的定义为
\begin{braceEq}
  \opLap\T{\Phi} &\defeq
    \opGrad\cdot\qty\big(\opGrad\tp\T{\Phi}) \comma \\
  \T{\Phi}\opLap &\defeq
    \qty\big(\T{\Phi}\tp\opGrad)\cdot\opGrad \fullstop
\end{braceEq}

\begin{myProof}
同样,我们以三阶张量为例进行计算。
代入旋度的定义,有
\begin{align}
  \opGrad \cp \qty\big(\opGrad\cp\T{\Phi})
  &=\qty(\V{g}^\nu\pdv{x^\nu})
    \cp \qty(\V{g}^\mu\cp\pdv{\T{\Phi}}{x^\mu}) \notag \\
  &=\V{g}^\nu \cp \pdv{x^\nu} \qty[ \vphantom{\frac{0}{0}}
      \V{g}^\mu\cp \qty(\coD{\mu}{\tc{\Phi}{^i_j^k}} \,
        \V{g}_i\tp\V{g}^j\tp\V{g}_k)] \notag \\
  &=\V{g}^\nu \cp \pdv{x^\nu} \qty( \vphantom{\frac{0}{0}}
      \coD{\mu}{\tc{\Phi}{^i_j^k} \LeviCivita{^\mu_i^s}} \,
      \V{g}_s\tp\V{g}^j\tp\V{g}_k) \notag \\
  &=\V{g}^\nu \cp \qty[ \vphantom{\frac{0}{0}}
    \qty(\coD{\nu}{
      \coD{\mu}{\tc{\Phi}{^i_j^k} \LeviCivita{^\mu_i^s}}}) \,
    \V{g}_s\tp\V{g}^j\tp\V{g}_k] \notag \\
  &=\LeviCivita{^\mu_i^s}\LeviCivita{^\nu_s^t} \qty(\coD{\nu}{
      \coD{\mu}{\tc{\Phi}{^i_j^k}}}) \,
    \V{g}_t\tp\V{g}^j\tp\V{g}_k \notag
  \intertext{为了使用“\emphB{前前后后,里里外外}”之法则
    (见 \ref{subsec:两种度量的关系}~小节),
    需要交换第二个 Eddington 张量的指标 $s$ 挪到最后,
    不要忘了添上负号:}
  &=-\LeviCivita{^\mu_i^s}\LeviCivita{^\nu^t_s} \qty(\coD{\nu}{
      \coD{\mu}{\tc{\Phi}{^i_j^k}}}) \,
    \V{g}_t\tp\V{g}^j\tp\V{g}_k \notag
  \intertext{这样就可以顺利用上口诀:}
  &=-\qty(g^{\mu\nu}\KroneckerDelta{t}{i}
      -\KroneckerDelta{\nu}{i}g^{\mu t})
    \qty(\coD{\nu}{
      \coD{\mu}{\tc{\Phi}{^i_j^k}}}) \,
    \V{g}_t\tp\V{g}^j\tp\V{g}_k \fullstop
\end{align}
接下来,形式上引入\emphA{逆变导数}
\begin{equation}
  \ctrD{\mu}{} \coloneq g^{\mu\nu}\coD{\nu}{} \comma
\end{equation}
则上式可化为
\begin{align}
  \opGrad \cp \qty\big(\opGrad\cp\T{\Phi})
  &=\KroneckerDelta{\nu}{i}g^{\mu t} \qty(\coD{\nu}{
      \coD{\mu}{\tc{\Phi}{^i_j^k}}}) \,
    \V{g}_t\tp\V{g}^j\tp\V{g}_k
    -g^{\mu\nu}\KroneckerDelta{t}{i} \qty(\coD{\nu}{
      \coD{\mu}{\tc{\Phi}{^i_j^k}}}) \,
    \V{g}_t\tp\V{g}^j\tp\V{g}_k \notag \\
  &=\qty(\coD{i}{\ctrD{t}{\tc{\Phi}{^i_j^k}}}) \,
      \V{g}_t\tp\V{g}^j\tp\V{g}_k
    -\qty(\ctrD{\mu}{\coD{\mu}{\tc{\Phi}{^i_j^k}}}) \,
      \V{g}_i\tp\V{g}^j\tp\V{g}_k \fullstop
  \label{eq:张量场旋度复合推导}
\end{align}
等式右边的第一项为
\begin{align}
  \opGrad \tp \qty\big(\opGrad\cdot\T{\Phi})
  &=\qty(\V{g}^\nu\pdv{x^\nu})
    \tp \qty(\V{g}^\mu\cdot\pdv{\T{\Phi}}{x^\mu}) \notag \\
  &=\V{g}^\nu \tp \pdv{x^\nu} \qty[ \vphantom{\frac{0}{0}}
      \V{g}^\mu \cdot \qty(\coD{\mu}{\tc{\Phi}{^i_j^k}} \,
        \V{g}_i\tp\V{g}^j\tp\V{g}_k)] \notag \\
  &=\V{g}^\nu \tp \pdv{x^\nu} \qty( \vphantom{\frac{0}{0}}
      \coD{\mu}{\tc{\Phi}{^i_j^k} \KroneckerDelta{\mu}{i}} \,
      \V{g}^j\tp\V{g}_k) \notag \\
  &=\V{g}^\nu \tp \qty[ \vphantom{\frac{0}{0}}
    \qty(\coD{\nu}{
      \coD{\mu}{\tc{\Phi}{^i_j^k} \KroneckerDelta{\mu}{i}}}) \,
    \V{g}^j\tp\V{g}_k] \notag \\
  &=\KroneckerDelta{\mu}{i} g^{\nu t} \qty(\coD{\nu}{
      \coD{\mu}{\tc{\Phi}{^i_j^k}}}) \,
    \V{g}_t\tp\V{g}^j\tp\V{g}_k \notag \\
  &=\qty(\ctrD{t}{\coD{i}{\tc{\Phi}{^i_j^k}}}) \,
      \V{g}_t\tp\V{g}^j\tp\V{g}_k \notag
  \intertext{根据式~\eqref{eq:混合协变导数定理},
    协变导数可以交换顺序;而逆变导数无非就是利用度量玩了下
    “指标升降游戏”,理应可以交换:}
  &=\qty(\coD{i}{\ctrD{t}{\tc{\Phi}{^i_j^k}}}) \,
    \V{g}_t\tp\V{g}^j\tp\V{g}_k \comma
\end{align}
而第二项为
\begin{align}
  \opLap\T{\Phi}
  \defeq \opGrad \cdot \qty\big(\opGrad\tp\T{\Phi})
  &=\qty(\V{g}^\nu\pdv{x^\nu})
    \cdot \qty(\V{g}^\mu\tp\pdv{\T{\Phi}}{x^\mu}) \notag \\
  &=\V{g}^\nu \cdot \pdv{x^\nu} \qty[ \vphantom{\frac{0}{0}}
      \V{g}^\mu \tp \qty(\coD{\mu}{\tc{\Phi}{^i_j^k}} \,
        \V{g}_i\tp\V{g}^j\tp\V{g}_k)] \notag \\
  &=\V{g}^\nu \cdot \pdv{x^\nu} \qty( \vphantom{\frac{0}{0}}
      \coD{\mu}{\tc{\Phi}{^i_j^k}} \,
      \V{g}^\mu\tp\V{g}_i\tp\V{g}^j\tp\V{g}_k) \notag \\
  &=\V{g}^\nu \cdot \qty[ \vphantom{\frac{0}{0}}
    \qty(\coD{\nu}{
      \coD{\mu}{\tc{\Phi}{^i_j^k}}}) \,
    \V{g}^\mu\tp\V{g}_i\tp\V{g}^j\tp\V{g}_k] \notag \\
  &=g^{\mu\nu} \qty(\coD{\nu}{
      \coD{\mu}{\tc{\Phi}{^i_j^k}}}) \,
    \V{g}_i\tp\V{g}^j\tp\V{g}_k \notag \\
  &=\qty(\ctrD{\mu}{\coD{\mu}{\tc{\Phi}{^i_j^k}}}) \,
      \V{g}_u\tp\V{g}^j\tp\V{g}_k \fullstop
\end{align}
将它们与 \eqref{eq:张量场旋度复合推导}~式比较,可得
\begin{equation}
  \opGrad\cp\qty\big(\opGrad\cp\T{\Phi})
  =\opGrad\tp\qty\big(\opGrad\cdot\T{\Phi}) 
    -\opLap\T{\Phi} \fullstop
\end{equation}

式~\eqref{eq:张量场旋度复合_2} 可以完全类似地证明,此处不再赘述。
\end{myProof}

    % 非完整基理论
    \chapter{非完整基理论}
\section{完整基与非完整基的概念}
在 \ref{sec:局部基}~节中,我们利用曲线坐标系 $\V{X}(\V{x})$
构造了 $\Rm$ 上的一组(局部协变)基
\begin{equation}
  \qty{\V{g}_i(\V{x})=\pdv{\V{X}}{x^i} (\V{x})}^m_{i=1}
  \subset\Rm \comma
\end{equation}
它们称为\emphA{完整基}\idx{完整基}。
与之对应,不是由曲线坐标系诱导的基,
称为\emphA{非完整基}\idx{非完整基}。

\begin{figure}[h]
  \centering
  \includegraphics{images/holonomic-nonholonomic-basis.png}
  \caption{完整基与非完整基}
  \label{fig:完整基与非完整基}
\end{figure}

如图~\ref{fig:完整基与非完整基},
$x^i$-线的\emphB{切向量}构成一组局部协变基
$\qty{\V{g}_i(\V{x})}^m_{i=1}$,
它和它的对偶 $\qty{\V{g}^i(\V{x})}^m_{i=1}$ 都是完整基。
除此以外,我们当然可以选取另外的基 $\qty{\V{g}_{(i)}(\V{x})}^m_{i=1}$
和 $\qty{\V{g}^{(i)}(\V{x})}^m_{i=1}$,它们不是由曲线坐标系诱导,
因而是非完整基。

\section{非完整基下的张量梯度} \label{sec:非完整基下的张量梯度}
下面我们来考察张量梯度在非完整基下的表达形式。
在 \ref{sec:张量场的梯度}~节中,我们已经推导出了张量场的(右)梯度:
\begin{equation}
  \qty\big(\T{\Phi}\tp\opGrad) (\V{x})
  \defeq \pdv{\T{\Phi}}{x^\mu} (\V{x})
    \tp \V{g}^\mu(\V{x})
  =\coD{\mu}{\tc{\Phi}{^i_j^k} (\V{x})} \,
    \V{g}_i (\V{x}) \tp \V{g}^j (\V{x})
    \tp \V{g}_k (\V{x}) \tp \V{g}^\mu (\V{x}) \fullstop
  \label{eq:张量梯度_非完整基}
\end{equation}
这是一个四阶张量,对应的张量分量可记作
\begin{equation}
  \tc{\qty\big(\T{\Phi}\tp\opGrad)}{^i_j^k_\mu} (\V{x})
  \coloneq \coD{\mu}{\tc{\Phi}{^i_j^k} (\V{x})} \fullstop
  \label{eq:张量梯度分量_非完整基}
\end{equation}
除此以外,其他的基当然也可以用来表示该张量,
比如前文提到过的 $\qty{\V{g}_{(i)}(\V{x})}^m_{i=1}$
和 $\qty{\V{g}^{(i)}(\V{x})}^m_{i=1}$,它们都是非完整基。

非完整基与完整基之间的关系,可以利用
\ref{subsec:相对不同基的张量分量之间的关系}~小节中引入的%
\emphB{坐标转换关系}\idx{坐标转换关系}来获得:
\begin{braceEq}
  \V{g}_{(i)} (\V{x})
    &= c^k_{(i)} (\V{x}) \, \V{g}_k (\V{x}) \comma \\
  \V{g}^{(i)} (\V{x})
    &= c_k^{(i)} (\V{x}) \, \V{g}^k (\V{x}) \semicolon \\
  \V{g}_i (\V{x})
    &= c^{(k)}_i (\V{x}) \, \V{g}_{(k)} (\V{x}) \comma \\
  \V{g}^i (\V{x})
    &= c_{(k)}^i (\V{x}) \, \V{g}^{(k)} (\V{x}) \fullstop
\end{braceEq}
\myPROBLEM[2017-01-20]{坐标转换关系}\\
其中的基转换系数都是已知量,它们的定义如下:\footnote{
  只有两个基转换系数的原因是内积具有交换律。}
\begin{braceEq}
  c^j_{(i)} (\V{x})
    &\coloneq \ipb{\V{g}_{(i)} (\V{x})}{\V{g}^j (\V{x})} \comma \\
  c_j^{(i)} (\V{x})
    &\coloneq \ipb{\V{g}^{(i)} (\V{x})}{\V{g}_j (\V{x})} \fullstop
\end{braceEq}
代入 \eqref{eq:张量梯度_非完整基}~式,可有\footnote{
  这里我们省略了“$(\V{x})$”。}
\begin{align}
  \T{\Phi}\tp\opGrad
  &=\coD{\mu}{\tc{\Phi}{^i_j^k}} \,
    \qty(\vphantom{\frac{0}{0}}
      \V{g}_i \tp \V{g}^j \tp \V{g}_k \tp \V{g}^\mu) \notag \\
  &=\coD{\mu}{\tc{\Phi}{^i_j^k} (\V{x})} \,
    \qty[\vphantom{\frac{0^0}{0^0}}
      \qty(c^{(p)}_i \, \V{g}_{(p)})
      \tp \qty(c_{(q)}^j \, \V{g}^{(q)})
      \tp \qty(c^{(r)}_k \, \V{g}_{(r)})
      \tp \qty(\vphantom{\frac{0}{0}}
        c_{(\alpha)}^\mu \, \V{g}^{(\alpha)})] \notag
  \intertext{根据线性性,提出系数:}
  &=\qty(c^{(p)}_i c_{(q)}^j c^{(r)}_k c_{(\alpha)}^\mu
      \coD{\mu}{\tc{\Phi}{^i_j^k}})
    \qty(\vphantom{\frac{0}{0}}
      \V{g}_{(p)} \tp \V{g}^{(q)} \tp \V{g}_{(r)}
      \tp \V{g}^{(\alpha)}) \notag
  \intertext{写成张量分量与简单张量“乘积”的形式,即为}
  &\eqcolon \tc{\qty\big(\T{\Phi}\tp\opGrad)}%
      {^{(p)}_{\!(q)}^{\!(r)}_{\!(\alpha)}} \,
    \qty(\vphantom{\frac{0}{0}}
      \V{g}_{(p)} \tp \V{g}^{(q)} \tp \V{g}_{(r)}
      \tp \V{g}^{(\alpha)}) \fullstop
\end{align}
这样,我们就获得了非完整基下张量梯度的表示。
再利用式~\eqref{eq:张量梯度分量_非完整基},可知
\begin{equation}
  \tc{\qty\big(\T{\Phi}\tp\opGrad)}%
    {^{(p)}_{\!(q)}^{\!(r)}_{\!(\alpha)}} \,
  =c^{(p)}_i c_{(q)}^j c^{(r)}_k c_{(\alpha)}^\mu \,
    \tc{\qty(\vphantom{0^0}
      \T{\Phi}\tp\opGrad)}{^i_j^k_\mu} \fullstop
  \label{eq:非完整基下的张量梯度分量}
\end{equation}
以上结果与 \ref{subsec:相对不同基的张量分量之间的关系}~小节
中的推导是完全一致的。

\section{非完整基的形式运算}
在 \ref{sec:非完整基下的张量梯度}~节中,
我们利用\emphB{坐标转换关系}获得了张量梯度在非完整基下的表示。
而在本节,我们将通过定义,建立所谓“形式理论”,
获得一套更统一、更连贯的表述。

\myPROBLEM[2017-01-31]{统一、连贯?}

首先需要给出一些定义。

\begin{myEnum}
\item \emphA{形式偏导数}:
\begin{equation}
  \pdv{x^{(\mu)}} \defeq c^l_{(\mu)} \pdv{x^l} \fullstop
  \label{eq:形式偏导数}
\end{equation}
注意 $\pdv*{x^{(\mu)}}$ 本身是不能用极限形式来定义的,
因为曲线坐标系中并不存在有 $x^{(\mu)}$ 坐标线。

\item \emphA{形式 Christoffel 符号}:
\begin{equation}
  \ChrB{(\alpha)}{(\beta)}{(\gamma)}
  \defeq c^i_{(\alpha)} c^j_{(\beta)} c^{(\gamma)}_k \ChrB{i}{j}{k}
    -c^i_{(\alpha)} c^j_{(\beta)} \pdv{c^{(\gamma)}_j}{x^i}
  =c^i_{(\alpha)} c^j_{(\beta)}
    \qty(c^{(\gamma)}_k \ChrB{i}{j}{k}
      -\pdv{c^{(\gamma)}_j}{x^i}) \fullstop
  \label{eq:形式Christoffel符号}
\end{equation}
\myPROBLEM[2017-01-31]{第一类形式 Christoffel 符号}

\item \emphA{形式协变导数}。我们以三阶张量 $\T{\Phi}$ 为例给出定义。
$\T{\Phi}$ 在非完整基下可以用混合分量表示如下:
\begin{equation}
  \tc{\Phi}{^{(\alpha)}_{\!(\beta)}^{\!(\gamma)}}
  \coloneq \T{\Phi} \qty(\V{g}^{(\alpha)},\,\V{g}_{(\beta)},\,
    \V{g}^{(\gamma)}) \fullstop
\end{equation}
它相对 $x^{(\mu)}$ 分量的形式协变导数为
\begin{equation}
  \coD{(\mu)}{\tc{\Phi}{^{(\alpha)}_{\!(\beta)}^{\!(\gamma)}}}
  \defeq \pdv{\tc{\Phi}{^{(\alpha)}_{\!(\beta)}^{\!(\gamma)}}}%
    {x^{(\mu)}}
  +\ChrB{(\mu)}{(\sigma)}{(\alpha)}
    \tc{\Phi}{^{(\sigma)}_{\!(\beta)}^{\!(\gamma)}}
  -\ChrB{(\mu)}{(\beta)}{(\sigma)}
    \tc{\Phi}{^{(\alpha)}_{\!(\sigma)}^{\!(\gamma)}}
  +\ChrB{(\mu)}{(\sigma)}{(\gamma)}
    \tc{\Phi}{^{(\alpha)}_{\!(\beta)}^{\!(\sigma)}} \fullstop
  \label{eq:形式协变导数}
\end{equation}
回顾 \ref{sec:张量场的偏导数_协变导数}~节,
\eqref{eq:协变导数定义}~式给出了完整基下协变导数的定义:
\begin{equation}
  \coD{l}{\tc{\Phi}{^i_j^k}} \defeq
  \pdv{\tc{\Phi}{^i_j^k}}{x^l}
  +\ChrB{l}{s}{i} \tc{\Phi}{^s_j^k}
  -\ChrB{l}{j}{s} \tc{\Phi}{^i_s^k}
  +\ChrB{l}{s}{k} \tc{\Phi}{^i_j^s} \fullstop
\end{equation}
可以看出\emphB{形式}协变导数的定义与它几乎一模一样。
\end{myEnum}

\blankline

接下来我们要证明
\begin{equation}
  \coD{(\mu)}{\tc{\Phi}{^{(\alpha)}_{\!(\beta)}^{\!(\gamma)}}}
  =c^l_{(\mu)} c^{(\alpha)}_i c^j_{(\beta)} c^{(\gamma)}_k \,
    \coD{l}{\tc{\Phi}{^i_j^k}} \fullstop
\end{equation}
代入式~\eqref{eq:张量梯度分量_非完整基} 和
\eqref{eq:非完整基下的张量梯度分量},即得
\begin{equation}
  \coD{(\mu)}{\tc{\Phi}{^{(\alpha)}_{\!(\beta)}^{\!(\gamma)}}}
  =\tc{\qty(\vphantom{0^0}
    \T{\Phi}\tp\opGrad)}{^{(p)}_{\!(q)}^{\!(r)}_{\!(\alpha)}}
  \fullstop
\end{equation}
换句话说,此处我们正是要验证这种“形式理论”与
\ref{sec:非完整基下的张量梯度}~节中坐标转换关系的一致性。

\begin{myProof}
左边按照 \eqref{eq:形式协变导数}~式展开,第一项为
\begin{align}
  \pdv{\tc{\Phi}{^{(\alpha)}_{\!(\beta)}^{\!(\gamma)}}}%
    {x^{(\mu)}}
  &=c^l_{(\mu)} \pdv{\tc{\Phi}{^{(\alpha)}_{\!(\beta)}%
      ^{\!(\gamma)}}}{x^l} \notag
  \intertext{这里用到了形式偏导数的定义 \eqref{eq:形式偏导数}~式。
  然后利用坐标转换关系展开张量分量:}
  &=c^l_{(\mu)} \pdv{x^l}
    \qty(c^{(\alpha)}_i c^j_{(\beta)} c^{(\gamma)}_k
      \tc{\Phi}{^i_j^k}) \notag
  %
  \intertext{再按照通常的偏导数法则直接打开:}
  &=c^l_{(\mu)} c^j_{(\beta)} c^{(\gamma)}_k
      \pdv{c^{(\alpha)}_i}{x^l} \tc{\Phi}{^i_j^k}
    +c^l_{(\mu)} c^{(\alpha)}_i c^{(\gamma)}_k
      \pdv{c^j_{(\beta)}}{x^l} \tc{\Phi}{^i_j^k}
    +c^l_{(\mu)} c^{(\alpha)}_i c^j_{(\beta)}
      \pdv{c^{(\gamma)}_k}{x^l} \tc{\Phi}{^i_j^k} \notag \\*
  &\alspace{}
    +c^l_{(\mu)} c^{(\alpha)}_i c^j_{(\beta)} c^{(\gamma)}_k
      \pdv{\tc{\Phi}{^i_j^k}}{x^l} \notag \\
  %
  &=c^l_{(\mu)} \tc{\Phi}{^i_j^k}
    \qty(c^j_{(\beta)} c^{(\gamma)}_k \pdv{c^{(\alpha)}_i}{x^l}
      +c^{(\alpha)}_i c^{(\gamma)}_k \pdv{c^j_{(\beta)}}{x^l}
      +c^{(\alpha)}_i c^j_{(\beta)} \pdv{c^{(\gamma)}_k}{x^l})
    \notag \\*
  &\alspace{}
    +c^l_{(\mu)} c^{(\alpha)}_i c^j_{(\beta)} c^{(\gamma)}_k
      \pdv{\tc{\Phi}{^i_j^k}}{x^l} \fullstop
  \label{eq:张量分量偏导数_非完整基形式运算}
\end{align}

接下来处理含有形式 Christoffel 符号的三项,分别是
\begin{mySubEq}
  \begin{align}
    \ChrB{(\mu)}{(\sigma)}{(\alpha)}
      \tc{\Phi}{^{(\sigma)}_{\!(\beta)}^{\!(\gamma)}}
    &=c^p_{(\mu)} c^q_{(\sigma)}
      \qty(c^{(\alpha)}_s \ChrB{p}{q}{s} - \pdv{c^{(\alpha)}_q}{x^p})
      \cdot \tc{\Phi}{^{(\sigma)}_{\!(\beta)}^{\!(\gamma)}} \notag \\
    &=c^p_{(\mu)} \hl{c^q_{(\sigma)}}
      \qty(c^{(\alpha)}_s \ChrB{p}{q}{s} - \pdv{c^{(\alpha)}_q}{x^p})
      \cdot \hl{c^{(\sigma)}_i} c^j_{(\beta)} c^{(\gamma)}_k
        \tc{\Phi}{^i_j^k} \notag
    \intertext{根据式~\eqref{eq:坐标转换系数的乘积},我们有
      $c^q_{(\sigma)} c^{(\sigma)}_i = \KroneckerDelta{q}{i}$,于是}
    &=c^p_{(\mu)} \tc{\Phi}{^i_j^k}
      \qty(c^{(\alpha)}_s c^j_{(\beta)} c^{(\gamma)}_k \ChrB{p}{i}{s}
        -c^j_{(\beta)} c^{(\gamma)}_k \pdv{c^{(\alpha)}_i}{x^p})
    \semicolon \\
    %
    -\ChrB{(\mu)}{(\beta)}{(\sigma)}
      \tc{\Phi}{^{(\alpha)}_{\!(\sigma)}^{\!(\gamma)}}
    &=-c^p_{(\mu)} c^q_{(\beta)}
      \qty(c^{(\sigma)}_s \ChrB{p}{q}{s} - \pdv{c^{(\sigma)}_q}{x^p})
      \cdot \tc{\Phi}{^{(\alpha)}_{\!(\sigma)}^{\!(\gamma)}} \notag \\
    &=-c^p_{(\mu)} c^q_{(\beta)}
      \qty(\hl{c^{(\sigma)}_s} \ChrB{p}{q}{s}
        -\pdv{c^{(\sigma)}_q}{x^p})
      \cdot c^{(\alpha)}_i \hl{c^j_{(\sigma)}} c^{(\gamma)}_k
        \tc{\Phi}{^i_j^k} \notag \\
    &=c^p_{(\mu)} \tc{\Phi}{^i_j^k}
      \qty(-c^{(\alpha)}_i c^q_{(\beta)} c^{(\gamma)}_k \ChrB{p}{q}{j}
        +c^{(\alpha)}_i c^q_{(\beta)} c^{(\gamma)}_k c^j_{(\sigma)}
          \pdv{c^{(\sigma)}_q}{x^p}) \semicolon \\
    %
    \ChrB{(\mu)}{(\sigma)}{(\gamma)}
      \tc{\Phi}{^{(\alpha)}_{\!(\beta)}^{\!(\sigma)}}
    &=c^p_{(\mu)} c^q_{(\sigma)}
      \qty(c^{(\gamma)}_s \ChrB{p}{q}{s} - \pdv{c^{(\gamma)}_q}{x^p})
      \cdot \tc{\Phi}{^{(\alpha)}_{\!(\beta)}^{\!(\sigma)}} \notag \\
    &=c^p_{(\mu)} \hl{c^q_{(\sigma)}}
      \qty(c^{(\gamma)}_s \ChrB{p}{q}{s} - \pdv{c^{(\gamma)}_q}{x^p})
      \cdot c^{(\alpha)}_i c^j_{(\beta)} \hl{c^{(\sigma)}_k}
        \tc{\Phi}{^i_j^k} \notag \\
    &=c^p_{(\mu)} \tc{\Phi}{^i_j^k}
      \qty(c^{(\alpha)}_i c^j_{(\beta)} c^{(\gamma)}_s \ChrB{p}{k}{s}
        -c^{(\alpha)}_i c^j_{(\beta)} \pdv{c^{(\gamma)}_k}{x^p})
    \fullstop
  \end{align}
\end{mySubEq}
以上三式都有公因子 $c^p_{(\mu)} \tc{\Phi}{^i_j^k}$。
为了进一步化简,不妨将哑标 $p$ 换为 $l$。这样可有
\begin{align}
  &\alspace \ChrB{(\mu)}{(\sigma)}{(\alpha)}
    \tc{\Phi}{^{(\sigma)}_{\!(\beta)}^{\!(\gamma)}}
  -\ChrB{(\mu)}{(\beta)}{(\sigma)}
    \tc{\Phi}{^{(\alpha)}_{\!(\sigma)}^{\!(\gamma)}}
  +\ChrB{(\mu)}{(\sigma)}{(\gamma)}
    \tc{\Phi}{^{(\alpha)}_{\!(\beta)}^{\!(\sigma)}} \notag \\
  &=c^l_{(\mu)} \tc{\Phi}{^i_j^k}
    \left[\vphantom{\pdv{c^{(\gamma)}_k}{x^l}} \qty(
      c^{(\alpha)}_s c^j_{(\beta)} c^{(\gamma)}_k \ChrB{l}{i}{s}
      -c^{(\alpha)}_i c^q_{(\beta)} c^{(\gamma)}_k \ChrB{l}{q}{j}
      +c^{(\alpha)}_i c^j_{(\beta)} c^{(\gamma)}_s \ChrB{l}{k}{s})
    \right. \notag \\
  &\alspace\phantom{c^l_{(\mu)} \tc{\Phi}{^i_j^k}\left[\right]}
    \left. {}-c^j_{(\beta)} c^{(\gamma)}_k \pdv{c^{(\alpha)}_i}{x^l}
      +c^{(\alpha)}_i c^q_{(\beta)} c^{(\gamma)}_k c^j_{(\sigma)}
        \pdv{c^{(\sigma)}_q}{x^l}
      -c^{(\alpha)}_i c^j_{(\beta)} \pdv{c^{(\gamma)}_k}{x^l}
    \right] \fullstop
\end{align}
该式与 \eqref{eq:张量分量偏导数_非完整基形式运算}~式相加,得
\begin{align}
  &\alspace \coD{(\mu)}{\tc{\Phi}
    {^{(\alpha)}_{\!(\beta)}^{\!(\gamma)}}}
  \defeq \pdv{\tc{\Phi}{^{(\alpha)}_{\!(\beta)}^{\!(\gamma)}}}%
    {x^{(\mu)}}
    +\ChrB{(\mu)}{(\sigma)}{(\alpha)}
      \tc{\Phi}{^{(\sigma)}_{\!(\beta)}^{\!(\gamma)}}
    -\ChrB{(\mu)}{(\beta)}{(\sigma)}
      \tc{\Phi}{^{(\alpha)}_{\!(\sigma)}^{\!(\gamma)}}
    +\ChrB{(\mu)}{(\sigma)}{(\gamma)}
      \tc{\Phi}{^{(\alpha)}_{\!(\beta)}^{\!(\sigma)}} \notag \\
  %
  &=c^l_{(\mu)} c^{(\alpha)}_i c^j_{(\beta)} c^{(\gamma)}_k
      \pdv{\tc{\Phi}{^i_j^k}}{x^l}
    +c^l_{(\mu)} \tc{\Phi}{^i_j^k} \left[
      \hl{c^j_{(\beta)} c^{(\gamma)}_k \pdv{c^{(\alpha)}_i}{x^l}}
      +c^{(\alpha)}_i c^{(\gamma)}_k \pdv{c^j_{(\beta)}}{x^l}
      +\hl[pink]{c^{(\alpha)}_i c^j_{(\beta)}
        \pdv{c^{(\gamma)}_k}{x^l}} \right. \notag \\*
  &\alspace
  \phantom{c^l_{(\mu)} c^{(\alpha)}_i c^j_{(\beta)} c^{(\gamma)}_k
      \pdv{\tc{\Phi}{^i_j^k}}{x^l}
    +c^l_{(\mu)} \tc{\Phi}{^i_j^k} \left[\right]}
    \left. {}
      +\qty(c^{(\alpha)}_s c^j_{(\beta)} c^{(\gamma)}_k \ChrB{l}{i}{s}
      -c^{(\alpha)}_i c^q_{(\beta)} c^{(\gamma)}_k \ChrB{l}{q}{j}
      +c^{(\alpha)}_i c^j_{(\beta)} c^{(\gamma)}_s \ChrB{l}{k}{s})
    \right. \notag \\*
  &\alspace
  \phantom{c^l_{(\mu)} c^{(\alpha)}_i c^j_{(\beta)} c^{(\gamma)}_k
      \pdv{\tc{\Phi}{^i_j^k}}{x^l}
    +c^l_{(\mu)} \tc{\Phi}{^i_j^k} \left[\right]}
    \left. {}
      -\hl{c^j_{(\beta)} c^{(\gamma)}_k \pdv{c^{(\alpha)}_i}{x^l}}
      +c^{(\alpha)}_i c^q_{(\beta)} c^{(\gamma)}_k c^j_{(\sigma)}
        \pdv{c^{(\sigma)}_q}{x^l}
      -\hl[pink]{c^{(\alpha)}_i c^j_{(\beta)}
        \pdv{c^{(\gamma)}_k}{x^l}} \right] \notag
  %
  \intertext{高亮部分相互抵消:}
  &=c^l_{(\mu)} c^{(\alpha)}_i c^j_{(\beta)} c^{(\gamma)}_k
      \pdv{\tc{\Phi}{^i_j^k}}{x^l}
    +c^l_{(\mu)} \tc{\Phi}{^i_j^k}
    \left[\vphantom{\pdv{c^{(\gamma)}_k}{x^l}} \qty(
      c^{(\alpha)}_s c^j_{(\beta)} c^{(\gamma)}_k \ChrB{l}{i}{s}
      -c^{(\alpha)}_i c^q_{(\beta)} c^{(\gamma)}_k \ChrB{l}{q}{j}
      +c^{(\alpha)}_i c^j_{(\beta)} c^{(\gamma)}_s \ChrB{l}{k}{s})
    \right. \notag \\
  &\alspace
  \phantom{c^l_{(\mu)} c^{(\alpha)}_i c^j_{(\beta)} c^{(\gamma)}_k
      \pdv{\tc{\Phi}{^i_j^k}}{x^l}
    +c^l_{(\mu)} \tc{\Phi}{^i_j^k} \left[\right]}
    \left. {}
      +c^{(\alpha)}_i c^{(\gamma)}_k \pdv{c^j_{(\beta)}}{x^l}
      +c^{(\alpha)}_i c^q_{(\beta)} c^{(\gamma)}_k c^j_{(\sigma)}
        \pdv{c^{(\sigma)}_q}{x^l} \right]
  \label{eq:非完整基形式理论推导}
\end{align}
注意到 $c^j_{(\beta)} = c^q_{(\beta)} \KroneckerDelta{j}{q}
  =c^q_{(\beta)} c^j_{(\sigma)} c^{(\sigma)}_q$,因此
\begin{equation}
  \pdv{c^j_{(\beta)}}{x^l}
  =\pdv{x^l} \qty(c^q_{(\beta)} c^j_{(\sigma)} c^{(\sigma)}_q)
  =c^j_{(\sigma)} c^{(\sigma)}_q \pdv{c^q_{(\beta)}}{x^l}
    +c^q_{(\beta)} c^{(\sigma)}_q \pdv{c^j_{(\sigma)}}{x^l}
    +c^q_{(\beta)} c^j_{(\sigma)} \pdv{c^{(\sigma)}_q}{x^l} \fullstop
\end{equation}
所以 \eqref{eq:非完整基形式理论推导}~式中最后一步的第二行就能够写成
\begin{align}
  &\alspace c^{(\alpha)}_i c^{(\gamma)}_k \pdv{c^j_{(\beta)}}{x^l}
    +c^{(\alpha)}_i c^q_{(\beta)} c^{(\gamma)}_k c^j_{(\sigma)}
      \pdv{c^{(\sigma)}_q}{x^l} \notag \\
  &=c^{(\alpha)}_i c^{(\gamma)}_k
    \qty(\pdv{c^j_{(\beta)}}{x^l}
      +c^q_{(\beta)} c^j_{(\sigma)} \pdv{c^{(\sigma)}_q}{x^l})
    \notag \\
  &=c^{(\alpha)}_i c^{(\gamma)}_k \qty(
      \hl{c^j_{(\sigma)}} c^{(\sigma)}_q \pdv{c^q_{(\beta)}}{x^l}
      +\hl[pink]{c^q_{(\beta)}} c^{(\sigma)}_q
        \pdv{c^j_{(\sigma)}}{x^l}
      +c^q_{(\beta)} \hl{c^j_{(\sigma)}} \pdv{c^{(\sigma)}_q}{x^l}
      +\hl[pink]{c^q_{(\beta)}} c^j_{(\sigma)}
        \pdv{c^{(\sigma)}_q}{x^l}) \notag
  \intertext{合并同类项:}
  &=c^{(\alpha)}_i c^{(\gamma)}_k \qty[
      c^j_{(\sigma)}
      \qty(c^{(\sigma)}_q \pdv{c^q_{(\beta)}}{x^l}
        +c^q_{(\beta)} \pdv{c^{(\sigma)}_q}{x^l})
    +c^q_{(\beta)}
      \qty(c^{(\sigma)}_q \pdv{c^j_{(\sigma)}}{x^l}
        +c^j_{(\sigma)} \pdv{c^{(\sigma)}_q}{x^l}) ] \notag \\
  &=c^{(\alpha)}_i c^{(\gamma)}_k \qty[
      \vphantom{\pdv{c^q_{(\beta)}}{x^l}}
      c^j_{(\sigma)} \pdv{x^l} \qty(c^{(\sigma)}_q c^q_{(\beta)})
      +c^q_{(\beta)} \pdv{x^l} \qty(c^{(\sigma)}_q c^j_{(\sigma)}) ]
    \notag
  \intertext{再次利用式~\eqref{eq:坐标转换系数的乘积},可得}
  &=c^{(\alpha)}_i c^{(\gamma)}_k \qty(
      c^j_{(\sigma)} \pdv{\KroneckerDelta{\sigma}{\beta}}{x^l}
      +c^q_{(\beta)} \pdv{\KroneckerDelta{j}{q}}{x^l} )
  =0 \fullstop
\end{align}
代回式~\eqref{eq:非完整基形式理论推导},有
\begin{align}
  &\alspace \coD{(\mu)}{\tc{\Phi}
    {^{(\alpha)}_{\!(\beta)}^{\!(\gamma)}}}
  =c^l_{(\mu)} c^{(\alpha)}_i c^j_{(\beta)} c^{(\gamma)}_k
      \pdv{\tc{\Phi}{^i_j^k}}{x^l}
    +c^l_{(\mu)} \tc{\Phi}{^i_j^k}
    \qty(\vphantom{\pdv{c^{(\gamma)}_k}{x^l}}
      c^{(\alpha)}_s c^j_{(\beta)} c^{(\gamma)}_k \ChrB{l}{i}{s}
      -c^{(\alpha)}_i c^q_{(\beta)} c^{(\gamma)}_k \ChrB{l}{q}{j}
      +c^{(\alpha)}_i c^j_{(\beta)} c^{(\gamma)}_s \ChrB{l}{k}{s})
    \notag \\
  &=c^l_{(\mu)} c^{(\alpha)}_i c^j_{(\beta)} c^{(\gamma)}_k
      \pdv{\tc{\Phi}{^i_j^k}}{x^l}
    +c^l_{(\mu)}
    \qty(\vphantom{\pdv{c^{(\gamma)}_k}{x^l}}
      c^{(\alpha)}_s c^j_{(\beta)} c^{(\gamma)}_k
        \ChrB{l}{i}{s} \tc{\Phi}{^i_j^k}
      -c^{(\alpha)}_i c^q_{(\beta)} c^{(\gamma)}_k
        \ChrB{l}{q}{j} \tc{\Phi}{^i_j^k}
      +c^{(\alpha)}_i c^j_{(\beta)} c^{(\gamma)}_s
        \ChrB{l}{k}{s} \tc{\Phi}{^i_j^k}) \notag
  \intertext{下面要对哑标进行重排。
    括号里的第一项:$s \leftrightarrow i$;
    第二项:$j \rightarrow s, q \rightarrow j$;
    第三项:$s \leftrightarrow k$。于是}
  &=c^l_{(\mu)} c^{(\alpha)}_i c^j_{(\beta)} c^{(\gamma)}_k
      \pdv{\tc{\Phi}{^i_j^k}}{x^l}
    +c^l_{(\mu)}
    \qty(\vphantom{\pdv{c^{(\gamma)}_k}{x^l}}
      c^{(\alpha)}_i c^j_{(\beta)} c^{(\gamma)}_k
        \ChrB{l}{s}{i} \tc{\Phi}{^s_j^k}
      -c^{(\alpha)}_i c^j_{(\beta)} c^{(\gamma)}_k
        \ChrB{l}{j}{s} \tc{\Phi}{^i_s^k}
      +c^{(\alpha)}_i c^j_{(\beta)} c^{(\gamma)}_k
        \ChrB{l}{s}{k} \tc{\Phi}{^i_j^s}) \notag \\
  &=c^l_{(\mu)} c^{(\alpha)}_i c^j_{(\beta)} c^{(\gamma)}_k
    \qty(\pdv{\tc{\Phi}{^i_j^k}}{x^l}
      +\ChrB{l}{s}{i} \tc{\Phi}{^s_j^k}
      -\ChrB{l}{j}{s} \tc{\Phi}{^i_s^k}
      +\ChrB{l}{s}{k} \tc{\Phi}{^i_j^s}) \notag \\
  &=c^l_{(\mu)} c^{(\alpha)}_i c^j_{(\beta)} c^{(\gamma)}_k \,
    \coD{l}{\tc{\Phi}{^i_j^k}} \fullstop
\end{align}
这就完成了证明。
\end{myProof}

\blankline

如前文所言,此种形式理论与我们在
\ref{sec:非完整基下的张量梯度}~节中所使用的方法(坐标转换)并无二致,
但它在某些特定情况下将会十分有用,这就是下一节要介绍的内容。

\section{单位正交基}
\subsection{选取非完整基}
实际情况下,为了计算的方便,
我们通常会取一组\emphA{正交基}\idx{正交基}作为完整基,它们满足
\begin{equation}
  \ipb{\V{g}_i}{\V{g}_j}=0, \quad i\neq j \fullstop
\end{equation}
注意此处的 $\V{g}_i$ 和 $\V{g}_j$ 都是协变基。

\myPROBLEM[2017-02-01]{为什么不直接取单位正交基}

这样,度量 $g_{ij}$ 就可以用矩阵形式写成
\begin{equation}
  \mqty[g_{ij}]=\mqty[\dmat{g_{11},\ddots,g_{mm}}] \comma
\end{equation}
它是一个对角矩阵。根据式~\eqref{eq:度量之积_矩阵形式},我们有
\begin{equation}
  \mqty[g_{ik}] \mqty[g^{kj}]
  =\mqty[\KroneckerDelta{j}{i}]=\Mat{I}_m \semicolon
\end{equation}
而根据线性代数的知识,对角矩阵的逆同样是对角阵,
因此 $\mqty[g^{ij}]$ 也是一个对角矩阵。
换句话说,逆变基同样是一组正交基。

出于量纲一致等因素的考虑,我们常常需要将正交基单位化,
使其成为\emphA{单位正交基}\idx{单位正交基}。
当然,之前的完整基也就成了非完整基:
\begin{equation}
  \V{g}_{(\alpha)} \defeq c^i_{(\alpha)}\,\V{g}_i \comma
\end{equation}
式中,
\begin{equation}
  c^i_{(\alpha)}=\begin{dcases}
    \frac{1}{\sqrt{g_{ii}}} \comma & i=\alpha \semicolon \\
    0 \comma & i\neq\alpha \fullstop
  \end{dcases}
  \label{eq:坐标转换系数_单位正交基_1}
\end{equation}
这里 $g_{ii}$ 中的指标 $i$ 不求和。\footnote{
  在 $i=\alpha$ 的情况下,该系数常被称作
  \emphA{Lamé 系数}\idx{Lamé 系数}。}
对于逆变基,也是同样的:
\begin{equation}
  \V{g}^{(\alpha)} \defeq c^{(\alpha)}_i\,\V{g}^i \comma
\end{equation}
其中的
\begin{equation}
  c^{(\alpha)}_i=\begin{dcases}
    \sqrt{g_{ii}} \comma & i=\alpha \semicolon \\
    0 \comma & i\neq\alpha \fullstop
  \end{dcases}
  \label{eq:坐标转换系数_单位正交基_2}
\end{equation}

\subsection{形式偏导数}
非完整基形式运算的第一步是考虑形式偏导数:
\begin{equation}
  \pdv{x^{(\mu)}} \defeq c^l_{(\mu)} \pdv{x^l} \fullstop
\end{equation}
一般来说,$\V{x}\in\Rm$,因而该式包含着 $m$ 项的求和。
但在完整基是正交基、非完整基是单位正交基的情况下,
系数 $c^l_{(\mu)}$ 仅在 $l=\mu$ 的时候才有非零值。所以
\begin{equation}
  \pdv{x^{(\mu)}}
  =c^\mu_{(\mu)} \pdv{x^\mu}
  =\frac{1}{\sqrt{g_{\mu\mu}}} \pdv{x^\mu} \fullstop
  \label{eq:形式偏导数_单位正交基}
\end{equation}
指标 $\mu$ 不求和,此式便只剩下了一项。

\subsection{形式 Christoffel 符号}
根据 \eqref{eq:形式Christoffel符号}~式,
\begin{align}
  \ChrB{(\alpha)}{(\beta)}{(\gamma)}
  &\defeq c^i_{(\alpha)} c^j_{(\beta)} c^{(\gamma)}_k \ChrB{i}{j}{k}
    -c^i_{(\alpha)} c^j_{(\beta)} \pdv{c^{(\gamma)}_j}{x^i} \notag
  \intertext{同样,特殊情况下只需要考虑非零值:}
  &=c^\alpha_{(\alpha)} c^\beta_{(\beta)} c^{(\gamma)}_\gamma
      \ChrB{\alpha}{\beta}{\gamma}
    -c^\alpha_{(\alpha)} c^\beta_{(\beta)}
      \pdv{c^{(\gamma)}_\beta}{x^\alpha} \notag
  \intertext{代入式~\eqref{eq:坐标转换系数_单位正交基_1} 和
    \eqref{eq:坐标转换系数_单位正交基_2},可得}
  &=\frac{1}{\sqrt{g_{\alpha\alpha}}}
      \frac{1}{\sqrt{g_{\beta\beta}}} \sqrt{g_{\gamma\gamma}}
      \cdot \ChrB{\alpha}{\beta}{\gamma}
    -\frac{1}{\sqrt{g_{\alpha\alpha}}}
      \frac{1}{\sqrt{g_{\beta\beta}}}
      \cdot \pdv{c^{(\gamma)}_\beta}{x^\alpha} \fullstop
\end{align}
此处的指标 $\alpha$、$\beta$、$\gamma$ 均不表示求和。

上式包含一个完整基下的 Christoffel 符号
$\ChrB{\alpha}{\beta}{\gamma}$。
根据 \ref{subsec:度量的性质_Christoffel符号的计算}~小节中的
\eqref{eq:第一类Christoffel符号与度量的关系}~式和
\eqref{eq:第二类Christoffel符号用第一类表示}~式,
很容易利用度量把它计算出来:
\begin{align}
  \ChrB{\alpha}{\beta}{\gamma}
  &=g^{\gamma s}\,\ChrA{\alpha}{\beta}{s}
  =g^{\gamma s} \cdot \frac{1}{2}\,
    \qty(\pdv{g_{\beta s}}{x^\alpha}+\pdv{g_{\alpha s}}{x^\beta}
      -\pdv{g_{\alpha\beta}}{x^s}) \notag
  \intertext{注意指标 $s$ \emphB{需要}求和!
    但是由于度量的非对角元均为零,所以可以直接写成}
  &=g^{\gamma\gamma}\,\ChrA{\alpha}{\beta}{\gamma}
  =\frac{1}{g_{\gamma\gamma}} \cdot \frac{1}{2}\,
    \qty(\pdv{g_{\beta\gamma}}{x^\alpha}
      +\pdv{g_{\alpha\gamma}}{x^\beta}
      -\pdv{g_{\alpha\beta}}{x^\gamma}) \fullstop
  \label{eq:形式Christoffel符号_单位正交基}
\end{align}
同样,指标都不表示求和。

现在我们来分 4 种情况,进一步化简 $\ChrB{\alpha}{\beta}{\gamma}$。

\begin{myEnum}
\item $\alpha\neq\beta\neq\gamma$。
前文已经提到,度量的非对角元均为零,即
\begin{equation}
  g_{\beta\gamma}=g_{\alpha\gamma}=g_{\alpha\beta}=0 \comma
\end{equation}
因此结果非常简单:
\begin{equation}
  \ChrB{\alpha}{\beta}{\gamma}=0 \fullstop
\end{equation}

\item $\alpha=\beta\neq\gamma$,即 $\ChrB{\alpha}{\alpha}{\gamma}$。
直接计算,可有
\begin{equation}
  \ChrB{\alpha}{\alpha}{\gamma}
  =g^{\gamma\gamma}\,\ChrA{\alpha}{\alpha}{\gamma}
  =\frac{1}{g_{\gamma\gamma}} \cdot \frac{1}{2}\,
    \qty(-\pdv{g_{\alpha\alpha}}{x^\gamma})
  =-\frac{1}{2} \frac{1}{g_{\gamma\gamma}}
    \pdv{g_{\alpha\alpha}}{x^\gamma} \fullstop
\end{equation}

\item $\alpha=\gamma\neq\beta$,即 $\ChrB{\alpha}{\beta}{\alpha}$。
根据式~\eqref{eq:第二类Christoffel符号指标交换},
它又等于 $\ChrB{\beta}{\alpha}{\alpha}$。同样,直接来进行计算:
\begin{equation}
  \ChrB{\alpha}{\beta}{\alpha}
  =g^{\alpha\alpha}\,\ChrA{\alpha}{\beta}{\alpha}
  =\frac{1}{g_{\alpha\alpha}} \cdot \frac{1}{2}\,
    \qty(\pdv{g_{\alpha\alpha}}{x^\beta})
  =\frac{1}{2} \frac{1}{g_{\alpha\alpha}}
    \pdv{g_{\alpha\alpha}}{x^\beta} \fullstop
\end{equation}

\item $\alpha=\beta=\gamma$,即 $\ChrB{\alpha}{\alpha}{\alpha}$。
指标只剩下了一个,喜闻乐见。
\begin{equation}
  \ChrB{\alpha}{\alpha}{\alpha}
  =g^{\alpha\alpha}\,\ChrA{\alpha}{\alpha}{\alpha}
  =\frac{1}{g_{\alpha\alpha}} \cdot \frac{1}{2}\,
    \qty(\pdv{g_{\alpha\alpha}}{x^\alpha})
  =\frac{1}{2} \frac{1}{g_{\alpha\alpha}}
    \pdv{g_{\alpha\alpha}}{x^\alpha} \fullstop
\end{equation}
\end{myEnum}

算好了完整基(正交基)下的 Christoffel 符号,
就可以考虑非完整基(单位正交基)下的情况了。
我们在式~\eqref{eq:形式Christoffel符号_单位正交基}
中已经计算出了非完整基下的 Christoffel 符号,
现在只要把以上四种情况逐一代入即可。

\begin{myEnum}
\item $\alpha\neq\beta\neq\gamma$。
已经知道 $\ChrB{\alpha}{\beta}{\gamma}=0$,
而根据 \eqref{eq:坐标转换系数_单位正交基_2}~式,
又有 $c^{(\gamma)}_\beta=0$,于是
\begin{equation}
  \ChrB{(\alpha)}{(\beta)}{(\gamma)}=0 \fullstop
\end{equation}

\item $\alpha=\beta\neq\gamma$。此时有
\begin{align}
  \ChrB{(\alpha)}{(\alpha)}{(\gamma)}
  &=\frac{1}{g_{\alpha\alpha}} \sqrt{g_{\gamma\gamma}}
      \cdot \ChrB{\alpha}{\alpha}{\gamma}
    -\frac{1}{g_{\alpha\alpha}}
      \cdot \pdv{c^{(\gamma)}_\alpha}{x^\alpha} \notag \\
  &=\frac{1}{g_{\alpha\alpha}} \sqrt{g_{\gamma\gamma}}
    \cdot \qty(-\frac{1}{2} \frac{1}{g_{\gamma\gamma}}
      \pdv{g_{\alpha\alpha}}{x^\gamma}) - 0 \notag \\
  &=-\frac{1}{\sqrt{g_{\gamma\gamma}}}
    \cdot \qty(\frac{1}{2g_{\alpha\alpha}}
      \pdv{g_{\alpha\alpha}}{x^\gamma}) \fullstop
\end{align}
考虑到
\begin{equation}
  \pdv{x} \ln\sqrt{f(x)}
  =\pdv{x} \qty[\frac{1}{2} \ln f(x)]
  =\frac{1}{2} \frac{1}{f(x)} \pdv{f(x)}{x} \comma
\end{equation}
于是
\begin{equation}
  \ChrB{(\alpha)}{(\alpha)}{(\gamma)}
  =-\frac{1}{\sqrt{g_{\gamma\gamma}}}
    \pdv{x^\gamma} \qty(\vphantom{\frac{0}{0}}
      \ln \sqrt{g_{\alpha\alpha}}) \fullstop
\end{equation}

\item $\alpha=\gamma\neq\beta$。此时
\begin{align}
  \ChrB{(\alpha)}{(\beta)}{(\alpha)}
  &=\frac{1}{\sqrt{g_{\beta\beta}}}
      \cdot \ChrB{\alpha}{\beta}{\alpha}
    -\frac{1}{\sqrt{g_{\alpha\alpha}}}
      \frac{1}{\sqrt{g_{\beta\beta}}}
      \cdot \pdv{c^{(\alpha)}_\beta}{x^\alpha} \notag \\
  &=\frac{1}{\sqrt{g_{\beta\beta}}}
    \cdot \qty(\frac{1}{2} \frac{1}{g_{\alpha\alpha}}
      \pdv{g_{\alpha\alpha}}{x^\beta}) - 0 \notag
  \intertext{同理,利用对数,可得}
  &=\frac{1}{\sqrt{g_{\beta\beta}}}
    \pdv{x^\beta} \qty(\vphantom{\frac{0}{0}}
      \ln \sqrt{g_{\alpha\alpha}}) \fullstop
\end{align}
注意这里没有负号。

\item[\theenumi*.] $\beta=\gamma\neq\alpha$,
即 $\ChrB{(\alpha)}{(\beta)}{(\beta)}$。
不过我们暂时先从 $\ChrB{(\beta)}{(\alpha)}{(\alpha)}$ 开始。

之前虽然已经计算了 $\ChrB{(\alpha)}{(\beta)}{(\alpha)}$,
但由于我们并未证明\emphB{形式} Christoffel 符号的下标可以交换
\footnote{所以这里也多了一种情况需要讨论。},
因而仍要从头来算:
\begin{align}
  \ChrB{(\beta)}{(\alpha)}{(\alpha)}
  &=\frac{1}{\sqrt{g_{\beta\beta}}}
      \cdot \ChrB{\beta}{\alpha}{\alpha}
    -\frac{1}{\sqrt{g_{\beta\beta}}}
      \frac{1}{\sqrt{g_{\alpha\alpha}}}
      \cdot \pdv{c^{(\alpha)}_\alpha}{x^\beta} \notag
  \intertext{交换 Christoffel 符号的下标,
    同时代入\eqref{eq:坐标转换系数_单位正交基_2}~式,可有}
  &=\frac{1}{\sqrt{g_{\beta\beta}}}
      \cdot \ChrB{\alpha}{\beta}{\alpha}
    -\frac{1}{\sqrt{g_{\beta\beta}}}
      \frac{1}{\sqrt{g_{\alpha\alpha}}}
      \cdot \pdv{x^\beta} \sqrt{g_{\alpha\alpha}} \notag \\
  &=\frac{1}{\sqrt{g_{\beta\beta}}}
    \cdot \qty(\frac{1}{2} \frac{1}{g_{\alpha\alpha}}
      \pdv{g_{\alpha\alpha}}{x^\beta})
    -\frac{1}{\sqrt{g_{\beta\beta}}}
      \frac{1}{\sqrt{g_{\alpha\alpha}}}
      \cdot \frac{1}{2\sqrt{g_{\alpha\alpha}}}
      \pdv{g_{\alpha\alpha}}{x^\beta} \notag \\
  &=0 \fullstop
\end{align}

回过头来,若要得到 $\ChrB{(\alpha)}{(\beta)}{(\beta)}$,
只需交换 $\alpha$、$\beta$,结果当然不变:
\begin{equation}
  \ChrB{(\alpha)}{(\beta)}{(\beta)}=0 \fullstop
\end{equation}

\item $\alpha=\beta=\gamma$。指标全部相同,有
\begin{align}
  \ChrB{(\alpha)}{(\alpha)}{(\alpha)}
  &=\frac{1}{\sqrt{g_{\alpha\alpha}}}
      \cdot \ChrB{\alpha}{\alpha}{\alpha}
    -\frac{1}{g_{\alpha\alpha}}
      \cdot \pdv{c^{(\alpha)}_\alpha}{x^\alpha} \notag \\
  &=\frac{1}{\sqrt{g_{\alpha\alpha}}}
    \cdot \qty(\frac{1}{2} \frac{1}{g_{\alpha\alpha}}
      \pdv{g_{\alpha\alpha}}{x^\alpha})
    -\frac{1}{g_{\alpha\alpha}}
      \cdot \pdv{x^\alpha} \sqrt{g_{\alpha\alpha}} \notag \\
  &=\frac{1}{2g_{\alpha\alpha} \sqrt{g_{\alpha\alpha}}}
      \pdv{g_{\alpha\alpha}}{x^\alpha}
    -\frac{1}{g_{\alpha\alpha}}
      \cdot \frac{1}{2\sqrt{g_{\alpha\alpha}}}
      \pdv{g_{\alpha\alpha}}{x^\alpha} \notag \\
  &=0 \comma
\end{align}
依然是个很漂亮的结果。
\end{myEnum}

到此,我们可以看到,只有两种形式的 Christoffel 符号非零:
\begin{braceEq}
  \ChrB{(\alpha)}{(\alpha)}{(\beta)}
  &=-\frac{1}{\sqrt{g_{\beta\beta}}}
    \pdv{x^\beta} \qty(\vphantom{\frac{0}{0}}
      \ln \sqrt{g_{\alpha\alpha}})
  =-\pdv{x^{(\beta)}} \ln\sqrt{g_{\alpha\alpha}} \comma \\
  \ChrB{(\alpha)}{(\beta)}{(\alpha)}
  &=\frac{1}{\sqrt{g_{\beta\beta}}}
    \pdv{x^\beta} \qty(\vphantom{\frac{0}{0}}
      \ln \sqrt{g_{\alpha\alpha}})
  =\pdv{x^{(\beta)}} \ln\sqrt{g_{\alpha\alpha}} \fullstop
\end{braceEq}
后一个等号利用了形式偏导数 \eqref{eq:形式偏导数_单位正交基}~式。

对于单位正交基,其度量满足
\begin{equation}
  g^{(\alpha)(\beta)}=g_{(\alpha)(\beta)}
  =\KroneckerDelta*{\alpha\beta} \comma
\end{equation}
因此协变基与逆变基只好相同。
这样一来,协变分量与逆变分量也就没有了差别,我们统一用尖括号标出。
于是上面的 Christoffel 符号就可以写成
\begin{braceEq}
  \ChrU{\alpha}{\alpha}{\beta}
  &=\ChrA{(\alpha)}{(\alpha)}{(\beta)}
    =\ChrB{(\alpha)}{(\alpha)}{(\beta)}
    =-\pdv{x^{(\beta)}} \ln\sqrt{g_{\alpha\alpha}} \comma \\
  \ChrU{\alpha}{\beta}{\alpha}
  &=\ChrA{(\alpha)}{(\beta)}{(\alpha)}
    =\ChrB{(\alpha)}{(\beta)}{(\alpha)}
    =\pdv{x^{(\beta)}} \ln\sqrt{g_{\alpha\alpha}} \fullstop
\end{braceEq}
显然,它们的第二、第三指标具有反对称性:
\begin{equation}
  \ChrU{\alpha}{\alpha}{\beta}
  =-\ChrU{\alpha}{\beta}{\alpha} \fullstop
  \label{eq:单位正交基下Christoffel符号的反对称性}
\end{equation}

\subsection{形式“协变”导数}
根据式~\eqref{eq:形式协变导数} 中的定义(仍以三阶张量为例),我们有
\begin{equation}
  \coD{(\mu)}{\tc{\Phi}{^{(\alpha)}_{\!(\beta)}^{\!(\gamma)}}}
  \defeq \pdv{\tc{\Phi}{^{(\alpha)}_{\!(\beta)}^{\!(\gamma)}}}%
    {x^{(\mu)}}
  +\ChrB{(\mu)}{(\sigma)}{(\alpha)}
    \tc{\Phi}{^{(\sigma)}_{\!(\beta)}^{\!(\gamma)}}
  -\ChrB{(\mu)}{(\beta)}{(\sigma)}
    \tc{\Phi}{^{(\alpha)}_{\!(\sigma)}^{\!(\gamma)}}
  +\ChrB{(\mu)}{(\sigma)}{(\gamma)}
    \tc{\Phi}{^{(\alpha)}_{\!(\beta)}^{\!(\sigma)}} \fullstop
\end{equation}
利用单位正交基不区分协变、逆变的特点,可以把它写成
\begin{align}
  \coD*{\mu}{\Phi\midscript{\orthIdx{\alpha\beta\gamma}}}
  &=\pdv{\Phi\midscript{\orthIdx{\alpha\beta\gamma}}}{x^{(\mu)}}
    +\ChrU{\mu}{\sigma}{\alpha} \,
      \Phi\midscript{\orthIdx{\sigma\beta\gamma}}
    -\ChrU{\mu}{\hl{\beta\sigma}}{} \,
      \Phi\midscript{\orthIdx{\alpha\sigma\gamma}}
    +\ChrU{\mu}{\sigma}{\gamma} \,
      \Phi\midscript{\orthIdx{\alpha\beta\sigma}} \notag
  \intertext{根据 Christoffel 符号的反对称性
    \eqref{eq:单位正交基下Christoffel符号的反对称性}~式,有}
  &=\pdv{\Phi\midscript{\orthIdx{\alpha\beta\gamma}}}{x^{(\mu)}}
    +\ChrU{\mu}{\sigma}{\alpha} \,
      \Phi\midscript{\orthIdx{\sigma\beta\gamma}}
    +\ChrU{\mu}{\hl{\sigma\beta}}{} \,
      \Phi\midscript{\orthIdx{\alpha\sigma\gamma}}
    +\ChrU{\mu}{\sigma}{\gamma} \,
      \Phi\midscript{\orthIdx{\alpha\beta\sigma}} \fullstop
\end{align}
这样,Christoffel 符号中的第二个指标就都是哑标 $\sigma$,
而哑标所取代的对应指标则放在第三个位置上。
同时,这种写法也免去了正负号的困扰。

    % 曲线上的标架及其运动方程
    \chapter{曲线上的标架及其运动方程}
\section{Frenet 标架(弧长参数)}
%CODE 20161130 TeX Studio bug 无法换行
% #1898 Freeze while typing \texorpdfstring
% https://sourceforge.net/p/texstudio/bugs/1898/
\subsection{\texorpdfstring{$\Rm$}{R\^{}m} 空间中曲线的表示}
$\Rm$ 空间中的\emphA{曲线}\idx{曲线},
就是一个\emphB{单参数}的向量值映照:
\begin{equation}
  \mmap{\V{X}(t)}{[\alpha,\,\beta]\ni t}
    {\V{X}(t)=\mqty[X^1(t) \\ \vdots \\ X^m(t)]\in\Rm} \fullstop
\end{equation}
考虑该向量值映照关于参数 $t$ 的变化率
\begin{equation}
  \dot{\V{X}}(t)\coloneq\dv{\V{X}}{t} (t)
  =\lim_{\incr t\to 0} \frac{\V{X}(t+\incr t)-\V{X}(t)}{\incr t}
  \eqcolon \mqty[\dot{X}^1(t) \\ \vdots \\ \dot{X}^m(t)] \comma
\end{equation}
按照物理上的习惯,我们用点表示对 $t$ 的导数。
若该极限存在,则称 $\V{X}(t)\in\Rm$ 在点 $t$
处\emphA{可微}\idx{可微}。此时,$\dot{\V{X}}(t)$ 称为曲线
$\V{X}(t)$ 的\emphA{切向量}\idx{切向量}。上述极限可以等价地表述为
\begin{equation}
  \V{X}(t+\incr t) = \V{X}(t)+\dv{\V{X}}{t} (t) \cdot\incr t
    +\sOv{\incr t} \fullstop
\end{equation}
在 $t_0$ 处,则可以写成
\begin{equation}
  \V{X}(t) =\V{X}\qty(t_0)+\dv{\V{X}}{t} \qty(t_0)
    \cdot\qty(t-t_0)+\sOv{t-t_0} \fullstop
\end{equation}
该方程表示一条直线,称为曲线 $\V{X}(t)$
在 $t_0$ 处的\emphA{切线}\idx{切线}。

在物理域中,\emphA{弧长}\idx{弧长} $s$ 可以表示为
\begin{equation}
  s=\int_\alpha^\beta \norm{\dv{\V{X}}{t} (t)} \dd{t} \comma
  \label{eq:弧长的定义}
\end{equation}
两边对 $t$ 求导,可有
\begin{equation}
  \dv{s}{t} (t) = \norm{\dv{\V{X}}{t} (t)} \fullstop
  \label{eq:弧长与一般参数的关系}
\end{equation}
在本节中,我们将采用弧长作为曲线的参数,即
\begin{equation}
  \mmap{\V{r}(s)}{[0,\,L]\ni s}{\V{r}(s)\in\Rm} \fullstop
\end{equation}
对应的切向量为
\begin{equation}
  \dot{\V{r}}(s)\coloneq\dv{\V{r}}{s} (s)
  =\lim_{\incr s\to 0} \frac{\V{r}(s+\incr s)-\V{r}(s)}{\incr s}
  \fullstop
\end{equation}
根据链式法则,
\begin{equation}
  \dv{\V{r}}{s} (s)
  =\dv{\V{r}}{t} (t) \cdot \dv{t}{s} (s)
  =\flatfrac{\dv{\V{r}}{t} (t)}{\dv{s}{t} (t)}
  =\flatfrac{\dv{\V{r}}{t} (t)}{\norm{\dv{\V{r}}{t} (t)}} \comma
\end{equation}
因而 $\dot{\V{r}}(s)$ 是一个单位向量,即
\begin{equation}
  \norm{\dot{\V{r}}(s)}=1 \fullstop
  \label{eq:r的一阶导数为单位向量_弧长参数}
\end{equation}
接下来继续对 $\dot{\V{r}}(s)$ 求导:
\begin{equation}
  \ddot{\V{r}}(s)\coloneq\dv{\dot{\V{r}}}{s} (s) \fullstop
\end{equation}
由于 $\norm{\dot{\V{r}}(s)}=1$,因此
\begin{equation}
  1=\norm{\dot{\V{r}}(s)}^2
  =\ipb{\vphantom{0^0} \dot{\V{r}}(s)}{\dot{\V{r}}(s)} \comma
\end{equation}
两边求导,则有
\begin{equation}
  0=\ipb{\vphantom{0^0} \ddot{\V{r}}(s)}{\dot{\V{r}}(s)}
    +\ipb{\vphantom{0^0} \dot{\V{r}}(s)}{\ddot{\V{r}}(s)}
  =2\cdot\ipb{\vphantom{0^0} \ddot{\V{r}}(s)}{\dot{\V{r}}(s)}
  \fullstop
\end{equation}
内积为零,就意味着正交:
\begin{equation}
  \ddot{\V{r}}(s) \perp \dot{\V{r}}(s) \fullstop
  \label{eq:r的一阶导数垂直二阶导数_弧长参数}
\end{equation}
采用一般参数时,\eqref{eq:r的一阶导数为单位向量_弧长参数}~式和
\eqref{eq:r的一阶导数垂直二阶导数_弧长参数} 未必成立。
解决方案见 \ref{sec:Frenet标架_一般参数}~节。

现在我们把目光限定在 $\realR^3$ 空间中。如前所述,
$\dot{\V{r}}(s)$ 已经是单位向量,我们将其记为 $\V{T}(s)$;
而 $\ddot{\V{r}}(s)$ 仍需作单位化处理,其结果记作 $\V{N}(s)$,即
\begin{equation}
  \V{N}(s)\coloneq\frac{\ddot{\V{r}}(s)}
    {\norm[\realR^3]{\ddot{\V{r}}(s)}}
  =\frac{\dv*{\dot{\V{r}}}{s}}
    {\norm[\realR^3]{\dv*{\dot{\V{r}}}{s}}}
  =\frac{\dot{\V{T}}(s)}{\norm[\realR^3]{\dot{\V{T}}(s)}} \fullstop
  \label{eq:主单位法向量定义}
\end{equation}
最后,只要再令
\begin{equation}
  \V{B}(s)\coloneq\V{T}(s)\cp\V{N}(s) \comma
  \label{eq:副单位法向量定义}
\end{equation}
我们便有了 $\realR^3$ 空间中的一组\emphB{单位正交基}:
\begin{equation}
  \qty\big{\V{T}(s),\,\V{N}(s),\,\V{B}(s)}
  \subset\realR^3 \comma
\end{equation}
它们称为\emphA{Frenet 标架}\idx{Frenet 标架}。
其中,$\V{T}(s)$、$\V{N}(s)$、$\V{B}(s)$,
分别叫做\emphA{单位切向量}\idx{单位切向量}、
\emphA{主单位法向量}\idx{主单位法向量}和%
\emphA{副单位法向量}\idx{副单位法向量}。

\subsection{标架运动方程}
考虑 Frenet 标架关于弧长参数 $s$ 的变化率,
即\emphB{标架运动方程}\idx{标架运动方程}:
\begin{equation}
  \qty\big{\dot{\V{T}}(s),\,\dot{\V{N}}(s),\,\dot{\V{B}}(s)}
  \subset\realR^3 \fullstop
\end{equation}
为此,我们需要先给出一个引理:设 $\qty{\V{e}_i(t)}^m_{i=1}$
是 $\Rm$ 空间中的一组\emphB{活动}单位正交基,它们满足
\begin{equation}
  \ipb{\V{e}_i(t)}{\V{e}_j(t)}=\KroneckerDelta*{ij} \fullstop
  \label{eq:活动单位正交基引理_单位正交性}
\end{equation}
这组基的导数仍位于 $\Rm$ 空间,用自身展开,可有
\begin{equation}
  \mqty[\dot{\V{e}}_1(t),\,\cdots,\,\dot{\V{e}}_m(t)]
  =\mqty[\V{e}_1(t),\,\cdots,\,\V{e}_m(t)] \Mat{P}(t) \fullstop
  \label{eq:活动单位正交基引理_导数}
\end{equation}
此时,我们有
\begin{equation}
  \Mat{P}(t)\in\Skw \comma
\end{equation}
即 $\Mat{P}(t)$ 是一个\emphB{反对称矩阵}。

\begin{myProof}
对式~\eqref{eq:活动单位正交基引理_单位正交性} 两边求导,得
\begin{equation}
  \ipb{\dot{\V{e}}_i(t)}{\V{e}_j(t)}
  +\ipb{\V{e}_i(t)}{\dot{\V{e}}_j(t)} = 0 \in\realR \comma
\end{equation}
写成矩阵形式,为
\begin{equation}
  \mqty[\dot{\V{e}}_1\trans(t) \\ \vdots \\ \dot{\V{e}}_m\trans(t)]
  \mqty[\V{e}_1(t),\,\cdots,\,\V{e}_m(t)]
  +\mqty[\V{e}_1\trans(t) \\ \vdots \\ \V{e}_m\trans(t)]
  \mqty[\dot{\V{e}}_1(t),\,\cdots,\,\dot{\V{e}}_m(t)]
  =\Mat{0}\in\realR^{m\times m} \fullstop
\end{equation}
引入矩阵 $\Mat{E}=\qty[\V{e}_1(t),\,\cdots,\,\V{e}_m(t)]$,
则上式与 \eqref{eq:活动单位正交基引理_导数}~式可以分别表示成
\begin{equation}
  \dot{\Mat{E}}\trans\Mat{E}+\Mat{E}\trans\dot{\Mat{E}}
  =\Mat{0}\in\realR^{m\times m}
\end{equation}
和
\begin{equation}
  \dot{\Mat{E}} = \Mat{E}\Mat{P}\in\realR^{m\times m} \fullstop
\end{equation}
两式联立,可有
\begin{align}
  \Mat{0}&=\dot{\Mat{E}}\trans\Mat{E}
    +\Mat{E}\trans\dot{\Mat{E}} \notag \\
  &=\qty\big(\Mat{E}\Mat{P})\trans \Mat{E}
    +\Mat{E}\trans \qty\big(\Mat{E}\Mat{P}) \notag \\
  &=\Mat{P}\trans \qty\big(\Mat{E}\trans\Mat{E})
    +\qty\big(\Mat{E}\trans\Mat{E}) \Mat{P}
  =\Mat{P}\trans+\Mat{P} \comma
\end{align}
即 $\Mat{P}\trans=-\Mat{P}$。\myPROBLEM{按照定义},便知
$\Mat{P}(t)\in\Skw$。
\end{myProof}

根据这一引理,便可有
\begin{equation}
  \mqty[\dot{\V{T}}(s),\,\dot{\V{N}}(s),\,\dot{\V{B}}(s)]
  =\mqty[\V{T}(s),\,\V{N}(s),\,\V{B}(s)] \Mat{P}(s) \comma
\end{equation}
其中的 $\Mat{P}(s)$ 是一个三阶反对称矩阵。显然,它的对角元均为零:
\begin{equation}
  \Mat{P}(s)=\mqty[0 & \ast & \ast \\
    \ast & 0 & \ast \\
    \ast & \ast & 0 ] \fullstop
\end{equation}
这里,我们用“$\ast$”表示待定元素。
由 \eqref{eq:主单位法向量定义}~式,可知
\begin{equation}
  \dot{\V{T}}(s)
  =\norm[\realR^3]{\dot{\V{T}}(s)} \V{N}(s)
  \eqcolon \kappa(s)\,\V{N}(s) \fullstop
\end{equation}
式中的 $\kappa(s)\coloneq\norm[\realR^3]{\dot{\V{T}}(s)}$。
于是,矩阵 $\Mat{P}(s)$ 的第一列就成为了
$\qty[0,\,\kappa(s),\,0]\trans$。利用\emphB{反}对称性,可有
\begin{equation}
  \Mat{P}(s)=\mqty[0 & -\kappa(s) & 0 \\
    \kappa(s) & 0 & -\tau(s) \\
    0 & \tau(s) & 0] \fullstop
\end{equation}
我们“强行”引入了 $\tau(s)$,用来取代占位符 $\ast$。
当然,它的具体形式仍然待定。

\subsection{曲率和挠率}
利用矩阵 $\Mat{P}(s)$,可以看出
\begin{equation}
  \dot{\V{B}}(s)=-\tau(s)\,\V{N}(s) \fullstop
\end{equation}
再与 $\V{N}(s)$ 做内积\footnote{
  为了表述的清晰,本小节中用“$\vdp$”来表示内积。},便有
\begin{equation}
  \dot{\V{B}}(s)\vdp\V{N}(s)=-\tau(s) \fullstop
\end{equation}
根据定义 \eqref{eq:副单位法向量定义}~式,
\begin{equation}
  \V{B}(s)=\V{T}(s)\cp\V{N}(s) \comma
\end{equation}
于是
\begin{align}
  \dot{\V{B}}(s)
  &=\dv{s}\qty\Big[\V{T}(s)\cp\V{N}(s)]
  =\dv{s}\qty[\dot{\V{r}}(s)\cp\frac{\ddot{\V{r}}(s)}
      {\norm[\realR^3]{\ddot{\V{r}}(s)}}] \notag \\
  &=\ddot{\V{r}}(s)\cp\frac{\ddot{\V{r}}(s)}
      {\norm[\realR^3]{\ddot{\V{r}}(s)}}
    +\dot{\V{r}}(s)\cp\frac{\dddot{\V{r}}(s)}
      {\norm[\realR^3]{\ddot{\V{r}}(s)}}
    +\dv{s}\qty(\frac{1}{\norm[\realR^3]{\ddot{\V{r}}(s)}}) \,
      \dot{\V{r}}(s)\cp\ddot{\V{r}}(s) \fullstop
\end{align}
显然,该式中的第一项为零。考虑 $\dot{\V{B}}(s)\vdp\V{N}(s)$,
注意到 $\V{N}(s)$ 与 $\ddot{\V{r}}(s)$ 平行,因此与
$\dot{\V{r}}(s)\cp\ddot{\V{r}}(s)$ 垂直,所以第三项在点乘
$\V{N}(s)$ 后也为零。这样便有
\begin{align}
  \tau(s)&=-\dot{\V{B}}(s)\vdp\V{N}(s) \notag \\
  &=-\dot{\V{r}}(s)\cp\frac{\dddot{\V{r}}(s)}
      {\norm[\realR^3]{\ddot{\V{r}}(s)}}
    \vdp\frac{\dddot{\V{r}}(s)}
      {\norm[\realR^3]{\ddot{\V{r}}(s)}} \notag \\
  &=-\frac{1}{\norm[\realR^3]{\ddot{\V{r}}(s)}^2}
    \qty\Big[\dot{\V{r}}(s) \cp \dddot{\V{r}}(s)
        \vdp \ddot{\V{r}}(s)] \notag
  \intertext{利用向量三重积的性质,再把负号移进来,可得}
  &=\frac{1}{\norm[\realR^3]{\ddot{\V{r}}(s)}^2}
    \det\!\mqty[\dot{\V{r}}(s),\,\ddot{\V{r}}(s),\,\dddot{\V{r}}(s)]
  \fullstop
\end{align}

\blankline

至此,我们就得到了 $\realR^3$ 空间中以\emphB{弧长}%
为参数的\emphB{Frenet 标架}\idx{Frenet 标架}:
\begin{braceEq*}{\label{eq:Frenet标架_弧长参数}}
  \V{T}(s) &= \dot{\V{r}}(s) \comma
  \label{eq:Frenet标架定义_T} \\
  \V{N}(s) &= \frac{\dot{\V{T}}(s)}{\norm[\realR^3]{\dot{\V{T}}(s)}}
    =\frac{\ddot{\V{r}}(s)}{\norm[\realR^3]{\ddot{\V{r}}(s)}} \comma
  \label{eq:Frenet标架定义_N} \\
  \V{B}(s) &= \V{T}(s)\cp\V{N}(s)
    =\frac{\dot{\V{r}}(s)\cp\ddot{\V{r}}(s)}
      {\norm[\realR^3]{\ddot{\V{r}}(s)}} \fullstop
  \label{eq:Frenet标架定义_B}
\end{braceEq*}
以及对应的标架运动方程\idx{标架运动方程}
\begin{braceEq*}{\label{eq:Frenet标架运动方程_弧长参数}}
  \dot{\V{T}}(s)&=\kappa(s)\,\V{N}(s) \comma
  \label{eq:Frenet标架的导数_T} \\
  \dot{\V{N}}(s)&=-\kappa(s)\,\V{T}(s)+\tau(s)\,\V{B}(s) \comma
  \label{eq:Frenet标架的导数_N} \\
  \dot{\V{B}}(s)&=-\tau(s)\,\V{N}(s) \comma
  \label{eq:Frenet标架的导数_B}
\end{braceEq*}
其中,
\begin{equation}
  \kappa(s)\defeq\norm[\realR^3]{\dot{\V{T}}(s)}
  =\norm[\realR^3]{\ddot{\V{r}}(s)}
  \label{eq:曲率的定义}
\end{equation}
称为\emphA{曲率}\idx{曲率!弧长参数下的\idxOmit},
\begin{equation}
  \tau(s)\defeq\frac{1}{\norm[\realR^3]{\ddot{\V{r}}(s)}^2}
    \det\!\mqty[\dot{\V{r}}(s),\,\ddot{\V{r}}(s),\,\dddot{\V{r}}(s)]
  =\frac{1}{\kappa^2(s)}
    \det\!\mqty[\dot{\V{r}}(s),\,\ddot{\V{r}}(s),\,\dddot{\V{r}}(s)]
  \label{eq:挠率的定义}
\end{equation}
称为\emphA{挠率}\idx{挠率!弧长参数下的\idxOmit}。

\subsection{Frenet 标架的几何意义}
利用 Taylor 公式,把 $\V{r}(s_0+\incr s)$ 展开至三阶,可得
\begin{equation}
  \V{r}(s_0+\incr s)
  =\V{r}(s_0)+\dot{\V{r}}(s_0)\cdot(\incr s)
    +\frac{1}{2}\,\ddot{\V{r}}(s_0)\cdot(\incr s)^2
    +\frac{1}{6}\,\dddot{\V{r}}(s_0)\cdot(\incr s)^3
    +\sOv{(\incr s)^3} \fullstop
\end{equation}
其中的一阶和二阶导数,根据式~\eqref{eq:Frenet标架_弧长参数},
可以分别表示为
\begin{equation}
  \dot{\V{r}}(s_0)=\V{T}(s_0)
\end{equation}
和
\begin{equation}
  \ddot{\V{r}}(s_0)=\norm[\realR^3]{\ddot{\V{r}}(s_0)} \V{N}(s_0)
  =\kappa(s_0)\,\V{N}(s_0) \fullstop
\end{equation}
至于三阶导数,可用一组单位正交基(此处当然要用 Frenet 标架)展开:
\begin{equation}
  \dddot{\V{r}}(s_0)
  =\qty[\dddot{\V{r}}(s_0)\vdp\V{T}(s_0)]\,\V{T}(s_0)
  +\qty[\dddot{\V{r}}(s_0)\vdp\V{N}(s_0)]\,\V{N}(s_0)
  +\qty[\dddot{\V{r}}(s_0)\vdp\V{B}(s_0)]\,\V{B}(s_0) \fullstop
\end{equation}
关于 $\V{T}(s_0)$、$\V{N}(s_0)$、$\V{B}(s_0)$ 合并同类项,可得
\begin{mySubEq}
  \begin{align}
    &\alspace\V{r}(s_0+\incr s) \notag \\
    &=\V{r}(s_0)
      +\qty[\incr s+\frac{1}{6}\,\dddot{\V{r}}(s_0)\vdp\V{T}(s_0)\,
        (\incr s)^3] \, \V{T}(s_0) \notag \\*
    &\alspace\phantom{\V{r}(s_0)}
      +\qty[\frac{1}{2}\,\kappa(s_0)\,(\incr s)^2
        +\frac{1}{6}\,\dddot{\V{r}}(s_0)\vdp\V{N}(s_0)\,
        (\incr s)^3] \, \V{N}(s_0) \notag \\*
    &\alspace\phantom{\V{r}(s_0)}
      +\qty[\frac{1}{6}\,\dddot{\V{r}}(s_0)\vdp\V{B}(s_0)\,
      (\incr s)^3] \, \V{B}(s_0) + \sOv{(\incr s)^3} \notag \\
    &=\V{r}(s_0)
      +\mqty[\V{T}(s_0),\,\V{N}(s_0),\,\V{B}(s_0)] \,
      \mqty[\displaystyle \incr s
          +\frac{1}{6}\,\dddot{\V{r}}(s_0)\vdp\V{T}(s_0)\,
          (\incr s)^3+\sO{(\incr s)^3} \\[1ex]
        \displaystyle \frac{1}{2}\,\kappa(s_0)\,(\incr s)^2
          +\frac{1}{6}\,\dddot{\V{r}}(s_0)\vdp\V{N}(s_0) \,
          (\incr s)^3+\sO{(\incr s)^3} \\[1ex]
        \displaystyle \frac{1}{6}
          \dddot{\V{r}}(s_0)\vdp\V{B}(s_0) \, (\incr s)^3
          +\sO{(\incr s)^3}]
    \intertext{若只精确到二阶无穷小量,则为}
    &=\V{r}(s_0)
      +\mqty[\V{T}(s_0),\,\V{N}(s_0),\,\V{B}(s_0)] \,
      \mqty[\incr s+\sO{(\incr s)^2} \\[1ex]
        \displaystyle \frac{1}{2}\kappa(s_0)\,(\incr s)^2
          +\sO{(\incr s)^2} \\[1ex]
        \sO{(\incr s)^3}] \fullstop
  \end{align}
\end{mySubEq}

以上推导说明,当误差限制在二阶无穷小量时,$\V{B}$ 方向分量为零。
这样,我们就可以认为曲线只在 $\V{T}$ 和 $\V{N}$ 张成的平面中运动,
此平面称为\emphA{密切平面}\idx{密切平面}。
设其横纵坐标分别为 $\tilde{x}$、$\tilde{y}$,则有
\begin{braceEq}
  \tilde{x}&=\incr s \comma \\
  \tilde{y}&=\frac{1}{2}\,\kappa(s_0)\,(\incr s)^2
    =\frac{1}{2}\,\kappa(s_0)\,\tilde{x}^2 \fullstop
\end{braceEq}
如图所示。
该抛物线与圆心位于 $\qty(0,\,\kappa^{-1}(s_0))$,
半径等于 $\kappa^{-1}(s_0)$ 的圆也是密切的。

\myPROBLEM{密切圆,图见“曲线上标架-Part 02”20 min}。

只有考虑三阶无穷小量时,才可以看出曲线偏离密切平面。
显然,这一偏离的“速率”将由挠率刻画。

\subsection{应用:速度与加速度}
三维空间中,质点的运动轨迹可以用 $\realR^3$ 中的曲线
$\V{r}(t)$ 来表示,其中的参数 $t$ 为时间。\idx{质点的运动}
质点运动的\emphA{速度}\idx{速度} $\V{v}(t)$,定义为
\begin{equation}
  \V{v}(t)\defeq\dot{\V{r}}(t)
  \coloneq\dv{\V{r}}{t} (t)
  =\dv{\V{r}}{s} (s) \cdot \dv{s}{t} (t)
  =\dot{\V{r}}(s)\cdot\dv{s}{t} (t) \fullstop
\end{equation}
由式~\eqref{eq:弧长的定义},弧长的定义为
\begin{equation}
  s(t)=\int_{t_0}^{t} \norm{\dot{\V{r}}(\xi)}\dd{\xi} \comma
\end{equation}
求导,可得
\begin{equation}
  \dv{s}{t} (t) = \norm{\dot{\V{r}}(t)} \fullstop
  \label{eq:弧长的导数}
\end{equation}
再代入 Frenet 标架的定义 \eqref{eq:Frenet标架定义_T}~式,便有
\begin{equation}
  \V{v}(t)=\norm{\dot{\V{r}}(t)}\,\V{T}(s) \fullstop
\end{equation}
可见,速度 $\V{v}(t)$ 的方向与 $\V{T}(t)$ 平行。另外由于 $\V{T}(t)$
是单位向量,因此速度的大小
\begin{equation}
  \norm{\V{v}(t)}=\norm{\dot{\V{r}}(t)} \comma
\end{equation}
我们称之为\emphA{速率}\idx{速率}。

速度相对时间的变化率称为\emphA{加速度}\idx{加速度}:
\begin{align}
  \V{a}(t)\defeq\dot{\V{v}}(t)
  &=\dv{t}\norm{\dot{\V{r}}(t)} \cdot \V{T}(s)
    +\norm{\dot{\V{r}}(t)} \cdot \dv{\V{T}}{t} (s) \notag \\
  &=\dv{t}\norm{\dot{\V{r}}(t)} \cdot \V{T}(s)
    +\norm{\dot{\V{r}}(t)} \cdot \dot{\V{T}}(s) \cdot
    \dv{s}{t} (t) \notag
  \intertext{代入标架运动方程 \eqref{eq:Frenet标架的导数_T}~式以及
    弧长的导数 \eqref{eq:弧长的导数}~式,可有}
  &=\dv{t}\norm{\dot{\V{r}}(t)} \cdot \V{T}(s)
    +\norm{\dot{\V{r}}(t)} \cdot \kappa(s)\,\V{N}(s)
      \cdot \norm{\dot{\V{r}}(t)} \notag
  \intertext{换成速率,则为}
  &=\dv{t}\norm{\V{v}(t)} \cdot \V{T}(s)
  +\norm{\V{v}(t)}^2\,\kappa(s)\,\V{N}(s) \fullstop
\end{align}
由此可知,加速度只有 $\V{T}$ 分量和 $\V{N}$ 分量;
当质点做\emphB{匀变速}运动时,则只有 $\V{N}$ 分量。

\section{Frenet 标架(一般参数)}
\label{sec:Frenet标架_一般参数}
本节我们重回\emphB{一般参数}下的映照形式,即
\begin{equation}
  \mmap{\V{r}(t)}{[\alpha,\,\beta]\ni t}{\V{r}(t)\in\realR^3} \comma
\end{equation}
其中的参数 $t$ 与弧长 $s$
的关系同式~\eqref{eq:弧长与一般参数的关系}:
\begin{equation}
  \dv{s}{t} (t)=\norm[\realR^3]{\dot{\V{r}}(t)} \fullstop
  \label{eq:弧长与一般参数的关系2}
\end{equation}

后文的推导需要用到向量的\emphA{内蕴正交分解}\idx{内蕴正交分解}:
$\forall\,\V{\xi},\,\V{e}\in\realR^3$ 且满足
$\norm[\realR^3]{\V{e}}=1$,有
\begin{equation}
  \V{\xi}=\qty\big(\V{\xi}\vdp\V{e})\,\V{e}
    -\qty\big(\V{\xi}\cp\V{e})\cp\V{e} \fullstop
  \label{eq:内蕴正交分解_Chapter6}
\end{equation}
\myPROBLEM{内蕴正交分解的证明}。

\subsection{标架的形式}
首先来处理单位切向量:\footnote{
  此时我们把 $s$ 作为参数,只不过 $s=s(t)$。所以 $\V{T}(s)$
  实际上和 $\V{T}(t)$ 相等。
  \label{fn:T(s)=T(t)}}
\begin{equation}
  \V{T}(s)\defeq\dot{\V{r}}(s)=\dot{\V{r}}(t)\cdot\dv{t}{s} (s)
  =\flatfrac{\dot{\V{r}}(t)}{\dv{s}{t} (t)}
  =\frac{\dot{\V{r}}(t)}{\norm[\realR^3]{\dot{\V{r}}(t)}} \fullstop
  \label{eq:T_一般参数}
\end{equation}
接下来计算它的导数 $\dot{\V{T}}(s)$:
\begin{align}
  \dot{\V{T}}(s)
  &=\dv{s}\qty[\frac{\dot{\V{r}}(t)}
    {\norm[\realR^3]{\dot{\V{r}}(t)}}] \notag \\
  &=\dv{t}\qty[\frac{\dot{\V{r}}(t)}
    {\norm[\realR^3]{\dot{\V{r}}(t)}}] \cdot \dv{t}{s} (s) \notag
  \intertext{代入 \eqref{eq:弧长与一般参数的关系2}~式,得}
  &=\frac{1}{\norm[\realR^3]{\dot{\V{r}}(t)}}
    \dv{t}\qty[\frac{\dot{\V{r}}(t)}
      {\norm[\realR^3]{\dot{\V{r}}(t)}}] \notag \\
  &=\frac{1}{\norm[\realR^3]{\dot{\V{r}}(t)}^2} \qty[\ddot{\V{r}}(t)
      -\frac{\dot{\V{r}}(t)}{\norm[\realR^3]{\dot{\V{r}}(t)}}
      \cdot\dv{t}\norm[\realR^3]{\dot{\V{r}}(t)}] \fullstop
  \label{eq:T的导数_一般参数_Part1}
\end{align}
此处涉及到了模的导数。考虑
\begin{equation}
  \dv{t}\qty\Big[\norm[\realR^3]{\dot{\V{r}}(t)}^2]
  =2\,\norm[\realR^3]{\dot{\V{r}}(t)}
    \cdot \dv{t}\norm[\realR^3]{\dot{\V{r}}(t)} \semicolon
\end{equation}
另一方面,
\begin{equation}
  \dv{t}\qty\Big[\norm[\realR^3]{\dot{\V{r}}(t)}^2]
  =\dv{t}\qty\big[\dot{\V{r}}(t)\vdp\dot{\V{r}}(t)]
  =2\,\dot{\V{r}}(t)\vdp\ddot{\V{r}}(t) \fullstop
\end{equation}
联立两式,便有
\begin{equation}
  \dv{t}\norm[\realR^3]{\dot{\V{r}}(t)}
  =\ddot{\V{r}}(t) \vdp
    \frac{\dot{\V{r}}(t)}{\norm[\realR^3]{\dot{\V{r}}(t)}} \fullstop
\end{equation}
代回 \eqref{eq:T的导数_一般参数_Part1}~式,继续推导:
\begin{align}
  \dot{\V{T}}(s)
  &=\frac{1}{\norm[\realR^3]{\dot{\V{r}}(t)}^2} \qty[\ddot{\V{r}}(t)
    -\qty(\ddot{\V{r}}(t) \vdp
      \frac{\dot{\V{r}}(t)}{\norm[\realR^3]{\dot{\V{r}}(t)}})
    \frac{\dot{\V{r}}(t)}{\norm[\realR^3]{\dot{\V{r}}(t)}}] \notag
  \intertext{式中的 $\flatfrac{\dot{\V{r}}(t)}{\norm{\dot{\V{r}}(t)}}$
    是一个单位向量,它相当于式~\eqref{eq:内蕴正交分解_Chapter6}
    中的 $\V{e}$,而 $\ddot{\V{r}}(t)$ 则相当于 $\V{\xi}$。因此,
    利用向量的内蕴正交分解,有}
  &=-\frac{1}{\norm[\realR^3]{\dot{\V{r}}(t)}^2}
    \qty[\qty(\ddot{\V{r}}(t)
      \cp \frac{\dot{\V{r}}(t)}{\norm[\realR^3]{\dot{\V{r}}(t)}})
    \cp \frac{\dot{\V{r}}(t)}{\norm[\realR^3]{\dot{\V{r}}(t)}}]
    \notag \\
  &=-\frac{\qty\big(\ddot{\V{r}}(t)\cp\dot{\V{r}}(t))
      \cp\dot{\V{r}}(t)}{\norm[\realR^3]{\dot{\V{r}}(t)}^4} \fullstop
  \label{eq:T的导数_一般参数_Part2}
\end{align}

根据曲率的定义 \eqref{eq:曲率的定义}~式,
\begin{align}
  \kappa(s)\defeq\norm[\realR^3]{\dot{\V{T}}(s)}
  =\frac{\norm[\realR^3]{\qty\big(\ddot{\V{r}}(t)\cp\dot{\V{r}}(t))
    \cp\dot{\V{r}}(t)}}{\norm[\realR^3]{\dot{\V{r}}(t)}^4} \fullstop
\end{align}
注意到 $\qty\big(\ddot{\V{r}}(t)\cp\dot{\V{r}}(t))
\perp\dot{\V{r}}(t)$,所以
\begin{equation}
  \norm[\realR^3]{\qty\big(\ddot{\V{r}}(t)\cp\dot{\V{r}}(t))
    \cp\dot{\V{r}}(t)}
  =\norm[\realR^3]{\ddot{\V{r}}(t)\cp\dot{\V{r}}(t)}
    \cdot\norm[\realR^3]{\dot{\V{r}}(t)} \fullstop
\end{equation}
这样,曲率就能够写成\idx{曲率!一般参数下的\idxOmit!推导过程}
\begin{equation}
  \kappa(s)=\frac{\norm[\realR^3]{\ddot{\V{r}}(t)\cp\dot{\V{r}}(t)}}
    {\norm[\realR^3]{\dot{\V{r}}(t)}^3} \fullstop
  \label{eq:曲率_一般参数}
\end{equation}
利用定义 \eqref{eq:Frenet标架定义_N}~式,
$\V{N}(s)$ 也便可以易如反掌地写出来了:
\begin{equation}
  \V{N}(s)\defeq\frac{\dot{\V{T}}(s)}
    {\norm[\realR^3]{\dot{\V{T}}(s)}}
  =\frac{\dot{\V{T}}(s)}{\kappa(s)}
  =-\frac{\qty\big(\ddot{\V{r}}(t)\cp\dot{\V{r}}(t))
      \cp\dot{\V{r}}(t)}
    {\norm[\realR^3]{\ddot{\V{r}}(t)\cp\dot{\V{r}}(t)}
      \norm[\realR^3]{\dot{\V{r}}(t)}} \fullstop
\end{equation}

最后轮到 $\V{B}(s)$ 了:
\begin{align}
  \V{B}(s)=\V{T}(s)\cp\V{N}(s)
  &=\frac{\dot{\V{r}}(t)}{\norm[\realR^3]{\dot{\V{r}}(t)}}
    \cp\qty[-\frac{\qty\big(\ddot{\V{r}}(t)\cp\dot{\V{r}}(t))
        \cp\dot{\V{r}}(t)}
      {\norm[\realR^3]{\ddot{\V{r}}(t)\cp\dot{\V{r}}(t)}
        \norm[\realR^3]{\dot{\V{r}}(t)}}] \notag \\
  &=\frac{\dot{\V{r}}(t)}
      {\norm[\realR^3]{\ddot{\V{r}}(t)\cp\dot{\V{r}}(t)}}
    \cp\qty[-\qty(\ddot{\V{r}}(t)
        \cp\frac{\dot{\V{r}}(t)}{\norm[\realR^3]{\dot{\V{r}}(t)}})
      \cp\frac{\dot{\V{r}}(t)}
        {\norm[\realR^3]{\dot{\V{r}}(t)}}] \notag
  \intertext{这样处理是为了倒过来应用\emphB{内蕴正交分解}:}
  &=\frac{\dot{\V{r}}(t)}
      {\norm[\realR^3]{\ddot{\V{r}}(t)\cp\dot{\V{r}}(t)}}
    \cp\qty[\ddot{\V{r}}(t)-\qty(\ddot{\V{r}}(t) \vdp
        \frac{\dot{\V{r}}(t)}{\norm[\realR^3]{\dot{\V{r}}(t)}})
      \frac{\dot{\V{r}}(t)}{\norm[\realR^3]{\dot{\V{r}}(t)}}] \notag
  \intertext{由于 $\dot{\V{r}}(t)\cp\dot{\V{r}}(t)=0$,
    因而方括号中的第二项可以略去,使得结果大为简化:}
  &=\frac{\dot{\V{r}}(t)\cp\ddot{\V{r}}(t)}
      {\norm[\realR^3]{\ddot{\V{r}}(t)\cp\dot{\V{r}}(t)}} \fullstop
  \label{eq:B_一般参数}
\end{align}

\subsection{曲率和挠率}
曲率在计算 $\V{N}$ 的时候已经顺带求过了,我们现在来求挠率。
根据式~\eqref{eq:挠率的定义},
有\idx{挠率!一般参数下的\idxOmit!推导过程|(}
\begin{equation}
  \tau(s)=\frac{1}{\kappa^2(s)}
    \det\!\mqty[\dot{\V{r}}(s),\,\ddot{\V{r}}(s),\,\dddot{\V{r}}(s)]
  \fullstop
\end{equation}
因此首先需要知道 $\V{r}$ 关于 $s$ 的一至三阶导数。
由 \eqref{eq:T_一般参数}~式,可知
\begin{equation}
  \dot{\V{r}}(s)=\V{T}(s)
  =\frac{\dot{\V{r}}(t)}{\norm[\realR^3]{\dot{\V{r}}(t)}} \semicolon
\end{equation}
二阶导数则要利用式~\eqref{eq:T的导数_一般参数_Part2}:
\begin{equation}
  \ddot{\V{r}}(s)=\dot{\V{T}}(s)
  =-\frac{\qty\big(\ddot{\V{r}}(t)\cp\dot{\V{r}}(t))
    \cp\dot{\V{r}}(t)}{\norm[\realR^3]{\dot{\V{r}}(t)}^4} \semicolon
\end{equation}
进而又可得到三阶导数:
\begin{equation}
  \dddot{\V{r}}(s)
  =\dv{\ddot{\V{r}}}{s} (s)
  =-\dv{\ddot{\V{r}}}{t} (s) \cdot \dv{t}{s}
  =-\flatfrac{\dv{\ddot{\V{r}}}{t} (s)}{\dv{s}{t} (t)}
  =-\dv{t}\qty[\frac{\qty\big(\ddot{\V{r}}(t)\cp\dot{\V{r}}(t))
      \cp\dot{\V{r}}(t)}{\norm[\realR^3]{\dot{\V{r}}(t)}^4}]
    \cdot\frac{1}{\norm[\realR^3]{\dot{\V{r}}(t)}} \fullstop
\end{equation}
最后一步仍然利用了 \eqref{eq:弧长与一般参数的关系2}~式。
我们知道,
\begin{equation*}
  \det\!\mqty[\dot{\V{r}}(s),\,\ddot{\V{r}}(s),\,\dddot{\V{r}}(s)]
  =\dot{\V{r}}(s)\cp\ddot{\V{r}}(s)\vdp\dddot{\V{r}}(s) \fullstop
\end{equation*}
首先考察叉乘项:
\begin{align}
  \dot{\V{r}}(s)\cp\ddot{\V{r}}(s)
  &=\frac{\dot{\V{r}}(t)}{\norm[\realR^3]{\dot{\V{r}}(t)}}
    \cp \qty[-\frac{\qty\big(\ddot{\V{r}}(t)\cp\dot{\V{r}}(t))
      \cp\dot{\V{r}}(t)}{\norm[\realR^3]{\dot{\V{r}}(t)}^4}] \notag
  \intertext{注意到该式的结构与 \eqref{eq:B_一般参数}~式的
    第二步非常相似,因此我们采用同样的办法处理,即先调整系数,
    再反向运用\emphB{内蕴正交分解}:}
  &=\frac{\dot{\V{r}}(t)}{\norm[\realR^3]{\dot{\V{r}}(t)}^3}
    \cp\qty[-\qty(\ddot{\V{r}}(t)
        \cp\frac{\dot{\V{r}}(t)}{\norm[\realR^3]{\dot{\V{r}}(t)}})
      \cp\frac{\dot{\V{r}}(t)}
        {\norm[\realR^3]{\dot{\V{r}}(t)}}] \notag \\
  &=\frac{\dot{\V{r}}(t)}{\norm[\realR^3]{\dot{\V{r}}(t)}^3}
    \cp\qty[\ddot{\V{r}}(t)-\qty(\ddot{\V{r}}(t) \vdp
        \frac{\dot{\V{r}}(t)}{\norm[\realR^3]{\dot{\V{r}}(t)}})
      \frac{\dot{\V{r}}(t)}
        {\norm[\realR^3]{\dot{\V{r}}(t)}}] \notag
  \intertext{同样,因为 $\dot{\V{r}}(t)\cp\dot{\V{r}}(t)=0$,
    所以又只剩下了第一项,即}
  &=\frac{\dot{\V{r}}(t)\cp\ddot{\V{r}}(t)}
      {\norm[\realR^3]{\dot{\V{r}}(t)}^3} \fullstop
\end{align}
再来看 $\dddot{\V{r}}(t)$。
\begin{align}
  \dddot{\V{r}}(t)
  &=-\dv{t}\qty[\frac{\qty\big(\ddot{\V{r}}(t)\cp\dot{\V{r}}(t))
      \cp\dot{\V{r}}(t)}{\norm[\realR^3]{\dot{\V{r}}(t)}^4}]
    \cdot\frac{1}{\norm[\realR^3]{\dot{\V{r}}(t)}} \notag \\
  &=-\frac{\displaystyle
      \dv{t}\qty\Big[\qty\big(\ddot{\V{r}}(t)\cp\dot{\V{r}}(t))
        \cp\dot{\V{r}}(t)] \cdot \norm[\realR^3]{\dot{\V{r}}(t)}^4
      +\qty\Big[\qty\big(\ddot{\V{r}}(t)\cp\dot{\V{r}}(t))
        \cp\dot{\V{r}}(t)]
        \cdot \dv{t}\qty\Big[\norm[\realR^3]{\dot{\V{r}}(t)}^4]}
    {\norm[\realR^3]{\dot{\V{r}}(t)}^8}
    \cdot\frac{1}{\norm[\realR^3]{\dot{\V{r}}(t)}} \notag \\
  &=-\frac{\displaystyle
      \dv{t}\qty\big(\ddot{\V{r}}(t)\cp\dot{\V{r}}(t))
        \cp\dot{\V{r}}(t)
      +\qty\big(\ddot{\V{r}}(t)\cp\dot{\V{r}}(t))
        \cp\dv{\V{\ddot{r}}}{t} (t)}
      {\norm[\realR^3]{\dot{\V{r}}(t)}^5}
    -\frac{\displaystyle
        \dv{t}\qty\Big[\norm[\realR^3]{\dot{\V{r}}(t)}^4]}
      {\norm[\realR^3]{\dot{\V{r}}(t)}^9}
      \cdot\qty\Big[\qty\big(\ddot{\V{r}}(t)\cp\dot{\V{r}}(t))
        \cp\dot{\V{r}}(t)] \notag \\
  &=-\frac{\qty\big(\dddot{\V{r}}(t)\cp\dot{\V{r}}(t)
        +\ddot{\V{r}}(t)\cp\ddot{\V{r}}(t)) \cp \dot{\V{r}}(t)
      +\qty\big(\ddot{\V{r}}(t)\cp\dot{\V{r}}(t))
        \cp \dddot{\V{r}}(t)}{\norm[\realR^3]{\dot{\V{r}}(t)}^5}
    -\frac{\displaystyle
        \dv{t}\qty\Big[\norm[\realR^3]{\dot{\V{r}}(t)}^4]}
      {\norm[\realR^3]{\dot{\V{r}}(t)}^9}
      \cdot\qty\Big[\qty\big(\ddot{\V{r}}(t)\cp\dot{\V{r}}(t))
        \cp\dot{\V{r}}(t)] \notag
  \intertext{第一项中,$\ddot{\V{r}}(t)\cp\ddot{\V{r}}(t)=0$。所以}
  &=-\frac{\qty\big(\dddot{\V{r}}(t)\cp\dot{\V{r}}(t))
        \cp\dot{\V{r}}(t)
      +\hl{\qty\big(\ddot{\V{r}}(t)\cp\dot{\V{r}}(t))}
        \cp\dddot{\V{r}}(t)}{\norm[\realR^3]{\dot{\V{r}}(t)}^5}
    -\frac{\displaystyle
        \dv{t}\qty\Big[\norm[\realR^3]{\dot{\V{r}}(t)}^4]}
      {\norm[\realR^3]{\dot{\V{r}}(t)}^9}
      \cdot\qty\Big[\hl{\qty\big(\ddot{\V{r}}(t)\cp\dot{\V{r}}(t))}
        \cp\dot{\V{r}}(t)] \fullstop
\end{align}
式中的高亮部分与另一个向量做叉乘后,垂直于
$\ddot{\V{r}}(t)\cp\dot{\V{r}}(t)$;
而另一方面,$\dot{\V{r}}(s)\cp\ddot{\V{r}}(s)$ 又平行于
$\dot{\V{r}}(t)\cp\ddot{\V{r}}(t)$。因此二者做点乘后即为零。
这样,我们就有
\begin{align}
  \dot{\V{r}}(s)\cp\ddot{\V{r}}(s)\vdp\dddot{\V{r}}(s)
  &=\frac{\dot{\V{r}}(t)\cp\ddot{\V{r}}(t)}
      {\norm[\realR^3]{\dot{\V{r}}(t)}^3}
    \vdp\qty[-\frac{\qty\big(\dddot{\V{r}}(t)\cp\dot{\V{r}}(t))
        \cp\dot{\V{r}}(t)}
      {\norm[\realR^3]{\dot{\V{r}}(t)}^5}] \notag \\
  &=\frac{\dot{\V{r}}(t)\cp\ddot{\V{r}}(t)}
      {\norm[\realR^3]{\dot{\V{r}}(t)}^6}
    \vdp\qty[-\qty(\dddot{\V{r}}(t)
        \cp\frac{\dot{\V{r}}(t)}{\norm[\realR^3]{\dot{\V{r}}(t)}})
      \cp\frac{\dot{\V{r}}(t)}
        {\norm[\realR^3]{\dot{\V{r}}(t)}}] \notag
  \intertext{照例,使用\emphB{内蕴正交分解}:}
  &=\frac{\dot{\V{r}}(t)\cp\ddot{\V{r}}(t)}
      {\norm[\realR^3]{\dot{\V{r}}(t)}^6}
    \vdp\qty[\dddot{\V{r}}(t)-\qty(\dddot{\V{r}}(t) \vdp
        \frac{\dot{\V{r}}(t)}{\norm[\realR^3]{\dot{\V{r}}(t)}})
      \frac{\dot{\V{r}}(t)}
        {\norm[\realR^3]{\dot{\V{r}}(t)}}] \notag
  \intertext{第二项点乘后为零:}
  &=\frac{\dot{\V{r}}(t)\cp\ddot{\V{r}}(t)\vdp\dddot{\V{r}}(t)}
      {\norm[\realR^3]{\dot{\V{r}}(t)}^6} \fullstop
\end{align}
此即
\begin{equation}
  \det\!\mqty[\dot{\V{r}}(s),\,\ddot{\V{r}}(s),\,\dddot{\V{r}}(s)]
  =\frac{1}{\norm[\realR^3]{\dot{\V{r}}(t)}^6}
    \det\!\mqty[\dot{\V{r}}(t),\,\ddot{\V{r}}(t),\,\dddot{\V{r}}(t)]
  \fullstop
\end{equation}
再代入一般参数下曲率的表达式~\eqref{eq:曲率_一般参数},
就可得到挠率:\idx{挠率!一般参数下的\idxOmit!推导过程|)}
\begin{align}
  \tau(s)&=\frac{1}{\kappa^2(s)}
    \cdot\frac{1}{\norm[\realR^3]{\dot{\V{r}}(t)}^6}
    \det\!\mqty[\dot{\V{r}}(t),\,\ddot{\V{r}}(t),\,\dddot{\V{r}}(t)]
    \notag \\
  &=\frac{\norm[\realR^3]{\dot{\V{r}}(t)}^6}
      {\norm[\realR^3]{\ddot{\V{r}}(t)\cp\dot{\V{r}}(t)}^2}
    \cdot\frac{1}{\norm[\realR^3]{\dot{\V{r}}(t)}^6}
    \det\!\mqty[\dot{\V{r}}(t),\,\ddot{\V{r}}(t),\,\dddot{\V{r}}(t)]
    \notag \\
  &=\frac{1}{\norm[\realR^3]{\ddot{\V{r}}(t)\cp\dot{\V{r}}(t)}^2}
    \det\!\mqty[\dot{\V{r}}(t),\,\ddot{\V{r}}(t),\,\dddot{\V{r}}(t)]
    \fullstop
\end{align}

\blankline

现在来总结一下\emphB{一般参数}下的 Frenet 标架:
\idx{Frenet 标架!一般参数下的\idxOmit}
\begin{braceEq*}{\label{eq:Frenet标架_一般参数}}
  \V{T}(t)&=\frac{\dot{\V{r}}(t)}{\norm[\realR^3]{\dot{\V{r}}(t)}}
    \comma \\
  \V{N}(t)&=-\frac{\qty\big(\ddot{\V{r}}(t)\cp\dot{\V{r}}(t))
      \cp\dot{\V{r}}(t)}
    {\norm[\realR^3]{\ddot{\V{r}}(t)\cp\dot{\V{r}}(t)}
      \norm[\realR^3]{\dot{\V{r}}(t)}} \comma \\
  \V{B}(t)&=\frac{\dot{\V{r}}(t)\cp\ddot{\V{r}}(t)}
    {\norm[\realR^3]{\ddot{\V{r}}(t)\cp\dot{\V{r}}(t)}} \fullstop
\end{braceEq*}
曲率和挠率分别为
\idx{曲率!一般参数下的\idxOmit}\idx{挠率!一般参数下的\idxOmit}
\begin{equation}
  \kappa(t)=\frac{\norm[\realR^3]{\ddot{\V{r}}(t)\cp\dot{\V{r}}(t)}}
    {\norm[\realR^3]{\dot{\V{r}}(t)}^3}
\end{equation}
和
\begin{equation}
  \tau(t)=
  \frac{1}{\norm[\realR^3]{\ddot{\V{r}}(t)\cp\dot{\V{r}}(t)}^2}
  \det\!\mqty[\dot{\V{r}}(t),\,\ddot{\V{r}}(t),\,\dddot{\V{r}}(t)]
  \fullstop
\end{equation}
注意我们把参数全部换成了 $t$。\footnote{
  根据第~\pageref{fn:T(s)=T(t)}~页的脚注~\ref{fn:T(s)=T(t)},
  $\V{T}(s)=\V{T}(t)$。类似地,还有 $\kappa(s)=\kappa(t)$ 等。
实际上,它们是同一个量在不同参数下的表示。
但 $\dot{\V{T}}(s)\neq\dot{\V{T}}(t)$。这是因为前者是对 $s$ 求导,
而后者则是对 $t$ 求导。不要被符号迷惑。}

至于标架运动方程,则可直接利用
\eqref{eq:Frenet标架运动方程_弧长参数}~式:
\begin{braceEq*}{\label{eq:Frenet标架运动方程_一般参数}}
  \dot{\V{T}}(t)&=\dot{\V{T}}(s)\cdot\dv{s}{t} (t)
    =\norm[\realR^3]{\dot{\V{r}}(t)} \cdot
      \qty\big[\kappa(t)\,\V{N}(t)] \comma \\
  \dot{\V{N}}(t)&=\dot{\V{N}}(s)\cdot\dv{s}{t} (t)
    =\norm[\realR^3]{\dot{\V{r}}(t)} \cdot
      \qty\big[-\kappa(t)\,\V{T}(t)+\tau(t)\,\V{B}(t)] \comma \\
  \dot{\V{B}}(t)&=\dot{\V{B}}(s)\cdot\dv{s}{t} (t)
    =\norm[\realR^3]{\dot{\V{r}}(t)} \cdot
      \qty\big[-\tau(t)\,\V{N}(t)] \fullstop
\end{braceEq*}

若 $t$ 取为 $s$,有
\begin{equation}
  \norm[\realR^3]{\dot{\V{r}}(t)}
  =\norm[\realR^3]{\dot{\V{r}}(s)}=1 \fullstop
\end{equation}
根据式~\eqref{eq:r的一阶导数垂直二阶导数_弧长参数},
$\ddot{\V{r}}(s) \perp \dot{\V{r}}(s)$,因此
\begin{equation}
  \norm[\realR^3]{\ddot{\V{r}}(t)\cp\dot{\V{r}}(t)}
  =\norm[\realR^3]{\ddot{\V{r}}(s)\cp\dot{\V{r}}(s)}
  =\norm[\realR^3]{\ddot{\V{r}}(t)}
    \cdot \norm[\realR^3]{\dot{\V{r}}(t)}
  =\norm[\realR^3]{\ddot{\V{r}}(t)} \fullstop
\end{equation}
利用内蕴正交分解,稍做计算,还可知
\begin{equation}
  \qty\big(\ddot{\V{r}}(t)\cp\dot{\V{r}}(t))\cp\dot{\V{r}}(t)
  =-\ddot{\V{r}}(t)+
    \qty\big(\ddot{\V{r}}(t)\vdp\dot{\V{r}}(t))\,\dot{\V{r}}(t)
  =-\ddot{\V{r}}(t) \fullstop
\end{equation}
此时,把 \eqref{eq:Frenet标架_一般参数}~\linkTilde
~\eqref{eq:Frenet标架运动方程_一般参数}~式
与 \eqref{eq:Frenet标架_弧长参数}~\linkTilde
~\eqref{eq:挠率的定义}~式进行比较,可以发现它们是完全一样的。


  \part{曲面上的张量场场论}
    % 曲面与曲面标架
    \chapter{曲面与曲面标架}
\section{曲面的定义;切空间}
\subsection{曲面的定义}
自变量维数比因变量维数低一维的向量值映照,都可以称为\emphA{曲面}。
常见的三维曲面,其自变量是二维的,正符合了该定义。

如图~\ref{fig:曲面的定义},
$m+1$ 维 Euclid 空间中的 $m$ 维曲面 $\V{\Sigma}(\V{x})$,
都可以用如下的向量值映照表示:
\begin{equation}
  \mmap{\V{\Sigma}(\V{x})}
    {\domD{\V{x}}\ni\V{x}=\mqty[x^1 \\ \vdots \\ x^m]}
    {\V{\Sigma}(\V{x})
      =\mqty[\Sigma^1 \\ \vdots \\ \Sigma^m \\ \Sigma^{m+1}](\V{x})
      \in\Rm*} \fullstop
  \label{eq:曲面的定义}
\end{equation}
式中的 $\domD{\V{x}}$ 代表\emphB{参数域}。

为了表述的清晰,本章中我们将用拉丁字母 $i$、$j$
代表指标 1 到 $m+1$,而用希腊字母 $\mu$、$\nu$ 代表指标 1 到 $m$。

\begin{figure}[h]
  \centering
  \includegraphics{images/surface-definition.png}
  \caption{$m+1$ 维 Euclid 空间中 $m$ 维曲面的定义}
  \label{fig:曲面的定义}
\end{figure}

\subsection{切向量与切空间}
循着与体积上张量场场论(见第\ref{chap:微分同胚}章)相同的思路,
我们先来计算 $\V{\Sigma}(\V{x})$ 的 Jacobi 矩阵:
\begin{equation}
  \JacobiD{\V{\Sigma}}(\V{x})
  =\mqty[\displaystyle \pdv{\V{\Sigma}}{x^1},\,
      \cdots,\,\pdv{\V{\Sigma}}{x^m}] (\V{x})
  \defeq\mqty[\V{g}_1,\,\cdots,\,\V{g}_m] (\V{x})
    \in\realR^{(m+1)\times m} \comma
  \label{eq:曲面映照的Jacobi矩阵}
\end{equation}
式中,
\begin{equation}
  \V{g}_\mu(\V{x}) \defeq \pdv{\V{\Sigma}}{x^\mu} (\V{x})
  =\lim_{\lambda\to 0}
    \frac{\V{\Sigma}\qty(\V{x}+\lambda\,\V{e}_\mu)-\V{\Sigma}(\V{x})}
    {\lambda} \in\Rm* \fullstop
  \label{eq:曲面切向量}
\end{equation}
需要注意,与微分同胚不同,此处的 Jacobi 矩阵\emphB{不是}方阵。

为了考察 $\V{g}_\mu(\V{x})$ 的几何意义,
我们在定义域空间 $\domD{\V{x}}\in\Rm$ 中过点 $\V{x}$ 作出
$x^\mu$-线,其上的任意一点均可表示为 $\V{x}+\lambda\,\V{e}_\mu$。
在映照 $\V{\Sigma}(\V{x})$ 的作用下,
点 $\V{x}$ 和 $\V{x}+\lambda\,\V{e}_\mu$ 分别被映照到值域空间
$\domD{\V{\Sigma}}$ 中的点 $\Sigma(\V{x})$ 和
$\Sigma\qty(\V{x}+\lambda\,\V{e}_\mu)$。同时,$x^\mu$-线也被映照到
$\domD{\V{\Sigma}}$ 中,成为一条曲线。

当 $\lambda\to 0$ 时,
\begin{equation}
  \frac{\V{\Sigma}\qty(\V{x}+\lambda\,\V{e}_\mu)-\V{\Sigma}(\V{x})}
  {\lambda}
  \to \pdv{\V{\Sigma}}{x^\mu} (\V{x}) = \V{g}_\mu(\V{x}) \comma
\end{equation}
这便是值域空间中 $x^\mu$-线的\emphA{切向量}。
切向量张成了 $\Rm*$ 空间的一个子空间:
\begin{equation}
  \vspan\qty{\V{g}_\mu(\V{x})}^m_{\mu=1} \subset\Rm* \fullstop
\end{equation}

如果要构造一组基,必须使其满足\emphB{线性无关}的要求。
由此,我们引出\emphA{正则点}的定义,
它指能够使 Jacobi 矩阵 $\JacobiD{\V{\Sigma}}(\V{x})$ 列满秩,即
\begin{equation*}
  \rank\JacobiD{\V{\Sigma}}(\V{x})=m
\end{equation*}
的点 $\V{x}\in\domD{\V{x}}$ 或 $\V{\Sigma}(\V{x})\in\Rm*$。
在正则点处,便有 $\qty{\V{g}_\mu(\V{x})}^m_{\mu=1} \subset\Rm*$
线性无关。此时,切向量张成的空间称为\emphA{切空间}
(或\emphA{切平面}),记作
\begin{equation}
  \Tspace{\V{\Sigma}}{\V{x}}
  \defeq\vspan\qty{\V{g}_\mu(\V{x})}^m_{\mu=1} \subset\Rm*
  \fullstop
\end{equation}
切空间 $\Tspace{\V{\Sigma}}{\V{x}}$ 的维度是 $m$。

\subsection{局部基}
设位于点 $\V{x}$ 处的 $m+1$ 维单位向量 $\V{n}(\V{x})$ 满足
$\norm[\Rm*]{\V{n}(\V{x})}=1$ 且
\begin{equation}
  (\JacobiD{\V{\Sigma}})\trans(\V{x}) \vdp \V{n}(\V{x})
  =\mqty[\V{g}_1,\,\cdots,\,\V{g}_m]\trans (\V{x}) \vdp \V{n}(\V{x})
  =\mqty[\qty(\V{g}_1)\trans \\ \vdots \\ \qty(\V{g}_m)\trans]
    \vdp \V{n}(\V{x}) = \V{0}\in\Rm \fullstop
  \label{eq:曲面协变基_法向量}
\end{equation}
当 $\JacobiD{\V{\Sigma}}(\V{x})$ 列满秩(即 $\V{x}$ 是正则点)时,
根据线性代数中的\emphB{基扩张定理}
\myPROBLEM[2017-02-13]{基扩张定理},
这样的单位向量 $\V{n}(\V{x})$ 是唯一存在的,称为\emphA{法向量}。

令 $\V{g}_{m+1}\defeq\V{n}(\V{x})$,则
\begin{equation}
  \qty{\V{g}_i(\V{x})}^{m+1}_{i=1}
  =\qty{\pdv{\V{\Sigma}}{x^1} (\V{x}),\,\cdots,\,
    \pdv{\V{\Sigma}}{x^m} (\V{x}),\,\V{n}(\V{x})}
\end{equation}
是 $\Rm*$ 空间中的一组基。
它的指标都在下面,因此是一组\emphA{局部协变基}。

\blankline

接下来研究对应的\emphA{局部逆变基}
$\qty{\V{g}^i(\V{x})}^{m+1}_{i=1}$,
它与局部协变基共同满足\emphA{对偶关系}:\footnote{
  以下在不引起混淆的地方将省去“$(\V{x})$”,
  但仍不要忘记正则点的要求。}
\begin{equation}
  \ipb[\Rm*]{\V{g}_i}{\V{g}^j} = \KroneckerDelta{j}{i} \fullstop
  \label{eq:曲面局部基_对偶关系}
\end{equation}
写成矩阵形式,为
\begin{equation}
  \qty[\begin{array}{@{}c@{}}
    \qty(\V{g}^1)\trans \\ \vdots \\ \qty\big(\V{g}^m)\trans \\[3pt]
    \hdashline \\[-12pt]
    \qty(\V{g}^{m+1})\trans
  \end{array}] \,
  \qty[\begin{array}{@{}c:c@{}}
    \V{g}_1,\,\cdots,\,\V{g}_m & \V{n}
  \end{array}]
  =\qty[\begin{array}{@{}c:c@{}}
    \mqty[\qty(\V{g}^\nu)\trans\vdp\V{g}_\mu]^m_{\mu,\,\nu=1} &
      \mqty[\qty(\V{g}^1)\trans \\ \vdots \\ \qty(\V{g}^m)\trans]
      \vdp \V{n} \\[3pt]
    \hdashline \\[-12pt]
    \qty(\V{g}^{m+1})\trans \vdp \mqty[\V{g}_1,\,\cdots,\,\V{g}_m] &
    \qty(\V{g}^{m+1})\trans \vdp \V{n}
  \end{array}]
  =\Mat{I}_{m+1} \fullstop
  \label{eq:曲面局部基_对偶关系_矩阵形式}
\end{equation}

按照分块矩阵的计算法则,显然有
$\qty(\V{g}^{m+1})\trans \vdp \V{n}=1$。于是
\begin{equation}
  \V{g}^{m+1} = \V{n}+\sum_{\mu=1}^{m} a_\mu\,\V{g}_\mu \comma
  \label{eq:曲面逆变基m+1_完整形式}
\end{equation}
式中的 $a_\mu$ 是待定系数。

考虑式~\eqref{eq:曲面局部基_对偶关系_矩阵形式} 中矩阵的左下角,有
\begin{align}
  \qty(\V{g}^{m+1})\trans \vdp \mqty[\V{g}_1,\,\cdots,\,\V{g}_m]
  &=\qty(\V{n}+\sum_{\mu=1}^{m} a_\mu\,\V{g}_\mu)\trans
    \vdp \mqty[\V{g}_1,\,\cdots,\,\V{g}_m] \notag
  \intertext{根据 \eqref{eq:曲面协变基_法向量}~式,
    $\V{n}$ 与 $\V{g}_\mu$ 正交:}
  &=\sum_{\mu=1}^{m} a_\mu\,\qty(\V{g}_\mu)\trans
    \vdp \mqty[\V{g}_1,\,\cdots,\,\V{g}_m] \notag \\
  &=\mqty[\displaystyle
      \sum_{\mu=1}^{m} a_\mu\,\qty(\V{g}_\mu)\trans
        \vdp\V{g}_1,\,\cdots,\,
      \sum_{\mu=1}^{m} a_\mu\,\qty(\V{g}_\mu)\trans
        \vdp\V{g}_m] \notag \\
  &\defeq \mqty[\displaystyle
      \sum_{\mu=1}^{m} a_\mu\,g_{1\mu},\,\cdots,\,
      \sum_{\mu=1}^{m} a_\mu\,g_{m\mu}] \notag \\
  &=\V{0}\in\realR^{1\times m} \fullstop
\end{align}
式中的 $g_{\mu\nu}\defeq\qty(\V{g}_\nu)\trans\vdp\V{g}_\mu
  =\ipb[\Rm*]{\V{g}_\mu}{\V{g}_\nu}$。转置一下,可得
\begin{align}
  \mqty[\displaystyle \sum_{\mu=1}^{m} a_\mu\,g_{1\mu} \\ \vdots \\
    \displaystyle \sum_{\mu=1}^{m} a_i\,g_{m\mu}]
  &=\mqty[g_{11} & \cdots & g_{1m} \\
      \vdots & \ddots & \vdots \\
      g_{m1} & \cdots & g_{mm}]\,
    \mqty[a_1 \\ \vdots \\ a_m] \notag \\
  &=\mqty[\qty(\V{g}_1)\trans \\ \vdots \\ \qty(\V{g}_m)\trans]
    \mqty[\V{g}_1,\,\cdots,\,\V{g}_m] \,
    \mqty[a_1 \\ \vdots \\ a_m] \notag \\
  &=\mqty[\V{g}_1,\,\cdots,\,\V{g}_m]\trans
    \mqty[\V{g}_1,\,\cdots,\,\V{g}_m] \,
    \mqty[a_1 \\ \vdots \\ a_m]
  \defeq\qty(\Mat{A}\trans\Mat{A}) \, \mqty[a_1 \\ \vdots \\ a_m]
  =\V{0}\in\Rm \fullstop
\end{align}
其中,矩阵 $\Mat{A}\defeq\qty[\V{g}_1,\,\cdots,\,\V{g}_m]
  \in\realR^{(m+1)\times m}$。
由于处在\emphB{正则点} $\V{x}$ 处,
$\qty{\V{g}_\mu}^m_{\mu=1}$ 线性无关,因此 $\rank\Mat{A}=m$。
根据线性代数的知识,
\begin{equation}
  \rank\qty(\Mat{A}\trans\Mat{A}) = \rank\Mat{A} = m \comma
\end{equation}
所以矩阵 $\Mat{A}\trans\Mat{A}$ 非奇异。
这样就必然有 $\qty[a_1,\,\cdots,\,a_m]\trans=\V{0}\in\Rm$,
即 $a_\mu=0$。代回到 \eqref{eq:曲面逆变基m+1_完整形式}~式,可知
\begin{equation}
  \V{g}^{m+1}=\V{n} \fullstop
\end{equation}

再来看矩阵的右上角:
\begin{equation}
  \mqty[\qty(\V{g}^1)\trans \\ \vdots \\ \qty(\V{g}^m)\trans]
    \vdp \V{n}
  =\mqty[\qty(\V{g}^1)\trans\vdp\V{n} \\ \vdots \\
    \qty(\V{g}^m)\trans\vdp\V{n}]
  =\V{0}\in\Rm \comma
\end{equation}
因此 $\V{g}^\mu\perp\V{n}$。又因为 $\V{n}$ 和
$\vspan\qty{\V{g}_i(\V{x})}^m_{\mu=1}$ 共同构成了基,所以
\begin{equation}
  \V{g}^\mu\in\vspan\qty{\V{g}_\mu(\V{x})}^m_{\mu=1} \fullstop
\end{equation}

最后轮到左上角:
\begin{equation}
  \mqty[\qty(\V{g}^\nu)\trans\vdp\V{g}_\mu]^m_{\mu,\,\nu=1}=\Mat{I}_m
  \iff \ipb[\Rm*]{\V{g}_\mu}{\V{g}^\nu}
    =\KroneckerDelta{\nu}{\mu} \fullstop
  \label{eq:曲面局部基_切空间_对偶关系}
\end{equation}
这是一个与式~\eqref{eq:曲面局部基_对偶关系} 类似的“对偶关系”,
不过请注意指标取值的不同。此处的“对偶关系”仅存在于\emphB{切空间}
$\Tspace{\V{\Sigma}}{\V{x}}
  =\vspan\qty{\V{g}_\mu(\V{x})}^m_{\mu=1}$ 中。

\begin{figure}[h]
  \centering
  \includegraphics{images/surface-local-basis.png}
  \caption{$m+1$ 维 Euclid 空间中 $m$ 维曲面上的局部基}
  \label{fig:曲面上的局部基}
\end{figure}

总结一下前面得到的结果。如图~\ref{fig:曲面上的局部基} 所示,
对于 $m+1$ 维空间中的 $m$ 维曲面而言,
其协变基由切向量与法向量共同组成:
\begin{equation}
  \qty{\V{g}_i(\V{x})}^m_{i=1}
  \defeq\qty{\pdv{\V{\Sigma}}{x^1} (\V{x}),\,\cdots,\,
    \pdv{\V{\Sigma}}{x^m} (\V{x}),\,\V{n}(\V{x})} \semicolon
\end{equation}
至于逆变基,它的前 $m$ 个向量由切空间上的对偶关系
\eqref{eq:曲面局部基_切空间_对偶关系}~式决定,而第 $m+1$ 个向量则为
\begin{equation}
  \V{g}^{m+1}(\V{x})=\V{g}_{m+1}(\V{x})=\V{n}(\V{x}) \fullstop
\end{equation}

\subsection{曲面上的曲线}
首先要在参数域 $\domD{\V{x}}$ 中定义曲线:
\begin{equation}
  \mmap{\V{\Gamma}_\V{x}(t)}{[a,\,b]\ni t}
    {\V{\Gamma}_\V{x}(t)
      =\mqty[\Gamma_\V{x}^1(t) \\ \vdots \\ \Gamma_\V{x}^m(t)]
      =\mqty[x^1(t) \\ \vdots \\ x^m(t)]
        \in\domD{\V{x}}\in\Rm} \fullstop
\end{equation}
这里用下标 $\V{x}$ 表示参数域。
于是,曲面(物理域)上的曲线就可以表示为
$\V{\Gamma}_\V{x}$ 与 $\V{\Sigma}$ 的复合:
\begin{equation}
  \mmap{\V{\Gamma}_\V{\Sigma}(t)}{[a,\,b]\ni t}
    {\V{\Gamma}_\V{\Sigma}(t)
      \defeq\V{\Sigma}\comp\V{\Gamma}_\V{x}(t)
      =\V{\Sigma}\qty\big[\V{x}(t)] \in\Rm*} \fullstop
\end{equation}
类似地,下标 $\V{\Sigma}$ 表示曲面。

下面计算曲面上曲线的切向量:
\begin{align}
  \dv{\V{\Gamma}_\V{\Sigma}}{t} (t)
  =\JacobiD{\V{\Gamma}_\V{\Sigma}} (t)
  &=\JacobiD{\V{\Sigma}} \qty\big(\V{x}(t))
    \vdp \JacobiD{\V{\Gamma}_\V{x}} (t) \notag
  \intertext{上面一个等号利用了\emphB{复合映照的可微性定理}。
    再根据式~\eqref{eq:曲面映照的Jacobi矩阵},
    $\V{\Sigma}(\V{x})$ 的 Jacobi 矩阵由(前 $m$ 个)
    局部协变基组成,而 $\V{\Gamma}_\V{x} (t)$ 的 Jacobi
    矩阵又可以直接写出来,于是}
  &=\mqty[\V{g}_1 \qty\big(\V{x}(t)),\,\cdots,\,
      \V{g}_m \qty\big(\V{x}(t))] \,
    \mqty[\dot{x}^1(t) \\ \vdots \\ \dot{x}^m(t)] \notag \\
  &=\dot{x}^\mu(t) \, \V{g}_\mu\qty\big(\V{x}(t)) \fullstop
\end{align}
不用忘记按照 Einstein 求和约定对哑标 $\mu$ 进行求和。可以发现,
\begin{equation}
  \dv{\V{\Gamma}_\V{\Sigma}}{t} (t)
  =\dot{x}^\mu(t) \, \V{g}_\mu\qty\big(\V{x}(t))
    \in\Tspace{\V{\Sigma}}{\V{x}}
      =\vspan\qty{\V{g}_\mu(\V{x})}^m_{\mu=1} \comma
\end{equation}
说明曲面上所有曲线的切向量都落在该曲面的切空间中。

在\emphB{体积上}的张量场场论中,如果非要使用典则基,也并非不可;
但在\emphB{曲面上}的张量场场论中,曲面上的局部基却是完全无法
被典则基所取代的。如果使用典则基,自变量的微小误差,
就会使得像点落在曲面之外。这在工业设计领域将是不可接受的。
\myPROBLEM[2017-02-25]{见 12-曲面定义及其切空间-Part 02.flv 最后}

\section{度量张量与曲率张量}
\label{sec:度量张量与曲率张量}
\subsection{第一基本形式;度量张量}
曲面的\emphA{第一基本形式}指切空间中的内积,即
\begin{equation}
  g_{\mu\nu}\defeq\ipb[\Rm*]{\V{g}_\mu}{\V{g}_\nu} \comma
  \label{eq:曲面第一基本形式}
\end{equation}
其中的 $\mu$、$\nu$ 表示 1 到 $m$。$g_{\mu\nu}$ 构成的矩阵为
\begin{equation}
  \Mat{G}\coloneq\mqty[g_{\mu\nu}]
  =\mqty[g_{11} & \cdots & g_{1m} \\
    \vdots & \ddots & \vdots \\
    g_{m1} & \cdots & g_{mm}]
  =\mqty[\qty(\V{g}_1)\trans \\ \vdots \\ \qty(\V{g}_m)\trans]\,
    \mqty[\V{g}_1,\,\cdots,\,\V{g}_m]
  =(\JacobiD{\V{\Sigma}})\trans \vdp (\JacobiD{\V{\Sigma}})
  \fullstop
\end{equation}
根据内积的对称性,$g_{\mu\nu}=g_{\nu\mu}$,于是矩阵 $\Mat{G}$
是一个对称阵。除此以外,在正则点处,它还是一个\emphB{正定}矩阵。

\begin{myProof}
对于任意的向量 $\V{\xi}\in\Rm$,有
\begin{align}
  \V{\xi}\trans\,\Mat{G}\,\V{\xi}
  &=\V{\xi}\trans \,
    \qty\Big[(\JacobiD{\V{\Sigma}})\trans
      \vdp (\JacobiD{\V{\Sigma}})] \,
    \V{\xi} \notag \\
  &=\qty\Big[\V{\xi}\trans (\JacobiD{\V{\Sigma}})\trans]
    \vdp \qty\Big[\qty(\JacobiD{\V{\Sigma}}) \, \V{\xi}]
  =\qty\Big[\qty(\JacobiD{\V{\Sigma}}) \, \V{\xi}]\trans
    \vdp \qty\Big[\qty(\JacobiD{\V{\Sigma}}) \, \V{\xi}] \notag \\
  &=\norm[\Rm*]{\vphantom{0^0}
      \qty(\JacobiD{\V{\Sigma}}) \, \V{\xi}}^2
  =\norm[\Rm*]{\xi^\mu\,\V{g}_\mu}^2 \geqslant 0 \fullstop
\end{align}
这样,若有 $\V{\xi}\trans\,\Mat{G}\,\V{\xi}=0$,则必有
$\xi^\mu\,\V{g}_\mu=0$。而在正则点处,
$\qty{\V{g}_\mu(\V{x})}^m_{\mu=1} \subset\Rm*$ 线性无关。
于是 $\V{g}_\mu$ 的系数就全都为零,即
$\xi^\mu=0\in\realR$,$\V{\xi}=\V{0}\in\Rm$。
这就证明了 $\Mat{G}$ 的正定性。
\end{myProof}

在曲面上的张量场场论中,我们以后的所有讨论都将只考虑处在正则点处的
情况。这相当于体积上张量场场论中\emphB{微分同胚}的条件。

\blankline

利用曲面上的第一基本形式,可以定义\emphA{度量张量}:
\begin{equation}
  \T{G} \defeq g_{\mu\nu}\,\V{g}^\mu\tp\V{g}^\nu
  \in\Tensors[\Rm*]{2} \comma
\end{equation}
其中的 $g_{\mu\nu}\defeq\ipb[\Rm*]{\V{g}_\mu}{\V{g}_\nu}$。

把 $\V{g}_\nu$ 用局部\emphB{逆变}基展开,可有
\begin{align}
  \V{g}_\nu &=\ipb[\Rm*]{\V{g}_\nu}{\V{g}_i}\,\V{g}^i \notag \\
  &=\ipb[\Rm*]{\V{g}_\nu}{\V{g}_\mu}\,\V{g}^\mu
    +\ipb[\Rm*]{\V{g}_\nu}{\V{n}}\,\V{n} \notag
  \intertext{由于 $\V{g}_\nu$ 与 $\V{n}$ 正交,因此}
  &=\ipb[\Rm*]{\V{g}_\nu}{\V{g}_\mu}\,\V{g}^\mu
  =g_{\mu\nu}\,\V{g}^\mu \fullstop
  \label{eq:曲面局部基_指标升降_1}
\end{align}
同理,还可以知道
\begin{equation}
  \V{g}^\nu=g^{\mu\nu}\,\V{g}_\mu \fullstop
  \label{eq:曲面局部基_指标升降_2}
\end{equation}
此处的 $g^{\mu\nu}$ 自然等于 $\ipb[\Rm*]{\V{g}^\mu}{\V{g}^\nu}$。

据此,我们便可以获得度量张量的其他形式:
\begin{mySubEq}
  \begin{align}
    \T{G} &\defeq g_{\mu\nu}\,\V{g}^\mu\tp\V{g}^\nu \notag
    \intertext{代入 \eqref{eq:曲面局部基_指标升降_1}~式,
      并利用 Krnonecker δ,可得}
    &=\V{g}_\nu\tp\V{g}^\nu
    =\KroneckerDelta{\mu}{\nu}\,\V{g}_\mu\tp\V{g}^\nu \fullstop
    \intertext{这是\emphB{混合分量}。代入
      \eqref{eq:曲面局部基_指标升降_2}~式,又有}
    \T{G} &= \V{g}_\nu\tp\V{g}^\nu
    =g^{\mu\nu}\,\V{g}_\mu\tp\V{g}_\nu \fullstop
  \end{align}
\end{mySubEq}
这是\emphB{逆变分量}。

\subsection{第二基本形式;曲率张量}
曲面的\emphA{第二基本形式}定义为
\begin{equation}
  b_{\mu\nu}\defeq\ipb[\Rm*]{\pdv{\V{g}_\nu}{x^\mu}}{\V{n}} \fullstop
  \label{eq:曲面的第二基本形式}
\end{equation}
与第一基本形式类似,矩阵 $\Mat{B}\coloneq\mqty[b_{\mu\nu}]$
也是对称的(但未必正定)。

\begin{myProof}
对称性的根源在于二阶偏导数可以交换次序:
\begin{align}
  \pdv{\V{g}_\nu}{x^\mu}
  =\pdv{x^\mu} \qty(\pdv{\V{\Sigma}}{x^\nu})
  =\pdv{\V{\Sigma}}{x^\mu}{x^\nu}
  =\pdv{\V{\Sigma}}{x^\nu}{x^\mu}
  =\pdv{x^\nu} \qty(\pdv{\V{\Sigma}}{x^\mu})
  =\pdv{\V{g}_\mu}{x^\nu} \fullstop
\end{align}
于是 $b_{\mu\nu}=b_{\nu\mu}$。
\end{myProof}

\blankline

对应于第二基本形式,我们可以定义\emphA{曲率张量}:
\begin{equation}
  \T{K} \defeq b_{\mu\nu}\,\V{g}^\mu\tp\V{g}^\nu
  \in\Tensors[\Rm*]{2} \comma
\end{equation}
式中的 $b_{\mu\nu}$ 由 \eqref{eq:曲面的第二基本形式}~式定义。

引入 $\tc{b}{^\mu_\nu}\coloneq g^{\mu\sigma} b_{\sigma\nu}$,
即可获得曲率张量在混合分量下的表示:
\begin{mySubEq}
  \begin{align}
    \T{K} &\defeq b_{\mu\nu}\,\V{g}^\mu\tp\V{g}^\nu \notag \\
    &=b_{\mu\nu}\,g^{\mu\sigma}\,\V{g}_\sigma\tp\V{g}^\nu \notag \\
    &=\tc{b}{^\sigma_\nu}\,\V{g}_\sigma\tp\V{g}^\nu
    =\tc{b}{^\mu_\nu}\,\V{g}_\mu\tp\V{g}^\nu \semicolon
    \intertext{再引入 $b^{\mu\nu}
      \coloneq g^{\nu\sigma} \tc{b}{^\mu_\sigma}$,
      又可获得逆变分量下的表示:}
    \T{K}&=\tc{b}{^\mu_\nu}\,\V{g}_\mu\tp\V{g}^\nu \notag \\
    &=\tc{b}{^\mu_\nu}\,g^{\nu\sigma}\,
      \V{g}_\mu\tp\V{g}^\sigma \notag \\
    &=b^{\mu\sigma}\,\V{g}_\mu\tp\V{g}_\sigma
    =b^{\mu\nu}\,\V{g}_\mu\tp\V{g}_\nu \fullstop
  \end{align}
\end{mySubEq}
以上推导中两式的最后一步都只是哑标的简单替换。

至于另一个混合分量 $\tc{b}{_\mu^\nu}$,根据对称性,可知
\begin{equation}
  \tc{b}{_\mu^\nu}
  \coloneq g_{\mu\sigma} b^{\sigma\nu}
  =g_{\mu\sigma} b^{\nu\sigma}=\tc{b}{^\nu_\mu} \fullstop
\end{equation}
所以我们把曲率张量 $\T{K}$ 的混合分量统一记为 $b^\mu_\nu$,即
\begin{equation}
  b^\mu_\nu \coloneq \tc{b}{^\mu_\nu} = \tc{b}{_\nu^\mu} \fullstop
\end{equation}

\section{Gauss 曲率与平均曲率}
\subsection{定义} \label{subsec:Gauss曲率与平均曲率_定义}
根据线性代数中的\emphA{同时对角化}定理,
\myPROBLEM[2017-02-25]{同时对角化}
由于第一、第二基本形式的表示矩阵 $\Mat{G}$ 和 $\Mat{B}$
均是对称矩阵,并且 $\Mat{G}$ 还具有正定性,因此必然唯一存在
一个非奇异的矩阵 $\Mat{S}\in\realR^{m\times m}$,使得
\begin{equation}
  \Mat{S}\trans\Mat{G}\Mat{S}=\Mat{I}_m \quad\text{且}\quad
  \Mat{S}\trans\Mat{B}\Mat{S}
  =\mqty[\dmat{\lambda_1,\ddots,\lambda_m}] \comma
\end{equation}
其中的 $\lambda_\mu$ 满足
\begin{equation}
  \det(\Mat{B}-\lambda_\mu\,\Mat{G})=0 \fullstop
\end{equation}
根据行列式的性质,可知
\begin{equation}
  \det(\Mat{G}^{-1}\Mat{B}-\lambda_\mu\,\Mat{I}_m)
  =\det\qty\Big[\Mat{G}^{-1} \qty(\Mat{B}-\lambda_\mu\,\Mat{G})]
  =\det(\Mat{G}^{-1}) \cdot \det(\Mat{B}-\lambda_\mu\,\Mat{G})
  =\det(\Mat{G}^{-1}) \cdot 0 = 0 \fullstop
\end{equation}
这说明 $\lambda_i$ 是矩阵 $\Mat{G}^{-1}\Mat{B}$ 的特征值。

定义\emphA{Gauss 曲率} $\KG$ 为这些特征值之积,它等于矩阵
$\Mat{G}^{-1}\Mat{B}$ 的行列式:
\begin{equation}
  \KG\defeq\prod_{\mu=1}^{m} \lambda_\mu = \det(\Mat{G}^{-1}\Mat{B})
  =\frac{\det\Mat{B}}{\det\Mat{G}} \semicolon
  \label{eq:Gauss曲率定义}
\end{equation}
定义\emphA{平均曲率} $H$ 为特征值之和的平均值,
它等于矩阵的迹除以 $m$:
\begin{equation}
  H\defeq\frac{1}{m}\sum_{\mu=1}^{m} \lambda_\mu
  =\frac{\tr(\Mat{G}^{-1}\Mat{B})}{m} \fullstop
  \label{eq:平均曲率定义}
\end{equation}

我们知道度量满足
\begin{equation}
  g_{\mu\sigma}\,g^{\sigma\nu}=\KroneckerDelta{\nu}{\mu} \fullstop
\end{equation}
考虑到 $\Mat{G}=\mqty[g_{\mu\nu}]$ 以及
$\Mat{B}=\mqty[b_{\mu\nu}]$,便有
\begin{equation}
  \Mat{G}^{-1}=\mqty[g_{\mu\nu}]^{-1}=\mqty[g^{\mu\nu}] \comma
\end{equation}
即度量张量逆变分量对应的矩阵。因此
\begin{equation}
  \Mat{G}^{-1}\Mat{B}=\mqty[g^{\mu\nu}]\mqty[b_{\mu\nu}]
  =\mqty[g^{\mu\sigma}\,b_{\sigma\nu}]
  =\mqty[b^\mu_\nu] \fullstop
\end{equation}
这是曲率张量\emphB{混合分量}所对应的矩阵。此时,
$\lambda_\mu$ 就成为了矩阵 $\mqty[b^\mu_\nu]$ 的特征值,
而两种曲率也可以分别写成
\begin{braceEq}
  \KG &=\det(\Mat{G}^{-1}\Mat{B})
    =\det\!\mqty[b^\mu_\nu] \comma \label{eq:Gauss曲率_表示} \\
  H &=\frac{1}{m}\frac{\tr(\Mat{G}^{-1}\Mat{B})}{m}
    =\frac{1}{m}\tr\mqty[b^\mu_\nu]
    =\frac{1}{m}\cdot b^\mu_\nu
    =\frac{1}{m}\qty\Big(b^1_1+b^2_2+\cdots+b^p_p) \fullstop
    \label{eq:平均曲率_表示}
\end{braceEq}
第二式的展开是根据了矩阵迹的定义。

\subsection{Gauss 映照}
以下叙述将按照三维空间中的曲面展开,当然若要推广到 $\Rm$
空间也并非难事。

设曲面 $\V{\Sigma}(\V{x})$ 上的每一点处都定义有法向量
$\V{n}(\V{x})$。如果 $\V{\Sigma}(\V{x})$ 是平坦的,
则各法向量就应当平行分布,类似于成片的小树林;
而如果 $\V{\Sigma}(\V{x})$ 是弯曲的,那么这些法向量就会分散开去,
好比刺猬身上张开的刺。

为了定量刻画曲面 $\V{\Sigma}(\V{x})$ 的弯曲程度,Gauss 构造了一个从
该曲面到二维单位球面 $S^2$ 的映照,
即\emphA{Gauss 映照}:\idx{Gauss 映照}
在 $P\in\V{\Sigma}$ 点附近取一小块邻域 $\Sigma (\V{x})$%
\footnote{注意“$\Sigma$”加粗与否代表着不同的意思。
  加粗的“$\V{\Sigma}$” 代表曲面(此处是一个 $\realR^2$ 到 $\realR^3$
  的映照);而不加粗的 “$\Sigma$”则仅仅表示曲面上的一块区域,
  不存在映照的概念。对于 $\sigma$ 也是一样的。
  另外注意 $\V{x}\in\domD{\V{x}}$ 是参数域中的点,可以表示为
  $\V{x}=\qty[x^1,\,x^2]\trans$。},
将该邻域中每一点处的单位法向量 $\V{n}$ \emphB{平行移动}到单位球所在的
坐标系中,并使该向量的起点位于坐标系原点。
这样,向量的终点就会落在单位球面 $S^2$ 上,并扫出一片区域
$\sigma (\V{x})$。如前文所言,由于曲面弯曲程度的不同,
这些法向量的张开程度也就不同。而这种“张开程度”,则可以用区域
$\sigma (\V{x})$ 相对于球心的立体角来描述。

\begin{figure}[h]
  \centering
  \includegraphics{images/gauss-mapping.png}
  \caption{Gauss 映照}
  \label{fig:Gauss映照}
\end{figure}

由此,曲面 $\V{\Sigma}$ 在 $P$ 点附近的弯曲程度就可以定义为
\begin{equation}
  K=\lim_{\norm[]{\Sigma (\V{x})}\to 0}
    \frac{\norm[]{\sigma(\V{x})}}{\norm[]{\Sigma(\V{x})}} \fullstop
  \label{eq:Gauss曲率计算式_1}
\end{equation}
式中的 $\norm[]{\Sigma (\V{x})}$ 和 $\norm[]{\sigma(\V{x})}$
分别表示 $\Sigma (\V{x})$ 和 $\sigma (\V{x})$ 两块区域的面积。

下面我们来计算 $K$。根据式~\eqref{eq:曲面的定义},
曲面 $\V{\Sigma}(\V{x})$ 可以用如下映照表示:
\begin{equation}
  \mmap{\V{\Sigma}(\V{x})}
    {\domD{\V{x}}\ni\V{x}=\mqty[x^1 \\ x^2]}
    {\V{\Sigma}(\V{x})
      =\mqty[\Sigma^1 \\ \Sigma^2 \\ \Sigma^3](\V{x}) \in\realR^3}
    \fullstop
\end{equation}
至于球面 $S^2$,它是由单位矢量绕原点旋转而扫出来的。
对于我们所感兴趣的区域 $\sigma (\V{x})$,单位矢量可以取为从曲面
$\V{\Sigma}$ 上平行移动过来的法向量 $\V{n}$。
于是球面 $\V{\sigma}(\V{x})$ (当然只有我们关注的一部分)
就可以表示为
\begin{equation}
  \mmap{\V{\sigma}(\V{x})}
    {\domD{\V{x}}\ni\V{x}=\mqty[x^1 \\ x^2]}
    {\V{\sigma}(\V{x})=\V{n}(\V{x})\in\realR^3} \fullstop
\end{equation}

在 $P$ 点附近,曲面 $\V{\Sigma}(\V{x})$ 可以用平面逼近。于是
\begin{equation}
  \norm[]{\Sigma (\V{x})}
  \approx \norm[\realR^3]{\pdv{\V{\Sigma}}{x^1} \cp
      \pdv{\V{\Sigma}}{x^2}} \fullstop
\end{equation}
同理,球面上区域 $\sigma (\V{x})$ 的面积近似为
\begin{equation}
  \norm[]{\sigma (\V{x})}
  \approx \norm[\realR^3]{\pdv{\V{n}}{x^1} \cp
      \pdv{\V{n}}{x^2}} \fullstop
\end{equation}
代入式~\eqref{eq:Gauss曲率计算式_1}。
由于我们只探讨正则点处的情形,因此取极限后等号就能够取到:
\begin{equation}
  K=\lim_{\norm[]{\Sigma (\V{x})}\to 0}
    \frac{\norm[]{\sigma(\V{x})}}{\norm[]{\Sigma(\V{x})}}
  =\frac{\displaystyle \norm[\realR^3]{\pdv{\V{n}}{x^1} \cp
      \pdv{\V{n}}{x^2}}}
    {\displaystyle \norm[\realR^3]{\pdv{\V{\Sigma}}{x^1} \cp
      \pdv{\V{\Sigma}}{x^2}}} \fullstop
  \label{eq:Gauss曲率计算式_2}
\end{equation}

考虑一般情况下法向量 $\V{n}$ 的偏导数。该偏导数仍为一个向量,
因此可以用局部(逆变)基展开:
\begin{equation}
  \pdv{\V{n}}{x^\mu}
  =\ipb[\Rm*]{\pdv{\V{n}}{x^\mu}}{\V{g}_\nu}\,\V{g}^\nu
    +\ipb[\Rm*]{\pdv{\V{n}}{x^\mu}}{\V{n}}\,\V{n} \fullstop
\end{equation}
式中的第一个内积可以写成
\begin{align}
  \ipb[\Rm*]{\pdv{\V{n}}{x^\mu}}{\V{g}_\nu}
  &=\pdv{x^\mu} \ipb[\Rm*]{\V{n}}{\V{g}_\nu}
    -\ipb[\Rm*]{\V{n}}{\pdv{\V{g}_\nu}{x^\mu}} \notag
  \intertext{代入 $\ipb[\Rm*]{\V{n}}{\V{g}_\nu}=0$,得}
  &=-\ipb[\Rm*]{\pdv{\V{g}_\nu}{x^\mu}}{\V{n}}
  =-b_{\mu\nu} \comma
\end{align}
这就是式~\eqref{eq:曲面的第二基本形式} 所定义的曲面第二基本形式;
第二个内积则等于零:
\begin{equation}
  \ipb[\Rm*]{\pdv{\V{n}}{x^\mu}}{\V{n}}
  =\frac{\displaystyle \ipb[\Rm*]{\pdv{\V{n}}{x^\mu}}{\V{n}}
    +\ipb[\Rm*]{\V{n}}{\pdv{\V{n}}{x^\mu}}}{2}
  =\frac{1}{2} \pdv{x^\mu} \ipb[\Rm*]{{\V{n}}}{{\V{n}}}
  =\frac{1}{2} \pdv{x^\mu} \norm[\Rm*]{\V{n}}^2
  =0 \fullstop
\end{equation}
所以
\begin{equation}
  \pdv{\V{n}}{x^\mu}
  =-b_{\mu\nu}\,\V{g}^\nu
  =-b^\tau_\mu\,g_{\tau\nu} \cdot g^{\sigma\nu}\,\V{g}_\sigma
  =-b^\tau_\mu\,\KroneckerDelta{\tau}{\sigma}\,\V{g}_\sigma
  =-b^\tau_\mu\,\V{g}_\tau=-b^\nu_\mu\,\V{g}_\nu \fullstop
\end{equation}
代回到 \eqref{eq:Gauss曲率计算式_2}~式的分子中去,可得
\begin{align}
  \norm[\realR^3]{\pdv{\V{n}}{x^1} \cp \pdv{\V{n}}{x^2}}
  &=\norm[\realR^3]{\qty(-b^\mu_1\,\V{g}_\mu)
      \cp \qty(-b^\nu_2\,\V{g}_\nu)} \notag \\
  &=\norm[\realR^3]{\qty(-b_1^1\,\V{g}_1 - b_1^2\,\V{g}_2)
      \cp \qty(-b_2^1\,\V{g}_1 - b_2^2\,\V{g}_2)} \notag
  \intertext{注意到 $\V{g}_1\cp\V{g}_1=\V{g}_2\cp\V{g}_2=0$,于是}
  &=\norm[\realR^3]{
      b_1^1 \, b_2^2 \qty(\V{g}_1 \cp \V{g}_2)
      +b_1^2 \, b_2^1 \qty(\V{g}_2 \cp \V{g}_1)} \notag \\
  &=\qty(b_1^1 \, b_2^2 - b_1^2 \, b_2^1)
    \cdot \norm[\realR^3]{\V{g}_1 \cp \V{g}_2} \notag \\
  &=\det\!\mqty[b_1^1 & b_1^2 \\[2pt] b_2^1 & b_2^2]
    \cdot \norm[\realR^3]{\V{g}_1 \cp \V{g}_2} \notag \\
  &=\det\!\mqty[b^\mu_\nu]
    \cdot \norm[\realR^3]{\V{g}_1 \cp \V{g}_2} \notag
  \intertext{根据切向量的定义 \eqref{eq:曲面切向量}~式,
    $\V{g}_\mu=\pdv*{\V{\Sigma}}{x^\mu}\in\realR^3$。所以}
  &=\det\!\mqty[b^\mu_\nu] \cdot
    \norm[\realR^3]{\pdv{\V{\Sigma}}{x^1} \cp \pdv{\V{\Sigma}}{x^2}}
    \comma
\end{align}
后面的一部分恰好是 \eqref{eq:Gauss曲率计算式_2}~式的分母。于是
\begin{equation}
  K=\det\!\mqty[b^\mu_\nu] \comma
\end{equation}
与 \eqref{eq:Gauss曲率_表示}~式相比较便可以看出,
这里的 $K$ 正是前一小节中所定义的 \emphB{Gauss 曲率} $\KG$。

\blankline

上面的计算过程中,分母最后消去了。当然它的值也可以算出来:
\begin{equation}
  \norm[\realR^3]{\pdv{\V{\Sigma}}{x^1} \cp \pdv{\V{\Sigma}}{x^2}}
  =\norm[\realR^3]{\V{g}_1 \cp \V{g}_2}
  =\qty(\V{g}_1 \cp \V{g}_2) \vdp \V{n}
  =\det\!\mqty[\V{g}_1,\,\V{g}_2,\,\V{n}] \fullstop
\end{equation}
把右边的行列式平方之后可得
\begin{align}
  &\mathrel{\phantom{=}}
    \qty(\det\!\mqty[\V{g}_1,\,\V{g}_2,\,\V{n}])^2
  =\det\!\mqty[\V{g}_1,\,\V{g}_2,\,\V{n}] \cdot
    \det\!\mqty[\V{g}_1,\,\V{g}_2,\,\V{n}] \notag
  \intertext{矩阵转置之后行列式不变:}
  &=\det\!\mqty[\V{g}_1,\,\V{g}_2,\,\V{n}]\trans \cdot
    \det\!\mqty[\V{g}_1,\,\V{g}_2,\,\V{n}]
  =\det\mqty[\V{g}_1\trans \\[2pt] \V{g}_2\trans \\[2pt]
      \V{n}\trans] \cdot
    \det\!\mqty[\V{g}_1,\,\V{g}_2,\,\V{n}]
  =\det \qty\Bigg(
      \mqty[\V{g}_1\trans \\[2pt] \V{g}_2\trans \\[2pt] \V{n}\trans]
      \mqty[\V{g}_1,\,\V{g}_2,\,\V{n}] ) \notag \\
  &=\det\!\mqty[
      \ipb[\realR^3]{\V{g}_1}{\V{g}_1} &
      \ipb[\realR^3]{\V{g}_1}{\V{g}_2} &
      \ipb[\realR^3]{\V{g}_1}{\V{n}} \\
      \ipb[\realR^3]{\V{g}_2}{\V{g}_1} &
      \ipb[\realR^3]{\V{g}_2}{\V{g}_2} &
      \ipb[\realR^3]{\V{g}_2}{\V{n}} \\
      \ipb[\realR^3]{\V{n}}{\V{g}_1} &
      \ipb[\realR^3]{\V{n}}{\V{g}_2} &
      \ipb[\realR^3]{\V{n}}{\V{n}} ] \notag
  \intertext{根据曲面第一基本形式的定义
    \eqref{eq:曲面第一基本形式}~式,并利用法向量与切向量正交的性质
    \eqref{eq:曲面协变基_法向量}~式,可有}
  &=\det\!\mqty[
      g_{11} & g_{12} & 0 \\
      g_{21} & g_{22} & 0 \\
      0 & 0 & 1 ]
  =\det\!\mqty[g_{11} & g_{12} \\ g_{21} & g_{22}]
  =\det\Mat{G} \fullstop
\end{align}
式中的 $\Mat{G}$ 就是曲面第一基本形式的表示矩阵。此时,
\begin{equation}
  \norm[\realR^3]{\pdv{\V{\Sigma}}{x^1} \cp \pdv{\V{\Sigma}}{x^2}}
  =\sqrt{\det\Mat{G}} \fullstop
\end{equation}


%  \part{微分流形}

\ifdefined\RELEASE
  \backmatter
    \listoffixmes
    \printindex
%   \bibliography{reference}
\fi
\end{document}
