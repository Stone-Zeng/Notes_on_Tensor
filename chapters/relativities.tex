\chapter{相对论简介}
\section{Lorentz 变换}
狭义相对论研究的空间是 \emphA{Minkowski 空间}(或称
\emphA{Minkowski 时空})$\minkM$。它是一个四维空间,第一个维度是时间,
之后的三个维度则是一般的 Euclid 空间。也就是说,Minkowski 空间中的
向量可以表示为
\begin{equation}
  \V{x} = \qty(x_0,\,x_1,\,x_2,\,x_3)
  = \qty(ct,\,x,\,y,\,z)\in\minkM \comma
\end{equation}
其中的 $c$ 是真空中的光速,它是一个常数。在自然单位制下,光速 $c=1$,
此时时间与空间具有相同的量纲。

根据惯例,我们用\emphB{希腊字母}指代任意一个指标($0,\,1,\,2,\,3$),
而用\emphB{拉丁字母}指代空间指标($1,\,2,\,3$)。

Minkowski 空间中也可定义协变基 $\qty{\V{g}_\mu}_{\mu=0}^3$ 与逆变基
$\qty{\V{g}^\mu_{\vphantom{\mu}}}_{\mu=0}^3$。向量 $\V{x}\in\minkM$
可以用它们展开:
\begin{equation}
  \V{x} = x^\mu\,\V{g}_\mu = x_\mu\,\V{g}^\mu \fullstop
\end{equation}
由基向量的内积可获得度量(物理上也称为\emphA{度规}):
\begin{equation}
  \eta_{\mu\nu} = \ipb[\minkM]{\V{g}_\mu}{\V{g}_\nu} \fullstop
\end{equation}
写成矩阵形式,为\footnote{
  在不同的著作中,度量的符号可以有不同的取法。此处为
  $(\mathord{+}\mathord{-}\mathord{-}\mathord{-})$,为粒子物理学家
  所偏爱;另一种是 $(\mathord{-}\mathord{+}\mathord{+}\mathord{+})$,
  在 $c\to\infty$ 的极限下自然退化到 Euclid 空间,数学家更喜欢用。
  当然,这两种符号约定并没有本质区别。}
\begin{equation}
  \qty[\eta_{\mu\nu}] = \mqty[\dmat{c^2, -1, -1, -1}] \fullstop
\end{equation}
根据度量满足的一般关系 \eqref{eq:度量之积}~式
\begin{equation}
  g_{ik}\,g^{kj} = \KroneckerDelta{j}{i} \comma
\end{equation}
可以知道
\begin{equation}
  \qty[\eta^{\mu\nu}] = \mqty[\dmat{c^{-2}, -1, -1, -1}] \fullstop
\end{equation}

%\begin{itemize}
%  \item 在本文的惯例中,
%  \item 在自然单位制下,光速 $c=1$,此时可将度量写为
%    \begin{equation*}
%      \qty[\eta_{\mu\nu}] = \mqty[\dmat{1, -1, -1, -1}] \fullstop
%    \end{equation*}
%\end{itemize}

%%\section{Lorentz 与 Poincaré 对称性}
%%
%%\begin{problem}{1.1}
%%\question{证明 Lorentz 变换满足 $\mat{\Lambda}\trans \mat{g} \mat{\Lambda} =\mat{g}$,
%%  并证明它们构成一个群。}
%%根据定义,变换后的向量
%%\[ {x'}^\mu = \Lambda\ids{^\mu_\nu} x^\nu \]
%%满足
%%\[ \qty(x')^2 = (x)^2 \]
%%因此
%%\[
%%  \qty(x')^2 = g_{\mu\nu} {x'}^\mu {x'}^\nu
%%  = g_{\mu\nu} \qty(\Lambda\ids{^\mu_\tau} x^\tau) \qty(\Lambda\ids{^\nu_\sigma} x^\sigma)
%%  = (x)^2 = g_{\tau\sigma} x^\tau x^\sigma
%%\]
%%由于 $\vb{x}$ 的任意性,可知
%%\[ g_{\mu\nu} \Lambda\ids{^\mu_\tau} \Lambda\ids{^\nu_\sigma} = g_{\tau\sigma} \]
%%为把该式写成矩阵形式,需要保证各指标依次排列,即
%%\[ \qty(\Lambda\trans)\ids{_\tau^\mu} g_{\mu\nu} \Lambda\ids{^\nu_\sigma} = g_{\tau\sigma} \]
%%这里用到了 $\qty(\Lambda\trans)\ids{_\nu^\mu} = \Lambda\ids{^\mu_\nu}$。此时可将上式写成
%%\[ \mat{\Lambda}\trans\mat{g}\mat{\Lambda} =\mat{g} \]
%%
%%群满足
%%\end{problem}
