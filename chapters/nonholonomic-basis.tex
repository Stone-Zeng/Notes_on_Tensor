\chapter{非完整基理论}
\section{完整基与非完整基的概念}
在 \ref{sec:局部基}~节中,我们利用曲线坐标系 $\V{X}(\V{x})$
构造了 $\Rm$ 上的一组(局部协变)基
\begin{equation}
  \qty{\V{g}_i(\V{x})=\pdv{\V{X}}{x^i} (\V{x})}^m_{i=1}
  \subset\Rm \comma
\end{equation}
它们称为\emphA{完整基}\idx{完整基}。
与之对应,不是由曲线坐标系诱导的基,
称为\emphA{非完整基}\idx{非完整基}。

\begin{figure}[h]
  \centering
  \includegraphics{images/holonomic-nonholonomic-basis.png}
  \caption{完整基与非完整基}
  \label{fig:完整基与非完整基}
\end{figure}

如图~\ref{fig:完整基与非完整基},
$x^i$-线的\emphB{切向量}构成一组局部协变基
$\qty{\V{g}_i(\V{x})}^m_{i=1}$,
它和它的对偶 $\qty{\V{g}^i(\V{x})}^m_{i=1}$ 都是完整基。
除此以外,我们当然可以选取另外的基 $\qty{\V{g}_{(i)}(\V{x})}^m_{i=1}$
和 $\qty{\V{g}^{(i)}(\V{x})}^m_{i=1}$,它们不是由曲线坐标系诱导,
因而是非完整基。

\section{非完整基下的张量梯度} \label{sec:非完整基下的张量梯度}
下面我们来考察张量梯度在非完整基下的表达形式。
在 \ref{sec:张量场的梯度}~节中,我们已经推导出了张量场的(右)梯度:
\begin{equation}
  \qty\big(\T{\Phi}\tp\opGrad) (\V{x})
  \defeq \pdv{\T{\Phi}}{x^\mu} (\V{x})
    \tp \V{g}^\mu(\V{x})
  =\coD{\mu}{\tc{\Phi}{^i_j^k} (\V{x})} \,
    \V{g}_i (\V{x}) \tp \V{g}^j (\V{x})
    \tp \V{g}_k (\V{x}) \tp \V{g}^\mu (\V{x}) \fullstop
  \label{eq:张量梯度_非完整基}
\end{equation}
这是一个四阶张量,对应的张量分量可记作
\begin{equation}
  \tc{\qty\big(\T{\Phi}\tp\opGrad)}{^i_j^k_\mu} (\V{x})
  \coloneq \coD{\mu}{\tc{\Phi}{^i_j^k} (\V{x})} \fullstop
  \label{eq:张量梯度分量_非完整基}
\end{equation}
除此以外,其他的基当然也可以用来表示该张量,
比如前文提到过的 $\qty{\V{g}_{(i)}(\V{x})}^m_{i=1}$
和 $\qty{\V{g}^{(i)}(\V{x})}^m_{i=1}$,它们都是非完整基。

非完整基与完整基之间的关系,可以利用
\ref{subsec:相对不同基的张量分量之间的关系}~小节中引入的%
\emphB{坐标转换关系}\idx{坐标转换关系}来获得:
\begin{braceEq}
  \V{g}_{(i)} (\V{x})
    &= c^k_{(i)} (\V{x}) \, \V{g}_k (\V{x}) \comma \\
  \V{g}^{(i)} (\V{x})
    &= c_k^{(i)} (\V{x}) \, \V{g}^k (\V{x}) \semicolon \\
  \V{g}_i (\V{x})
    &= c^{(k)}_i (\V{x}) \, \V{g}_{(k)} (\V{x}) \comma \\
  \V{g}^i (\V{x})
    &= c_{(k)}^i (\V{x}) \, \V{g}^{(k)} (\V{x}) \fullstop
\end{braceEq}
\myPROBLEM[2017-01-20]{坐标转换关系}\\
其中的基转换系数都是已知量,它们的定义如下:\footnote{
  只有两个基转换系数的原因是内积具有交换律。}
\begin{braceEq}
  c^j_{(i)} (\V{x})
    &\coloneq \ipb{\V{g}_{(i)} (\V{x})}{\V{g}^j (\V{x})} \comma \\
  c_j^{(i)} (\V{x})
    &\coloneq \ipb{\V{g}^{(i)} (\V{x})}{\V{g}_j (\V{x})} \fullstop
\end{braceEq}
代入 \eqref{eq:张量梯度_非完整基}~式,可有\footnote{
  这里我们省略了“$(\V{x})$”。}
\begin{align}
  \T{\Phi}\tp\opGrad
  &=\coD{\mu}{\tc{\Phi}{^i_j^k}} \,
    \qty(\vphantom{\frac{0}{0}}
      \V{g}_i \tp \V{g}^j \tp \V{g}_k \tp \V{g}^\mu) \notag \\
  &=\coD{\mu}{\tc{\Phi}{^i_j^k} (\V{x})} \,
    \qty[\vphantom{\frac{0^0}{0^0}}
      \qty(c^{(p)}_i \, \V{g}_{(p)})
      \tp \qty(c_{(q)}^j \, \V{g}^{(q)})
      \tp \qty(c^{(r)}_k \, \V{g}_{(r)})
      \tp \qty(\vphantom{\frac{0}{0}}
        c_{(\alpha)}^\mu \, \V{g}^{(\alpha)})] \notag
  \intertext{根据线性性,提出系数:}
  &=\qty(c^{(p)}_i c_{(q)}^j c^{(r)}_k c_{(\alpha)}^\mu
      \coD{\mu}{\tc{\Phi}{^i_j^k}})
    \qty(\vphantom{\frac{0}{0}}
      \V{g}_{(p)} \tp \V{g}^{(q)} \tp \V{g}_{(r)}
      \tp \V{g}^{(\alpha)}) \notag
  \intertext{写成张量分量与简单张量“乘积”的形式,即为}
  &\eqcolon \tc{\qty\big(\T{\Phi}\tp\opGrad)}%
      {^{(p)}_{\!(q)}^{\!(r)}_{\!(\alpha)}} \,
    \qty(\vphantom{\frac{0}{0}}
      \V{g}_{(p)} \tp \V{g}^{(q)} \tp \V{g}_{(r)}
      \tp \V{g}^{(\alpha)}) \fullstop
\end{align}
这样,我们就获得了非完整基下张量梯度的表示。
再利用式~\eqref{eq:张量梯度分量_非完整基},可知
\begin{equation}
  \tc{\qty\big(\T{\Phi}\tp\opGrad)}%
    {^{(p)}_{\!(q)}^{\!(r)}_{\!(\alpha)}} \,
  =c^{(p)}_i c_{(q)}^j c^{(r)}_k c_{(\alpha)}^\mu \,
    \tc{\qty(\vphantom{0^0}
      \T{\Phi}\tp\opGrad)}{^i_j^k_\mu} \fullstop
  \label{eq:非完整基下的张量梯度分量}
\end{equation}
以上结果与 \ref{subsec:相对不同基的张量分量之间的关系}~小节
中的推导是完全一致的。

\section{非完整基的形式运算}
在 \ref{sec:非完整基下的张量梯度}~节中,
我们利用\emphB{坐标转换关系}获得了张量梯度在非完整基下的表示。
而在本节,我们将通过定义,建立所谓“形式理论”,
获得一套更统一、更连贯的表述。

\myPROBLEM[2017-01-31]{统一、连贯?}

首先需要给出一些定义。

\begin{myEnum}
\item \emphA{形式偏导数}:
\begin{equation}
  \pdv{x^{(\mu)}} \defeq c^l_{(\mu)} \pdv{x^l} \fullstop
  \label{eq:形式偏导数}
\end{equation}
注意 $\pdv*{x^{(\mu)}}$ 本身是不能用极限形式来定义的,
因为曲线坐标系中并不存在有 $x^{(\mu)}$ 坐标线。

\item \emphA{形式 Christoffel 符号}:
\begin{equation}
  \ChrB{(\alpha)}{(\beta)}{(\gamma)}
  \defeq c^i_{(\alpha)} c^j_{(\beta)} c^{(\gamma)}_k \ChrB{i}{j}{k}
    -c^i_{(\alpha)} c^j_{(\beta)} \pdv{c^{(\gamma)}_j}{x^i}
  =c^i_{(\alpha)} c^j_{(\beta)}
    \qty(c^{(\gamma)}_k \ChrB{i}{j}{k}
      -\pdv{c^{(\gamma)}_j}{x^i}) \fullstop
  \label{eq:形式Christoffel符号}
\end{equation}
\myPROBLEM[2017-01-31]{第一类形式 Christoffel 符号}

\item \emphA{形式协变导数}。我们以三阶张量 $\T{\Phi}$ 为例给出定义。
$\T{\Phi}$ 在非完整基下可以用混合分量表示如下:
\begin{equation}
  \tc{\Phi}{^{(\alpha)}_{\!(\beta)}^{\!(\gamma)}}
  \coloneq \T{\Phi} \qty(\V{g}^{(\alpha)},\,\V{g}_{(\beta)},\,
    \V{g}^{(\gamma)}) \fullstop
\end{equation}
它相对 $x^{(\mu)}$ 分量的形式协变导数为
\begin{equation}
  \coD{(\mu)}{\tc{\Phi}{^{(\alpha)}_{\!(\beta)}^{\!(\gamma)}}}
  \defeq \pdv{\tc{\Phi}{^{(\alpha)}_{\!(\beta)}^{\!(\gamma)}}}%
    {x^{(\mu)}}
  +\ChrB{(\mu)}{(\sigma)}{(\alpha)}
    \tc{\Phi}{^{(\sigma)}_{\!(\beta)}^{\!(\gamma)}}
  -\ChrB{(\mu)}{(\beta)}{(\sigma)}
    \tc{\Phi}{^{(\alpha)}_{\!(\sigma)}^{\!(\gamma)}}
  +\ChrB{(\mu)}{(\sigma)}{(\gamma)}
    \tc{\Phi}{^{(\alpha)}_{\!(\beta)}^{\!(\sigma)}} \fullstop
  \label{eq:形式协变导数}
\end{equation}
回顾 \ref{sec:张量场的偏导数_协变导数}~节,
\eqref{eq:协变导数定义}~式给出了完整基下协变导数的定义:
\begin{equation}
  \coD{l}{\tc{\Phi}{^i_j^k}} \defeq
  \pdv{\tc{\Phi}{^i_j^k}}{x^l}
  +\ChrB{l}{s}{i} \tc{\Phi}{^s_j^k}
  -\ChrB{l}{j}{s} \tc{\Phi}{^i_s^k}
  +\ChrB{l}{s}{k} \tc{\Phi}{^i_j^s} \fullstop
\end{equation}
可以看出\emphB{形式}协变导数的定义与它几乎一模一样。
\end{myEnum}

\blankline

接下来我们要证明
\begin{equation}
  \coD{(\mu)}{\tc{\Phi}{^{(\alpha)}_{\!(\beta)}^{\!(\gamma)}}}
  =c^l_{(\mu)} c^{(\alpha)}_i c^j_{(\beta)} c^{(\gamma)}_k \,
    \coD{l}{\tc{\Phi}{^i_j^k}} \fullstop
\end{equation}
代入式~\eqref{eq:张量梯度分量_非完整基} 和
\eqref{eq:非完整基下的张量梯度分量},即得
\begin{equation}
  \coD{(\mu)}{\tc{\Phi}{^{(\alpha)}_{\!(\beta)}^{\!(\gamma)}}}
  =\tc{\qty(\vphantom{0^0}
    \T{\Phi}\tp\opGrad)}{^{(p)}_{\!(q)}^{\!(r)}_{\!(\alpha)}}
  \fullstop
\end{equation}
换句话说,此处我们正是要验证这种“形式理论”与
\ref{sec:非完整基下的张量梯度}~节中坐标转换关系的一致性。

\begin{myProof}
左边按照 \eqref{eq:形式协变导数}~式展开,第一项为
\begin{align}
  \pdv{\tc{\Phi}{^{(\alpha)}_{\!(\beta)}^{\!(\gamma)}}}%
    {x^{(\mu)}}
  &=c^l_{(\mu)} \pdv{\tc{\Phi}{^{(\alpha)}_{\!(\beta)}%
      ^{\!(\gamma)}}}{x^l} \notag
  \intertext{这里用到了形式偏导数的定义 \eqref{eq:形式偏导数}~式。
  然后利用坐标转换关系展开张量分量:}
  &=c^l_{(\mu)} \pdv{x^l}
    \qty(c^{(\alpha)}_i c^j_{(\beta)} c^{(\gamma)}_k
      \tc{\Phi}{^i_j^k}) \notag
  %
  \intertext{再按照通常的偏导数法则直接打开:}
  &=c^l_{(\mu)} c^j_{(\beta)} c^{(\gamma)}_k
      \pdv{c^{(\alpha)}_i}{x^l} \tc{\Phi}{^i_j^k}
    +c^l_{(\mu)} c^{(\alpha)}_i c^{(\gamma)}_k
      \pdv{c^j_{(\beta)}}{x^l} \tc{\Phi}{^i_j^k}
    +c^l_{(\mu)} c^{(\alpha)}_i c^j_{(\beta)}
      \pdv{c^{(\gamma)}_k}{x^l} \tc{\Phi}{^i_j^k} \notag \\*
  &\alspace{}
    +c^l_{(\mu)} c^{(\alpha)}_i c^j_{(\beta)} c^{(\gamma)}_k
      \pdv{\tc{\Phi}{^i_j^k}}{x^l} \notag \\
  %
  &=c^l_{(\mu)} \tc{\Phi}{^i_j^k}
    \qty(c^j_{(\beta)} c^{(\gamma)}_k \pdv{c^{(\alpha)}_i}{x^l}
      +c^{(\alpha)}_i c^{(\gamma)}_k \pdv{c^j_{(\beta)}}{x^l}
      +c^{(\alpha)}_i c^j_{(\beta)} \pdv{c^{(\gamma)}_k}{x^l})
    \notag \\*
  &\alspace{}
    +c^l_{(\mu)} c^{(\alpha)}_i c^j_{(\beta)} c^{(\gamma)}_k
      \pdv{\tc{\Phi}{^i_j^k}}{x^l} \fullstop
  \label{eq:张量分量偏导数_非完整基形式运算}
\end{align}

接下来处理含有形式 Christoffel 符号的三项,分别是
\begin{mySubEq}
  \begin{align}
    \ChrB{(\mu)}{(\sigma)}{(\alpha)}
      \tc{\Phi}{^{(\sigma)}_{\!(\beta)}^{\!(\gamma)}}
    &=c^p_{(\mu)} c^q_{(\sigma)}
      \qty(c^{(\alpha)}_s \ChrB{p}{q}{s} - \pdv{c^{(\alpha)}_q}{x^p})
      \cdot \tc{\Phi}{^{(\sigma)}_{\!(\beta)}^{\!(\gamma)}} \notag \\
    &=c^p_{(\mu)} \hl{c^q_{(\sigma)}}
      \qty(c^{(\alpha)}_s \ChrB{p}{q}{s} - \pdv{c^{(\alpha)}_q}{x^p})
      \cdot \hl{c^{(\sigma)}_i} c^j_{(\beta)} c^{(\gamma)}_k
        \tc{\Phi}{^i_j^k} \notag
    \intertext{根据式~\eqref{eq:坐标转换系数的乘积},我们有
      $c^q_{(\sigma)} c^{(\sigma)}_i = \KroneckerDelta{q}{i}$,于是}
    &=c^p_{(\mu)} \tc{\Phi}{^i_j^k}
      \qty(c^{(\alpha)}_s c^j_{(\beta)} c^{(\gamma)}_k \ChrB{p}{i}{s}
        -c^j_{(\beta)} c^{(\gamma)}_k \pdv{c^{(\alpha)}_i}{x^p})
    \semicolon \\
    %
    -\ChrB{(\mu)}{(\beta)}{(\sigma)}
      \tc{\Phi}{^{(\alpha)}_{\!(\sigma)}^{\!(\gamma)}}
    &=-c^p_{(\mu)} c^q_{(\beta)}
      \qty(c^{(\sigma)}_s \ChrB{p}{q}{s} - \pdv{c^{(\sigma)}_q}{x^p})
      \cdot \tc{\Phi}{^{(\alpha)}_{\!(\sigma)}^{\!(\gamma)}} \notag \\
    &=-c^p_{(\mu)} c^q_{(\beta)}
      \qty(\hl{c^{(\sigma)}_s} \ChrB{p}{q}{s}
        -\pdv{c^{(\sigma)}_q}{x^p})
      \cdot c^{(\alpha)}_i \hl{c^j_{(\sigma)}} c^{(\gamma)}_k
        \tc{\Phi}{^i_j^k} \notag \\
    &=c^p_{(\mu)} \tc{\Phi}{^i_j^k}
      \qty(-c^{(\alpha)}_i c^q_{(\beta)} c^{(\gamma)}_k \ChrB{p}{q}{j}
        +c^{(\alpha)}_i c^q_{(\beta)} c^{(\gamma)}_k c^j_{(\sigma)}
          \pdv{c^{(\sigma)}_q}{x^p}) \semicolon \\
    %
    \ChrB{(\mu)}{(\sigma)}{(\gamma)}
      \tc{\Phi}{^{(\alpha)}_{\!(\beta)}^{\!(\sigma)}}
    &=c^p_{(\mu)} c^q_{(\sigma)}
      \qty(c^{(\gamma)}_s \ChrB{p}{q}{s} - \pdv{c^{(\gamma)}_q}{x^p})
      \cdot \tc{\Phi}{^{(\alpha)}_{\!(\beta)}^{\!(\sigma)}} \notag \\
    &=c^p_{(\mu)} \hl{c^q_{(\sigma)}}
      \qty(c^{(\gamma)}_s \ChrB{p}{q}{s} - \pdv{c^{(\gamma)}_q}{x^p})
      \cdot c^{(\alpha)}_i c^j_{(\beta)} \hl{c^{(\sigma)}_k}
        \tc{\Phi}{^i_j^k} \notag \\
    &=c^p_{(\mu)} \tc{\Phi}{^i_j^k}
      \qty(c^{(\alpha)}_i c^j_{(\beta)} c^{(\gamma)}_s \ChrB{p}{k}{s}
        -c^{(\alpha)}_i c^j_{(\beta)} \pdv{c^{(\gamma)}_k}{x^p})
    \fullstop
  \end{align}
\end{mySubEq}
以上三式都有公因子 $c^p_{(\mu)} \tc{\Phi}{^i_j^k}$。
为了进一步化简,不妨将哑标 $p$ 换为 $l$。这样可有
\begin{align}
  &\alspace \ChrB{(\mu)}{(\sigma)}{(\alpha)}
    \tc{\Phi}{^{(\sigma)}_{\!(\beta)}^{\!(\gamma)}}
  -\ChrB{(\mu)}{(\beta)}{(\sigma)}
    \tc{\Phi}{^{(\alpha)}_{\!(\sigma)}^{\!(\gamma)}}
  +\ChrB{(\mu)}{(\sigma)}{(\gamma)}
    \tc{\Phi}{^{(\alpha)}_{\!(\beta)}^{\!(\sigma)}} \notag \\
  &=c^l_{(\mu)} \tc{\Phi}{^i_j^k}
    \left[\vphantom{\pdv{c^{(\gamma)}_k}{x^l}} \qty(
      c^{(\alpha)}_s c^j_{(\beta)} c^{(\gamma)}_k \ChrB{l}{i}{s}
      -c^{(\alpha)}_i c^q_{(\beta)} c^{(\gamma)}_k \ChrB{l}{q}{j}
      +c^{(\alpha)}_i c^j_{(\beta)} c^{(\gamma)}_s \ChrB{l}{k}{s})
    \right. \notag \\
  &\alspace\phantom{c^l_{(\mu)} \tc{\Phi}{^i_j^k}\left[\right]}
    \left. {}-c^j_{(\beta)} c^{(\gamma)}_k \pdv{c^{(\alpha)}_i}{x^l}
      +c^{(\alpha)}_i c^q_{(\beta)} c^{(\gamma)}_k c^j_{(\sigma)}
        \pdv{c^{(\sigma)}_q}{x^l}
      -c^{(\alpha)}_i c^j_{(\beta)} \pdv{c^{(\gamma)}_k}{x^l}
    \right] \fullstop
\end{align}
该式与 \eqref{eq:张量分量偏导数_非完整基形式运算}~式相加,得
\begin{align}
  &\alspace \coD{(\mu)}{\tc{\Phi}
    {^{(\alpha)}_{\!(\beta)}^{\!(\gamma)}}}
  \defeq \pdv{\tc{\Phi}{^{(\alpha)}_{\!(\beta)}^{\!(\gamma)}}}%
    {x^{(\mu)}}
    +\ChrB{(\mu)}{(\sigma)}{(\alpha)}
      \tc{\Phi}{^{(\sigma)}_{\!(\beta)}^{\!(\gamma)}}
    -\ChrB{(\mu)}{(\beta)}{(\sigma)}
      \tc{\Phi}{^{(\alpha)}_{\!(\sigma)}^{\!(\gamma)}}
    +\ChrB{(\mu)}{(\sigma)}{(\gamma)}
      \tc{\Phi}{^{(\alpha)}_{\!(\beta)}^{\!(\sigma)}} \notag \\
  %
  &=c^l_{(\mu)} c^{(\alpha)}_i c^j_{(\beta)} c^{(\gamma)}_k
      \pdv{\tc{\Phi}{^i_j^k}}{x^l}
    +c^l_{(\mu)} \tc{\Phi}{^i_j^k} \left[
      \hl{c^j_{(\beta)} c^{(\gamma)}_k \pdv{c^{(\alpha)}_i}{x^l}}
      +c^{(\alpha)}_i c^{(\gamma)}_k \pdv{c^j_{(\beta)}}{x^l}
      +\hl[pink]{c^{(\alpha)}_i c^j_{(\beta)}
        \pdv{c^{(\gamma)}_k}{x^l}} \right. \notag \\*
  &\alspace
  \phantom{c^l_{(\mu)} c^{(\alpha)}_i c^j_{(\beta)} c^{(\gamma)}_k
      \pdv{\tc{\Phi}{^i_j^k}}{x^l}
    +c^l_{(\mu)} \tc{\Phi}{^i_j^k} \left[\right]}
    \left. {}
      +\qty(c^{(\alpha)}_s c^j_{(\beta)} c^{(\gamma)}_k \ChrB{l}{i}{s}
      -c^{(\alpha)}_i c^q_{(\beta)} c^{(\gamma)}_k \ChrB{l}{q}{j}
      +c^{(\alpha)}_i c^j_{(\beta)} c^{(\gamma)}_s \ChrB{l}{k}{s})
    \right. \notag \\*
  &\alspace
  \phantom{c^l_{(\mu)} c^{(\alpha)}_i c^j_{(\beta)} c^{(\gamma)}_k
      \pdv{\tc{\Phi}{^i_j^k}}{x^l}
    +c^l_{(\mu)} \tc{\Phi}{^i_j^k} \left[\right]}
    \left. {}
      -\hl{c^j_{(\beta)} c^{(\gamma)}_k \pdv{c^{(\alpha)}_i}{x^l}}
      +c^{(\alpha)}_i c^q_{(\beta)} c^{(\gamma)}_k c^j_{(\sigma)}
        \pdv{c^{(\sigma)}_q}{x^l}
      -\hl[pink]{c^{(\alpha)}_i c^j_{(\beta)}
        \pdv{c^{(\gamma)}_k}{x^l}} \right] \notag
  %
  \intertext{高亮部分相互抵消:}
  &=c^l_{(\mu)} c^{(\alpha)}_i c^j_{(\beta)} c^{(\gamma)}_k
      \pdv{\tc{\Phi}{^i_j^k}}{x^l}
    +c^l_{(\mu)} \tc{\Phi}{^i_j^k}
    \left[\vphantom{\pdv{c^{(\gamma)}_k}{x^l}} \qty(
      c^{(\alpha)}_s c^j_{(\beta)} c^{(\gamma)}_k \ChrB{l}{i}{s}
      -c^{(\alpha)}_i c^q_{(\beta)} c^{(\gamma)}_k \ChrB{l}{q}{j}
      +c^{(\alpha)}_i c^j_{(\beta)} c^{(\gamma)}_s \ChrB{l}{k}{s})
    \right. \notag \\
  &\alspace
  \phantom{c^l_{(\mu)} c^{(\alpha)}_i c^j_{(\beta)} c^{(\gamma)}_k
      \pdv{\tc{\Phi}{^i_j^k}}{x^l}
    +c^l_{(\mu)} \tc{\Phi}{^i_j^k} \left[\right]}
    \left. {}
      +c^{(\alpha)}_i c^{(\gamma)}_k \pdv{c^j_{(\beta)}}{x^l}
      +c^{(\alpha)}_i c^q_{(\beta)} c^{(\gamma)}_k c^j_{(\sigma)}
        \pdv{c^{(\sigma)}_q}{x^l} \right]
  \label{eq:非完整基形式理论推导}
\end{align}
注意到 $c^j_{(\beta)} = c^q_{(\beta)} \KroneckerDelta{j}{q}
  =c^q_{(\beta)} c^j_{(\sigma)} c^{(\sigma)}_q$,因此
\begin{equation}
  \pdv{c^j_{(\beta)}}{x^l}
  =\pdv{x^l} \qty(c^q_{(\beta)} c^j_{(\sigma)} c^{(\sigma)}_q)
  =c^j_{(\sigma)} c^{(\sigma)}_q \pdv{c^q_{(\beta)}}{x^l}
    +c^q_{(\beta)} c^{(\sigma)}_q \pdv{c^j_{(\sigma)}}{x^l}
    +c^q_{(\beta)} c^j_{(\sigma)} \pdv{c^{(\sigma)}_q}{x^l} \fullstop
\end{equation}
所以 \eqref{eq:非完整基形式理论推导}~式中最后一步的第二行就能够写成
\begin{align}
  &\alspace c^{(\alpha)}_i c^{(\gamma)}_k \pdv{c^j_{(\beta)}}{x^l}
    +c^{(\alpha)}_i c^q_{(\beta)} c^{(\gamma)}_k c^j_{(\sigma)}
      \pdv{c^{(\sigma)}_q}{x^l} \notag \\
  &=c^{(\alpha)}_i c^{(\gamma)}_k
    \qty(\pdv{c^j_{(\beta)}}{x^l}
      +c^q_{(\beta)} c^j_{(\sigma)} \pdv{c^{(\sigma)}_q}{x^l})
    \notag \\
  &=c^{(\alpha)}_i c^{(\gamma)}_k \qty(
      \hl{c^j_{(\sigma)}} c^{(\sigma)}_q \pdv{c^q_{(\beta)}}{x^l}
      +\hl[pink]{c^q_{(\beta)}} c^{(\sigma)}_q
        \pdv{c^j_{(\sigma)}}{x^l}
      +c^q_{(\beta)} \hl{c^j_{(\sigma)}} \pdv{c^{(\sigma)}_q}{x^l}
      +\hl[pink]{c^q_{(\beta)}} c^j_{(\sigma)}
        \pdv{c^{(\sigma)}_q}{x^l}) \notag
  \intertext{合并同类项:}
  &=c^{(\alpha)}_i c^{(\gamma)}_k \qty[
      c^j_{(\sigma)}
      \qty(c^{(\sigma)}_q \pdv{c^q_{(\beta)}}{x^l}
        +c^q_{(\beta)} \pdv{c^{(\sigma)}_q}{x^l})
    +c^q_{(\beta)}
      \qty(c^{(\sigma)}_q \pdv{c^j_{(\sigma)}}{x^l}
        +c^j_{(\sigma)} \pdv{c^{(\sigma)}_q}{x^l}) ] \notag \\
  &=c^{(\alpha)}_i c^{(\gamma)}_k \qty[
      \vphantom{\pdv{c^q_{(\beta)}}{x^l}}
      c^j_{(\sigma)} \pdv{x^l} \qty(c^{(\sigma)}_q c^q_{(\beta)})
      +c^q_{(\beta)} \pdv{x^l} \qty(c^{(\sigma)}_q c^j_{(\sigma)}) ]
    \notag
  \intertext{再次利用式~\eqref{eq:坐标转换系数的乘积},可得}
  &=c^{(\alpha)}_i c^{(\gamma)}_k \qty(
      c^j_{(\sigma)} \pdv{\KroneckerDelta{\sigma}{\beta}}{x^l}
      +c^q_{(\beta)} \pdv{\KroneckerDelta{j}{q}}{x^l} )
  =0 \fullstop
\end{align}
代回式~\eqref{eq:非完整基形式理论推导},有
\begin{align}
  &\alspace \coD{(\mu)}{\tc{\Phi}
    {^{(\alpha)}_{\!(\beta)}^{\!(\gamma)}}}
  =c^l_{(\mu)} c^{(\alpha)}_i c^j_{(\beta)} c^{(\gamma)}_k
      \pdv{\tc{\Phi}{^i_j^k}}{x^l}
    +c^l_{(\mu)} \tc{\Phi}{^i_j^k}
    \qty(\vphantom{\pdv{c^{(\gamma)}_k}{x^l}}
      c^{(\alpha)}_s c^j_{(\beta)} c^{(\gamma)}_k \ChrB{l}{i}{s}
      -c^{(\alpha)}_i c^q_{(\beta)} c^{(\gamma)}_k \ChrB{l}{q}{j}
      +c^{(\alpha)}_i c^j_{(\beta)} c^{(\gamma)}_s \ChrB{l}{k}{s})
    \notag \\
  &=c^l_{(\mu)} c^{(\alpha)}_i c^j_{(\beta)} c^{(\gamma)}_k
      \pdv{\tc{\Phi}{^i_j^k}}{x^l}
    +c^l_{(\mu)}
    \qty(\vphantom{\pdv{c^{(\gamma)}_k}{x^l}}
      c^{(\alpha)}_s c^j_{(\beta)} c^{(\gamma)}_k
        \ChrB{l}{i}{s} \tc{\Phi}{^i_j^k}
      -c^{(\alpha)}_i c^q_{(\beta)} c^{(\gamma)}_k
        \ChrB{l}{q}{j} \tc{\Phi}{^i_j^k}
      +c^{(\alpha)}_i c^j_{(\beta)} c^{(\gamma)}_s
        \ChrB{l}{k}{s} \tc{\Phi}{^i_j^k}) \notag
  \intertext{下面要对哑标进行重排。
    括号里的第一项:$s \leftrightarrow i$;
    第二项:$j \rightarrow s, q \rightarrow j$;
    第三项:$s \leftrightarrow k$。于是}
  &=c^l_{(\mu)} c^{(\alpha)}_i c^j_{(\beta)} c^{(\gamma)}_k
      \pdv{\tc{\Phi}{^i_j^k}}{x^l}
    +c^l_{(\mu)}
    \qty(\vphantom{\pdv{c^{(\gamma)}_k}{x^l}}
      c^{(\alpha)}_i c^j_{(\beta)} c^{(\gamma)}_k
        \ChrB{l}{s}{i} \tc{\Phi}{^s_j^k}
      -c^{(\alpha)}_i c^j_{(\beta)} c^{(\gamma)}_k
        \ChrB{l}{j}{s} \tc{\Phi}{^i_s^k}
      +c^{(\alpha)}_i c^j_{(\beta)} c^{(\gamma)}_k
        \ChrB{l}{s}{k} \tc{\Phi}{^i_j^s}) \notag \\
  &=c^l_{(\mu)} c^{(\alpha)}_i c^j_{(\beta)} c^{(\gamma)}_k
    \qty(\pdv{\tc{\Phi}{^i_j^k}}{x^l}
      +\ChrB{l}{s}{i} \tc{\Phi}{^s_j^k}
      -\ChrB{l}{j}{s} \tc{\Phi}{^i_s^k}
      +\ChrB{l}{s}{k} \tc{\Phi}{^i_j^s}) \notag \\
  &=c^l_{(\mu)} c^{(\alpha)}_i c^j_{(\beta)} c^{(\gamma)}_k \,
    \coD{l}{\tc{\Phi}{^i_j^k}} \fullstop
\end{align}
这就完成了证明。
\end{myProof}

\blankline

如前文所言,此种形式理论与我们在
\ref{sec:非完整基下的张量梯度}~节中所使用的方法(坐标转换)并无二致,
但它在某些特定情况下将会十分有用,这就是下一节要介绍的内容。

\section{单位正交基}
\subsection{选取非完整基}
实际情况下,为了计算的方便,
我们通常会取一组\emphA{正交基}\idx{正交基}作为完整基,它们满足
\begin{equation}
  \ipb{\V{g}_i}{\V{g}_j}=0, \quad i\neq j \fullstop
\end{equation}
注意此处的 $\V{g}_i$ 和 $\V{g}_j$ 都是协变基。

\myPROBLEM[2017-02-01]{为什么不直接取单位正交基}

这样,度量 $g_{ij}$ 就可以用矩阵形式写成
\begin{equation}
  \mqty[g_{ij}]=\mqty[\dmat{g_{11},\ddots,g_{mm}}] \comma
\end{equation}
它是一个对角矩阵。根据式~\eqref{eq:度量之积_矩阵形式},我们有
\begin{equation}
  \mqty[g_{ik}] \mqty[g^{kj}]
  =\mqty[\KroneckerDelta{j}{i}]=\Mat{I}_m \semicolon
\end{equation}
而根据线性代数的知识,对角矩阵的逆同样是对角阵,
因此 $\mqty[g^{ij}]$ 也是一个对角矩阵。
换句话说,逆变基同样是一组正交基。

出于量纲一致等因素的考虑,我们常常需要将正交基单位化,
使其成为\emphA{单位正交基}\idx{单位正交基}。
当然,之前的完整基也就成了非完整基:
\begin{equation}
  \V{g}_{(\alpha)} \defeq c^i_{(\alpha)}\,\V{g}_i \comma
\end{equation}
式中,
\begin{equation}
  c^i_{(\alpha)}=\begin{dcases}
    \frac{1}{\sqrt{g_{ii}}} \comma & i=\alpha \semicolon \\
    0 \comma & i\neq\alpha \fullstop
  \end{dcases}
  \label{eq:坐标转换系数_单位正交基_1}
\end{equation}
这里 $g_{ii}$ 中的指标 $i$ 不求和。\footnote{
  在 $i=\alpha$ 的情况下,该系数常被称作
  \emphA{Lamé 系数}\idx{Lamé 系数}。}
对于逆变基,也是同样的:
\begin{equation}
  \V{g}^{(\alpha)} \defeq c^{(\alpha)}_i\,\V{g}^i \comma
\end{equation}
其中的
\begin{equation}
  c^{(\alpha)}_i=\begin{dcases}
    \sqrt{g_{ii}} \comma & i=\alpha \semicolon \\
    0 \comma & i\neq\alpha \fullstop
  \end{dcases}
  \label{eq:坐标转换系数_单位正交基_2}
\end{equation}

\subsection{形式偏导数}
非完整基形式运算的第一步是考虑形式偏导数:
\begin{equation}
  \pdv{x^{(\mu)}} \defeq c^l_{(\mu)} \pdv{x^l} \fullstop
\end{equation}
一般来说,$\V{x}\in\Rm$,因而该式包含着 $m$ 项的求和。
但在完整基是正交基、非完整基是单位正交基的情况下,
系数 $c^l_{(\mu)}$ 仅在 $l=\mu$ 的时候才有非零值。所以
\begin{equation}
  \pdv{x^{(\mu)}}
  =c^\mu_{(\mu)} \pdv{x^\mu}
  =\frac{1}{\sqrt{g_{\mu\mu}}} \pdv{x^\mu} \fullstop
  \label{eq:形式偏导数_单位正交基}
\end{equation}
指标 $\mu$ 不求和,此式便只剩下了一项。

\subsection{形式 Christoffel 符号}
根据 \eqref{eq:形式Christoffel符号}~式,
\begin{align}
  \ChrB{(\alpha)}{(\beta)}{(\gamma)}
  &\defeq c^i_{(\alpha)} c^j_{(\beta)} c^{(\gamma)}_k \ChrB{i}{j}{k}
    -c^i_{(\alpha)} c^j_{(\beta)} \pdv{c^{(\gamma)}_j}{x^i} \notag
  \intertext{同样,特殊情况下只需要考虑非零值:}
  &=c^\alpha_{(\alpha)} c^\beta_{(\beta)} c^{(\gamma)}_\gamma
      \ChrB{\alpha}{\beta}{\gamma}
    -c^\alpha_{(\alpha)} c^\beta_{(\beta)}
      \pdv{c^{(\gamma)}_\beta}{x^\alpha} \notag
  \intertext{代入式~\eqref{eq:坐标转换系数_单位正交基_1} 和
    \eqref{eq:坐标转换系数_单位正交基_2},可得}
  &=\frac{1}{\sqrt{g_{\alpha\alpha}}}
      \frac{1}{\sqrt{g_{\beta\beta}}} \sqrt{g_{\gamma\gamma}}
      \cdot \ChrB{\alpha}{\beta}{\gamma}
    -\frac{1}{\sqrt{g_{\alpha\alpha}}}
      \frac{1}{\sqrt{g_{\beta\beta}}}
      \cdot \pdv{c^{(\gamma)}_\beta}{x^\alpha} \fullstop
\end{align}
此处的指标 $\alpha$、$\beta$、$\gamma$ 均不表示求和。

上式包含一个完整基下的 Christoffel 符号
$\ChrB{\alpha}{\beta}{\gamma}$。
根据 \ref{subsec:度量的性质_Christoffel符号的计算}~小节中的
\eqref{eq:第一类Christoffel符号与度量的关系}~式和
\eqref{eq:第二类Christoffel符号用第一类表示}~式,
很容易利用度量把它计算出来:
\begin{align}
  \ChrB{\alpha}{\beta}{\gamma}
  &=g^{\gamma s}\,\ChrA{\alpha}{\beta}{s}
  =g^{\gamma s} \cdot \frac{1}{2}\,
    \qty(\pdv{g_{\beta s}}{x^\alpha}+\pdv{g_{\alpha s}}{x^\beta}
      -\pdv{g_{\alpha\beta}}{x^s}) \notag
  \intertext{注意指标 $s$ \emphB{需要}求和!
    但是由于度量的非对角元均为零,所以可以直接写成}
  &=g^{\gamma\gamma}\,\ChrA{\alpha}{\beta}{\gamma}
  =\frac{1}{g_{\gamma\gamma}} \cdot \frac{1}{2}\,
    \qty(\pdv{g_{\beta\gamma}}{x^\alpha}
      +\pdv{g_{\alpha\gamma}}{x^\beta}
      -\pdv{g_{\alpha\beta}}{x^\gamma}) \fullstop
  \label{eq:形式Christoffel符号_单位正交基}
\end{align}
同样,指标都不表示求和。

现在我们来分 4 种情况,进一步化简 $\ChrB{\alpha}{\beta}{\gamma}$。

\begin{myEnum}
\item $\alpha\neq\beta\neq\gamma$。
前文已经提到,度量的非对角元均为零,即
\begin{equation}
  g_{\beta\gamma}=g_{\alpha\gamma}=g_{\alpha\beta}=0 \comma
\end{equation}
因此结果非常简单:
\begin{equation}
  \ChrB{\alpha}{\beta}{\gamma}=0 \fullstop
\end{equation}

\item $\alpha=\beta\neq\gamma$,即 $\ChrB{\alpha}{\alpha}{\gamma}$。
直接计算,可有
\begin{equation}
  \ChrB{\alpha}{\alpha}{\gamma}
  =g^{\gamma\gamma}\,\ChrA{\alpha}{\alpha}{\gamma}
  =\frac{1}{g_{\gamma\gamma}} \cdot \frac{1}{2}\,
    \qty(-\pdv{g_{\alpha\alpha}}{x^\gamma})
  =-\frac{1}{2} \frac{1}{g_{\gamma\gamma}}
    \pdv{g_{\alpha\alpha}}{x^\gamma} \fullstop
\end{equation}

\item $\alpha=\gamma\neq\beta$,即 $\ChrB{\alpha}{\beta}{\alpha}$。
根据式~\eqref{eq:第二类Christoffel符号指标交换},
它又等于 $\ChrB{\beta}{\alpha}{\alpha}$。同样,直接来进行计算:
\begin{equation}
  \ChrB{\alpha}{\beta}{\alpha}
  =g^{\alpha\alpha}\,\ChrA{\alpha}{\beta}{\alpha}
  =\frac{1}{g_{\alpha\alpha}} \cdot \frac{1}{2}\,
    \qty(\pdv{g_{\alpha\alpha}}{x^\beta})
  =\frac{1}{2} \frac{1}{g_{\alpha\alpha}}
    \pdv{g_{\alpha\alpha}}{x^\beta} \fullstop
\end{equation}

\item $\alpha=\beta=\gamma$,即 $\ChrB{\alpha}{\alpha}{\alpha}$。
指标只剩下了一个,喜闻乐见。
\begin{equation}
  \ChrB{\alpha}{\alpha}{\alpha}
  =g^{\alpha\alpha}\,\ChrA{\alpha}{\alpha}{\alpha}
  =\frac{1}{g_{\alpha\alpha}} \cdot \frac{1}{2}\,
    \qty(\pdv{g_{\alpha\alpha}}{x^\alpha})
  =\frac{1}{2} \frac{1}{g_{\alpha\alpha}}
    \pdv{g_{\alpha\alpha}}{x^\alpha} \fullstop
\end{equation}
\end{myEnum}

算好了完整基(正交基)下的 Christoffel 符号,
就可以考虑非完整基(单位正交基)下的情况了。
我们在式~\eqref{eq:形式Christoffel符号_单位正交基}
中已经计算出了非完整基下的 Christoffel 符号,
现在只要把以上四种情况逐一代入即可。

\begin{myEnum}
\item $\alpha\neq\beta\neq\gamma$。
已经知道 $\ChrB{\alpha}{\beta}{\gamma}=0$,
而根据 \eqref{eq:坐标转换系数_单位正交基_2}~式,
又有 $c^{(\gamma)}_\beta=0$,于是
\begin{equation}
  \ChrB{(\alpha)}{(\beta)}{(\gamma)}=0 \fullstop
\end{equation}

\item $\alpha=\beta\neq\gamma$。此时有
\begin{align}
  \ChrB{(\alpha)}{(\alpha)}{(\gamma)}
  &=\frac{1}{g_{\alpha\alpha}} \sqrt{g_{\gamma\gamma}}
      \cdot \ChrB{\alpha}{\alpha}{\gamma}
    -\frac{1}{g_{\alpha\alpha}}
      \cdot \pdv{c^{(\gamma)}_\alpha}{x^\alpha} \notag \\
  &=\frac{1}{g_{\alpha\alpha}} \sqrt{g_{\gamma\gamma}}
    \cdot \qty(-\frac{1}{2} \frac{1}{g_{\gamma\gamma}}
      \pdv{g_{\alpha\alpha}}{x^\gamma}) - 0 \notag \\
  &=-\frac{1}{\sqrt{g_{\gamma\gamma}}}
    \cdot \qty(\frac{1}{2g_{\alpha\alpha}}
      \pdv{g_{\alpha\alpha}}{x^\gamma}) \fullstop
\end{align}
考虑到
\begin{equation}
  \pdv{x} \ln\sqrt{f(x)}
  =\pdv{x} \qty[\frac{1}{2} \ln f(x)]
  =\frac{1}{2} \frac{1}{f(x)} \pdv{f(x)}{x} \comma
\end{equation}
于是
\begin{equation}
  \ChrB{(\alpha)}{(\alpha)}{(\gamma)}
  =-\frac{1}{\sqrt{g_{\gamma\gamma}}}
    \pdv{x^\gamma} \qty(\vphantom{\frac{0}{0}}
      \ln \sqrt{g_{\alpha\alpha}}) \fullstop
\end{equation}

\item $\alpha=\gamma\neq\beta$。此时
\begin{align}
  \ChrB{(\alpha)}{(\beta)}{(\alpha)}
  &=\frac{1}{\sqrt{g_{\beta\beta}}}
      \cdot \ChrB{\alpha}{\beta}{\alpha}
    -\frac{1}{\sqrt{g_{\alpha\alpha}}}
      \frac{1}{\sqrt{g_{\beta\beta}}}
      \cdot \pdv{c^{(\alpha)}_\beta}{x^\alpha} \notag \\
  &=\frac{1}{\sqrt{g_{\beta\beta}}}
    \cdot \qty(\frac{1}{2} \frac{1}{g_{\alpha\alpha}}
      \pdv{g_{\alpha\alpha}}{x^\beta}) - 0 \notag
  \intertext{同理,利用对数,可得}
  &=\frac{1}{\sqrt{g_{\beta\beta}}}
    \pdv{x^\beta} \qty(\vphantom{\frac{0}{0}}
      \ln \sqrt{g_{\alpha\alpha}}) \fullstop
\end{align}
注意这里没有负号。

\item[\theenumi*.] $\beta=\gamma\neq\alpha$,
即 $\ChrB{(\alpha)}{(\beta)}{(\beta)}$。
不过我们暂时先从 $\ChrB{(\beta)}{(\alpha)}{(\alpha)}$ 开始。

之前虽然已经计算了 $\ChrB{(\alpha)}{(\beta)}{(\alpha)}$,
但由于我们并未证明\emphB{形式} Christoffel 符号的下标可以交换
\footnote{所以这里也多了一种情况需要讨论。},
因而仍要从头来算:
\begin{align}
  \ChrB{(\beta)}{(\alpha)}{(\alpha)}
  &=\frac{1}{\sqrt{g_{\beta\beta}}}
      \cdot \ChrB{\beta}{\alpha}{\alpha}
    -\frac{1}{\sqrt{g_{\beta\beta}}}
      \frac{1}{\sqrt{g_{\alpha\alpha}}}
      \cdot \pdv{c^{(\alpha)}_\alpha}{x^\beta} \notag
  \intertext{交换 Christoffel 符号的下标,
    同时代入\eqref{eq:坐标转换系数_单位正交基_2}~式,可有}
  &=\frac{1}{\sqrt{g_{\beta\beta}}}
      \cdot \ChrB{\alpha}{\beta}{\alpha}
    -\frac{1}{\sqrt{g_{\beta\beta}}}
      \frac{1}{\sqrt{g_{\alpha\alpha}}}
      \cdot \pdv{x^\beta} \sqrt{g_{\alpha\alpha}} \notag \\
  &=\frac{1}{\sqrt{g_{\beta\beta}}}
    \cdot \qty(\frac{1}{2} \frac{1}{g_{\alpha\alpha}}
      \pdv{g_{\alpha\alpha}}{x^\beta})
    -\frac{1}{\sqrt{g_{\beta\beta}}}
      \frac{1}{\sqrt{g_{\alpha\alpha}}}
      \cdot \frac{1}{2\sqrt{g_{\alpha\alpha}}}
      \pdv{g_{\alpha\alpha}}{x^\beta} \notag \\
  &=0 \fullstop
\end{align}

回过头来,若要得到 $\ChrB{(\alpha)}{(\beta)}{(\beta)}$,
只需交换 $\alpha$、$\beta$,结果当然不变:
\begin{equation}
  \ChrB{(\alpha)}{(\beta)}{(\beta)}=0 \fullstop
\end{equation}

\item $\alpha=\beta=\gamma$。指标全部相同,有
\begin{align}
  \ChrB{(\alpha)}{(\alpha)}{(\alpha)}
  &=\frac{1}{\sqrt{g_{\alpha\alpha}}}
      \cdot \ChrB{\alpha}{\alpha}{\alpha}
    -\frac{1}{g_{\alpha\alpha}}
      \cdot \pdv{c^{(\alpha)}_\alpha}{x^\alpha} \notag \\
  &=\frac{1}{\sqrt{g_{\alpha\alpha}}}
    \cdot \qty(\frac{1}{2} \frac{1}{g_{\alpha\alpha}}
      \pdv{g_{\alpha\alpha}}{x^\alpha})
    -\frac{1}{g_{\alpha\alpha}}
      \cdot \pdv{x^\alpha} \sqrt{g_{\alpha\alpha}} \notag \\
  &=\frac{1}{2g_{\alpha\alpha} \sqrt{g_{\alpha\alpha}}}
      \pdv{g_{\alpha\alpha}}{x^\alpha}
    -\frac{1}{g_{\alpha\alpha}}
      \cdot \frac{1}{2\sqrt{g_{\alpha\alpha}}}
      \pdv{g_{\alpha\alpha}}{x^\alpha} \notag \\
  &=0 \comma
\end{align}
依然是个很漂亮的结果。
\end{myEnum}

到此,我们可以看到,只有两种形式的 Christoffel 符号非零:
\begin{braceEq}
  \ChrB{(\alpha)}{(\alpha)}{(\beta)}
  &=-\frac{1}{\sqrt{g_{\beta\beta}}}
    \pdv{x^\beta} \qty(\vphantom{\frac{0}{0}}
      \ln \sqrt{g_{\alpha\alpha}})
  =-\pdv{x^{(\beta)}} \ln\sqrt{g_{\alpha\alpha}} \comma \\
  \ChrB{(\alpha)}{(\beta)}{(\alpha)}
  &=\frac{1}{\sqrt{g_{\beta\beta}}}
    \pdv{x^\beta} \qty(\vphantom{\frac{0}{0}}
      \ln \sqrt{g_{\alpha\alpha}})
  =\pdv{x^{(\beta)}} \ln\sqrt{g_{\alpha\alpha}} \fullstop
\end{braceEq}
后一个等号利用了形式偏导数 \eqref{eq:形式偏导数_单位正交基}~式。

对于单位正交基,其度量满足
\begin{equation}
  g^{(\alpha)(\beta)}=g_{(\alpha)(\beta)}
  =\KroneckerDelta*{\alpha\beta} \comma
\end{equation}
因此协变基与逆变基只好相同。
这样一来,协变分量与逆变分量也就没有了差别,我们统一用尖括号标出。
于是上面的 Christoffel 符号就可以写成
\begin{braceEq}
  \ChrU{\alpha}{\alpha}{\beta}
  &=\ChrA{(\alpha)}{(\alpha)}{(\beta)}
    =\ChrB{(\alpha)}{(\alpha)}{(\beta)}
    =-\pdv{x^{(\beta)}} \ln\sqrt{g_{\alpha\alpha}} \comma \\
  \ChrU{\alpha}{\beta}{\alpha}
  &=\ChrA{(\alpha)}{(\beta)}{(\alpha)}
    =\ChrB{(\alpha)}{(\beta)}{(\alpha)}
    =\pdv{x^{(\beta)}} \ln\sqrt{g_{\alpha\alpha}} \fullstop
\end{braceEq}
显然,它们的第二、第三指标具有反对称性:
\begin{equation}
  \ChrU{\alpha}{\alpha}{\beta}
  =-\ChrU{\alpha}{\beta}{\alpha} \fullstop
  \label{eq:单位正交基下Christoffel符号的反对称性}
\end{equation}

\subsection{形式“协变”导数}
根据式~\eqref{eq:形式协变导数} 中的定义(仍以三阶张量为例),我们有
\begin{equation}
  \coD{(\mu)}{\tc{\Phi}{^{(\alpha)}_{\!(\beta)}^{\!(\gamma)}}}
  \defeq \pdv{\tc{\Phi}{^{(\alpha)}_{\!(\beta)}^{\!(\gamma)}}}%
    {x^{(\mu)}}
  +\ChrB{(\mu)}{(\sigma)}{(\alpha)}
    \tc{\Phi}{^{(\sigma)}_{\!(\beta)}^{\!(\gamma)}}
  -\ChrB{(\mu)}{(\beta)}{(\sigma)}
    \tc{\Phi}{^{(\alpha)}_{\!(\sigma)}^{\!(\gamma)}}
  +\ChrB{(\mu)}{(\sigma)}{(\gamma)}
    \tc{\Phi}{^{(\alpha)}_{\!(\beta)}^{\!(\sigma)}} \fullstop
\end{equation}
利用单位正交基不区分协变、逆变的特点,可以把它写成
\begin{align}
  \coD*{\mu}{\Phi\midscript{\orthIdx{\alpha\beta\gamma}}}
  &=\pdv{\Phi\midscript{\orthIdx{\alpha\beta\gamma}}}{x^{(\mu)}}
    +\ChrU{\mu}{\sigma}{\alpha} \,
      \Phi\midscript{\orthIdx{\sigma\beta\gamma}}
    -\ChrU{\mu}{\hl{\beta\sigma}}{} \,
      \Phi\midscript{\orthIdx{\alpha\sigma\gamma}}
    +\ChrU{\mu}{\sigma}{\gamma} \,
      \Phi\midscript{\orthIdx{\alpha\beta\sigma}} \notag
  \intertext{根据 Christoffel 符号的反对称性
    \eqref{eq:单位正交基下Christoffel符号的反对称性}~式,有}
  &=\pdv{\Phi\midscript{\orthIdx{\alpha\beta\gamma}}}{x^{(\mu)}}
    +\ChrU{\mu}{\sigma}{\alpha} \,
      \Phi\midscript{\orthIdx{\sigma\beta\gamma}}
    +\ChrU{\mu}{\hl{\sigma\beta}}{} \,
      \Phi\midscript{\orthIdx{\alpha\sigma\gamma}}
    +\ChrU{\mu}{\sigma}{\gamma} \,
      \Phi\midscript{\orthIdx{\alpha\beta\sigma}} \fullstop
\end{align}
这样,Christoffel 符号中的第二个指标就都是哑标 $\sigma$,
而哑标所取代的对应指标则放在第三个位置上。
同时,这种写法也免去了正负号的困扰。
