\chapter{预备知识}

\section{Euclid 空间}

我们生活在三维 \emphA{Euclid 空间}\idx{Euclid 空间}中。
Euclid 空间的存在,使我们得以完成平面几何与立体几何的计算与证明,
进而能够刻画诸多物理现象。

Euclid 空间具有很好的性质:
\begin{itemize}
  \item 封闭性:使得平移、缩放等操作可以完成;
  \item 完备性:允许微积分运算;
  \item 平坦:这是 Euclid 空间的根本属性;
  \item 数学结构:可以明确距离、角度和旋转等概念。
\end{itemize}

为了较为严格地刻画以上这些性质,我们需要引入一些定义。

\subsection{向量空间}

\emphA{向量空间}\idx{向量空间} $\vecV$ 是定义在数域\idx{数域} $\numF$
\footnote{这里不再给出域的严格定义,只需知道 $\numF$ 可以取为实数域
  $\realR$ 或复数域 $\comC$。}
上的一个集合。其上需要定义有两种二元运算,一种是%
\emphA{向量加法}\idx{向量加法}(简称\emphA{加法}):
\begin{equation}
  \mmap{+}{\vecV\times\vecV \ni (\V{u},\,\V{v})}%
    {\V{u}+\V{v} \in \vecV} \semicolon
\end{equation}
另一种是\emphA{标量乘法}\idx{标量乘法}(也称\emphA{数乘}):
\begin{equation} \label{eq:标量乘法}
  \mmap{\cdot}{\numF\times\vecV \ni (a,\,\V{u})}%
  {a\cdot\V{u} \in \vecV} \fullstop
\end{equation}
它们需要满足如下公理:
\begin{itemize}
  \item 加法结合律:对于 $\forall\,\V{u},\,\V{v},\,\V{w}\in\vecV$,
    满足
    \begin{equation}
      (\V{u}+\V{v})+\V{w} = \V{u}+(\V{v}+\V{w}) \fullstop
    \end{equation}
  \item 加法交换律:对于 $\forall\,\V{u},\,\V{v}\in\vecV$,满足
    \begin{equation}
      \V{u}+\V{v} = \V{v}+\V{u} \fullstop
    \end{equation}
  \item 加法单位元:存在 $\V{0}\in\vecV$,使得
    \begin{equation}
      \forall \, \V{u}\in\vecV \qc \V{u} + \V{0} = \V{u} \fullstop
    \end{equation}
  \item 加法逆元:对于 $\forall\,\V{x}\in\vecV$,存在逆元
    $-\V{x}\in\vecV$,满足
    \begin{equation}
      \V{x}+(-\V{x})=\V{0} \fullstop
    \end{equation}
    注意此处的“$-$”不表示减法,因为我们并没有给出减法的定义。
  \item 标量乘法与数域乘法相容:对于
    $\forall\,\alpha,\,\beta\in\numF$ 和 $\V{u}\in\vecV$,满足
    \begin{equation}
      \alpha \, (\beta\,\V{u}) = (\alpha\beta) \, \V{u} \fullstop
    \end{equation}
    式中,左侧括号里的为 \eqref{eq:标量乘法}~式中定义的标量乘法,
    而右侧括号里的则为数域上的普通乘法。
  \item 标量乘法单位元:存在 $1\in\numF$,使得
    \begin{equation}
      \forall \V{u}\in\vecV \qc 1 \cdot \V{u} = \V{u} \fullstop
    \end{equation}
  \item 标量乘法关于加法的分配律:对于 $\forall\,\alpha,\,\beta
    \in\numF$ 和 $\V{u},\,\V{v}\in\vecV$,满足
    \begin{mySubEq}
      \begin{align}
        (\alpha+\beta) \, \V{u}
          &= \alpha \, \V{u} + \beta \, \V{u} \comma \\
        \alpha \, (\V{u}+\V{v})
          &= \alpha \, \V{u} + \alpha \, \V{v} \fullstop
      \end{align}
    \end{mySubEq}
    其中,第一式为关于\emphB{向量加法}的分配律,第二式则为关于%
    \emphB{数域加法}的分配律。
\end{itemize}

前四条公理 说明向量空间 $\vecV$ 是一个\emphB{交换群}\idx{交换群}。

\subsection{代数结构}

向量空间上可以定一些数学结构,包括\emphA{度量}(也称\emphA{度规})、
\emphA{范数}、\emphA{内积}等。

\emphA{度量}是对距离的刻画。向量空间 $\vecV$ 上的度量是一个映照:
\begin{equation}
  \mmap{d}{\vecV\times\vecV \ni (\V{x},\,\V{y})}%
    {d(\V{x},\,\V{y}) \in \realR} \comma
\end{equation}
它满足
\begin{itemize}
  \item 非负性:对于 $\forall\,\V{x},\,\V{y}\in\vecV$,满足
    \begin{equation}
      d(\V{x},\,\V{y}) \geqslant 0 \comma
    \end{equation}
    当且仅当 $\V{x}=\V{y}$ 时取等号;
  \item 对称性:对于 $\forall\,\V{x},\,\V{y}\in\vecV$,满足
    \begin{equation}
      d(\V{x},\,\V{y}) = d(\V{y},\,\V{x}) \semicolon
    \end{equation}
  \item 三角不等式:对于 $\forall\,\V{x},\,\V{y},\,\V{z}\in\vecV$,
    满足
    \begin{equation}
      d(\V{x},\,\V{z}) \leqslant d(\V{x},\,\V{y}) + d(\V{y},\,\V{z})
      \fullstop
    \end{equation}
\end{itemize}

\emphA{范数}反映了向量的“长度”。它也是一个映照:
\begin{equation}
  \mmap{\norm[\vecV]{\cdot}}{\vecV \ni \V{x}}%
    {\norm[\vecV]{\V{x}} \in \realR} \comma
\end{equation}
并满足
\begin{itemize}
  \item 非负性:对于 $\forall\,\V{x}\in\vecV$,满足
    \begin{equation}
      \norm[\vecV]{\V{x}} \geqslant 0 \comma
    \end{equation}
    当且仅当 $\V{x}=\V{0}$ 时取等号;
  \item 正齐次性:对于 $\forall\,\V{x}\in\vecV$ 和 $\forall\,\lambda\in\realR$,
    满足
    \begin{equation}
     \norm[\vecV]{\lambda\V{x}} = \abs{\lambda} \norm[\vecV]{\V{x}} \semicolon
    \end{equation}
  \item 三角不等式:对于 $\forall\,\V{x},\,\V{y}\in\vecV$,
    满足
    \begin{equation}
      \norm[\vecV]{\V{x}+\V{y}} \leqslant
      \norm[\vecV]{\V{x}} + \norm[\vecV]{\V{y}} \fullstop
    \end{equation}
\end{itemize}

\emphA{内积}用来刻画向量的之间的“夹角”。它同样用映照来定义:
\begin{equation}
  \mmap{\ipb[\vecV]{\cdot}{\cdot}}{\vecV \ni (\V{x},\,\V{y})}%
    {\ipb[\vecV]{\V{x}}{\V{y}} \in \realR} \comma
\end{equation}
满足
\begin{itemize}
  \item 非负性:对于 $\forall\,\V{x}\in\vecV$,满足
    \begin{equation}
      \ipb[\vecV]{\V{x}}{\V{x}} \geqslant 0 \comma
    \end{equation}
    当且仅当 $\V{x}=\V{0}$ 时取等号;
  \item 对称性:对于 $\forall\,\V{x},\,\V{y}\in\vecV$,满足
    \begin{equation}
      \ipb[\vecV]{\V{x}}{\V{y}} = \ipb[\vecV]{\V{y}}{\V{x}} \semicolon
    \end{equation}
  \item 对第一个元素的线性性:对于 $\forall\,\V{x},\,\V{y},\,\V{z}\in\vecV$ 和
    $\forall\,\alpha,\,\beta\in\realR$,满足
    \begin{equation}
      \ipb[\vecV]{\alpha\V{x}+\beta\V{y}}{\V{z}}
      = \alpha \ipb[\vecV]{\V{x}}{\V{z}} + \beta \ipb[\vecV]{\V{y}}{\V{z}}
      \fullstop
    \end{equation}
    根据对称性,内积对第二个元素显然也具有线性性:
    \begin{equation}
      \ipb[\vecV]{\V{x}}{\alpha\V{y}+\beta\V{z}}
      = \ipb[\vecV]{\alpha\V{y}+\beta\V{z}}{\V{x}}
      = \alpha \ipb[\vecV]{\V{y}}{\V{x}} + \beta \ipb[\vecV]{\V{z}}{\V{x}}
      = \alpha \ipb[\vecV]{\V{x}}{\V{y}} + \beta \ipb[\vecV]{\V{x}}{\V{z}}.
    \end{equation}
\end{itemize}

以上三种代数结构并非是独立的。事实上,内积可以诱导范数,范数又可以诱导度量。
如果线性空间 $\vecV$ 上装备有内积,则可以定义度量:
\begin{equation}
  \norm[\vecV]{\V{x}} = \sqrt{\ipb[\vecV]{\V{x}}{\V{x}}} \qc
  \forall\,\V{x}\in\vecV.
\end{equation}
