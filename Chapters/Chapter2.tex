\section{张量积}
	\emphA{张量积}也叫\emphA{张量并},用符号“$\tp$”表示。
	在 \ref{subsec:张量的表示与简单张量} 小节给出简单张量的定义时,
	实际上就用到了张量积。张量积的定义为:
	\begin{align}
		\forall\,\Tens{\Phi}\in\Tensors{p},\,\Tens{\Psi}\in\Tensors{q},
		&\mathrel{\phantom{=}} \Tens{\Phi}\tp\Tens{\Psi}
			\in\Tensors{p+q} \notag \\
		&=\qty(\Phi^{i_1 \cdots i_p} \,
				\V{g}_{i_1}\tp\cdots\tp\V{g}_{i_p})
			\tp \qty(\Psi_{j_1 \cdots j_q} \,
				\V{g}^{j_1}\tp\cdots\tp\V{g}^{j_q}) \notag \\
		&\defeq \Phi^{i_1 \cdots i_p} \:
			\Psi_{j_1 \cdots j_q}\,
			\qty(\V{g}_{i_1}\tp\cdots\tp\V{g}_{i_p})
			\tp \qty(\V{g}^{j_1}\tp\cdots
				\tp\V{g}^{j_q}_{\phantom{i_p}}) \fullstop
	\end{align}
	由该定义可以知道,关于简单张量 $\qty(\V{g}_{i_1}\tp\cdots
		\tp\V{g}_{i_p}) \tp \qty(\V{g}^{j_1}\tp\cdots
		\tp\V{g}^{j_q}_{\phantom{i_p}})$,相应的张量分量为
	\begin{equation}
		\tensor{\qty\big(\Phi\tp\Psi)}
			{^{i_1 \cdots i_p}_{j_1 \cdots j_q}} \fullstop
	\end{equation}
	
\section{\texorpdfstring{$e$ 点积}{e 点积}}
	张量的 \emphA{$e$ 点积}可以用符号“$\edp$”表示。
	从这个符号可以看出 $e$ 点积的作用:前 $e$ 个指标缩并,后面的点乘。
	
	对于任意的 $\Tens{\Phi}\in\Tensors{p},\,
		\Tens{\Psi}\in\Tensors{q},\,
		e\leqslant\min\qty{p,\,q}\in\natN$,$e$ 点积是这样定义的:
	\begin{align}
		&\mathrel{\phantom{=}} \Tens{\Phi}\edp\Tens{\Psi} \notag \\
		&=\qty(\Phi^{i_1 \cdots i_{p-e} i_{p-e+1} \cdots i_p} \,
			\V{g}_{i_1}\tp\cdots\tp\V{g}_{i_{p-e}}
			\tp\hl{\V{g}_{i_{p-e+1}}\tp\cdots\tp\V{g}_{i_p}}
			) \notag \\
		&\mathrel{\phantom{=}}\quad\edp
			\qty(\Psi^{j_1 \cdots j_e j_{e+1} \cdots j_q} \,
			\hl{\V{g}_{j_1}\tp\cdots\tp\V{g}_{j_e}}
			\tp\V{g}_{j_{e+1}}\tp\cdots\tp\V{g}_{j_q}) \notag
		\intertext{把高亮的部分做内积,得到\emphB{度量}:}
		&\defeq\Phi^{i_1 \cdots i_{p-e} i_{p-e+1} \cdots i_p} \:
			\Psi^{j_1 \cdots j_e j_{e+1} \cdots j_q} \notag \\
		&\mathrel{\phantom{=}}\quad\cdot
			g_{i_{p-e+1} j_1} \cdots g_{i_p j_e} \,
			\qty(\V{g}_{i_1}\tp\cdots\tp\V{g}_{i_{p-e}})
			\tp\qty(\V{g}_{j_{e+1}}\tp\cdots\tp\V{g}_{j_q}) \notag
		\intertext{玩一下“指标升降游戏”(注意有两种结合方式:
			与 $\Phi$ 或 $\Psi$),可得}
		&=\left\{\begin{lgathered}
				\tensor{\Phi}{^{i_1 \cdots i_{p-e}}_{\hl{j_1 \cdots j_e}}}
				\Psi^{\hl{j_1 \cdots j_e} j_{e+1} \cdots j_q} \\
				\Phi^{i_1 \cdots i_{p-e} \hl{i_{p-e+1} \cdots i_p}}
				\tensor{\Psi}{_{\hl{i_{p-e+1} \cdots i_p}}^{j_{e+1}
					\cdots j_q}}
			\end{lgathered}\right\}
			\qty(\V{g}_{i_1}\tp\cdots\tp\V{g}_{i_{p-e}})
			\tp\qty(\V{g}_{j_{e+1}}\tp\cdots\tp\V{g}_{j_q}) \fullstop
	\end{align}
	最后一式的大括号中,高亮的 $j_1 \cdots j_e$
	和 $i_{p-e+1} \cdots i_p$ 都是哑标,所以可以求和求掉。因此有
	\begin{equation}
		\Tens{\Phi}\edp\Tens{\Psi} \in \Tensors{p+q-2e} \fullstop
	\end{equation}