\section{张量积} \label{sec:张量积}
\emphA{张量积}也叫\emphA{张量并},用符号“$\tp$”表示。
在 \ref{subsec:张量的表示与简单张量}~小节给出简单张量的定义时,
实际上就用到了张量积。张量积的定义为:
\begin{align}
	\forall\,\T{\Phi}\in\Tensors{p},\,\T{\Psi}\in\Tensors{q},
	&\alspace \T{\Phi}\tp\T{\Psi}
		\in\Tensors{p+q} \notag \\
	&=\qty(\Phi^{i_1 \cdots i_p} \,
			\V{g}_{i_1}\tp\cdots\tp\V{g}_{i_p})
		\tp \qty(\Psi_{j_1 \cdots j_q} \,
			\V{g}^{j_1}\tp\cdots\tp\V{g}^{j_q}) \notag \\
	&\defeq \Phi^{i_1 \cdots i_p} \,
		\Psi_{j_1 \cdots j_q}\,
		\qty(\V{g}_{i_1}\tp\cdots\tp\V{g}_{i_p})
		\tp \qty(\V{g}^{j_1}\tp\cdots
			\tp\V{g}^{j_q}_{\phantom{i_p}}) \fullstop
\end{align}
由该定义可以知道,关于简单张量 $\qty(\V{g}_{i_1}\tp\cdots
	\tp\V{g}_{i_p}) \tp \qty(\V{g}^{j_1}\tp\cdots
	\tp\V{g}^{j_q}_{\phantom{i_p}})$,相应的张量分量为
\begin{equation}
	\tensor{\qty\big(\Phi\tp\Psi)}
		{^{i_1 \cdots i_p}_{j_1 \cdots j_q}} \fullstop
\end{equation}

\section{\texorpdfstring{$e$ 点积}{e 点积}}
张量的 \emphA{$e$ 点积}可以用符号“$\edp$”表示。
从这个符号可以看出 $e$ 点积的作用:前 $e$ 个指标缩并,后面的点乘。

对于任意的 $\T{\Phi}\in\Tensors{p},\,
	\T{\Psi}\in\Tensors{q},\,
	e\leqslant\min\qty{p,\,q}\in\natN$,$e$ 点积是这样定义的:
\begin{align}
	&\alspace \T{\Phi}\edp\T{\Psi} \notag \\
	&=\qty(\Phi^{i_1 \cdots i_{p-e} i_{p-e+1} \cdots i_p} \,
		\V{g}_{i_1}\tp\cdots\tp\V{g}_{i_{p-e}}
		\tp\hl{\V{g}_{i_{p-e+1}}\tp\cdots\tp\V{g}_{i_p}}
		) \notag \\
	&\alspace\quad\edp
		\qty(\Psi^{j_1 \cdots j_e j_{e+1} \cdots j_q} \,
		\hl{\V{g}_{j_1}\tp\cdots\tp\V{g}_{j_e}}
		\tp\V{g}_{j_{e+1}}\tp\cdots\tp\V{g}_{j_q}) \notag
	\intertext{把高亮的部分做内积,得到\emphB{度量}:}
	&\defeq\Phi^{i_1 \cdots i_{p-e} i_{p-e+1} \cdots i_p} \,
		\Psi^{j_1 \cdots j_e j_{e+1} \cdots j_q} \notag \\
	&\alspace\quad\cdotp
		g_{i_{p-e+1} j_1} \cdots g_{i_p j_e} \,
		\qty(\V{g}_{i_1}\tp\cdots\tp\V{g}_{i_{p-e}})
		\tp\qty(\V{g}_{j_{e+1}}\tp\cdots\tp\V{g}_{j_q}) \notag
	\intertext{玩一下“指标升降游戏”(注意有两种结合方式:
		与 $\Phi$ 或 $\Psi$),可得}
	&=\left\{\begin{lgathered}
			\tensor{\Phi}{^{i_1 \cdots i_{p-e}}_{\hl{j_1 \cdots j_e}}} \,
			\Psi^{\hl{j_1 \cdots j_e} j_{e+1} \cdots j_q} \\
			\Phi^{i_1 \cdots i_{p-e} \hl{i_{p-e+1} \cdots i_p}} \,
			\tensor{\Psi}{_{\hl{i_{p-e+1} \cdots i_p}}^{j_{e+1}
				\cdots j_q}}
		\end{lgathered}\right\}
		\qty(\V{g}_{i_1}\tp\cdots\tp\V{g}_{i_{p-e}})
		\tp\qty(\V{g}_{j_{e+1}}\tp\cdots\tp\V{g}_{j_q}) \fullstop
\end{align}
最后一步的花括号中,高亮的 $j_1 \cdots j_e$
和 $i_{p-e+1} \cdots i_p$ 都是哑标,可以通过求和求掉。因此有
\begin{equation}
	\T{\Phi}\edp\T{\Psi} \in \Tensors{p+q-2e} \fullstop
\end{equation}
换句话说,$e$ 点积的作用就是将指标\emphB{哑标化}。

作为一个特殊的应用,接下来我们介绍\emphA{全点积},用符号“\fdp”表示。
对于任意的 $\T{\Phi},\,\T{\Psi}\in\Tensors{p}$,有
\begin{align}
	&\alspace \T{\Phi}\fdp\T{\Psi}
		\defeq \T{\Phi}\edp[p]\T{\Psi} \notag \\
	&=\qty(\Phi^{i_1 \cdots i_p}\,\V{g}_{i_1}\tp\cdots\tp\V{g}_{i_p})
		\edp[p]
		\qty(\Psi^{j_1 \cdots j_p}\,\V{g}_{j_1}\tp\cdots\tp\V{g}_{j_p})
		\notag \\
	&=\Phi^{i_1 \cdots i_p} \, \Psi^{j_1 \cdots j_p} \,
		g_{i_1 j_1} \cdots g_{i_p j_p} \notag \\
	&=\left\{\begin{lgathered}
			\Phi_{j_1 \cdots j_p} \, \Psi^{j_1 \cdots j_p} \\
			\Phi^{i_1 \cdots i_p} \, \Psi_{i_1 \cdots i_p}
		\end{lgathered}\right.
		\in\realR \fullstop
\end{align}
可见,全点积将\emphB{全部}指标哑标化。

张量自身和自身的全点积,定义为它的\emphA{范数}:
\begin{equation}
	\T{\Phi}\fdp\T{\Phi}
	=\Phi^{i_1 \cdots i_p} \, \Phi_{i_1 \cdots i_p}
	\eqcolon \qty|\T{\Phi}|^2_{\Tensors{p}} \fullstop
\end{equation}

\section{叉乘}
张量的\emphA{叉乘}要求底空间为 $\realR^3$。
对于任意的 $\T{\Phi}\in\Tensors[\realR^3]{p},\,
\T{\Psi}\in\Tensors[\realR^3]{q}$,叉乘的定义如下:
\begin{align}
	&\alspace \T{\Phi}\cp\T{\Psi} \notag \\
	&=\qty(\Phi^{i_1 \cdots i_{p-1} i_p} \,
			\V{g}_{i_1}\tp\cdots\tp\V{g}_{i_{p-1}}\tp\V{g}_{i_p})
		\cp \qty(\Psi_{j_1 j_2 \cdots j_q} \,
			\V{g}^{j_1}\tp\V{g}^{j_2}\cdots\tp\V{g}^{j_q}) \notag \\
	&\defeq \Phi^{i_1 \cdots i_p} \, \Psi_{j_1 \cdots j_p} \,
		\V{g}_{i_1}\tp\cdots\tp\V{g}_{i_{p-1}}
		\tp\qty(\V{g}_{i_p}\cp\V{g}^{j_1})
		\tp\V{g}^{j_2}\cdots\tp\V{g}^{j_q}
		\in\Tensors[\realR^3]{p+q-1} \fullstop
\end{align}
注意到,此时简单张量的维数已经降了一阶。

利用\emphA{Levi-Civita 记号},可以进一步展开上式。
\begin{align}
	\V{g}_{i_p}\cp\V{g}^{j_1}
	=\LeviCivita{_{i_p}^{j_1}_s}\,\V{g}^s \comma
\end{align}
式中的
\begin{equation}
	\LeviCivita{_{i_p}^{j_1}_s}
	=\det[\V{g}_{i_p},\,\V{g}^{j_1},\,\V{g}_s] \fullstop
\end{equation}
于是
\begin{equation}
	\T{\Phi}\cp\T{\Psi} \,
	=\LeviCivita{_{i_p}^{j_1}_s}\,
		\Phi^{i_1 \cdots i_p} \, \Psi_{j_1 \cdots j_p}
		\V{g}_{i_1}\tp\cdots\tp\V{g}_{i_{p-1}} \tp\V{g}^s
		\tp\V{g}^{j_2}\cdots\tp\V{g}^{j_q} \fullstop
\end{equation}

下面我们再来类比地定义一种混合积“$\edp[\cp]$”。
对于任意的 $\T{\Phi},\,\T{\Psi}\in\Tensors{3}$,定义
\begin{align}
	\T{\Phi}\edp[\cp]\T{\Psi}
	&=\qty(\Phi^{ijk}\,\V{g}_i\tp\V{g}_j\tp\V{g}_k)
		\edp[\cp]\qty(\Psi_{pqr}\,\V{g}^p\tp\V{g}^q\tp\V{g}^r)\notag \\
	&\defeq \Phi^{ijk}\,\Psi_{pqr} \,
		\KroneckerDelta{q}{j} \,
		\V{g}_i\tp\qty(\V{g}_k\cp\V{g}^p)\tp\V{g}^r \notag
	\intertext{缩并掉 Kronecker δ,
		同时利用 Levi-Civita 记号展开叉乘项,可有}
	&=\LeviCivita{_k^p_s}\,\Phi^{ijk}\,\Psi_{pjr}\,
		\V{g}_i\tp\V{g}^s\tp\V{g}^r \comma
\end{align}
式中的
\begin{equation}
	\LeviCivita{_k^p_s}=\det[\V{g}_k,\,\V{g}^p,\,\V{g}_s] \fullstop
\end{equation}

对于这种混合积,并没有一般的约定。不同的研究者往往会采用不同的写法及表示。

\section{置换(一)}
本节主要介绍\emphA{置换运算}的定义及相关概念,
这将使我们暂时离开张量运算的主线。

置换运算实际上是一种交换位置或者改变次序的运算。
之后我们还将引入针对张量的\emph{置换算子},它是外积运算和外微分运算的基础。
这些运算是现代张量分析与微分几何的支柱。

\subsection{置换的定义}
我们从一个例子开始。下面是一个 $2 \times 7$ 的“矩阵”:
\begin{equation}
	\Perm{\sigma}=\mqty[
		\circNum{1} & \circNum{2} & \circNum{3} & \circNum{4} &
			\circNum{5} & \circNum{6} & \circNum{7} \\
		\circNum{7} & \circNum{4} & \circNum{5} & \circNum{1} &
			\circNum{6} & \circNum{2} & \circNum{3}
	] \fullstop
	\label{eq:置换序号定义}
\end{equation}
矩阵里面的每一个数字表示一个位置。可以想象成 7 把椅子,
先是按第一行的顺序依次排列,再按照第二行的顺序打乱,重新排列。
于是这就成为一个\emphA{7 阶置换}。这个定义等价于
\begin{mySubEq}
	\begin{gather}
		\Perm{\sigma}=\mqty*(
			4 & 9 & 2 & 7 & 5 & 8 & 3 \\
			3 & 7 & 5 & 4 & 8 & 9 & 2
		) \comma \label{eq:置换元素表示_数字}
		\intertext{自然也等价于}
		\Perm{\sigma}=\mqty*(
			\spadesuit & \heartsuit & \diamondsuit & \clubsuit &
				\varspadesuit & \varheartsuit & \vardiamondsuit \\
			\vardiamondsuit & \clubsuit & \varspadesuit & \spadesuit &
				\varheartsuit & \heartsuit & \diamondsuit
		) \comma \label{eq:置换元素表示_符号}
	\end{gather}
\end{mySubEq}
当然,换用任何元素也都是可以的。

通常我们用方括号表示置换的\emphA{序号定义},即标号的排列轮换;
用圆括号表示\emphA{元素定义},即标号对应元素的轮换。

\subsection{置换的符号}
接着来定义置换的\emphA{符号} $\sgn\Perm{\sigma}$。
这里我们把每次交换两个数字称为一次“操作”。
如果经过\emphB{偶数次}“操作”,可以把经置换后的序列恢复为原来的顺序,
那么该置换的符号 $\sgn\Perm{\sigma} = 1$;
而如果经过\emphB{奇数次}“操作”才可以复原,则 $\sgn\Perm{\sigma}=-1$。
若用一个式子表示,则为
\begin{equation}
	\sgn\Perm{\sigma} = (-1)^n \comma
\end{equation}
其中的 $n$ 是恢复原本顺序所需“操作”的次数.

下面我们以式~\eqref{eq:置换序号定义} 所定义的 $\Perm{\sigma}$ 为例,
演示求置换符号的过程。这里的关键是通过两两交换,
按如下步骤把式~\eqref{eq:置换元素表示_符号} 的第二行变换成第一行:
\begin{gather*}
	\mqty{
		\hl{\vardiamondsuit} & \hlw{\clubsuit} & \hlw{\varspadesuit} &
			\hl{\spadesuit} & \hlw{\varheartsuit} & \hlw{\heartsuit} &
			\hlw{\diamondsuit}
	} \\
	\mqty{ & & & \Downarrow & & & } \\
	\mqty{
		\hl[pink]{\spadesuit} & \hl{\clubsuit} & \hlw{\varspadesuit} &
			\hl[pink]{\vardiamondsuit} & \hlw{\varheartsuit} &
			\hl{\heartsuit} & \hlw{\diamondsuit}
	} \\
	\mqty{ & & & \Downarrow & & & } \\
	\mqty{
		\hlw{\spadesuit} & \hl[pink]{\heartsuit} & \hl{\varspadesuit} &
			\hlw{\vardiamondsuit} & \hlw{\varheartsuit} &
			\hl[pink]{\clubsuit} & \hl{\diamondsuit}
	} \\
	\mqty{ & & & \Downarrow & & & } \\
	\mqty{
		\hlw{\spadesuit} & \hlw{\heartsuit} & \hl[pink]{\diamondsuit} &
			\hl{\vardiamondsuit} & \hlw{\varheartsuit} & \hl{\clubsuit} &
			\hl[pink]{\varspadesuit}
	} \\
	\mqty{ & & & \Downarrow & & & } \\
	\mqty{
		\hlw{\spadesuit} & \hlw{\heartsuit} & \hlw{\diamondsuit} &
			\hl[pink]{\clubsuit} & \hl{\varheartsuit} &
			\hl[pink]{\vardiamondsuit} & \hl{\varspadesuit}
	} \\
	\mqty{ & & & \Downarrow & & & } \\
	\phantom{\mspace{10mu}}\mqty{
		\hlw{\spadesuit} & \hlw{\heartsuit} & \hlw{\diamondsuit} &
			\hlw{\clubsuit} & \hl[pink]{\varspadesuit} &
			\hl{\vardiamondsuit} &
			\hl{\hl[pink]{\varheartsuit}}
	} \\
	\mqty{ & & & \Downarrow & & & } \\
	\mqty{
		\hlw{\spadesuit} & \hlw{\heartsuit} & \hlw{\diamondsuit} &
			\hlw{\clubsuit} & \hlw{\varspadesuit} &
			\hl[pink]{\varheartsuit} & \hl[pink]{\vardiamondsuit}
	}
\end{gather*}
一共进行了 6 次两两交换,因此 $\sgn\Perm{\sigma}=1$。

\subsection{置换的复合}
再定义一个置换
\begin{equation}
	\Perm{\tau}=\mqty[
		1 & 2 & 3 & 4 & 5 & 6 & 7 \\
		5 & 1 & 7 & 3 & 6 & 4 & 2
	] \fullstop
\end{equation}
注意这里用了方括号,因此它是一个\emphB{序号定义}。
方便起见,以后的序号我们都只用不带圈的普通数字表示。
考虑之前定义的置换
\begin{equation}
	\Perm{\sigma}=\mqty[
		1 & 2 & 3 & 4 & 5 & 6 & 7 \\
		7 & 4 & 5 & 1 & 6 & 2 & 3
	] \comma
\end{equation}
则 $\Perm{\tau}$ 与 $\Perm{\sigma}$ 的复合
\begin{equation}
	\Perm{\tau}\comp\Perm{\sigma}=
	\qty(\begin{array}{@{}ccccccc@{}}
		\dicei & \diceii & \diceiii & \diceiv & \dicev &
			\dicevi & \circledtwodots \\
		\circledtwodots & \diceiv & \dicev & \dicei & \dicevi &
			\diceii & \diceiii \\
		\hdashline
		\dicevi & \circledtwodots & \diceiii & \dicev & \diceii &
			\dicei & \diceiv
	\end{array})
	\quad\mqty{\\ \leftarrow\Perm{\sigma} \\ \leftarrow\Perm{\tau}}
\end{equation}
与函数、线性变换等的复合类似,这里也用小圆圈“$\comp$”表示置换的复合。

假设经过置换 $\Perm{\sigma}$、$\Perm{\tau}$ 作用后得到的序列,
分别需要 $p$ 次和 $q$ 次两两交换才能复原为原来的序列。
那么很显然,经过复合置换 $\Perm{\tau}\comp\Perm{\sigma}$ 作用后的序列,
经过 $q+p$ 次两两交换也一定可以复原。因此,复合置换的符号
\begin{equation}
	\sgn\qty(\Perm{\tau}\comp\Perm{\sigma})
	=(-1)^{q+p}=(-1)^q \cdotp (-1)^p
	=\sgn\Perm{\tau}\cdotp\sgn\Perm{\sigma} \fullstop
	\label{eq:置换复合的符号}
\end{equation}

\subsection{逆置换}
逆置换 $\Perm{\sigma}^{-1}$ 的定义为
\begin{equation}
	\Perm{\sigma}^{-1}\comp\Perm{\sigma} = \Id \comma
\end{equation}
其中的“$\Id$”是\emphA{恒等映照}。

仍然使用式~\eqref{eq:置换元素表示_符号}:
\begin{equation}
	\Perm{\sigma}=\mqty*(
		\spadesuit & \heartsuit & \diamondsuit & \clubsuit &
			\varspadesuit & \varheartsuit & \vardiamondsuit \\
		\vardiamondsuit & \clubsuit & \varspadesuit & \spadesuit &
			\varheartsuit & \heartsuit & \diamondsuit
	) \comma
\end{equation}
那么自然有
\begin{equation}
	\Perm{\sigma}^{-1}=\mqty*(
		\vardiamondsuit & \clubsuit & \varspadesuit & \spadesuit &
			\varheartsuit & \heartsuit & \diamondsuit \\
		\spadesuit & \heartsuit & \diamondsuit & \clubsuit &
			\varspadesuit & \varheartsuit & \vardiamondsuit
	) \fullstop
\end{equation}
显然,我们有 $\Perm{\sigma}^{-1}\comp\Perm{\sigma} = \Id$。

回忆一下逆矩阵的定义。矩阵 $\Mat{A}$ 的逆 $\Mat{A}^{-1}$ 既要满足
$\Mat{A}^{-1}\Mat{A}=\Mat{I}$,又要满足
$\Mat{A}\Mat{A}^{-1}=\Mat{I}$。对于置换也是如此,
因此我们需要检查 $\Perm{\sigma}\comp\Perm{\sigma}^{-1}$:\footnote{%
	该式中的数字角标用来澄清原始序号。}
\begin{equation}
	\Perm{\sigma}\comp\Perm{\sigma}^{-1}=
	\qty(\begin{array}{@{}ccccccc@{}}
		\vardiamondsuit & \clubsuit & \varspadesuit & \spadesuit &
			\varheartsuit & \heartsuit & \diamondsuit \\
		\spadesuit_1 & \heartsuit_2 & \diamondsuit_3 & \clubsuit_4 &
			\varspadesuit_5 & \varheartsuit_6 & \vardiamondsuit_7 \\
		\hdashline
		\vardiamondsuit_7 & \clubsuit_4 & \varspadesuit_5 &
			\spadesuit_1 & \varheartsuit_6 &
			\heartsuit_2 & \diamondsuit_3
	\end{array})
	\quad\mqty{
		\\ \leftarrow\Perm{\sigma}^{-1} \\
		\leftarrow\Perm{\sigma}\phantom{^{-1}}
	}
\end{equation}
可见的确有 $\Perm{\sigma}\comp\Perm{\sigma}^{-1}=\Id$。

另外,由于恒等映照 $\Id$ 作用后序列不发生变化,
复原所需的交换次数为 0,因此
\begin{equation}
	\sgn\Id=(-1)^0=1 \fullstop
\end{equation}
而根据定义,
\begin{equation}
	\Id=\Perm{\sigma}^{-1}\comp\Perm{\sigma} \comma
\end{equation}
故有
\begin{equation}
	\sgn\Perm{\sigma} \cdotp \sgn\Perm{\sigma}^{-1} = 1 \fullstop
\end{equation}
由此,可以推知
\begin{equation}
	\sgn\Perm{\sigma}=\sgn\Perm{\sigma}^{-1} \comma
	\label{eq:逆置换的符号}
\end{equation}
即置换与它的逆具有\emphB{相同}的符号。

\section{置换(二)}
本节将介绍置换运算的基本性质。

\subsection{置换的穷尽} \label{subsec:置换的穷尽}
先要做一点铺垫。设有序数组
\begin{equation*}
	\qty{i_1,\,i_2,\,\cdots,\,i_r}
\end{equation*}
经置换 $\Perm{\sigma}$ 作用后成为
\begin{equation*}
	\qty{\Perm{\sigma}(i_1),\,\Perm{\sigma}(i_2),\,
		\cdots,\,\Perm{\sigma}(i_r)} \comma
\end{equation*}
则根据之前的元素定义(圆括号),可以把 $\Perm{\sigma}$ 记为
\begin{equation}
	\Perm{\sigma}=\mqty*(
		i_1 & i_2 & \cdots & i_r \\
		\Perm{\sigma}(i_1) & \Perm{\sigma}(i_2) &
			\cdots & \Perm{\sigma}(i_r)
	)\fullstop
\end{equation}
每次置换都将得到一个有序数组。把它们组合到一起,就可以得到集合
\begin{equation}
	\set[\bigg]
		{\qty(\Perm{\sigma}(i_1),\,\Perm{\sigma}(i_2),\,
			\cdots,\,\Perm{\sigma}(i_r))}
		{\forall\,\Perm{\sigma}\in\Permutations{r}} \fullstop
\end{equation}
其中的 $\Permutations{r}$ 表示 $r$ 阶置换的全体。
根据排列组合原理,$r$ 阶置换的总数等于 $r$ 个元素的\emphB{全排列数}。
即该集合共有 $r!$ 个元素。

下面我们要证明
\begin{mySubEq}
	\begin{align}
		&\alspace\set[\bigg]
			{\qty(\Perm{\sigma}(i_1),\,\Perm{\sigma}(i_2),\,
				\cdots,\,\Perm{\sigma}(i_r))}
			{\forall\,\Perm{\sigma}\in\Permutations{r}} \notag \\
		%
		&=\set[\bigg]
			{\qty(
				\Perm{\tau}\comp\Perm{\sigma}(i_1),\,
				\Perm{\tau}\comp\Perm{\sigma}(i_2),\,\cdots,\,
				\Perm{\tau}\comp\Perm{\sigma}(i_r) )}
			{\forall\,\Perm{\sigma},\,\Perm{\tau}\in\Permutations{r}}
			\label{eq:置换的穷尽_复合1} \\
		&=\set[\bigg]
			{\qty(
				\Perm{\sigma}\comp\Perm{\tau}(i_1),\,
				\Perm{\sigma}\comp\Perm{\tau}(i_2),\,\cdots,\,
				\Perm{\sigma}\comp\Perm{\tau}(i_r) )}
			{\forall\,\Perm{\sigma},\,\Perm{\tau}\in\Permutations{r}}
			\label{eq:置换的穷尽_复合2} \\
		&=\set[\bigg]
			{\qty(
				\Perm{\sigma}^{-1}(i_1),\,
				\Perm{\sigma}^{-1}(i_2),\,\cdots,\,
				\Perm{\sigma}^{-1}(i_r))}
			{\forall\,\Perm{\sigma}\in\Permutations{r}} 
			\label{eq:置换的穷尽_逆} \fullstop
	\end{align}
\end{mySubEq}

所谓“穷尽”,就是将 $\Permutations{r}$ 中的所有置换 $\Perm{\sigma}$
全部枚举出来。关于 $\Perm{\sigma}$ 的求和就是一个例子。
以上这条性质说明,置换 $\Perm{\sigma}$ 如果作为一个广义上的“哑标”,
那么穷尽的结果与用 $\Perm{\tau}\comp\Perm{\sigma}$、
$\Perm{\sigma}\comp\Perm{\tau}$ 或 $\Perm{\sigma}^{-1}$
代替该“哑标”的结果是一样的。

\myPROBLEM{这说明置换构成了置换群。?}

\begin{myProof}
证明的思路是说明集合互相包含。

对于式~\eqref{eq:置换的穷尽_复合1},
右边的 $\Perm{\tau}\comp\Perm{\sigma}$ 也是一个 $r$ 阶置换,
自然符合左边集合的定义,因此 $\text{右边}\subset\text{左边}$。
由于这一步是相当显然的,以下的几个证明我们将略去该步。
另一方面,左边的 $\Perm{\sigma}$ 可以表示成
\begin{equation}
	\Perm{\sigma}
	=\Id\comp\Perm{\sigma}
	=\qty(\Perm{\tau}\comp\Perm{\tau}^{-1}) \comp\Perm{\sigma}
	=\Perm{\tau}\comp \qty(\Perm{\tau}^{-1}\comp\Perm{\sigma})
	\comma
\end{equation}
这就是右边集合的定义,因此 $\text{左边}\subset\text{右边}$。
故可证得等式成立。

对于式~\eqref{eq:置换的穷尽_复合2},我们有
\begin{equation}
	\Perm{\sigma}
	=\Perm{\sigma}\comp\Id
	=\Perm{\sigma}\comp \qty(\Perm{\tau}^{-1}\comp\Perm{\tau})
	=\qty(\Perm{\sigma}\comp\Perm{\tau}^{-1}) \comp\Perm{\tau}
	\comma
\end{equation}
它符合了右边集合的定义,因此 $\text{左边}\subset\text{右边}$。
于是等式成立。

对于式~\eqref{eq:置换的穷尽_逆},我们有
\begin{equation}
	\Perm{\sigma}=\qty(\Perm{\sigma}^{-1})^{-1} \comma
\end{equation}
它符合了右边集合的定义,因此 $\text{左边}\subset\text{右边}$。
于是等式成立。
\end{myProof}

\subsection{数组元素的乘积} \label{subsec:数组元素的乘积}
设有序数组 $\qty{i_1,\,i_2,\,\cdots,\,i_r}$、
$\qty{j_1,\,j_2,\,\cdots,\,j_r}$ 和 $\qty{k_1,\,k_2,\,\cdots,\,k_r}$
经 $r$ 阶置换 $\Perm{\sigma}$ 作用后分别成为
$\qty{\Perm{\sigma}(i_1),\,\Perm{\sigma}(i_2),\,
	\cdots,\,\Perm{\sigma}(i_r)}$、
$\qty{\Perm{\sigma}(j_1),\,\Perm{\sigma}(j_2),\,
	\cdots,\,\Perm{\sigma}(j_r)}$ 和
$\qty{\Perm{\sigma}(k_1),\,\Perm{\sigma}(k_2),\,
	\cdots,\,\Perm{\sigma}(k_r)}$,也就是说
\begin{equation}
	\Perm{\sigma}=\mqty*(
		i_1 & i_2 & \cdots & i_r \\
		\Perm{\sigma}(i_1) & \Perm{\sigma}(i_2) &
			\cdots & \Perm{\sigma}(i_r) )
	=\mqty*(
		j_1 & j_2 & \cdots & j_r \\
		\Perm{\sigma}(j_1) & \Perm{\sigma}(j_2) &
			\cdots & \Perm{\sigma}(j_r) )
	=\mqty*(
		k_1 & k_2 & \cdots & k_r \\
		\Perm{\sigma}(k_1) & \Perm{\sigma}(k_2) &
			\cdots & \Perm{\sigma}(k_r) ) \fullstop
\end{equation}
我们有如下结论:
\begin{equation}
	\forall\,\Perm{\sigma}\in\Permutations{r}\, ,\quad
	A_{i_1 j_1 k_1} A_{i_2 j_2 k_2} \cdots
		A_{i_r j_r k_r}
	=A_{\Perm{\sigma}(i_1)\,\Perm{\sigma}(j_1)\,\Perm{\sigma}(k_1)}
		A_{\Perm{\sigma}(i_2)\,\Perm{\sigma}(j_2)\,\Perm{\sigma}(k_2)}
		\cdots
		A_{\Perm{\sigma}(i_r)\,\Perm{\sigma}(j_r)\,\Perm{\sigma}(k_r)}
		\comma
\end{equation}
式中的 $A_{ijk}$ 表示三维数组 $\Mat{A}$ 的一个元素,其指标为 $ijk$。

下面通过一个例子来说明这一条性质。还是用式~\eqref{eq:置换元素表示_数字} 和
\eqref{eq:置换元素表示_符号} 所定义的置换 $\Perm{\sigma}$:
\begin{equation}
	\Perm{\sigma}=\mqty*(
		4 & 9 & 2 & 7 & 5 & 8 & 3 \\
		3 & 7 & 5 & 4 & 8 & 9 & 2 )
	=\mqty*(
		\spadesuit & \heartsuit & \diamondsuit & \clubsuit &
			\varspadesuit & \varheartsuit & \vardiamondsuit \\
		\vardiamondsuit & \clubsuit & \varspadesuit & \spadesuit &
			\varheartsuit & \heartsuit & \diamondsuit ) \fullstop
			\label{eq:置换元素表示_数组元素的乘积举例}
\end{equation}
随意写出一个数组元素乘积:
\begin{equation}
	A_{379}A_{264}A_{157}A_{483}A_{698}
	A_{\diamondsuit\clubsuit\heartsuit}
	A_{\vardiamondsuit\varspadesuit\varheartsuit} \fullstop
	\label{eq:数组元素乘积举例}
\end{equation}
三组下标分别为
\begin{equation}
	\left\{\begin{lgathered}
		3,\,2,\,1,\,4,\,6,\,\diamondsuit,\,\vardiamondsuit;\\
		7,\,6,\,5,\,8,\,9,\,\clubsuit,\,\varspadesuit;\\
		9,\,4,\,7,\,3,\,8,\,\heartsuit,\,\varheartsuit.\\
	\end{lgathered}
	\right.
\end{equation}
考虑 $\Perm{\sigma}$ 的\emphB{序号定义}式~\eqref{eq:置换序号定义}:
\begin{equation}
	\Perm{\sigma}=\mqty[
		1 & 2 & 3 & 4 & 5 & 6 & 7 \\
		7 & 4 & 5 & 1 & 6 & 2 & 3
	] \fullstop
\end{equation}
所谓序号只是位置的抽象表示,而不代表任何真实的元素。
请记住:置换始终是\emphB{位置}的变换,而非\emphB{元素}的变换,
不要被式~\eqref{eq:置换元素表示_数组元素的乘积举例} 给迷惑了。
把 $\Perm{\sigma}$ 作用在这三组下标上,可得
\begin{equation}
	\left\{\begin{lgathered}
		\vardiamondsuit,\,4,\,6,\,3,\,\diamondsuit,\,2,\,1;\\
		\varspadesuit,\,8,\,9,\,7,\,\clubsuit,\,6,\,5;\\
		\varheartsuit,\,3,\,8,\,9,\,\heartsuit,\,4,\,7.\\
	\end{lgathered}
	\right.
\end{equation}
于是之前的数组元素乘积就变成了
\begin{equation}
	A_{\vardiamondsuit\varspadesuit\varheartsuit}
	A_{483}A_{698}A_{379}
	A_{\diamondsuit\clubsuit\heartsuit}
	A_{264}A_{157} \fullstop
\end{equation}
比对一下各元素,可见与式~\eqref{eq:数组元素乘积举例} 的确是完全一样的。

\subsection{哑标的穷尽} \label{subsec:哑标的穷尽}
考虑如下集合:
\begin{equation}
	\set[\bigg]
	{\qty(i_1,\,i_2,\,\cdots,\,i_r)}
	{\qty{i_1,\,i_2,\,\cdots,\,i_r}
		\text{\ 可取\ } 1,\,2,\,\cdots,\,m}
	\fullstop
\end{equation}
每个 $i_k$ 都有 $m$ 种取法,而 $i_k$ 又有 $r$ 个,
因此该集合一共有 $m^r$ 元素。我们有
\begin{mySubEq}
	\begin{align}
		\forall\,\Perm{\sigma}\in\Permutations{r}\, ,
		&\alspace\set[\bigg]
		{\qty(i_1,\,i_2,\,\cdots,\,i_r)}
		{\qty{i_1,\,i_2,\,\cdots,\,i_r}
			\text{\ 可取\ } 1,\,2,\,\cdots,\,m} \notag \\
		&=\set[\bigg]
		{\qty(\Perm{\sigma}(i_1),\,\Perm{\sigma}(i_2),\,
			\cdots,\,\Perm{\sigma}(i_r) )}
		{\qty{i_1,\,i_2,\,\cdots,\,i_r}
			\text{\ 可取\ } 1,\,2,\,\cdots,\,m}
		\label{eq:哑标的穷尽_置换} \\
		&=\set[\bigg]
		{\qty(\Perm{\sigma}^{-1}(i_1),\,\Perm{\sigma}^{-1}(i_2),\,
			\cdots,\,\Perm{\sigma}^{-1}(i_r) )}
		{\qty{i_1,\,i_2,\,\cdots,\,i_r}
			\text{\ 可取\ } 1,\,2,\,\cdots,\,m} \fullstop
		\label{eq:哑标的穷尽_逆置换}
	\end{align}
\end{mySubEq}
这里,$i_k$ 起的就是\emphB{哑标}的作用。

\begin{myProof}
无论怎样置换,$\Perm{\sigma}(i_k)$ 都是 $1,\,2,\,\cdots,\,m$ 中的数。
因此,对于 $\forall\,\Perm{\sigma}\in\Permutations{r}$,
\begin{equation}
	\qty(\Perm{\sigma}(i_1),\,\Perm{\sigma}(i_2),\,
		\cdots,\,\Perm{\sigma}(i_r) )
	\in\set[\bigg]
		{\qty(i_1,\,i_2,\,\cdots,\,i_r)}
		{\qty{i_1,\,i_2,\,\cdots,\,i_r}
			\text{\ 可取\ } 1,\,2,\,\cdots,\,m} \comma
\end{equation}
即
\begin{align}
	&\mathrel{\phantom{\subset}}\set[\bigg]
	{\qty(\Perm{\sigma}(i_1),\,\Perm{\sigma}(i_2),\,
		\cdots,\,\Perm{\sigma}(i_r) )}
	{\qty{i_1,\,i_2,\,\cdots,\,i_r}
		\text{\ 可取\ } 1,\,2,\,\cdots,\,m} \notag \\
	&\subset\set[\bigg]
	{\qty(i_1,\,i_2,\,\cdots,\,i_r)}
	{\qty{i_1,\,i_2,\,\cdots,\,i_r}
		\text{\ 可取\ } 1,\,2,\,\cdots,\,m} \fullstop
\end{align}
另一方面,由于 $\Id=\Perm{\sigma}^{-1}\comp\Perm{\sigma}$,即
\begin{equation}
	\qty(i_1,\,i_2,\,\cdots,\,i_r)
	=\qty(\Perm{\sigma}^{-1}\comp\Perm{\sigma}(i_1),\,
		\Perm{\sigma}^{-1}\comp\Perm{\sigma}(i_2),\,
		\cdots,\,\Perm{\sigma}^{-1}\comp\Perm{\sigma}(i_r) ) \comma
\end{equation}
而进行一次逆置换仍然使得元素不离开原有的范围,也就是说
\begin{equation}
	\qty(i_1,\,i_2,\,\cdots,\,i_r)
	\in\set[\bigg]
		{\qty(\Perm{\sigma}(i_1),\,\Perm{\sigma}(i_2),\,
			\cdots,\,\Perm{\sigma}(i_r) )}
		{\qty{i_1,\,i_2,\,\cdots,\,i_r}
			\text{\ 可取\ } 1,\,2,\,\cdots,\,m} \comma
\end{equation}
即
\begin{align}
	&\mathrel{\phantom{\subset}}\set[\bigg]
	{\qty(i_1,\,i_2,\,\cdots,\,i_r)}
	{\qty{i_1,\,i_2,\,\cdots,\,i_r}
		\text{\ 可取\ } 1,\,2,\,\cdots,\,m} \notag \\
	&\subset\set[\bigg]
	{\qty(\Perm{\sigma}(i_1),\,\Perm{\sigma}(i_2),\,
		\cdots,\,\Perm{\sigma}(i_r) )}
	{\qty{i_1,\,i_2,\,\cdots,\,i_r}
		\text{\ 可取\ } 1,\,2,\,\cdots,\,m} \fullstop
\end{align}
两个集合互相包含,也就证得了式~\eqref{eq:哑标的穷尽_置换}。

用相同的方法也可证得关于逆置换的 \eqref{eq:哑标的穷尽_逆置换}~式,
此处从略。
\end{myProof}

\section{置换(三)}
本节将给出置换运算在线性代数中的一些应用。

\subsection{行列式}

\section{置换(四)}
本节将重回张量运算的主线,引入\emphA{置换算子}。

\subsection{置换算子;对称张量与反对称张量}
对于任意的置换 $\Perm{\sigma}\in\Permutations{r}$,定义\emphA{置换算子}
\begin{equation}
	\mmap{\opPerm}
		{\Tensors{r}\ni\T{\Phi}}
		{\opPerm(\T{\Phi})\in\Tensors{r}} \comma
\end{equation}
式中
\begin{equation}
	\opPerm(\T{\Phi})\qty(\V{u}_1,\,\V{u}_2,\,\cdots,\,\V{u}_r)
	\defeq\T{\Phi}\qty(\V{u}_{\Perm{\sigma}(1)},\,
		\V{u}_{\Perm{\sigma}(2)},\,\cdots,\,
		\V{u}_{\Perm{\sigma}(r)})
	\in\realR \fullstop
\end{equation}
这里的“$\cdots\in\realR$”是根据张量的定义:\emphB{多重线性函数}。

如果我们的置换
\begin{equation}
	\Perm{\sigma}=\mqty*(
		i_1 & i_2 & \cdots & i_r \\
		\Perm{\sigma}(i_1) & \Perm{\sigma}(i_1) & \cdots
			& \Perm{\sigma}(i_1)
	) \comma
\end{equation}
那么对应的置换算子将满足
\begin{equation}
	\opPerm(\T{\Phi})\qty(\V{u}_{i_1},\,\V{u}_{i_2},\,
		\cdots,\,\V{u}_{i_r})
	\defeq\T{\Phi}\qty(\V{u}_{\Perm{\sigma}(i_1)},\,
		\V{u}_{\Perm{\sigma}(i_2)},\,\cdots,\,
		\V{u}_{\Perm{\sigma}(i_r)}) \fullstop
\end{equation}

根据张量的线性性,容易知道置换算子也具有线性性:
\begin{align}
	\forall\,\T{\Phi},\,\T{\Psi}\in\Tensors{r}
		\text{\ 以及\ } \alpha,\,\beta\in\realR,\quad
	\opPerm(\alpha\,\T{\Phi}+\beta\,\T{\Psi})
	=\alpha\,\opPerm(\T{\Phi})
		+\beta\,\alpha\opPerm(\T{\Psi}) \fullstop
	\label{eq:置换算子的线性性}
\end{align}

\begin{myProof}
\begin{align}
	\opPerm(\alpha\,\T{\Phi}+\beta\,\T{\Psi})
		\qty(\V{u}_1,\,\cdots,\,\V{u}_r)
	&=(\alpha\,\T{\Phi}+\beta\,\T{\Psi})
		\qty(\V{u}_{\Perm{\sigma}(1)},\,\cdots,\,
			\V{u}_{\Perm{\sigma}(r)}) \notag \\
	&=\alpha\,\T{\Phi}\qty(\V{u}_{\Perm{\sigma}(1)},\,\cdots,\,
			\V{u}_{\Perm{\sigma}(r)})
		+\beta\,\T{\Psi}\qty(\V{u}_{\Perm{\sigma}(1)},\,\cdots,\,
			\V{u}_{\Perm{\sigma}(r)}) \notag \\
	&=\alpha\opPerm(\T{\Phi})
			\qty(\V{u}_1,\,\cdots,\,\V{u}_r)
		+\beta\opPerm(\T{\Psi})
			\qty(\V{u}_1,\,\cdots,\,\V{u}_r) \notag \\
	&=\qty[\alpha\opPerm(\T{\Phi})
			+\beta\opPerm(\T{\Psi})]
		\qty(\V{u}_1,\,\cdots,\,\V{u}_r) \fullstop
\end{align}
\end{myProof}
两个置换算子复合的结果也是很显然的:
\begin{equation}
	\forall\,\Perm{\sigma},\,\Perm{\tau}\in\Permutations{r},\quad
	\opPerm\comp\opPerm[\Perm{\tau}]
	=\opPerm[\Perm{\sigma}\comp\Perm{\tau}] \fullstop
	\label{eq:置换算子的复合}
\end{equation}

\begin{myProof}
\begin{align}
	\opPerm\comp\opPerm[\Perm{\tau}](\T{\Phi})
		\qty(\V{u}_{i_1},\,\cdots,\,\V{u}_{i_r})
	&=\opPerm(\T{\Phi})
			\qty(\V{u}_{\Perm{\tau}(1)},\,
			\cdots,\,\V{u}_{\Perm{\tau}(r)}) \notag \\
	&=\T{\Phi}\qty(\V{u}_{\Perm{\sigma}\comp\Perm{\tau}(1)},\,
		\cdots,\,\V{u}_{\Perm{\sigma}\comp\Perm{\tau}(r)}) \notag \\
	&=\opPerm[\Perm{\sigma}\comp\Perm{\tau}](\T{\Phi})
		\qty(\V{u}_{i_1},\,\cdots,\,\V{u}_{i_r}) \fullstop
\end{align}
\end{myProof}

\blankline

有了置换算子,我们就可以来定义\emphA{对称张量}和\emphA{反对称张量}。
对称张量的全体记为 $\Sym$,反对称张量的全体记为 $\Skw$。
如果以 $\Rm$ 为底空间,
又分别可以记为 $\SymTensors{r}$ 和 $\SkwTensors{r}$。

对于任意的 $\T{\Phi}\in\Tensors{r}$,如果
\begin{equation}
	\opPerm(\T{\Phi})=\T{\Phi} \comma
	\label{eq:对称张量的定义}
\end{equation}
则称 $\T{\Phi}$ 为\emphB{对称张量},
即 $\T{\Phi}\in\Sym \text{\ 或\ } \SymTensors{r}$;如果
\begin{equation}
	\opPerm(\T{\Phi})=\sgn\Perm{\sigma}\cdotp\T{\Phi} \comma
	\label{eq:反对称张量的定义}
\end{equation}
则称 $\T{\Phi}$ 为\emphB{反对称张量},
即 $\T{\Phi}\in\Skw \text{\ 或\ } \SkwTensors{r}$。

有些书中采用分量形式来定义(反)对称张量。这与此处的定义是等价的:
\begin{mySubEq}
	\begin{align}
		\opPerm(\T{\Phi})=\T{\Phi} &\iff
			\Phi_{\Perm{\sigma}(i_1)\cdots\Perm{\sigma}(i_p)}
				=\Phi_{i_1 \cdots i_p} \comma\\
		\opPerm(\T{\Phi})=\sgn\Perm{\sigma}\cdotp\T{\Phi} &\iff
			\Phi_{\Perm{\sigma}(i_1)\cdots\Perm{\sigma}(i_p)}
				=\sgn\Perm{\sigma}\cdotp\Phi_{i_1 \cdots i_p}
				\fullstop
	\end{align}
\end{mySubEq}
反对称张量与我们熟知的\emphB{行列式}有些类似:
交换两列(对于张量就是两个分量),符号相反。
全部分量两两交换一遍,前面的系数自然是置换的符号。
而如果无论怎么交换分量(当然需要全部两两交换一遍),符号都不变,
那这样的张量就是对称张量。

\myPROBLEM{一个二阶张量的协变(或逆变)分量,可以用一个矩阵表示。}如果这个张量是一个反对称张量,交换任意两个分量要添加负号;对于矩阵而言,这就意味着交换两行(或两列)……

\subsection{置换算子的表示}
根据上文给出的定义,我们有
\begin{equation}
	\opPerm(\T{\Phi})\qty(\V{u}_{i_1},\,\cdots,\,\V{u}_{i_r})
	\defeq\T{\Phi}\qty(\V{u}_{\Perm{\sigma}(i_1)},\,\cdots,\,
		\V{u}_{\Perm{\sigma}(i_r)}) \fullstop
\end{equation}
首先回忆一下 \ref{subsec:张量的表示与简单张量}~小节中张量的表示:
选一组基(协变、逆变均可),然后把张量用这组基表示。于是
\begin{align}
	&\alspace\opPerm(\T{\Phi})
		\qty(\V{u}_{i_1},\,\cdots,\,\V{u}_{i_r})
	=\T{\Phi}\qty(\V{u}_{\Perm{\sigma}(i_1)},\,\cdots,\,
		\V{u}_{\Perm{\sigma}(i_r)}) \notag
	\intertext{把向量用协变基表示:}
	&=\T{\Phi}\qty(u^{i_1}_{\Perm{\sigma}(i_1)}\,\V{g}_{i_1},\,
		\cdots,\,u^{i_r}_{\Perm{\sigma}(i_r)}\,\V{g}_{i_r}) \notag
	\intertext{根据张量的线性性,提出系数:}
	&=\T{\Phi}\qty(\V{g}_{i_1},\,\cdots,\,\V{g}_{i_r}) \cdotp
		\qty(u^{i_1}_{\Perm{\sigma}(i_1)} \cdots
			u^{i_r}_{\Perm{\sigma}(i_r)}) \notag
	\intertext{前半部分可以用张量分量表示;
		而后半部分是一组逆变分量,可以写成内积的形式}
	&=\Phi_{i_1 \cdots i_p}
		\qty[\ipb{\V{u}_{\Perm{\sigma}(i_1)}}{\V{g}^{i_1}}\cdots
			\ipb{\V{u}_{\Perm{\sigma}(i_r)}}{\V{g}^{i_r}}]
	\addtocounter{equation}{1}
	\tag{\theequation*}
	\label{eq:置换算子的表示_中间步骤}
	\intertext{注意到方括号中的其实是简单张量的定义,这就有}
	&=\Phi_{i_1 \cdots i_p}
		\V{g}^{i_1}\tp\cdots\tp\V{g}^{i_r}
		\qty(\V{u}_{\Perm{\sigma}(i_1)},\,\cdots,\,
			\V{u}_{\Perm{\sigma}(i_r)}) \fullstop
	\addtocounter{equation}{-1}
\end{align}
最后一步仍然没能回到 $\qty(\V{u}_{i_1},\,\cdots,\,\V{u}_{i_r})$,
因此以上推导只是简单地展开了 $\T{\Phi}$,并没有获得实质性的结果。

然而,只要稍作改动,情况就会大不相同。
考虑一下 \ref{subsec:数组元素的乘积}~小节中置换运算%
有关\emphB{数组元素乘积}的性质:
\begin{equation}
	\forall\,\Perm{\tau}\in\Permutations{r},\quad
	A_{i_1 j_1} \cdots A_{i_r j_r}
	=A_{\Perm{\tau}(i_1) \Perm{\tau}(j_1)} \cdots
		A_{\Perm{\tau}(i_r) \Perm{\tau}(j_r)} \comma
	\label{eq:置换算子的表示_推导_置换性质}
\end{equation}
式中
\begin{equation}
	\Perm{\tau}
	=\mqty*(
		i_1 & \cdots & i_r \\
		\Perm{\tau}(i_1) & \cdots & \Perm{\tau}(i_r) )
	=\mqty*(
		j_1 & \cdots & j_r \\
		\Perm{\tau}(j_1) & \cdots & \Perm{\tau}(j_r) ) \fullstop
\end{equation}
由此可以看出,式~\eqref{eq:置换算子的表示_中间步骤} 方括号中的部分
其实是由 $\Perm{\sigma}(i_k)$ 和 $i_k$ 两套指标确定的一组数:
\begin{equation}
	A_{\Perm{\sigma}(i_k)\,i_k}
	=\ipb{\V{u}_{\Perm{\sigma}(i_k)}}{\V{g}^{i_k}} \semicomma
\end{equation}
另一方面,显然有 $\Perm{\sigma}^{-1}\in\Permutations{r}$。于是
\begin{align}
	&\alspace\opPerm(\T{\Phi})
		\qty(\V{u}_{i_1},\,\cdots,\,\V{u}_{i_r}) \notag \\
	&=\Phi_{i_1 \cdots i_r}
		\qty[\ipb{\V{u}_{\Perm{\sigma}(i_1)}}{\V{g}^{i_1}}\cdots
			\ipb{\V{u}_{\Perm{\sigma}(i_r)}}{\V{g}^{i_r}}] \notag
	\intertext{应用置换的性质 \eqref{eq:置换算子的表示_推导_置换性质}~式:}
	&=\Phi_{i_1 \cdots i_r}
		\qty[
			\ipb{\V{u}_{\Perm{\sigma}^{-1}\comp\Perm{\sigma}(i_1)}}
				{\V{g}^{\Perm{\sigma}^{-1}(i_1)}} \cdots
			\ipb{\V{u}_{\Perm{\sigma}^{-1}\comp\Perm{\sigma}(i_r)}}
				{\V{g}^{\Perm{\sigma}^{-1}(i_r)}}
		] \notag \\
	&=\Phi_{i_1 \cdots i_r}
		\qty[
			\ipb{\V{u}_{i_1}}{\V{g}^{\Perm{\sigma}^{-1}(i_1)}} \cdots
			\ipb{\V{u}_{i_2}}{\V{g}^{\Perm{\sigma}^{-1}(i_r)}}
		] \notag
	\intertext{同样,用简单张量表示,可得}
	&=\Phi_{i_1 \cdots i_r}
		\V{g}^{\Perm{\sigma}^{-1}(i_1)}\tp\cdots
			\tp\V{g}^{\Perm{\sigma}^{-1}(i_r)}
		\qty(\V{u}_{i_1},\,\cdots,\,\V{u}_{i_r}) \fullstop
\end{align}
这样,我们就得到了置换算子的一种表示:
\begin{align}
	\opPerm(\T{\Phi})
	&=\opPerm\qty(\Phi_{i_1 \cdots i_r}
		\V{g}^{i_1}\tp\cdots\tp\V{g}^{i_r}) \notag \\
	&=\Phi_{i_1 \cdots i_r}
		\V{g}^{\Perm{\sigma}^{-1}(i_1)}\tp\cdots
			\tp\V{g}^{\Perm{\sigma}^{-1}(i_r)} \fullstop
	\label{eq:置换算子的表示_逆置换在简单张量上}
\end{align}

在式~\eqref{eq:置换算子的表示_逆置换在简单张量上} 中,
$i_1,\,\cdots,\,i_r$ 都是哑标,要被求和求掉。
张量 $\T{\Phi}$ 的底空间是 $\Rm$,所以每个 $i_k$ 都有 $m$ 个取值。
考虑一下 \ref{subsec:哑标的穷尽}~小节中置换运算%
有关\emphB{哑标穷尽}的性质,有
\begin{align}
	\forall\,\Perm{\sigma}\in\Permutations{r}\, ,
	&\alspace\set[\bigg]
	{\qty(i_1,\,i_2,\,\cdots,\,i_r)}
	{\qty{i_1,\,i_2,\,\cdots,\,i_r}
		\text{\ 可取\ } 1,\,2,\,\cdots,\,m} \notag \\
	&=\set[\bigg]
	{\qty(\Perm{\sigma}(i_1),\,\Perm{\sigma}(i_2),\,
		\cdots,\,\Perm{\sigma}(i_r) )}
	{\qty{i_1,\,i_2,\,\cdots,\,i_r}
		\text{\ 可取\ } 1,\,2,\,\cdots,\,m} \fullstop
\end{align}
因此,我们可以把式~\eqref{eq:置换算子的表示_逆置换在简单张量上} 中的指标
$i_k$ 换成 $\Perm{\sigma}(i_k)$:
\begin{align}
	\opPerm(\T{\Phi})
	&=\Phi_{i_1 \cdots i_r}
		\V{g}^{\Perm{\sigma}^{-1}(i_1)}\tp\cdots
			\tp\V{g}^{\Perm{\sigma}^{-1}(i_r)} \notag \\
	&=\Phi_{\Perm{\sigma}(i_1) \cdots \Perm{\sigma}(i_r)}
		\V{g}^{\Perm{\sigma}^{-1}\comp\Perm{\sigma}(i_1)}\tp\cdots
			\tp\V{g}^{\Perm{\sigma}^{-1}\comp
				\Perm{\sigma}(i_r)} \notag \\
	&=\Phi_{\Perm{\sigma}(i_1) \cdots \Perm{\sigma}(i_r)}
		\V{g}^{i_1}\tp\cdots\tp\V{g}^{i_r} \fullstop
\end{align}
这是置换算子的另一种表示。

综上,要获得置换算子的表示,
若是对\emphB{张量分量}进行操作,就直接使用对分量指标使用置换;
若是对\emphB{简单张量}进行操作,则要对其指标使用逆置换:\footnote{%
	这里稍有改动,用了张量的逆变分量,不过实质都是一样的。%
	使用协变分量还是逆变分量,这个嘛,悉听尊便。}
\begin{mySubEq}
	\begin{align}
		\opPerm(\T{\Phi})
		&=\opPerm\qty(\Phi^{i_1 \cdots i_r}
			\V{g}_{i_1}\tp\cdots\tp\V{g}_{i_r}) \notag \\
		&=\Phi^{\Perm{\sigma}(i_1)\cdots\Perm{\sigma}(i_r)}
			\V{g}_{i_1}\tp\cdots\tp\V{g}_{i_r}
		\label{eq:置换算子的表示_总结_对张量分量} \\
		&=\Phi^{i_1 \cdots i_r}
			\V{g}_{\Perm{\sigma}^{-1}(i_1)}\tp\cdots
				\tp\V{g}_{\Perm{\sigma}^{-1}(i_r)} \fullstop
		\label{eq:置换算子的表示_总结_对简单张量} 
	\end{align}
\end{mySubEq}

\section{对称化算子与反对称化算子}
\subsection{定义}
\emphA{对称化算子 $\opSym$} 和\emphA{反对称化算子 $\opSkw$} 的定义分别为
\begin{align}
	\opSym(\T{\Phi})
		&\defeq \frac{1}{r!} \sum_{\Perm{\sigma}\in\Permutations{r}}
			\opPerm(\T{\Phi})
	\intertext{和}
	\opSkw(\T{\Phi})
		&\defeq \frac{1}{r!} \sum_{\Perm{\sigma}\in\Permutations{r}}
			\sgn\Perm{\sigma}\cdotp\opPerm(\T{\Phi}) \comma
\end{align}
式中,$\T{\Phi}\in\Tensors{r}$。
根据置换算子的线性性,很容易知道对称化算子与反对称化算子也具有线性性。

对于任意的 $\T{\Phi}\in\Tensors{r}$,我们有
\begin{braceEq}
	\opSym(\T{\Phi}) &\in \Sym \comma \\
	\opSkw(\T{\Phi}) &\in \Skw \fullstop
\end{braceEq}
这说明任意一个张量,对它作用对称化算子之后,将变为对称张量;
反之,作用反对称算子之后,将变为反对称张量。\footnote{%
	换一个角度,(反)对称张量实际上可以用(反)对称化算子来定义。}

\begin{myProof}
要判断 $\opSym(\T{\Phi})$ 是不是对称张量,
首先需要在其上作用一个置换算子 $\opPerm[\Perm{\tau}]$:
\begin{align}
	\opPerm[\Perm{\tau}]\qty\big[\opSym(\T{\Phi})]
	&=\opPerm[\Perm{\tau}]\qty[\frac{1}{r!}
			\sum_{\Perm{\sigma}\in\Permutations{r}}
			\opPerm(\T{\Phi})] \notag
	\intertext{根据置换算子的线性性 \eqref{eq:置换算子的线性性}~式,可有}
	&=\frac{1}{r!} \sum_{\Perm{\sigma}\in\Permutations{r}}
		\opPerm[\Perm{\tau}]\comp\opPerm(\T{\Phi}) \notag
	\intertext{再用一下 \eqref{eq:置换算子的复合}~式,得到}
	&=\frac{1}{r!} \sum_{\Perm{\sigma}\in\Permutations{r}}
		\opPerm[\Perm{\tau}\comp\Perm{\sigma}](\T{\Phi}) \notag
	\intertext{这里求和的作用就是把置换 $\Perm{\sigma}$ 穷尽了。
		根据 \ref{subsec:置换的穷尽}~小节中的内容,
		再在 $\Perm{\sigma}$ 上复合一个置换 $\Perm{\tau}$,
		结果将保持不变:}
	&=\frac{1}{r!} \sum_{\Perm{\sigma}\in\Permutations{r}}
		\opPerm(\T{\Phi})
	=\opSym(\T{\Phi}) \fullstop
\end{align}
对照一下对称张量的定义 \eqref{eq:对称张量的定义}~式,可见的确有
$\opSym(\T{\Phi})\in\Sym$。

类似地,
\begin{align}
	\opPerm[\Perm{\tau}]\qty\big[\opSkw(\T{\Phi})]
	&=\opPerm[\Perm{\tau}]\qty[\frac{1}{r!}
			\sum_{\Perm{\sigma}\in\Permutations{r}}
			\sgn\Perm{\sigma}\cdotp\opPerm(\T{\Phi})] \notag \\
	&=\frac{1}{r!} \sum_{\Perm{\sigma}\in\Permutations{r}}
		\sgn\Perm{\sigma}\cdotp
		\qty\bigg[\opPerm[\Perm{\tau}]\comp\opPerm(\T{\Phi})]
		\notag \\
	&=\frac{1}{r!} \sum_{\Perm{\sigma}\in\Permutations{r}}
		\sgn\Perm{\sigma}\cdotp
		\opPerm[\Perm{\tau}\comp\Perm{\sigma}](\T{\Phi}) \notag
	\intertext{根据式~\eqref{eq:置换复合的符号},
		$\sgn\Perm{\tau}\cdotp\sgn\Perm{\sigma}
		=\sgn(\Perm{\tau}\comp\Perm{\sigma})$,于是}
	&=\frac{1}{r!} \sum_{\Perm{\sigma}\in\Permutations{r}}
		\frac{\sgn(\Perm{\tau}\comp\Perm{\sigma})}{\sgn\Perm{\tau}}
		\cdotp\opPerm[\Perm{\tau}\comp\Perm{\sigma}](\T{\Phi}) \notag
	\intertext{注意到始终成立
		$\sgn\Perm{\tau}\cdotp\sgn\Perm{\tau}=1$
		(因为 $\sgn\Perm{\tau}=\pm 1$),又有}
	&=\frac{\sgn\Perm{\tau}}{r!}
		\sum_{\Perm{\sigma}\in\Permutations{r}}
		\sgn(\Perm{\tau}\comp\Perm{\sigma})
		\cdotp\opPerm[\Perm{\tau}\comp\Perm{\sigma}](\T{\Phi}) \notag
	\intertext{利用置换的穷尽,
		$\Perm{\tau}\comp\Perm{\sigma}$ 与 $\Perm{\sigma}$ 相比,
		结果将保持不变:}
	&=\sgn\tau\cdotp\qty[\frac{1}{r!}
		\sum_{\Perm{\sigma}\in\Permutations{r}}
		\sgn\Perm{\sigma}\cdotp\opPerm(\T{\Phi})]
	=\sgn\tau\cdotp\opSkw(\T{\Phi}) \fullstop
\end{align}
与反对称张量的定义 \eqref{eq:反对称张量的定义}~式相比,可见的确有
$\opSym(\T{\Phi})\in\Skw$。

这里的操作直接对张量本身进行,没有采用涉及到张量“自变量”(向量)的繁琐计算,
因而显得更加干净利落。
\end{myProof}

\subsection{反对称化算子的性质}
上文已经定义了\emphA{反对称化算子 $\opSkw$}:
\begin{equation}
	\forall\,\T{\Phi}\in\Tensors{r},\quad
	\opSkw(\T{\Phi})\defeq
	\frac{1}{r!} \sum_{\Perm{\sigma}\in\Permutations{r}}
	\sgn\Perm{\sigma}\cdotp\opPerm(\T{\Phi})
	\in\Skw \text{\ 或\ } \SkwTensors{r} \fullstop
\end{equation}
即任意一个 $r$ 阶张量,作用反对称化算子后就变成了 $r$ 阶反对称张量。
$r$ 阶反对称张量也称为 \emphA{$r$-form} (\emphA{$r$-形式})。

下面列出反对称化算子的几条性质。

\begin{myEnum}
\item 反对称化算子若重复作用,仅相当于一次作用:
\begin{equation}
	\opSkw^2 \coloneq \opSkw\comp\opSkw = \opSkw \fullstop
\end{equation}
根据数学归纳法,显然有
\begin{equation}
	\forall\,p\in\natN,\quad
	\opSkw^p \coloneq \underbrace{\opSkw\comp\cdots\comp\opSkw}_
		{\text{$p$ 个 $\opSkw$}} = \opSkw \fullstop
\end{equation}

\begin{myProof}
\begin{align}
	\opSkw^2 &= \opSkw\qty\big[\opSkw(\T{\Phi})] \notag \\
	&\defeq \opSkw\qty[\frac{1}{r!}
		\sum_{\Perm{\sigma}\in\Permutations{r}}
		\sgn\Perm{\sigma}\cdotp\opPerm(\T{\Phi})] \notag \\
	&\defeq \frac{1}{r!}
		\sum_{\Perm{\tau}\in\Permutations{r}}
		\sgn\Perm{\tau}\cdotp\opPerm[\Perm{\tau}]\qty[\frac{1}{r!}
			\sum_{\Perm{\sigma}\in\Permutations{r}}
			\sgn\Perm{\sigma}\cdotp\opPerm(\T{\Phi})] \notag
	\intertext{根据线性性,可有}
	&=\frac{1}{(r!)^2} \sum_{\Perm{\tau}\in\Permutations{r}}
		\sum_{\Perm{\sigma}\in\Permutations{r}}
		\sgn\Perm{\tau}\sgn\Perm{\sigma}
		\cdotp\opPerm[\Perm{\tau}]\comp\opPerm(\T{\Phi}) \notag
	\intertext{根据式~\eqref{eq:置换复合的符号}
		和式~\eqref{eq:置换算子的复合},有}
	&=\frac{1}{(r!)^2} \sum_{\Perm{\tau}\in\Permutations{r}}
		\sum_{\Perm{\sigma}\in\Permutations{r}}
		\sgn\qty(\Perm{\tau}\comp\Perm{\sigma}) \cdotp
		\opPerm[\Perm{\tau}\comp\Perm{\sigma}](\T{\Phi}) \notag \\
	&=\frac{1}{(r!)^2} \sum_{\Perm{\tau}\in\Permutations{r}}
		\qty[\sum_{\Perm{\sigma}\in\Permutations{r}}
			\sgn\qty(\Perm{\tau}\comp\Perm{\sigma}) \cdotp
			\opPerm[\Perm{\tau}\comp\Perm{\sigma}](\T{\Phi})] \notag
	\intertext{注意到方括号中的部分穷尽了置换 $\Perm{\sigma}$,
		因此可以用 $\Perm{\sigma}$ 取代“指标”
		$\Perm{\tau}\comp\Perm{\sigma}$:}
	&=\frac{1}{(r!)^2} \sum_{\Perm{\tau}\in\Permutations{r}}
		\qty[\sum_{\Perm{\sigma}\in\Permutations{r}}
			\sgn\Perm{\sigma}\cdotp\opPerm(\T{\Phi})] \notag
	\intertext{回到定义,有}
	&=\frac{1}{r!} \sum_{\Perm{\tau}\in\Permutations{r}}
		\opSkw(\T{\Phi})
	=\frac{1}{r!} \cdotp r! \opSkw(\T{\Phi})
	=\opSkw(\T{\Phi}) \fullstop
\end{align}
\end{myProof}

\blankline

\item 对任意两个张量 $\T{\Phi}\in\Tensors{p}$ 和
$\T{\Psi}\in\Tensors{q}$ 的并施加反对称化算子,可以得到如下结果:
\begin{mySubEq}
	\begin{align}
		\opSkw\qty\big(\T{\Phi}\tp\T{\Psi})
		&=\opSkw\qty\big[\opSkw(\T{\Phi})\tp\T{\Psi}]
		\label{eq:对张量并施加反对称化算子_1} \\
		&=\opSkw\qty\big[\T{\Phi}\tp\opSkw(\T{\Psi})]
		\label{eq:对张量并施加反对称化算子_2} \\
		&=\opSkw\qty\big[\opSkw(\T{\Phi})\tp\opSkw(\T{\Psi})]
		\fullstop
		\label{eq:对张量并施加反对称化算子_3}
	\end{align}
\end{mySubEq}

\begin{myProof}
这里只给出式~\eqref{eq:对张量并施加反对称化算子_2} 的证明。
另外两式的证明是类似的。
\begin{align}
	\opSkw\qty\big[\T{\Phi}\tp\opSkw(\T{\Psi})]
	&=\opSkw\,\qty[\T{\Phi}\tp
		\qty\Bigg(\frac{1}{q!}
			\sum_{\Perm{\tau}\in\Permutations{q}}\sgn\Perm{\tau}
			\cdotp\opPerm[\Perm{\tau}] (\T{\Psi}) ) ] \notag
	\intertext{根据张量积的线性性提出系数:}
	&=\opSkw\,\qty[\frac{1}{q!}
		\sum_{\Perm{\tau}\in\Permutations{q}} \sgn\Perm{\tau}
		\cdotp\T{\Phi}\tp\opPerm[\Perm{\tau}](\T{\Psi})] \notag
	\intertext{利用置换的穷尽,
		可以把 $\Perm{\tau}$ 换作 $\Perm{\tau}^{-1}$:}
	&=\opSkw\,\qty[\frac{1}{q!}
		\sum_{\Perm{\tau}\in\Permutations{q}}
		\sgn\Perm{\tau}^{-1} \cdotp\T{\Phi}\tp
		\opPerm[\Perm{\tau}^{-1}](\T{\Psi})] \notag
	\intertext{注意到
		$\sgn\Perm{\tau}=\sgn\Perm{\tau}^{-1}$,于是}
	&=\opSkw\,\qty[\frac{1}{q!}
		\sum_{\Perm{\tau}\in\Permutations{q}}
		\sgn\Perm{\tau} \cdotp\T{\Phi}\tp
		\opPerm[\Perm{\tau}^{-1}](\T{\Psi})] \comma
	\label{eq:对张量并施加反对称化算子_证明_Part1}
\end{align}
式中,
\begin{equation}
	\opPerm[\Perm{\tau}^{-1}](\T{\Psi})
	=\Psi^{j_1 \cdots j_q} \,
		\V{g}_{\Perm{\tau}(j_1)}\tp\cdots
		\tp\V{g}_{\Perm{\tau}(j_q)} \fullstop
\end{equation}
于是有
\begin{equation}
	\T{\Phi}\tp\opPerm[\Perm{\tau}^{-1}](\T{\Psi})
	=\Phi^{i_1 \cdots i_p}\,\Psi^{j_1 \cdots j_q}\,
		\qty(\V{g}_{i_1}\tp\cdots\tp\V{g}_{i_p}) \tp
		\qty(\V{g}_{\Perm{\tau}(j_1)}\tp\cdots
		\tp\V{g}_{\Perm{\tau}(j_q)}) \fullstop
\end{equation}

置换 $\Perm{\tau}\in\Permutations{q}$ 的元素定义为
\begin{equation}
	\Perm{\tau}=\mqty*(
		j_1 & \cdots & j_q \\
		\Perm{\tau}(j_1) & \cdots & \Perm{\tau}(j_q)
	) \fullstop
\end{equation}
引入它的“延拓”(或曰“增广”)置换
$\Perm{\hat{\tau}}\in\Permutations{p+q}$,其定义为
\begin{equation}
	\Perm{\hat{\tau}}=\mqty*(
		i_1 & \cdots & i_p & j_1 & \cdots & j_q \\
		i_1 & \cdots & i_p &
			\Perm{\tau}(j_1) & \cdots & \Perm{\tau}(j_q)
	) \fullstop
\end{equation}
这样一来,就有
\begin{align}
	\T{\Phi}\tp\opPerm[\Perm{\tau}^{-1}](\T{\Psi})
	&=\Phi^{i_1 \cdots i_p}\,\Psi^{j_1 \cdots j_q}\,
		\qty(\V{g}_{i_1}\tp\cdots\tp\V{g}_{i_p}) \tp
		\qty(\V{g}_{\Perm{\tau}(j_1)}\tp\cdots
		\tp\V{g}_{\Perm{\tau}(j_q)}) \notag \\
	&=\Phi^{i_1 \cdots i_p}\,\Psi^{j_1 \cdots j_q}\,
		\qty(\V{g}_{\Perm{\hat{\tau}}(i_1)}\tp\cdots
		\tp\V{g}_{\Perm{\hat{\tau}}(i_p)}) \tp
		\qty(\V{g}_{\Perm{\hat{\tau}}(j_1)}\tp\cdots
		\tp\V{g}_{\Perm{\hat{\tau}}(j_q)}) \notag \\
	&=\opPerm[\Perm{\hat{\tau}}^{-1}]
		\qty\big(\T{\Phi}\tp\T{\Psi}) \fullstop
\end{align}
另一方面,参与轮换的元素只有后面的 $q$ 个,
因此 $\Perm{\hat{\tau}}$ 起到的作用
实际上等同于 $\Perm{\tau}$(当然两者作用范围不同)。所以可知
\begin{equation}
	\sgn\Perm{\hat{\tau}} = \sgn\Perm{\tau} \fullstop
\end{equation}
把以上这两点代入%
式~\eqref{eq:对张量并施加反对称化算子_证明_Part1} 的推导,有
\begin{align}
	\opSkw\qty\big[\T{\Phi}\tp\opSkw(\T{\Psi})]
	&=\opSkw\,\qty[\frac{1}{q!}
		\sum_{\Perm{\tau}\in\Permutations{q}}
		\sgn\Perm{\tau} \cdotp\T{\Phi}\tp
		\opPerm[\Perm{\tau}^{-1}](\T{\Psi})] \notag \\
	&=\opSkw\,\qty[\frac{1}{q!}
		\sum_{\Perm{\tau}\in\Permutations{q}}
		\sgn\Perm{\hat{\tau}} \cdotp
		\opPerm[\Perm{\hat{\tau}}^{-1}]
		\qty\big(\T{\Phi}\tp\T{\Psi}) ] \notag
	\intertext{再用一次线性性,可得}
	&=\frac{1}{q!} \sum_{\Perm{\tau}\in\Permutations{q}}
		\sgn\Perm{\hat{\tau}} \cdotp
		\opSkw\qty\Big[\opPerm[\Perm{\hat{\tau}}^{-1}]
			\qty\big(\T{\Phi}\tp\T{\Psi})] \fullstop
	\label{eq:对张量并施加反对称化算子_证明_Part2}
\end{align}

$\T{\Phi}\tp\T{\Psi}$ 是一个 $p+q$ 阶张量,
它作用置换算子后阶数当然保持不变。
根据反对称化算子的定义,可有
\begin{align}
	\opSkw\qty\Big[\opPerm[\Perm{\hat{\sigma}}^{-1}]
		\qty\big(\T{\Phi}\tp\T{\Psi})]
	&=\frac{1}{(p+q)!}
		\sum_{\Perm{\hat{\sigma}}\in\Permutations{p+q}}
		\sgn\Perm{\hat{\sigma}} \cdotp
		\qty\Big[\opPerm[\Perm{\hat{\sigma}}]
			\comp \opPerm[\Perm{\hat{\tau}}^{-1}]
			\qty\big(\T{\Phi}\tp\T{\Psi})] \notag \\
	&=\frac{1}{(p+q)!}
		\sum_{\Perm{\hat{\sigma}}\in\Permutations{p+q}}
		\sgn\Perm{\hat{\sigma}} \cdotp
		\qty\Big[\opPerm[\Perm{\hat{\sigma}}
				\comp\Perm{\hat{\tau}}^{-1}]
			\qty\big(\T{\Phi}\tp\T{\Psi})] \comma
	\label{eq:对张量并施加反对称化算子_证明_Part3}
\end{align}
式中的 $\Perm{\hat{\sigma}}$ 和之前定义的
$\Perm{\hat{\tau}}$ 含义相同,只是为了确保哑标不重复,
我们采用了不同的字母来表示。
该式~\eqref{eq:对张量并施加反对称化算子_证明_Part3} 中的
$\sgn\Perm{\hat{\sigma}}$ 可以写成
\begin{equation}
	\sgn\Perm{\hat{\sigma}}
	=\frac{\sgn\qty(
			\Perm{\hat{\sigma}}\comp\Perm{\hat{\tau}}^{-1})}
		{\sgn\Perm{\hat{\tau}}^{-1}}
	=\frac{\sgn\qty(
			\Perm{\hat{\sigma}}\comp\Perm{\hat{\tau}}^{-1})}
		{\sgn\Perm{\hat{\tau}}} \fullstop
	\label{eq:对张量并施加反对称化算子_证明_Part4}
\end{equation}
第二个等号是根据式~\eqref{eq:逆置换的符号}。

注意到 \eqref{eq:对张量并施加反对称化算子_证明_Part2}~式中
也有一个 $\sgn\Perm{\hat{\tau}}$,因此
\begin{align}
	\opSkw\qty\big[\T{\Phi}\tp\opSkw(\T{\Psi})]
	&=\frac{1}{q!} \sum_{\Perm{\tau}\in\Permutations{q}}
		\sgn\Perm{\hat{\tau}} \cdotp
		\opSkw\qty\Big[\opPerm[\Perm{\hat{\tau}}^{-1}]
			\qty\big(\T{\Phi}\tp\T{\Psi})] \notag \\
	&=\frac{1}{q!} \sum_{\Perm{\tau}\in\Permutations{q}}
		\sgn\Perm{\hat{\tau}} \cdotp
		\qty[\frac{1}{(p+q)!}
			\sum_{\Perm{\hat{\sigma}}\in\Permutations{p+q}}
			\sgn\Perm{\hat{\sigma}} \cdotp
			\qty\Big(\opPerm[\Perm{\hat{\sigma}}
					\comp\Perm{\hat{\tau}}^{-1}]
				\qty\big(\T{\Phi}\tp\T{\Psi}))] \notag
	\intertext{用式~\eqref{eq:对张量并施加反对称化算子_证明_Part4}
		合并掉 $\sgn\Perm{\hat{\tau}}$ 和
		$\sgn\Perm{\hat{\sigma}}$:}
	&=\frac{1}{q!} \sum_{\Perm{\tau}\in\Permutations{q}}
		\frac{1}{(p+q)!}
		\sum_{\Perm{\hat{\sigma}}\in\Permutations{p+q}}
		\sgn\qty(\Perm{\hat{\sigma}}\comp\Perm{\hat{\tau}}^{-1})
		\cdotp\qty\Big[\opPerm[\Perm{\hat{\sigma}}
				\comp\Perm{\hat{\tau}}^{-1}]
				\qty\big(\T{\Phi}\tp\T{\Psi})] \notag
	\intertext{再次利用置换穷尽的性质改变“哑标”置换:}
	&=\frac{1}{q!} \sum_{\Perm{\tau}\in\Permutations{q}}
		\frac{1}{(p+q)!}
		\sum_{\Perm{\hat{\sigma}}\in\Permutations{p+q}}
		\sgn\Perm{\hat{\sigma}}
		\cdotp\qty\Big[\opPerm[\Perm{\hat{\sigma}}]
				\qty\big(\T{\Phi}\tp\T{\Psi})] \notag
	\intertext{终于拨开云雾见青天,看到了似曾相识的定义:}
	&=\frac{1}{q!} \sum_{\Perm{\tau}\in\Permutations{q}}
		\opSkw\qty\big(\T{\Phi}\tp\T{\Psi}) \notag \\
	&=\frac{1}{q!} \cdotp q!
		\opSkw\qty\big(\T{\Phi}\tp\T{\Psi})
	=\opSkw\qty\big(\T{\Phi}\tp\T{\Psi}) \fullstop
\end{align}
\end{myProof}

\blankline

\item 反对称化算子具有所谓\emphA{反导性}:
\begin{equation}
	\forall\,\T{\Phi}\in\Tensors{p},\,\T{\Psi}\in\Tensors{q},\quad
	\opSkw\qty\big(\T{\Phi}\tp\T{\Psi})
	=(-1)^{pq}\cdotp\opSkw\qty\big(\T{\Psi}\tp\T{\Phi}) \fullstop
\end{equation}

\begin{myProof}
首先单独把反对称算子展开。它所作用的张量为 $p+q$ 阶,因而相应的置换
$\Perm{\sigma}\in\Permutations{p+q}$:
\begin{align}
	\opSkw&=\frac{1}{(p+q)!} \sum_{\Perm{\sigma}\in\Permutations{p+q}}
		\sgn\Perm{\sigma}\cdotp\opPerm \notag \\
	&=\frac{1}{(p+q)!} \sum_{\Perm{\sigma}\in\Permutations{p+q}}
		\sgn\qty(\Perm{\sigma}\comp\Perm{\tau}^{-1})
		\cdotp\opPerm[\Perm{\sigma}\comp\Perm{\tau}^{-1}] \fullstop
\end{align}
第二的等号与之前一样,利用了置换的穷尽。
这里的 $\Perm{\tau}$ 是 $\Permutations{p+q}$ 中一个任意的置换。
利用置换符号的性质,有
\begin{equation}
	\sgn\qty(\Perm{\sigma}\comp\Perm{\tau}^{-1})
	=\sgn\Perm{\sigma} \cdotp \sgn\Perm{\tau}^{-1}
	=\sgn\Perm{\sigma} \cdotp \sgn\Perm{\tau} \fullstop
\end{equation}
因此
\begin{align}
	\opSkw\qty\big(\T{\Phi}\tp\T{\Psi})
	&=\frac{1}{(p+q)!} \sum_{\Perm{\sigma}\in\Permutations{p+q}}
		\sgn\qty(\Perm{\sigma}\comp\Perm{\tau}^{-1})
		\cdotp\opPerm[\Perm{\sigma}\comp\Perm{\tau}^{-1}]
			\qty\big(\T{\Phi}\tp\T{\Psi}) \notag \\
	&=\frac{\sgn\Perm{\tau}}{(p+q)!}
		\sum_{\Perm{\sigma}\in\Permutations{p+q}}
		\sgn\Perm{\sigma}
		\cdotp\opPerm[\Perm{\sigma}\comp\Perm{\tau}^{-1}]
			\qty\big(\T{\Phi}\tp\T{\Psi}) \fullstop
	\label{eq:反对称化算子的反导性_证明_Part1}
\end{align}

把张量展开成分量形式,可以有
\begin{align}
	\opPerm[\Perm{\sigma}\comp\Perm{\tau}^{-1}]
		\qty\big(\T{\Phi}\tp\T{\Psi})
	&=\opPerm[\Perm{\sigma}\comp\Perm{\tau}^{-1}]
		\qty[\Phi^{i_1 \cdots i_p}\,\Psi^{j_1 \cdots j_q}\,
			\qty(\V{g}_{i_1}\tp\cdots\tp\V{g}_{i_p}) \tp
			\qty(\V{g}_{j_1}\tp\cdots\tp\V{g}_{j_q})] \notag
	\intertext{根据式~\eqref{eq:置换算子的复合},可得}
	&=\opPerm\comp\opPerm[\Perm{\tau}^{-1}]
		\qty[\Phi^{i_1 \cdots i_p}\,\Psi^{j_1 \cdots j_q}\,
			\qty(\V{g}_{i_1}\tp\cdots\tp\V{g}_{i_p}) \tp
			\qty(\V{g}_{j_1}\tp\cdots\tp\V{g}_{j_q})] \notag
	\intertext{利用置换算子的表示
		\eqref{eq:置换算子的表示_总结_对简单张量}~一式,对简单张量进行操作:}
	&=\opPerm\qty[
			\Phi^{i_1 \cdots i_p}\,\Psi^{j_1 \cdots j_q}\,
			\qty(\V{g}_{\Perm{\tau}(i_1)}\tp\cdots
				\tp\V{g}_{\Perm{\tau}(i_p)}) \tp
			\qty(\V{g}_{\Perm{\tau}(j_1)}\tp\cdots
				\tp\V{g}_{\Perm{\tau}(j_q)})] \fullstop
\end{align}

根据 $\Perm{\tau}$ 的任意性,不妨取\footnote{%
	矩阵中的 $i_p$ 和 $j_q$ 等未必是对齐的,这里的写法只是为了表示方便。}
\begin{equation}
	\Perm{\tau}=\mqty*(
		i_1 & \cdots & i_p & j_1 & \cdots & j_q \\
		j_1 & \cdots & j_q & i_1 & \cdots & i_p ) \fullstop
\end{equation}
这种取法恰好可以使指标为 $i$ 和 $j$ 的向量交换一下位置。于是
\begin{align}
	\opPerm[\Perm{\sigma}\comp\Perm{\tau}^{-1}]
		\qty\big(\T{\Phi}\tp\T{\Psi})
	&=\opPerm\qty[
			\Phi^{i_1 \cdots i_p}\,\Psi^{j_1 \cdots j_q}\,
			\qty(\V{g}_{\Perm{\tau}(j_1)}\tp\cdots
				\tp\V{g}_{\Perm{\tau}(j_q)}) \tp
			\qty(\V{g}_{\Perm{\tau}(i_1)}\tp\cdots
				\tp\V{g}_{\Perm{\tau}(i_p)})] \notag \\
	&=\opPerm\qty[
			\Phi^{i_1 \cdots i_p}\,\Psi^{j_1 \cdots j_q}\,
			\qty(\V{g}_{i_1}\tp\cdots\tp\V{g}_{i_p}) \tp
			\qty(\V{g}_{j_1}\tp\cdots\tp\V{g}_{j_q})] \notag
	\intertext{张量分量作为\emphB{数},可交换顺序自然无需多言:}
	&=\opPerm\qty[
			\Psi^{j_1 \cdots j_q}\,\Phi^{i_1 \cdots i_p}\,
			\qty(\V{g}_{i_1}\tp\cdots\tp\V{g}_{i_p}) \tp
			\qty(\V{g}_{j_1}\tp\cdots\tp\V{g}_{j_q})] \notag \\
	&=\opPerm \qty\big(\T{\Psi}\tp\T{\Phi}) \fullstop
\end{align}
下面再考虑一下 $\Perm{\tau}$ 的符号:$j_1$ 先和 $i_p$ 交换,
再和 $i_{p-1}$ 交换,以此类推,直到移动至 $i_1$ 的位置,一共交换了 $p$ 次。
而 $j_2,\,\cdots,\,j_q$ 也是同理,各需进行 $p$ 次交换。
所以总共是 $p \cdotp q$ 次两两交换。因此,
\begin{equation}
	\sgn\Perm{\tau}=(-1)^{pq} \fullstop
\end{equation}

回到式~\eqref{eq:反对称化算子的反导性_证明_Part1} 的推导,有
\begin{align}
	\opSkw\qty\big(\T{\Phi}\tp\T{\Psi})
	&=\frac{\sgn\Perm{\tau}}{(p+q)!}
		\sum_{\Perm{\sigma}\in\Permutations{p+q}}
		\sgn\Perm{\sigma}
		\cdotp\opPerm[\Perm{\sigma}\comp\Perm{\tau}^{-1}]
			\qty\big(\T{\Phi}\tp\T{\Psi}) \notag \\
	&=\frac{(-1)^{pq}}{(p+q)!}
		\sum_{\Perm{\sigma}\in\Permutations{p+q}}
		\sgn\Perm{\sigma}
		\cdotp\opPerm\qty\big(\T{\Psi}\tp\T{\Phi}) \notag \\
	&=(-1)^{pq}\cdotp\opSkw\qty\big(\T{\Psi}\tp\T{\Phi}) \fullstop
\end{align}
\end{myProof}

\end{myEnum}