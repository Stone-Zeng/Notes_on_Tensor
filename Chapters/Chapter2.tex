\section{张量积}
	\emphA{张量积}也叫\emphA{张量并},用符号“$\tp$”表示。
	在 \ref{subsec:张量的表示与简单张量} 小节给出简单张量的定义时,
	实际上就用到了张量积。张量积的定义为:
	\begin{align}
		\forall\,\Tens{\Phi}\in\Tensors{p},\,\Tens{\Psi}\in\Tensors{q},
		&\mathrel{\phantom{=}} \Tens{\Phi}\tp\Tens{\Psi}
			\in\Tensors{p+q} \notag \\
		&=\qty(\Phi^{i_1 \cdots i_p} \,
				\V{g}_{i_1}\tp\cdots\tp\V{g}_{i_p})
			\tp \qty(\Psi_{j_1 \cdots j_q} \,
				\V{g}^{j_1}\tp\cdots\tp\V{g}^{j_q}) \notag \\
		&\defeq \Phi^{i_1 \cdots i_p} \,
			\Psi_{j_1 \cdots j_q}\,
			\qty(\V{g}_{i_1}\tp\cdots\tp\V{g}_{i_p})
			\tp \qty(\V{g}^{j_1}\tp\cdots
				\tp\V{g}^{j_q}_{\phantom{i_p}}) \fullstop
	\end{align}
	由该定义可以知道,关于简单张量 $\qty(\V{g}_{i_1}\tp\cdots
		\tp\V{g}_{i_p}) \tp \qty(\V{g}^{j_1}\tp\cdots
		\tp\V{g}^{j_q}_{\phantom{i_p}})$,相应的张量分量为
	\begin{equation}
		\tensor{\qty\big(\Phi\tp\Psi)}
			{^{i_1 \cdots i_p}_{j_1 \cdots j_q}} \fullstop
	\end{equation}
	
\section{\texorpdfstring{$e$ 点积}{e 点积}}
	张量的 \emphA{$e$ 点积}可以用符号“$\edp$”表示。
	从这个符号可以看出 $e$ 点积的作用:前 $e$ 个指标缩并,后面的点乘。
	
	对于任意的 $\Tens{\Phi}\in\Tensors{p},\,
		\Tens{\Psi}\in\Tensors{q},\,
		e\leqslant\min\qty{p,\,q}\in\natN$,$e$ 点积是这样定义的:
	\begin{align}
		&\mathrel{\phantom{=}} \Tens{\Phi}\edp\Tens{\Psi} \notag \\
		&=\qty(\Phi^{i_1 \cdots i_{p-e} i_{p-e+1} \cdots i_p} \,
			\V{g}_{i_1}\tp\cdots\tp\V{g}_{i_{p-e}}
			\tp\hl{\V{g}_{i_{p-e+1}}\tp\cdots\tp\V{g}_{i_p}}
			) \notag \\
		&\mathrel{\phantom{=}}\quad\edp
			\qty(\Psi^{j_1 \cdots j_e j_{e+1} \cdots j_q} \,
			\hl{\V{g}_{j_1}\tp\cdots\tp\V{g}_{j_e}}
			\tp\V{g}_{j_{e+1}}\tp\cdots\tp\V{g}_{j_q}) \notag
		\intertext{把高亮的部分做内积,得到\emphB{度量}:}
		&\defeq\Phi^{i_1 \cdots i_{p-e} i_{p-e+1} \cdots i_p} \,
			\Psi^{j_1 \cdots j_e j_{e+1} \cdots j_q} \notag \\
		&\mathrel{\phantom{=}}\quad\cdot
			g_{i_{p-e+1} j_1} \cdots g_{i_p j_e} \,
			\qty(\V{g}_{i_1}\tp\cdots\tp\V{g}_{i_{p-e}})
			\tp\qty(\V{g}_{j_{e+1}}\tp\cdots\tp\V{g}_{j_q}) \notag
		\intertext{玩一下“指标升降游戏”(注意有两种结合方式:
			与 $\Phi$ 或 $\Psi$),可得}
		&=\left\{\begin{lgathered}
				\tensor{\Phi}{^{i_1 \cdots i_{p-e}}_{\hl{j_1 \cdots j_e}}} \,
				\Psi^{\hl{j_1 \cdots j_e} j_{e+1} \cdots j_q} \\
				\Phi^{i_1 \cdots i_{p-e} \hl{i_{p-e+1} \cdots i_p}} \,
				\tensor{\Psi}{_{\hl{i_{p-e+1} \cdots i_p}}^{j_{e+1}
					\cdots j_q}}
			\end{lgathered}\right\}
			\qty(\V{g}_{i_1}\tp\cdots\tp\V{g}_{i_{p-e}})
			\tp\qty(\V{g}_{j_{e+1}}\tp\cdots\tp\V{g}_{j_q}) \fullstop
	\end{align}
	最后一步的大括号中,高亮的 $j_1 \cdots j_e$
	和 $i_{p-e+1} \cdots i_p$ 都是哑标,可以通过求和求掉。因此有
	\begin{equation}
		\Tens{\Phi}\edp\Tens{\Psi} \in \Tensors{p+q-2e} \fullstop
	\end{equation}
	换句话说,$e$ 点积的作用就是将指标\emphB{哑标化}。
	
	作为一个特殊的应用,接下来我们介绍\emphA{全点积},用符号“\fdp”表示。
	对于任意的 $\Tens{\Phi},\,\Tens{\Psi}\in\Tensors{p}$,有
	\begin{align}
		&\mathrel{\phantom{=}} \Tens{\Phi}\fdp\Tens{\Psi}
			\defeq \Tens{\Phi}\edp[p]\Tens{\Psi} \notag \\
		&=\qty(\Phi^{i_1 \cdots i_p}\,\V{g}_{i_1}\tp\cdots\tp\V{g}_{i_p})
			\edp[p]
			\qty(\Psi^{j_1 \cdots j_p}\,\V{g}_{j_1}\tp\cdots\tp\V{g}_{j_p})
			\notag \\
		&=\Phi^{i_1 \cdots i_p} \, \Psi^{j_1 \cdots j_p} \,
			g_{i_1 j_1} \cdots g_{i_p j_p} \notag \\
		&=\left\{\begin{lgathered}
				\Phi_{j_1 \cdots j_p} \, \Psi^{j_1 \cdots j_p} \\
				\Phi^{i_1 \cdots i_p} \, \Psi_{i_1 \cdots i_p}
			\end{lgathered}\right.
			\in\realR \fullstop
	\end{align}
	可见,全点积将\emphB{全部}指标哑标化。
	
	张量自身和自身的全点积,定义为它的\emphA{范数}:
	\begin{equation}
		\Tens{\Phi}\fdp\Tens{\Phi}
		=\Phi^{i_1 \cdots i_p} \, \Phi_{i_1 \cdots i_p}
		\eqcolon \qty|\Tens{\Phi}|^2_{\Tensors{p}} \fullstop
	\end{equation}
	
\section{叉乘}
	张量的\emphA{叉乘}要求底空间为 $\realR^3$。
	对于任意的 $\Tens{\Phi}\in\Tensors[\realR^3]{p},\,
	\Tens{\Psi}\in\Tensors[\realR^3]{q}$,叉乘的定义如下:
	\begin{align}
		&\mathrel{\phantom{=}} \Tens{\Phi}\cp\Tens{\Psi} \notag \\
		&=\qty(\Phi^{i_1 \cdots i_{p-1} i_p} \,
				\V{g}_{i_1}\tp\cdots\tp\V{g}_{i_{p-1}}\tp\V{g}_{i_p})
			\cp \qty(\Psi_{j_1 j_2 \cdots j_q} \,
				\V{g}^{j_1}\tp\V{g}^{j_2}\cdots\tp\V{g}^{j_q}) \notag \\
		&\defeq \Phi^{i_1 \cdots i_p} \, \Psi_{j_1 \cdots j_p} \,
			\V{g}_{i_1}\tp\cdots\tp\V{g}_{i_{p-1}}
			\tp\qty(\V{g}_{i_p}\cp\V{g}^{j_1})
			\tp\V{g}^{j_2}\cdots\tp\V{g}^{j_q}
			\in\Tensors[\realR^3]{p+q-1} \fullstop
	\end{align}
	注意到,此时简单张量的维数已经降了一阶。
	
	利用\emphA{Levi-Civita 记号},可以进一步展开上式。
	\begin{align}
		\V{g}_{i_p}\cp\V{g}^{j_1}
		=\LeviCivita{_{i_p}^{j_1}_s}\,\V{g}^s \comma
	\end{align}
	式中的
	\begin{equation}
		\LeviCivita{_{i_p}^{j_1}_s}
		=\det[\V{g}_{i_p},\,\V{g}^{j_1},\,\V{g}_s] \fullstop
	\end{equation}
	于是
	\begin{equation}
		\Tens{\Phi}\cp\Tens{\Psi} \,
		=\LeviCivita{_{i_p}^{j_1}_s}\,
			\Phi^{i_1 \cdots i_p} \, \Psi_{j_1 \cdots j_p}
			\V{g}_{i_1}\tp\cdots\tp\V{g}_{i_{p-1}} \tp\V{g}^s
			\tp\V{g}^{j_2}\cdots\tp\V{g}^{j_q} \fullstop
	\end{equation}
	
	下面我们再来类比地定义一种混合积“$\edp[\cp]$”。
	对于任意的 $\Tens{\Phi},\,\Tens{\Psi}\in\Tensors{3}$,定义
	\begin{align}
		\Tens{\Phi}\edp[\cp]\Tens{\Psi}
		&=\qty(\Phi^{ijk}\,\V{g}_i\tp\V{g}_j\tp\V{g}_k)
			\edp[\cp]\qty(\Psi_{pqr}\,\V{g}^p\tp\V{g}^q\tp\V{g}^r)\notag \\
		&\defeq \Phi^{ijk}\,\Psi_{pqr} \,
			\KroneckerDelta{q}{j} \,
			\V{g}_i\tp\qty(\V{g}_k\cp\V{g}^p)\tp\V{g}^r \notag
		\intertext{缩并掉 Kronecker δ,
			同时利用 Levi-Civita 记号展开叉乘项,可有}
		&=\LeviCivita{_k^p_s}\,\Phi^{ijk}\,\Psi_{pjr}\,
			\V{g}_i\tp\V{g}^s\tp\V{g}^r \comma
	\end{align}
	式中的
	\begin{equation}
		\LeviCivita{_k^p_s}=\det[\V{g}_k,\,\V{g}^p,\,\V{g}_s] \fullstop
	\end{equation}
	
	对于这种混合积,并没有一般的约定。不同的研究者往往会采用不同的写法及表示。