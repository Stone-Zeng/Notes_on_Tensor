\section{对偶基,度量}
\subsection{对偶基} \label{subsec:对偶基}
设 $\qty{\V{g}_i}^m_{i=1}$ 是 $\Rm$ 空间中的一组\emphA{基},
即极大线性无关向量组。此时,$\Rm$ 中将\emphB{唯一}存在另一组基
$\qty{\V{g}^i}^m_{i=1}$,二者满足\emphA{对偶关系}:
\begin{equation}
	\ipb{\V{g}_i}{\V{g}^j}=\KroneckerDelta{j}{i}
	=\left\{\begin{aligned}
		1,\quad j&=i \semicomma\\
		0,\quad j&\neq i \fullstop
	\end{aligned}\right.
	\label{eq:对偶关系}
\end{equation}
式中,$\ipb{\V{g}_i}{\V{g}^j}$ 的 $\KroneckerDelta{j}{i}$ 是 
\emphA{Kronecker δ 函数}。

\begin{myProof}
根据内积的定义,
\begin{equation}
	\ipb{\V{g}_i}{\V{g}^j}=\qty(\V{g}^j)\trans\V{g}_i
	=\KroneckerDelta{j}{i} \comma
\end{equation}
其中的 $i$、$j$ 可取 $1,\,2,\,\cdots,\,m$。写成矩阵形式,为\footnote{
	除非特殊说明,本文中的所有向量均取\emphB{列向量}。}
\begin{equation}
	\mqty[\qty(\V{g}^1)\trans \\ \vdots \\ \qty(\V{g}^m)\trans]
	\mqty[\V{g}_1,\,\cdots,\,\V{g}_m] = \Mat{I}_m \fullstop
\end{equation}
左边的第一个矩阵拆分成了 $m$ 行,每行是一个 $m$ 维行向量;
第二个矩阵则拆分成了 $m$ 列,每行是一个 $m$ 维列向量。
根据分块矩阵的乘法,所得结果对角元为 $\KroneckerDelta{i}{i}$,
非对角元则为 $\KroneckerDelta{i}{j}$(其中 $i\neq j$),即单位阵。

把第一个矩阵的转置挪到外面,可有
\begin{equation}
	\mqty[\V{g}^1,\,\cdots,\,\V{g}^m]\trans
	\mqty[\V{g}_1,\,\cdots,\,\V{g}_m] = \Mat{I}_m \fullstop
\end{equation}
$\qty{\V{g}_i}^m_{i=1}$ 作为基,必然满足
$\mqty[\V{g}_1,\,\cdots,\,\V{g}_m]$ 非奇异。因此
\begin{equation}
	\mqty[\V{g}^1,\,\cdots,\,\V{g}^m]\trans
	=\mqty[\V{g}_1,\,\cdots,\,\V{g}_m]^{-1} \comma
\end{equation}
即
\begin{equation}
	\mqty[\V{g}^1,\,\cdots,\,\V{g}^m]
	=\mqty[\V{g}_1,\,\cdots,\,\V{g}_m]\invTrans \fullstop
	\label{eq:逆变基用协变基表示}
\end{equation}
逆矩阵(及其转置)是存在且唯一的,这就证明了对偶基的存在性和唯一性。
\end{myProof}

我们把指标写在\emphB{下面}的基 $\qty{\V{g}_i}^m_{i=1}$
称为\emphA{协变基};指标写在\emphB{上面}的基
$\qty{\V{g}^i}^m_{i=1}$ 称为\emphA{逆变基}。
式~\eqref{eq:逆变基用协变基表示} 明确指出了逆变基与协变基的关系。

\subsection{度量} \label{subsec:度量}
下面引入\emphA{度量}的概念。其定义为
\begin{braceEq}
	g_{ij} &\defeq \ipb{\V{g}_i}{\V{g}_j} \comma 
	\label{eq:度量的定义_协变} \\
	g^{ij} &\defeq \ipb{\V{g}^i}{\V{g}^j} \fullstop
	\label{eq:度量的定义_逆变}
\end{braceEq}
这两种度量满足
\begin{equation}
	g_{ik}\,g^{kj} = \KroneckerDelta{j}{i} \fullstop
	\label{eq:度量之积}
\end{equation}
也可以写成矩阵的形式:
\begin{equation}
	\mqty[g_{ik}]\mqty[g^{kj}]
	=\mqty[\KroneckerDelta{j}{i}]=\Mat{I}_m \comma
\end{equation}
其中的 $\Mat{I}_m$ 是 $m$ 阶单位阵。该式的证明将在稍后给出。

由于内积具有交换律,因而度量的两个指标显然可以交换:
\begin{equation}
	g_{ij}=g_{ji}, \quad g^{ij}=g^{ji} \fullstop
\end{equation}

利用度量,可以获得\emphA{基向量转换关系}。
第 $i$ 个协变基向量 $\V{g}_i$ 既然是向量,
就必然可以用协变基或逆变基来表示\footnote{%
	所谓用某组基来“表示”一个向量,就是把它朝各个基的方向做投影,然后再求和。}。
根据对偶关系式~\eqref{eq:对偶关系} 和度量的定义
式~\eqref{eq:度量的定义_协变}、\eqref{eq:度量的定义_逆变},可知
\begin{braceEq}
	\V{g}_i &=\qty(\V{g}_i,\,\V{g}_k)_{\Rm} \,\V{g}^k
	=g_{ik}\,\V{g}^k \comma \label{eq:基向量转换关系1} \\
	\V{g}_i &=\qty(\V{g}_i,\,\V{g}^k)_{\Rm} \,\V{g}_k
	=\KroneckerDelta{k}{i}\V{g}_k \label{eq:基向量转换关系2}
\end{braceEq}
以及
\begin{braceEq}
	\V{g}^i &=\qty(\V{g}^i,\,\V{g}_k)_{\Rm} \,\V{g}^k
	=\KroneckerDelta{i}{k}\,\V{g}^k \comma \label{eq:基向量转换关系3} \\
	\V{g}^i &=\qty(\V{g}^i,\,\V{g}^k)_{\Rm} \,\V{g}_k
	=g^{ik}\V{g}_k \fullstop \label{eq:基向量转换关系4}
\end{braceEq}
这四个式子中,式~\eqref{eq:基向量转换关系2} 和
\eqref{eq:基向量转换关系3} 是平凡的,
而式~\eqref{eq:基向量转换关系1} 和 \eqref{eq:基向量转换关系4}
则通过\emphB{度量}建立起了协变基与逆变基之间的关系。
这就称为\emphB{基向量转换关系},也可以叫做“指标升降游戏”。

需要说明的是,根据\emphA{Einstein 求和约定},
重复指标(即\emphB{哑标},这里是 $k$)且一上一下时,已经暗含了求和。
后文除非特殊说明,也都是如此。

\blankline

现在我们来证明式~\eqref{eq:度量之积}:
\begin{equation}
	g_{ik}\,g^{kj} = \KroneckerDelta{j}{i} \fullstop
\end{equation}

\begin{myProof}
\begin{equation}
	g_{ik}\,g^{kj}
	=\ipb{\V{g}_i}{\V{g}_k} \, g^{kj}
	=\ipb{\V{g}_i}{g^{kj}\V{g}_k}
\end{equation}
根据式~\eqref{eq:基向量转换关系4},有
\begin{equation}
	g^{kj}\V{g}_k=g^{jk}\V{g}_k=\V{g}^j
\end{equation}
因此可得
\begin{equation}
	g_{ik}\,g^{kj}=\ipb{\V{g}_i}{\V{g}^j}
	=\KroneckerDelta{j}{i} \fullstop
\end{equation}

\end{myProof}

\subsection{向量的分量}
对于任意的向量 $\V{\xi} \in \Rm$,它可以用协变基表示:
\begin{mySubEq}
	\begin{align}
		\V{\xi}&=\ipb{\V{\xi}}{\V{g}^k}\,\V{g}_k
		=\xi^k \V{g}_k \comma
		\intertext{也可以用逆变基表示:}
		\V{\xi}&=\ipb{\V{\xi}}{\V{g}_k}\,\V{g}^k
		=\xi_k \V{g}^k \fullstop
	\end{align}
\end{mySubEq}
式中,$\xi^k$ 是 $\V{\xi}$ 与第 $k$ 个\emphB{逆变基}做内积的结果,
称为 $\V{\xi}$ 的第 $k$ 个\emphA{逆变分量};
而 $\xi_k$ 是 $\V{\xi}$ 与第 $k$ 个\emphB{协变基}做内积的结果,
称为 $\V{\xi}$ 的第 $k$ 个\emphA{协变分量}。

以后凡是指标在下的(下标),均称为\emphB{协变}某某;
指标在上的(上标),称为\emphB{逆变}某某。

\section{张量的表示}
\subsection{张量的表示与简单张量} \label{subsec:张量的表示与简单张量}
所谓\emphA{张量},即\emphA{多重线性函数}。

首先用三阶张量举个例子。考虑任意的 $\T{\Phi}\in\Tensors{3}$,
其中的 $\Tensors{3}$ 表示以 $\Rm$ 为底空间的三阶张量全体。
所谓三阶(或三重)线性函数,指“吃掉”三个向量之后变成\emphB{实数},
并且“吃法”具有线性性。

一般地,$r$ 阶张量的定义如下:
\begin{equation}
	\mmap{\T{\Phi}}
		{\underbrace{\Rm\times\Rm\times\cdots\times\Rm}_
			{\text{$r$ 个 $\Rm$}}
			\ni\qty{\V{u}_1,\,\V{u}_2,\,\cdots,\,\V{u}_r}}
		{\T{\Phi}\qty(\V{u}_1,\,\V{u}_2,\,\cdots,\,\V{u}_r)}
			\in\realR \comma
\end{equation}
式中的 $\T{\Phi}$ 满足
\begin{align}
	\forall\,\alpha,\,\beta\in\realR,\,
	&\alspace\T{\Phi}\qty(\V{u}_1,\,\cdots,\,
		\alpha\tilde{\V{u}}_i+\beta\hat{\V{u}}_i,\,\cdots,\,\V{u}_r)
		\notag \\
	&=\alpha\,\T{\Phi}\qty(\V{u}_1,\,\cdots,\,
		\tilde{\V{u}}_i,\,\cdots,\,\V{u_r})
	+\beta\,\T{\Phi}\qty(\V{u}_1,\,\cdots,\,
		\hat{\V{u}}_i,\,\cdots,\,\V{u_r}) \comma
\end{align}
即所谓“\emphB{对第 $i$ 个变元的线性性}”。这里的 $i$ 可取
$1,\,2,\,\cdots,\,r$。

在张量空间 $\Tensors{r}$ 上,我们引入线性结构:
\begin{align}
	\forall\,\alpha,\,\beta\in\realR
	\text{\ 和\ } \T{\Phi},\,\T{\Psi}\in\Tensors{r},
	&\alspace\qty(\alpha\,\T{\Phi}+\beta\,\T{\Psi})
	\qty(\V{u}_1,\,\V{u}_2,\,\cdots,\,\V{u}_r) \notag \\
	&\mathrel{\defeq}
		\alpha\,\T{\Phi}\qty(\V{u}_1,\,\V{u}_2,\,\cdots,\,\V{u}_r)
		+\beta\,\T{\Psi}\qty(\V{u}_1,\,\V{u}_2,\,\cdots,\,\V{u}_r)
	\comma
\end{align}
于是
\begin{equation}
	\alpha\,\T{\Phi}+\beta\,\T{\Psi} \in \Tensors{r} \fullstop
\end{equation}

下面我们要获得 $\T{\Phi}$ 的表示。
根据之前任意向量用协变基或逆变基的表示,有
\begin{align}
	\forall\,\V{u},\,\V{v},\,\V{w}\in\Rm,
	&\alspace
	\T{\Phi}\qty(\V{u},\,\V{v},\,\V{w}) \notag\\
	&=\T{\Phi}\qty(u^i\V{g}_i,\,v_j\V{g}^j,\,w^k\V{g}_k) \notag
	\intertext{考虑到 $\T{\Phi}$ 对第一变元的线性性,可得}
	&=u^i\,\T{\Phi}\qty(\V{g}_i,\,v_j\V{g}^j,\,w^k\V{g}_k) \notag
	\intertext{同理,}
	&=u^i v_j w^k\,\T{\Phi}\qty(\V{g}_i,\,\V{g}^j,\,\V{g}_k)
	\fullstop
	\label{eq:张量的表示1}
\end{align}
注意这里自然需要满足 Einstein 求和约定。

上式中的 $\T{\Phi}\qty(\V{g}_i,\,\V{g}^j,\,\V{g}_k)$ 是一个数。
它是张量 $\T{\Phi}$ “吃掉”三个基向量的结果。
至于 $u^i v_j w^k$ 部分,三项分别是 $\V{u}$ 的第 $i$ 个逆变分量、
$\V{v}$ 的第 $j$ 个协变分量和 $\V{w}$ 的第 $k$ 个逆变分量。
根据向量分量的定义,可知
\begin{equation}
	u^i v_j w^k
	= \ipb{\V{u}}{\V{g}^i}
	\cdotp \ipb{\V{v}}{\V{g}_j}
	\cdotp \ipb{\V{w}}{\V{g}^k} \fullstop
	\label{eq:张量的表示2}
\end{equation}

\blankline

暂时中断一下思路,先给出\emphA{简单张量}的定义。
\begin{equation}
	\forall\,\V{u},\,\V{v},\,\V{w}\in\Rm,\quad
	\V{\xi}\tp\V{\eta}\tp\V{\zeta}\qty(\V{u},\,\V{v},\,\V{w})
	\defeq \ipb{\V{\xi}}{\V{u}}
	\cdotp \ipb{\V{\eta}}{\V{v}}
	\cdotp \ipb{\V{\zeta}}{\V{w}} \in\realR \comma
\end{equation}
式中的 $\V{\xi},\,\V{\eta},\,\V{\zeta}\in\Rm$。
“$\tp$”的定义将在 \ref{sec:张量积}~节中给出,
现在可以暂时把 $\V{\xi}\tp\V{\eta}\tp\V{\zeta}$ 理解为一种记号。
简单张量作为一个映照,组成它的三个向量分别与它们“吃掉”的第一、二、三个变元
做内积并相乘,结果为一个实数。

考虑到内积的线性性,便有(以第二个变元为例)
\begin{align}
	\V{\xi}\tp\V{\eta}\tp\V{\zeta}
	\qty(\V{u},\,\alpha\tilde{\V{v}}+\beta\hat{\V{v}},\,\V{w})
	&\defeq \ipb{\V{\xi}}{\V{u}}
	\cdotp \ipb{\V{\eta}}{\alpha\tilde{\V{v}}+\beta\hat{\V{v}}}
	\cdotp \ipb{\V{\zeta}}{\V{w}} \in\realR \notag
	\intertext{注意到
		$\ipb{\V{\eta}}{\alpha\tilde{\V{v}}+\beta\hat{\V{v}}}
			=\alpha\ipb{\V{\eta}}{\tilde{\V{v}}}
			+\beta\ipb{\V{\eta}}{\hat{\V{v}}}$,
		同时再次利用简单张量的定义,可得}
	&= \alpha \V{\xi}\tp\V{\eta}\tp\V{\zeta}
		\qty(\V{u},\,\tilde{\V{v}},\,\V{w})
		+\beta \V{\xi}\tp\V{\eta}\tp\V{\zeta}
		\qty(\V{u},\,\hat{\V{v}},\,\V{w}) \fullstop
\end{align}
类似地,对第一变元和第三变元,同样具有线性性。因此,可以知道
\begin{equation}
	\V{\xi}\tp\V{\eta}\tp\V{\zeta}
	\in\Tensors{3} \fullstop
\end{equation}
可见,“简单张量”的名字是名副其实的,它的确是一个特殊的张量。

回过头来看 \eqref{eq:张量的表示2}~式。很明显,它可以用简单张量来表示。
要注意,由于内积的对称性,可以有两种\footnote{%
	这里只考虑把 $\V{u}$、$\V{v}$、$\V{w}$%
	和 $\V{g}^i$、$\V{g}_j$、$\V{g}^k$ 分别放在一起的情况。}表示方法:
\begin{mySubEq}
	\begin{gather}
		\V{g}^i\tp\V{g}_j\tp\V{g}^k
		\qty(\V{u},\,\V{v},\,\V{w})
		\intertext{或者}
		\V{u}\tp\V{v}\tp\V{w}
		\qty(\V{g}^i,\,\V{g}_j,\,\V{g}^k) \comma
	\end{gather}
\end{mySubEq}
我们这里取上面一种。代入式~\eqref{eq:张量的表示1},得
\begin{align}
	&\alspace
		\T{\Phi}\qty(\V{u},\,\V{v},\,\V{w}) \notag\\
	&=\T{\Phi}\qty(\V{g}_i,\,\V{g}^j,\,\V{g}_k)
		\cdotp\V{g}^i\tp\V{g}_j\tp\V{g}^k
		\qty(\V{u},\,\V{v},\,\V{w}) \notag
	\intertext{由于
		$\T{\Phi}\qty(\V{g}_i,\,\V{g}^j,\,\V{g}_k) \in\Rm$,因此}
	&=\qty[\T{\Phi}\qty(\V{g}_i,\,\V{g}^j,\,\V{g}_k)
		\V{g}^i\tp\V{g}_j\tp\V{g}^k]
		\qty(\V{u},\,\V{v},\,\V{w}) \fullstop
\end{align}
方括号里的部分,就是根据 Einstein 求和约定,
用 $\T{\Phi}\qty(\V{g}_i,\,\V{g}^j,\,\V{g}_k)$
对 $\V{g}^i\tp\V{g}_j\tp\V{g}^k$ 进行线性组合。

由于 $\V{u}$、$\V{v}$、$\V{w}$ 选取的任意性,可以引入如下记号:
\begin{equation}
	\T{\Phi}
	=\T{\Phi}\qty(\V{g}_i,\,\V{g}^j,\,\V{g}_k) \,
		\V{g}^i\tp\V{g}_j\tp\V{g}^k
	\eqcolon \tensor{\Phi}{_i^j_k} \,
		\V{g}^i\tp\V{g}_j\tp\V{g}^k \comma
\end{equation}
即
\begin{equation}
	\tensor{\Phi}{_i^j_k}
	\coloneq \T{\Phi}\qty(\V{g}_i,\,\V{g}^j,\,\V{g}_k) \comma
\end{equation}
这称为张量的\emphA{分量}。
它说明一个张量可以用\emphB{张量分量}和基向量组成的\emphB{简单张量}来表示。

指标 $i$、$j$、$k$ 的上下是任意的。这里,
它有赖于式~\eqref{eq:张量的表示1} 中基向量的选取。
实际上,对于这里的三阶张量,指标的上下一共有 8 种可能。
指标全部在下面的,称为\emphA{协变分量}:
\begin{equation}
	\tensor{\Phi}{_i_j_k} \coloneq
	\T{\Phi}\qty(\V{g}_i,\,\V{g}_j,\,\V{g}_k) \semicomma
\end{equation}
指标全部在上面的,称为\emphA{逆变分量}:
\begin{equation}
	\tensor{\Phi}{^i^j^k} \coloneq
	\T{\Phi}\qty(\V{g}^i,\,\V{g}^j,\,\V{g}^k) \semicomma
\end{equation}
其余 6 种,称为\emphA{混合分量}。
对于一个 $r$ 阶张量,显然共有 $2^r$ 种分量表示,
其中协变分量与逆变分量各一种,混合分量 $2^r-2$ 种。

\subsection{张量分量之间的关系}
我们已经知道,对于任意一个向量 $\V{\xi}\in\Rm$,它可以用协变基或逆变基表示:
\begin{equation}
	\V{\xi}=\left\{\begin{aligned}
		\xi^i\V{g}_i \comma \\
		\xi_i\V{g}^i \fullstop
	\end{aligned}\right.
\end{equation}
式中,协变分量与逆变分量满足\emphB{坐标转换关系}:
\begin{braceEq}
	\xi^i &=\ipb{\V{\xi}}{\V{g}^i}
	=\ipb{\V{\xi}}{g^{ik}\V{g}_k}
	=g^{ik}\ipb{\V{\xi}}{\V{g}_k}
	=g^{ik}\xi_k \comma \\
	\xi_i &=\ipb{\V{\xi}}{\V{g}_i}
	=\ipb{\V{\xi}}{g_{ik}\V{g}^k}
	=g_{ik}\ipb{\V{\xi}}{\V{g}^k}
	=g_{ik}\xi^k \fullstop
\end{braceEq}
每一式的第二个等号都用到了\emphB{基向量转换关系},
见式~\eqref{eq:基向量转换关系1} 和 \eqref{eq:基向量转换关系4}。

现在再来考虑张量的分量。仍以上文中的张量 $\tensor{\Phi}{_i^j_k}
	\coloneq \T{\Phi}\qty(\V{g}_i,\,\V{g}^j,\,\V{g}_k)$ 为例,
我们想要知道它与张量 $\tensor{\Phi}{^p_q^r} \coloneq
	\T{\Phi}\qty(\V{g}^p,\,\V{g}_q,\,\V{g}^r)$ 之间的关系。
利用基向量转换关系,可有
\begin{align}
	\tensor{\Phi}{_i^j_k}
	&\coloneq\T{\Phi}\qty(\V{g}_i,\,\V{g}^j,\,\V{g}_k) \notag \\
	&=\T{\Phi}
		\qty(g_{ip}\V{g}^p,\,g^{jq}\V{g}_q,\,g_{kr}\V{g}^r) \notag
	\intertext{又利用张量的线性性,得}
	&=g_{ip}g^{jq}g_{kr}
		\T{\Phi}\qty(\V{g}^p,\,\V{g}_q,\,\V{g}^r) \notag \\
	&=g_{ip}g^{jq}g_{kr} \tensor{\Phi}{^p_q^r} \fullstop
\end{align}
可见,张量的分量与向量的分量类似,其指标升降可通过\emphB{度量}来实现。
用同样的手法,还可以得到诸如
$\tensor{\Phi}{^i^j^k}=g^{jp}\tensor{\Phi}{^i_p^k}$、
$\tensor{\Phi}{^i_j^k}=g_{jp}g^{kq}\tensor{\Phi}{^i^p_k}$
这样的关系式。

\subsection{相对不同基的张量分量之间的关系}
$\Rm$ 空间中,除了 $\qty{\V{g}_i}^m_{i=1}$ 和相应的
对偶基 $\qty{\V{g}^i}^m_{i=1}$ 之外,当然还可以有其他的基,
比如带括号的 $\qty{\V{g}_{(i)}}^m_{i=1}$ 以及对应的
对偶基 $\qty{\V{g}^{(i)}}^m_{i=1}$。
前者对应形如 $\tensor{\Phi}{^i_j^k}
	\coloneq \T{\Phi}\qty(\V{g}^i,\,\V{g}_j,\,\V{g}^k)$ 的张量,
后者则对应带括号的张量,如 $\tensor{\Phi}{^{(p)}_{(q)}^{(r)}} \coloneq
	\T{\Phi}\qty(\V{g}^{(p)},\,\V{g}_{(q)},\,\V{g}^{(r)})$。
下面我们来探讨这两个张量的关系。

首先来建立基之间的关系。带括号的第 $i$ 个基向量
$\V{g}_{(i)}$,作为 $\Rm$ 空间中的一个向量,自然可以用另一组基来表示:
\begin{equation}
	\V{g}_{(i)}=\left\{\begin{aligned}
		\ipb{\V{g}_{(i)}}{\V{g}_k}\,\V{g}^k \comma \\
		\ipb{\V{g}_{(i)}}{\V{g}^k}\,\V{g}_k \fullstop
	\end{aligned}\right.
\end{equation}
同理,自然还有它的对偶基:
\begin{equation}
	\V{g}^{(i)}=\left\{\begin{aligned}
		\ipb{\V{g}^{(i)}}{\V{g}_k}\,\V{g}^k \comma \\
		\ipb{\V{g}^{(i)}}{\V{g}^k}\,\V{g}_k \fullstop
	\end{aligned}\right.
\end{equation}
引入记号 $c^k_{(i)} \coloneq \ipb{\V{g}_{(i)}}{\V{g}^k}$
和 $c^{(i)}_k \coloneq \ipb{\V{g}^{(i)}}{\V{g}_k}$,那么有
\begin{braceEq}
	\V{g}_{(i)} &= c^k_{(i)}\V{g}_k \comma \\
	\V{g}^{(i)} &= c^{(i)}_k\V{g}^k \fullstop
\end{braceEq}

容易看出,这两个系数具有如下性质:
\begin{equation}
	c^{(i)}_k c^k_{(j)} = \KroneckerDelta{i}{j} \fullstop
\end{equation}
写成矩阵形式\footnote{%
	通常我们约定上面的标号作为行号,下面的标号作为列号。},为
\begin{equation}
	\qty[c^{(i)}_k]\qty[c^k_{(j)}]
	=\qty[\KroneckerDelta{j}{i}]=\Mat{I}_m \fullstop
\end{equation}
换句话说,两个系数矩阵是互逆的。
\begin{myProof}
\begin{align}
	c^{(i)}_k c^k_{(j)}
	&=\ipb{\V{g}^{(i)}}{\V{g}_k}\,c^k_{(j)} \notag
	\intertext{利用内积的线性性,有}
	&=\ipb{\V{g}^{(i)}}{c^k_{(j)} \V{g}_k} \notag
	\intertext{根据 $c^k_{(j)}$ 的定义,得到}
	&=\ipb{\V{g}^{(i)}}{\V{g}_{(j)}} \fullstop
\end{align}
带括号的基同样满足对偶关系 \eqref{eq:对偶关系}~式,于是得证。
\end{myProof}

上面我们用不带括号的基表示了带括号的基。反之也是可以的:
\begin{braceEq}
	\V{g}_i &= \ipb{\V{g}_i}{\V{g}^{(k)}}{\Rm}\,\V{g}_{(k)}
		=c^{(k)}_i\V{g}_{(k)} \comma \\
	\V{g}^i &= \ipb{\V{g}^i}{\V{g}_{(k)}}{\Rm}\,\V{g}^{(k)}
		=c^i_{(k)}\V{g}^{(k)} \fullstop
\end{braceEq}
这样一来,就建立起了不同基之间的转换关系。

现在我们回到张量。根据张量分量的定义,
\begin{align}
	\tensor{\Phi}{^i_j^k}
	&\coloneq \T{\Phi}\qty(\V{g}^i,\,\V{g}_j,\,\V{g}^k) \notag
	\intertext{利用之前推导的不同基向量之间的转换关系,得}
	&=\T{\Phi}\qty(
		c^i_{(p)}\V{g}^{(p)},\,c^{(q)}_j\V{g}_{(q)},\,
		c^k_{(r)}\V{g}^{(r)}) \notag
	\intertext{由张量的线性性,提出系数:}
	&=c^i_{(p)} c^{(q)}_j c^k_{(r)} \,
		\T{\Phi}\qty(\V{g}^{(p)},\,\V{g}_{(q)},\V{g}^{(r)}) \notag\\
	&=c^i_{(p)} c^{(q)}_j c^k_{(r)} \,
		\tensor{\Phi}{^{(p)}_{(q)}^{(r)}} \fullstop
\end{align}
完全类似,还可以有
\begin{equation}
	\tensor{\Phi}{^{(i)}_{(j)}^{(k)}}
	=c^{(i)}_p c^g_{(j)} c^{(k)}_r \tensor{\Phi}{^p_q^r} \fullstop
\end{equation}

\blankline

总结一下这两小节得到的结果。
对于同一组基下的张量分量,其指标升降通过\emphB{度量}来实现;
对于不同基下的张量分量,其指标转换则通过不同基之间的转换系数来完成。