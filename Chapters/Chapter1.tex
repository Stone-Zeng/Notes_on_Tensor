\section{对偶基,度量}
	$\Rm$ 空间中的基可分为两类:指标在\emphB{下面}的基
	\begin{equation}
		\qty{\V{g}_i}^m_{i=1} \subset \Rm
	\end{equation}
	称为\emphA{协变基},指标在\emphB{上面}的基
	\begin{equation}
		\qty{\V{g}^i}^m_{i=1} \subset \Rm
	\end{equation}称为\emphA{逆变基}。
	它们满足\emphA{对偶关系}:
	\begin{equation}
		\qty(\V{g}^i,\,\V{g}_j)_{\Rm}=\kroneckerDelta{i}{j}
		=\left\{\begin{aligned}
			1,\quad i&=j \semicomma\\
			0,\quad i&\neq j \fullstop
		\end{aligned}\right.
		\label{eq:对偶关系}
	\end{equation}
	这里的 $\kroneckerDelta{i}{j}$ 是 \emphA{Kronecker δ 函数}。
	
	下面引入\emphA{度量}的概念。其定义为
	\begin{braceEq}
		g_{ij} &\defeq \qty(\V{g}_i,\,\V{g}_j)_{\Rm} \comma 
		\label{eq:度量的定义_协变} \\
		g^{ij} &\defeq \qty(\V{g}^i,\,\V{g}^j)_{\Rm} \fullstop
		\label{eq:度量的定义_逆变}
	\end{braceEq}% \label{}
	
	下面证明
	\begin{equation}
		g_{ik}\,g^{kj} = \kroneckerDelta{j}{i} \fullstop
		\label{eq:度量之积}
	\end{equation}
	它也可以写成矩阵的形式:
	\begin{equation}
	\qty[g_{ik}]\qty[g^{kj}]
	=\qty[\kroneckerDelta{j}{i}]=\Mat{I}_m \comma
	\end{equation}
	其中的 $\Mat{I}_m$ 是 $m$ 阶单位阵。
	\begin{myProof}
		\begin{equation}
			g_{ik}\,g^{kj}
			=\qty(\V{g}_i,\,\V{g}_k)_{\Rm} \, g^{kj}
			=\qty(\V{g}_i,\,g^{kj}\V{g}_k)_{\Rm}
		\end{equation}
		后文将说明 $g^{kj}\V{g}_k=\V{g}^j$,因此可得
		\begin{equation}
			g_{ik}\,g^{kj}=\qty(\V{g}_i,\,\V{g}^j)_{\Rm}
			=\kroneckerDelta{j}{i} \fullstop
		\end{equation}
		
		要注意的是,这里的指标 $k$ 是\emphB{哑标}。
		根据\emphA{Einstein 求和约定},
		重复指标并且一上一下时,就表示对它求和。后文除非特殊说明,也均是如此。
	\end{myProof}
	
	现在澄清\emphA{基向量转换关系}。第 $i$ 个协变基向量 $\V{g}_i$ 既然是向量,
	就必然可以用协变基或逆变基来表示。
	根据对偶关系式~\eqref{eq:对偶关系} 和度量的定义
	式~\eqref{eq:度量的定义_协变}、\eqref{eq:度量的定义_逆变},可知
	\begin{braceEq}
		\V{g}_i &=\qty(\V{g}_i,\,\V{g}_k)_{\Rm} \,\V{g}^k
		=g_{ik}\,\V{g}^k \comma \label{eq:基向量转换关系1} \\
		\V{g}_i &=\qty(\V{g}_i,\,\V{g}^k)_{\Rm} \,\V{g}_k
		=\kroneckerDelta{k}{i}\V{g}_k \label{eq:基向量转换关系2}
	\end{braceEq}
	以及
	\begin{braceEq}
		\V{g}^i &=\qty(\V{g}^i,\,\V{g}_k)_{\Rm} \,\V{g}^k
		=\kroneckerDelta{i}{k}\,\V{g}^k \comma \label{eq:基向量转换关系3} \\
		\V{g}^i &=\qty(\V{g}^i,\,\V{g}^k)_{\Rm} \,\V{g}_k
		=g^{ik}\V{g}_k \fullstop \label{eq:基向量转换关系4}
	\end{braceEq}
	这四个式子中,式~\eqref{eq:基向量转换关系2} 和
	\eqref{eq:基向量转换关系3} 是平凡的,
	而式~\eqref{eq:基向量转换关系1} 和 \eqref{eq:基向量转换关系4}
	则通过\emphB{度量}建立起了协变基与逆变基之间的关系。
	这就称为基向量转换关系。
	
	对于任意的向量 $\V{\xi} \in \Rm$,它可以用协变基表示:
	\begin{equation}
		\V{\xi}=\ipb{\V{\xi}}{\V{g}^k}\,\V{g}_k
		=\xi^k \V{g}_k \comma
	\end{equation}
	也可以用逆变基表示:
	\begin{equation}
		\V{\xi}=\ipb{\V{\xi}}{\V{g}_k}\,\V{g}^k
		=\xi_k \V{g}^k \comma
	\end{equation}
	式中,$\xi^k$ 是 $\V{\xi}$ 与第 $k$ 个\emphB{逆变基}做内积的结果,
	称为 $\V{\xi}$ 的第 $k$ 个\emphA{逆变分量};
	而 $\xi_k$ 是 $\V{\xi}$ 与第 $k$ 个\emphB{协变基}做内积的结果,
	称为 $\V{\xi}$ 的第 $k$ 个\emphA{协变分量}。
	
	以后凡是指标在下的(下标),均称为\emphB{协变}某某;
	指标在上的(上标),称为\emphB{逆变}某某。
	
\section{张量的表示}
	所谓\emphA{张量},即指\emphA{多重线性函数}。
	
	以三阶张量为例。考虑任意的 $\Tens{\Phi}\in\Tensors{3}$,
	其中的 $\Tensors{3}$ 表示以 $\Rm$ 为底空间的三阶张量全体。
	所谓三阶(或三重)线性函数,指“吃掉”三个向量之后变成数,
	并且“吃法”具有线性性。

	对于一般地张量空间 $\Tensors{r}$,我们引入了线性结构:
	\begin{equation}
		\forall\,\alpha,\,\beta\in\realR,\,
		\Tens{\Phi},\,\Tens{\Psi}\in\Tensors{r},\quad
		\qty(\alpha\Tens{\Phi}+\beta\Tens{\Psi})
		\qty(\V{u}_1,\,\V{u}_2,\,\cdots,\,\V{u}_r)
		\defeq \alpha\Tens{\Phi}\qty(\V{u}_1,\,\V{u}_2,\,\cdots,\,\V{u}_r)
			+\beta\Tens{\Psi}\qty(\V{u}_1,\,\V{u}_2,\,\cdots,\,\V{u}_r)
		\comma
	\end{equation}
	于是
	\begin{equation}
		\alpha\Tens{\Phi}+\beta\Tens{\Psi} \in \Tensors{r} \fullstop
	\end{equation}
	
	下面我们要获得 $\Tens{\Phi}$ 的表示。
	根据之前任意向量用协变基或逆变基的表示,有
	\begin{align}
		\forall\,\V{u},\,\V{v},\,\V{w}\in\Rm,
		&\mathrel{\phantom{=}}
		\Tens{\Phi}\qty(\V{u},\,\V{v},\,\V{w}) \notag\\
		&=\Tens{\Phi}\qty(u^i\V{g}_i,\,v_j\V{g}^j,\,w^k\V{g}_k) \notag
		\intertext{考虑到 $\Tens{\Phi}$ 对第一变元的线性性,可得}
		&=u^i\,\Tens{\Phi}\qty(\V{g}_i,\,v_j\V{g}^j,\,w^k\V{g}_k) \notag
		\intertext{同理,}
		&=u^i v_j w^k\,\Tens{\Phi}\qty(\V{g}_i,\,\V{g}^j,\,\V{g}_k)
		\fullstop
		\label{eq:张量的表示1}
	\end{align}
	注意这里自然需要满足 Einstein 求和约定。
	
	上式中的 $\Tens{\Phi}\qty(\V{g}_i,\,\V{g}^j,\,\V{g}_k)$ 是一个数。
	它是张量 $\Tens{\Phi}$ “吃掉”三个基向量的结果。
	至于 $u^i v_j w^k$ 部分,三项分别是 $\V{u}$ 的第 $i$ 个逆变分量、
	$\V{v}$ 的第 $j$ 个协变分量和 $\V{w}$ 的第 $k$ 个逆变分量。
	根据向量分量的定义,可知
	\begin{equation}
		u^i v_j w^k
		= \ipb{\V{u}}{\V{g}^i}{\Rm}
		\cdot \ipb{\V{v}}{\V{g}_j}{\Rm}
		\cdot \ipb{\V{w}}{\V{g}^k}{\Rm} \fullstop
		\label{eq:张量的表示2}
	\end{equation}
	
	暂时中断一下思路,先给出\emphA{简单张量}的定义。
	\begin{equation}
		\forall\,\V{u},\,\V{v},\,\V{w}\in\Rm,\quad
		\V{\xi}\otimes\V{\eta}\otimes\V{\zeta}\qty(\V{u},\,\V{v},\,\V{w})
		\defeq \ipb{\V{\xi}}{\V{u}}{\Rm}
		\cdot \ipb{\V{\eta}}{\V{v}}{\Rm}
		\cdot \ipb{\V{\zeta}}{\V{w}}{\Rm} \in\realR \comma
	\end{equation}
	式中 $\V{\xi},\,\V{\eta},\,\V{\zeta}\in\Rm$,
	而暂时把 $\V{\xi}\otimes\V{\eta}\otimes\V{\zeta}$ 理解为一种记号。
	简单张量作为一个映照,组成它的三个向量分别与它们“吃掉”的第一、二、三个变元
	做内积并相乘,结果为一个实数。
	
	考虑到内积的线性性,便有(以第二个变元为例)
	\begin{align}
		\V{\xi}\otimes\V{\eta}\otimes\V{\zeta}
		\qty(\V{u},\,\alpha\tilde{\V{v}}+\beta\hat{\V{v}},\,\V{w})
		&\defeq \ipb{\V{\xi}}{\V{u}}{\Rm}
		\cdot \ipb{\V{\eta}}{\alpha\tilde{\V{v}}+\beta\hat{\V{v}}}{\Rm}
		\cdot \ipb{\V{\zeta}}{\V{w}}{\Rm} \in\realR \notag
		\intertext{注意到
			$\ipb{\V{\eta}}{\alpha\tilde{\V{v}}+\beta\hat{\V{v}}}{\Rm}
				=\alpha\ipb{\V{\eta}}{\tilde{\V{v}}}{\Rm}
				+\beta\ipb{\V{\eta}}{\hat{\V{v}}}{\Rm}$,
			同时再次利用简单张量的定义,可得}
		&= \alpha \V{\xi}\otimes\V{\eta}\otimes\V{\zeta}
			\qty(\V{u},\,\tilde{\V{v}},\,\V{w})
			+\beta \V{\xi}\otimes\V{\eta}\otimes\V{\zeta}
			\qty(\V{u},\,\hat{\V{v}},\,\V{w}) \fullstop
	\end{align}
	类似地,对第一变元和第三变元,同样具有线性性。因此,可以知道
	\begin{equation}
		\V{\xi}\otimes\V{\eta}\otimes\V{\zeta}
		\in\Tensors{3} \fullstop
	\end{equation}
	可见,“简单张量”的名字是名副其实的,它的确是一个特殊的张量。
	
	回过头来看 \eqref{eq:张量的表示2}~式。很明显,它可以用简单张量来表示。
	要注意,由于内积的对称性,可以有两种\footnote{%
		这里只考虑把 $\V{u}$、$\V{v}$、$\V{w}$%
		和 $\V{g}^i$、$\V{g}_j$、$\V{g}^k$ 分别放在一起的情况。}表示方法:
	\begin{gather}
		\V{g}^i\otimes\V{g}_j\otimes\V{g}^k
		\qty(\V{u},\,\V{v},\,\V{w})
		\intertext{或者}
		\V{u}\otimes\V{v}\otimes\V{w}
		\qty(\V{g}^i,\,\V{g}_j,\,\V{g}^k) \comma
	\end{gather}
	我们这里取上面一种。代入式~\eqref{eq:张量的表示1},得
	\begin{align}
		&\mathrel{\phantom{=}}
			\Tens{\Phi}\qty(\V{u},\,\V{v},\,\V{w}) \notag\\
		&=\Tens{\Phi}\qty(\V{g}_i,\,\V{g}^j,\,\V{g}_k)
			\cdot\V{g}^i\otimes\V{g}_j\otimes\V{g}^k
			\qty(\V{u},\,\V{v},\,\V{w}) \notag
		\intertext{由于
			$\Tens{\Phi}\qty(\V{g}_i,\,\V{g}^j,\,\V{g}_k) \in\Rm$,因此}
		&=\qty[\Tens{\Phi}\qty(\V{g}_i,\,\V{g}^j,\,\V{g}_k)
			\V{g}^i\otimes\V{g}_j\otimes\V{g}^k]
			\qty(\V{u},\,\V{v},\,\V{w}) \fullstop
	\end{align}
	方括号里的部分,就是根据 Einstein 求和约定,
	用 $\Tens{\Phi}\qty(\V{g}_i,\,\V{g}^j,\,\V{g}_k)$
	对 $\V{g}^i\otimes\V{g}_j\otimes\V{g}^k$ 进行线性组合。
	
	由于 $\V{u}$、$\V{v}$、$\V{w}$ 选取的任意性,可以引入如下记号:
	\begin{equation}
		\Tens{\Phi}
		=\Tens{\Phi}\qty(\V{g}_i,\,\V{g}^j,\,\V{g}_k) \,
			\V{g}^i\otimes\V{g}_j\otimes\V{g}^k
		\eqcolon \tensor{\Tens{\Phi}}{_i^j_k} \,
			\V{g}^i\otimes\V{g}_j\otimes\V{g}^k \comma
	\end{equation}
	即
	\begin{equation}
		\tensor{\Tens{\Phi}}{_i^j_k}
		\coloneqq \Tens{\Phi}\qty(\V{g}_i,\,\V{g}^j,\,\V{g}_k) \comma
	\end{equation}
	这称为张量的\emphA{分量}。
	它说明一个张量可以用\emphB{张量分量}和基向量组成的\emphB{简单张量}来表示。