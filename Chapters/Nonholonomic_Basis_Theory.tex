\section{完整基与非完整基的概念}
在 \ref{sec:局部基}~节中,我们利用曲线坐标系 $\V{X}(\V{x})$
构造了 $\Rm$ 上的一组(局部协变)基
\begin{equation}
	\qty{\V{g}_i(\V{x})=\pdv{\V{X}}{x^i} (\V{x})}^m_{i=1}
	\subset\Rm \comma
\end{equation}
它们称为\emphA{完整基}。
与之对应,不是由曲线坐标系诱导的基,称为\emphA{非完整基}。

\begin{figure}[h]
	\centering
	\includegraphics{Images/Holonomic_Nonholonomic_Basis.PNG}
	\caption{完整基与非完整基}
	\label{fig:完整基与非完整基}
\end{figure}

如图~\ref{fig:完整基与非完整基},
$x^i$-线的\emphB{切向量}构成一组局部协变基
$\qty{\V{g}_i(\V{x})}^m_{i=1}$,
它和它的对偶 $\qty{\V{g}^i(\V{x})}^m_{i=1}$ 都是完整基。
除此以外,我们当然可以选取另外的基 $\qty{\V{g}_{(i)}(\V{x})}^m_{i=1}$
和 $\qty{\V{g}^{(i)}(\V{x})}^m_{i=1}$,它们不是由曲线坐标系诱导,
因而是非完整基。

\section{非完整基下的张量梯度} \label{sec:非完整基下的张量梯度}
下面我们来考察张量梯度在非完整基下的表达形式。
在 \ref{sec:张量场的梯度}~节中,我们已经推导出了张量场的(右)梯度:
\begin{equation}
	\qty\big(\T{\Phi}\tp\opGrad) (\V{x})
	\defeq \pdv{\T{\Phi}}{x^\mu} (\V{x})
		\tp \V{g}^\mu(\V{x})
	=\coD{\mu}{\tc{\Phi}{^i_j^k} (\V{x})} \,
		\V{g}_i (\V{x}) \tp \V{g}^j (\V{x})
		\tp \V{g}_k (\V{x}) \tp \V{g}^\mu (\V{x}) \fullstop
	\label{eq:张量梯度_非完整基}
\end{equation}
这是一个四阶张量,对应的张量分量可记作
\begin{equation}
	\tc{\qty\big(\T{\Phi}\tp\opGrad)}{^i_j^k_\mu} (\V{x})
	\coloneq \coD{\mu}{\tc{\Phi}{^i_j^k} (\V{x})} \fullstop
	\label{eq:张量梯度分量_非完整基}
\end{equation}
除此以外,其他的基当然也可以用来表示该张量,
比如前文提到过的 $\qty{\V{g}_{(i)}(\V{x})}^m_{i=1}$
和 $\qty{\V{g}^{(i)}(\V{x})}^m_{i=1}$,它们都是非完整基。

非完整基与完整基之间的关系,可以利用
\ref{subsec:相对不同基的张量分量之间的关系}~小节中引入的%
\emphB{坐标转换关系}来获得:
\begin{braceEq}
	\V{g}_{(i)} (\V{x})
		&= c^k_{(i)} (\V{x}) \, \V{g}_k (\V{x}) \comma \\
	\V{g}^{(i)} (\V{x})
		&= c_k^{(i)} (\V{x}) \, \V{g}^k (\V{x}) \semicolon \\
	\V{g}_i (\V{x})
		&= c^{(k)}_i (\V{x}) \, \V{g}_{(k)} (\V{x}) \comma \\
	\V{g}^i (\V{x})
		&= c_{(k)}^i (\V{x}) \, \V{g}^{(k)} (\V{x}) \fullstop
\end{braceEq}
\myPROBLEM[2017-01-20]{坐标转换关系}\\
其中的基转换系数都是已知量,它们的定义如下:\footnote{
	只有两个基转换系数的原因是内积具有交换律。}
\begin{braceEq}
	c^j_{(i)} (\V{x})
		&\coloneq \ipb{\V{g}_{(i)} (\V{x})}{\V{g}^j (\V{x})} \comma \\
	c_j^{(i)} (\V{x})
		&\coloneq \ipb{\V{g}^{(i)} (\V{x})}{\V{g}_j (\V{x})} \fullstop
\end{braceEq}
代入 \eqref{eq:张量梯度_非完整基}~式,可有\footnote{
	这里我们省略了“$(\V{x})$”。}
\begin{align}
	\T{\Phi}\tp\opGrad
	&=\coD{\mu}{\tc{\Phi}{^i_j^k}} \,
		\qty(\vphantom{\frac{0}{0}}
			\V{g}_i \tp \V{g}^j \tp \V{g}_k \tp \V{g}^\mu) \notag \\
	&=\coD{\mu}{\tc{\Phi}{^i_j^k} (\V{x})} \,
		\qty[\vphantom{\frac{0^0}{0^0}}
			\qty(c^{(p)}_i \, \V{g}_{(p)})
			\tp \qty(c_{(q)}^j \, \V{g}^{(q)})
			\tp \qty(c^{(r)}_k \, \V{g}_{(r)})
			\tp \qty(\vphantom{\frac{0}{0}}
				c_{(\alpha)}^\mu \, \V{g}^{(\alpha)})] \notag
	\intertext{根据线性性,提出系数:}
	&=\qty(c^{(p)}_i c_{(q)}^j c^{(r)}_k c_{(\alpha)}^\mu
			\coD{\mu}{\tc{\Phi}{^i_j^k}})
		\qty(\vphantom{\frac{0}{0}}
			\V{g}_{(p)} \tp \V{g}^{(q)} \tp \V{g}_{(r)}
			\tp \V{g}^{(\alpha)}) \notag
	\intertext{写成张量分量与简单张量“乘积”的形式,即为}
	&\eqcolon \tc{\qty\big(\T{\Phi}\tp\opGrad)}%
			{^{(p)}_{\!(q)}^{\!(r)}_{\!(\alpha)}} \,
		\qty(\vphantom{\frac{0}{0}}
			\V{g}_{(p)} \tp \V{g}^{(q)} \tp \V{g}_{(r)}
			\tp \V{g}^{(\alpha)}) \fullstop
\end{align}
这样,我们就获得了非完整基下张量梯度的表示。
再利用式~\eqref{eq:张量梯度分量_非完整基},可知
\begin{equation}
	\tc{\qty\big(\T{\Phi}\tp\opGrad)}%
		{^{(p)}_{\!(q)}^{\!(r)}_{\!(\alpha)}} \,
	=c^{(p)}_i c_{(q)}^j c^{(r)}_k c_{(\alpha)}^\mu \,
		\tc{\qty(\vphantom{0^0}
			\T{\Phi}\tp\opGrad)}{^i_j^k_\mu} \fullstop
	\label{eq:非完整基下的张量梯度分量}
\end{equation}
以上结果与 \ref{subsec:相对不同基的张量分量之间的关系}~小节
中的推导是完全一致的。

\section{非完整基的形式运算}
在 \ref{sec:非完整基下的张量梯度}~节中,
我们利用\emphB{坐标转换关系}获得了张量梯度在非完整基下的表示。
而在本节,我们将通过定义,建立所谓“形式理论”,
获得一套更统一、更连贯的表述。

\myPROBLEM[2017-01-31]{统一、连贯?}

首先需要给出一些定义。

\begin{myEnum}
\item \emphA{形式偏导数}:
\begin{equation}
	\pdv{x^{(\mu)}} \defeq c^m_{(\mu)} \pdv{x^m} \fullstop
	\label{eq:形式偏导数}
\end{equation}
注意 $\pdv*{x^{(\mu)}}$ 本身是不能用极限形式来定义的,
因为曲线坐标系中并不存在有 $x^{(\mu)}$ 坐标线。

\item \emphA{形式 Christoffel 符号}:
\begin{equation}
	\ChrB{(\alpha)}{(\beta)}{(\gamma)}
	\defeq c^i_{(\alpha)} c^j_{(\beta)} c^{(\gamma)}_k \ChrB{i}{j}{k}
		-c^i_{(\alpha)} c^j_{(\beta)} \pdv{c^{(\gamma)}_j}{x^i}
	=c^i_{(\alpha)} c^j_{(\beta)}
		\qty(c^{(\gamma)}_k \ChrB{i}{j}{k}
			-\pdv{c^{(\gamma)}_j}{x^i}) \fullstop
\end{equation}
\myPROBLEM[2017-01-31]{第一类形式 Christoffel 符号}

\item \emphA{形式协变导数}。我们以三阶张量 $\T{\Phi}$ 为例给出定义。
$\T{\Phi}$ 在非完整基下可以用混合分量表示如下:
\begin{equation}
	\tc{\Phi}{^{(\alpha)}_{\!(\beta)}^{\!(\gamma)}}
	\coloneq \T{\Phi} \qty(\V{g}^{(\alpha)},\,\V{g}_{(\beta)},\,
		\V{g}^{(\gamma)}) \fullstop
\end{equation}
它相对 $x^{(\mu)}$ 分量的形式协变导数为
\begin{equation}
	\coD{(\mu)}{\tc{\Phi}{^{(\alpha)}_{\!(\beta)}^{\!(\gamma)}}}
	\defeq \pdv{\tc{\Phi}{^{(\alpha)}_{\!(\beta)}^{\!(\gamma)}}}%
		{x^{(\mu)}}
	+\ChrB{(\mu)}{(\sigma)}{(\alpha)}
		\tc{\Phi}{^{(\sigma)}_{\!(\beta)}^{\!(\gamma)}}
	-\ChrB{(\mu)}{(\beta)}{(\sigma)}
		\tc{\Phi}{^{(\alpha)}_{\!(\sigma)}^{\!(\gamma)}}
	+\ChrB{(\mu)}{(\sigma)}{(\gamma)}
		\tc{\Phi}{^{(\alpha)}_{\!(\beta)}^{\!(\sigma)}} \fullstop
	\label{eq:形式协变导数}
\end{equation}
回顾 \ref{sec:张量场的偏导数_协变导数}~节,
\eqref{eq:协变导数定义}~式给出了完整基下协变导数的定义:
\begin{equation}
	\coD{\mu}{\tc{\Phi}{^i_j^k}} \defeq
	\pdv{\tc{\Phi}{^i_j^k}}{x^\mu}
	+\ChrB{\mu}{s}{i} \tc{\Phi}{^s_j^k}
	-\ChrB{\mu}{j}{s} \tc{\Phi}{^i_s^k}
	+\ChrB{\mu}{s}{k} \tc{\Phi}{^i_j^s} \fullstop
\end{equation}
可以看出\emphB{形式}协变导数的定义与它几乎一模一样。
\end{myEnum}

\blankline

接下来我们要证明
\begin{equation}
	\coD{(\mu)}{\tc{\Phi}{^{(\alpha)}_{\!(\beta)}^{\!(\gamma)}}}
	=c^m_{(\mu)} c^{(\alpha)}_i c^j_{(\beta)} c^{(\gamma)}_k \,
		\coD{m}{\tc{\Phi}{^i_j^k}} \fullstop
\end{equation}
代入式~\eqref{eq:张量梯度分量_非完整基} 和
\eqref{eq:非完整基下的张量梯度分量},即得
\begin{equation}
	\coD{(\mu)}{\tc{\Phi}{^{(\alpha)}_{\!(\beta)}^{\!(\gamma)}}}
	=\tc{\qty(\vphantom{0^0}
		\T{\Phi}\tp\opGrad)}{^{(p)}_{\!(q)}^{\!(r)}_{\!(\alpha)}}
	\fullstop
\end{equation}
换句话说,此处我们正是要验证这种“形式理论”与
\ref{sec:非完整基下的张量梯度}~节中坐标转换关系的一致性。

\begin{myProof}
左边按照 \eqref{eq:形式协变导数}~式展开,第一项为
\begin{align}
	\pdv{\tc{\Phi}{^{(\alpha)}_{\!(\beta)}^{\!(\gamma)}}}%
		{x^{(\mu)}}
	&=c^m_{(\mu)} \pdv{\tc{\Phi}{^{(\alpha)}_{\!(\beta)}%
			^{\!(\gamma)}}}{x^m} \notag
	\intertext{这里用到了形式偏导数的定义 \eqref{eq:形式偏导数}~式。
	然后利用坐标转换关系展开张量分量:}
	&=c^m_{(\mu)} \pdv{x^m}
		\qty(c^{(\alpha)}_i c^j_{(\beta)} c^{(\gamma)}_k
			\tc{\Phi}{^i_j^k}) \notag
	\intertext{按照通常的偏导数法则直接打开:}
	&=c^m_{(\mu)} c^j_{(\beta)} c^{(\gamma)}_k
			\pdv{c^{(\alpha)}_i}{x^m} \tc{\Phi}{^i_j^k}
		+c^m_{(\mu)} c^{(\alpha)}_i c^{(\gamma)}_k
			\pdv{c^j_{(\beta)}}{x^m} \tc{\Phi}{^i_j^k}
		+c^m_{(\mu)} c^{(\alpha)}_i c^j_{(\beta)}
			\pdv{c^{(\gamma)}_k}{x^m} \tc{\Phi}{^i_j^k} \notag \\*
	&\alspace{}
		+c^m_{(\mu)} c^{(\alpha)}_i c^j_{(\beta)} c^{(\gamma)}_k
			\pdv{\tc{\Phi}{^i_j^k}}{x^m} \notag \\
	&=c^m_{(\mu)} \tc{\Phi}{^i_j^k}
		\qty(c^j_{(\beta)} c^{(\gamma)}_k \pdv{c^{(\alpha)}_i}{x^m}
			+c^{(\alpha)}_i c^{(\gamma)}_k \pdv{c^j_{(\beta)}}{x^m}
			+c^{(\alpha)}_i c^j_{(\beta)} \pdv{c^{(\gamma)}_k}{x^m})
		\notag \\*
	&\alspace{}
		+c^m_{(\mu)} c^{(\alpha)}_i c^j_{(\beta)} c^{(\gamma)}_k
			\pdv{\tc{\Phi}{^i_j^k}}{x^m} \fullstop
	\label{eq:张量分量偏导数_非完整基形式运算}
\end{align}

接下来处理含有形式 Christoffel 符号的三项,分别是
\begin{mySubEq}
	\begin{align}
		\ChrB{(\mu)}{(\sigma)}{(\alpha)}
			\tc{\Phi}{^{(\sigma)}_{\!(\beta)}^{\!(\gamma)}}
		&=c^p_{(\mu)} c^q_{(\sigma)}
			\qty(c^{(\alpha)}_s \ChrB{p}{q}{s} - \pdv{c^{(\alpha)}_q}{x^p})
			\cdot \tc{\Phi}{^{(\sigma)}_{\!(\beta)}^{\!(\gamma)}} \notag \\
		&=c^p_{(\mu)} \hl{c^q_{(\sigma)}}
			\qty(c^{(\alpha)}_s \ChrB{p}{q}{s} - \pdv{c^{(\alpha)}_q}{x^p})
			\cdot \hl{c^{(\sigma)}_i} c^j_{(\beta)} c^{(\gamma)}_k
				\tc{\Phi}{^i_j^k} \notag
		\intertext{根据式~\eqref{eq:坐标转换系数的乘积},我们有
			$c^q_{(\sigma)} c^{(\sigma)}_i = \KroneckerDelta{q}{i}$,于是}
		&=c^p_{(\mu)} \tc{\Phi}{^i_j^k}
			\qty(c^{(\alpha)}_s c^j_{(\beta)} c^{(\gamma)}_k \ChrB{p}{i}{s}
				-c^j_{(\beta)} c^{(\gamma)}_k \pdv{c^{(\alpha)}_i}{x^p})
		\semicolon \\
		%
		-\ChrB{(\mu)}{(\beta)}{(\sigma)}
			\tc{\Phi}{^{(\alpha)}_{\!(\sigma)}^{\!(\gamma)}}
		&=-c^p_{(\mu)} c^q_{(\beta)}
			\qty(c^{(\sigma)}_s \ChrB{p}{q}{s} - \pdv{c^{(\sigma)}_q}{x^p})
			\cdot \tc{\Phi}{^{(\alpha)}_{\!(\sigma)}^{\!(\gamma)}} \notag \\
		&=-c^p_{(\mu)} c^q_{(\beta)}
			\qty(\hl{c^{(\sigma)}_s} \ChrB{p}{q}{s}
				-\pdv{c^{(\sigma)}_q}{x^p})
			\cdot c^{(\alpha)}_i \hl{c^j_{(\sigma)}} c^{(\gamma)}_k
				\tc{\Phi}{^i_j^k} \notag \\
		&=c^p_{(\mu)} \tc{\Phi}{^i_j^k}
			\qty(-c^{(\alpha)}_i c^q_{(\beta)} c^{(\gamma)}_k \ChrB{p}{q}{j}
				+c^{(\alpha)}_i c^q_{(\beta)} c^{(\gamma)}_k c^j_{(\sigma)}
					\pdv{c^{(\sigma)}_q}{x^p}) \semicolon \\
		%
		\ChrB{(\mu)}{(\sigma)}{(\gamma)}
			\tc{\Phi}{^{(\alpha)}_{\!(\beta)}^{\!(\sigma)}}
		&=c^p_{(\mu)} c^q_{(\sigma)}
			\qty(c^{(\gamma)}_s \ChrB{p}{q}{s} - \pdv{c^{(\gamma)}_q}{x^p})
			\cdot \tc{\Phi}{^{(\alpha)}_{\!(\beta)}^{\!(\sigma)}} \notag \\
		&=c^p_{(\mu)} \hl{c^q_{(\sigma)}}
			\qty(c^{(\gamma)}_s \ChrB{p}{q}{s} - \pdv{c^{(\gamma)}_q}{x^p})
			\cdot c^{(\alpha)}_i c^j_{(\beta)} \hl{c^{(\sigma)}_k}
				\tc{\Phi}{^i_j^k} \notag \\
		&=c^p_{(\mu)} \tc{\Phi}{^i_j^k}
			\qty(c^{(\alpha)}_i c^j_{(\beta)} c^{(\gamma)}_s \ChrB{p}{k}{s}
				-c^{(\alpha)}_i c^j_{(\beta)} \pdv{c^{(\gamma)}_k}{x^p})
		\fullstop
	\end{align}
\end{mySubEq}
以上三式都有公因子 $c^p_{(\mu)} \tc{\Phi}{^i_j^k}$。
为了进一步化简,不妨将哑标 $p$ 换为 $m$。这样可有
\begin{align}
	&\alspace \ChrB{(\mu)}{(\sigma)}{(\alpha)}
		\tc{\Phi}{^{(\sigma)}_{\!(\beta)}^{\!(\gamma)}}
	-\ChrB{(\mu)}{(\beta)}{(\sigma)}
		\tc{\Phi}{^{(\alpha)}_{\!(\sigma)}^{\!(\gamma)}}
	+\ChrB{(\mu)}{(\sigma)}{(\gamma)}
		\tc{\Phi}{^{(\alpha)}_{\!(\beta)}^{\!(\sigma)}} \notag \\
	&=c^m_{(\mu)} \tc{\Phi}{^i_j^k}
		\left[\vphantom{\pdv{c^{(\gamma)}_k}{x^m}} \qty(
			c^{(\alpha)}_s c^j_{(\beta)} c^{(\gamma)}_k \ChrB{m}{i}{s}
			-c^{(\alpha)}_i c^q_{(\beta)} c^{(\gamma)}_k \ChrB{m}{q}{j}
			+c^{(\alpha)}_i c^j_{(\beta)} c^{(\gamma)}_s \ChrB{m}{k}{s})
		\right. \notag \\
	&\alspace\phantom{c^m_{(\mu)} \tc{\Phi}{^i_j^k}\left[\right]}
		\left. {}-c^j_{(\beta)} c^{(\gamma)}_k \pdv{c^{(\alpha)}_i}{x^m}
			+c^{(\alpha)}_i c^q_{(\beta)} c^{(\gamma)}_k c^j_{(\sigma)}
				\pdv{c^{(\sigma)}_q}{x^m}
			-c^{(\alpha)}_i c^j_{(\beta)} \pdv{c^{(\gamma)}_k}{x^m}
		\right] \fullstop
\end{align}
该式与 \eqref{eq:张量分量偏导数_非完整基形式运算}~式相加,得
\begin{align}
	\coD{(\mu)}{\tc{\Phi}{^{(\alpha)}_{\!(\beta)}^{\!(\gamma)}}}
	&\defeq \pdv{\tc{\Phi}{^{(\alpha)}_{\!(\beta)}^{\!(\gamma)}}}%
		{x^{(\mu)}}
		+\ChrB{(\mu)}{(\sigma)}{(\alpha)}
			\tc{\Phi}{^{(\sigma)}_{\!(\beta)}^{\!(\gamma)}}
		-\ChrB{(\mu)}{(\beta)}{(\sigma)}
			\tc{\Phi}{^{(\alpha)}_{\!(\sigma)}^{\!(\gamma)}}
		+\ChrB{(\mu)}{(\sigma)}{(\gamma)}
			\tc{\Phi}{^{(\alpha)}_{\!(\beta)}^{\!(\sigma)}} \notag \\
	%
	&=c^m_{(\mu)} c^{(\alpha)}_i c^j_{(\beta)} c^{(\gamma)}_k
			\pdv{\tc{\Phi}{^i_j^k}}{x^m}
		+c^m_{(\mu)} \tc{\Phi}{^i_j^k} \left[
			\hl{c^j_{(\beta)} c^{(\gamma)}_k \pdv{c^{(\alpha)}_i}{x^m}}
			+c^{(\alpha)}_i c^{(\gamma)}_k \pdv{c^j_{(\beta)}}{x^m}
			+\hl[pink]{c^{(\alpha)}_i c^j_{(\beta)}
				\pdv{c^{(\gamma)}_k}{x^m}} \right. \notag \\*
	&\alspace
	\phantom{c^m_{(\mu)} c^{(\alpha)}_i c^j_{(\beta)} c^{(\gamma)}_k
			\pdv{\tc{\Phi}{^i_j^k}}{x^m}
		+c^m_{(\mu)} \tc{\Phi}{^i_j^k} \left[\right]}
		\left. {}
			+\qty(c^{(\alpha)}_s c^j_{(\beta)} c^{(\gamma)}_k \ChrB{m}{i}{s}
			-c^{(\alpha)}_i c^q_{(\beta)} c^{(\gamma)}_k \ChrB{m}{q}{j}
			+c^{(\alpha)}_i c^j_{(\beta)} c^{(\gamma)}_s \ChrB{m}{k}{s})
		\right. \notag \\*
	&\alspace
	\phantom{c^m_{(\mu)} c^{(\alpha)}_i c^j_{(\beta)} c^{(\gamma)}_k
			\pdv{\tc{\Phi}{^i_j^k}}{x^m}
		+c^m_{(\mu)} \tc{\Phi}{^i_j^k} \left[\right]}
		\left. {}
			-\hl{c^j_{(\beta)} c^{(\gamma)}_k \pdv{c^{(\alpha)}_i}{x^m}}
			+c^{(\alpha)}_i c^q_{(\beta)} c^{(\gamma)}_k c^j_{(\sigma)}
				\pdv{c^{(\sigma)}_q}{x^m}
			-\hl[pink]{c^{(\alpha)}_i c^j_{(\beta)}
				\pdv{c^{(\gamma)}_k}{x^m}} \right] \notag
	%
	\intertext{高亮部分相互抵消:}
	&=c^m_{(\mu)} c^{(\alpha)}_i c^j_{(\beta)} c^{(\gamma)}_k
			\pdv{\tc{\Phi}{^i_j^k}}{x^m}
		+c^m_{(\mu)} \tc{\Phi}{^i_j^k}
		\left[\vphantom{\pdv{c^{(\gamma)}_k}{x^m}} \qty(
			c^{(\alpha)}_s c^j_{(\beta)} c^{(\gamma)}_k \ChrB{m}{i}{s}
			-c^{(\alpha)}_i c^q_{(\beta)} c^{(\gamma)}_k \ChrB{m}{q}{j}
			+c^{(\alpha)}_i c^j_{(\beta)} c^{(\gamma)}_s \ChrB{m}{k}{s})
		\right. \notag \\
	&\alspace
	\phantom{c^m_{(\mu)} c^{(\alpha)}_i c^j_{(\beta)} c^{(\gamma)}_k
			\pdv{\tc{\Phi}{^i_j^k}}{x^m}
		+c^m_{(\mu)} \tc{\Phi}{^i_j^k} \left[\right]}
		\left. {}
			+c^{(\alpha)}_i c^{(\gamma)}_k \pdv{c^j_{(\beta)}}{x^m}
			+c^{(\alpha)}_i c^q_{(\beta)} c^{(\gamma)}_k c^j_{(\sigma)}
				\pdv{c^{(\sigma)}_q}{x^m} \right]
	\label{eq:非完整基形式理论推导}
\end{align}
注意到 $c^j_{(\beta)} = c^q_{(\beta)} \KroneckerDelta{j}{q}
	=c^q_{(\beta)} c^j_{(\sigma)} c^{(\sigma)}_q$,因此
\begin{equation}
	\pdv{c^j_{(\beta)}}{x^m}
	=\pdv{x^m} \qty(c^q_{(\beta)} c^j_{(\sigma)} c^{(\sigma)}_q)
	=c^j_{(\sigma)} c^{(\sigma)}_q \pdv{c^q_{(\beta)}}{x^m}
		+c^q_{(\beta)} c^{(\sigma)}_q \pdv{c^j_{(\sigma)}}{x^m}
		+c^q_{(\beta)} c^j_{(\sigma)} \pdv{c^{(\sigma)}_q}{x^m} \fullstop
\end{equation}
所以 \eqref{eq:非完整基形式理论推导}~式中最后一步的第二行就能够写成
\begin{align}
	&\alspace c^{(\alpha)}_i c^{(\gamma)}_k \pdv{c^j_{(\beta)}}{x^m}
		+c^{(\alpha)}_i c^q_{(\beta)} c^{(\gamma)}_k c^j_{(\sigma)}
			\pdv{c^{(\sigma)}_q}{x^m} \notag \\
	&=c^{(\alpha)}_i c^{(\gamma)}_k
		\qty(\pdv{c^j_{(\beta)}}{x^m}
			+c^q_{(\beta)} c^j_{(\sigma)} \pdv{c^{(\sigma)}_q}{x^m})
		\notag \\
	&=c^{(\alpha)}_i c^{(\gamma)}_k \qty(
			\hl{c^j_{(\sigma)}} c^{(\sigma)}_q \pdv{c^q_{(\beta)}}{x^m}
			+\hl[pink]{c^q_{(\beta)}} c^{(\sigma)}_q
				\pdv{c^j_{(\sigma)}}{x^m}
			+c^q_{(\beta)} \hl{c^j_{(\sigma)}} \pdv{c^{(\sigma)}_q}{x^m}
			+\hl[pink]{c^q_{(\beta)}} c^j_{(\sigma)}
				\pdv{c^{(\sigma)}_q}{x^m}) \notag
	\intertext{合并同类项:}
	&=c^{(\alpha)}_i c^{(\gamma)}_k \qty[
			c^j_{(\sigma)}
			\qty(c^{(\sigma)}_q \pdv{c^q_{(\beta)}}{x^m}
				+c^q_{(\beta)} \pdv{c^{(\sigma)}_q}{x^m})
		+c^q_{(\beta)}
			\qty(c^{(\sigma)}_q \pdv{c^j_{(\sigma)}}{x^m}
				+c^j_{(\sigma)} \pdv{c^{(\sigma)}_q}{x^m}) ] \notag \\
	&=c^{(\alpha)}_i c^{(\gamma)}_k \qty[
			\vphantom{\pdv{c^q_{(\beta)}}{x^m}}
			c^j_{(\sigma)} \pdv{x^m} \qty(c^{(\sigma)}_q c^q_{(\beta)})
			+c^q_{(\beta)} \pdv{x^m} \qty(c^{(\sigma)}_q c^j_{(\sigma)}) ]
		\notag
	\intertext{再次利用式~\eqref{eq:坐标转换系数的乘积},可得}
	&=c^{(\alpha)}_i c^{(\gamma)}_k \qty(
			c^j_{(\sigma)} \pdv{\KroneckerDelta{\sigma}{\beta}}{x^m}
			+c^q_{(\beta)} \pdv{\KroneckerDelta{j}{q}}{x^m} )
	=0 \fullstop
\end{align}
代回式~\eqref{eq:非完整基形式理论推导},有
\begin{align}
	\coD{(\mu)}{\tc{\Phi}{^{(\alpha)}_{\!(\beta)}^{\!(\gamma)}}}
	&=c^m_{(\mu)} c^{(\alpha)}_i c^j_{(\beta)} c^{(\gamma)}_k
			\pdv{\tc{\Phi}{^i_j^k}}{x^m}
		+c^m_{(\mu)} \tc{\Phi}{^i_j^k}
		\qty(\vphantom{\pdv{c^{(\gamma)}_k}{x^m}}
			c^{(\alpha)}_s c^j_{(\beta)} c^{(\gamma)}_k \ChrB{m}{i}{s}
			-c^{(\alpha)}_i c^q_{(\beta)} c^{(\gamma)}_k \ChrB{m}{q}{j}
			+c^{(\alpha)}_i c^j_{(\beta)} c^{(\gamma)}_s \ChrB{m}{k}{s})
		\notag \\
	&=c^m_{(\mu)} c^{(\alpha)}_i c^j_{(\beta)} c^{(\gamma)}_k
			\pdv{\tc{\Phi}{^i_j^k}}{x^m}
		+c^m_{(\mu)}
		\qty(\vphantom{\pdv{c^{(\gamma)}_k}{x^m}}
			c^{(\alpha)}_s c^j_{(\beta)} c^{(\gamma)}_k
				\ChrB{m}{i}{s} \tc{\Phi}{^i_j^k}
			-c^{(\alpha)}_i c^q_{(\beta)} c^{(\gamma)}_k
				\ChrB{m}{q}{j} \tc{\Phi}{^i_j^k}
			+c^{(\alpha)}_i c^j_{(\beta)} c^{(\gamma)}_s
				\ChrB{m}{k}{s} \tc{\Phi}{^i_j^k}) \notag
	\intertext{下面要对哑标进行重排。
		括号里的第一项:$s \leftrightarrow i$;
		第二项:$j \rightarrow s, q \rightarrow j$;
		第三项:$s \leftrightarrow k$。于是}
	&=c^m_{(\mu)} c^{(\alpha)}_i c^j_{(\beta)} c^{(\gamma)}_k
			\pdv{\tc{\Phi}{^i_j^k}}{x^m}
		+c^m_{(\mu)}
		\qty(\vphantom{\pdv{c^{(\gamma)}_k}{x^m}}
			c^{(\alpha)}_i c^j_{(\beta)} c^{(\gamma)}_k
				\ChrB{m}{s}{i} \tc{\Phi}{^s_j^k}
			-c^{(\alpha)}_i c^j_{(\beta)} c^{(\gamma)}_k
				\ChrB{m}{j}{s} \tc{\Phi}{^i_s^k}
			+c^{(\alpha)}_i c^j_{(\beta)} c^{(\gamma)}_k
				\ChrB{m}{s}{k} \tc{\Phi}{^i_j^s}) \notag \\
	&=c^m_{(\mu)} c^{(\alpha)}_i c^j_{(\beta)} c^{(\gamma)}_k
		\qty(\pdv{\tc{\Phi}{^i_j^k}}{x^m}
			+\ChrB{m}{s}{i} \tc{\Phi}{^s_j^k}
			-\ChrB{m}{j}{s} \tc{\Phi}{^i_s^k}
			+\ChrB{m}{s}{k} \tc{\Phi}{^i_j^s}) \notag \\
	&=c^m_{(\mu)} c^{(\alpha)}_i c^j_{(\beta)} c^{(\gamma)}_k \,
		\coD{m}{\tc{\Phi}{^i_j^k}} \fullstop
\end{align}
这就完成了证明。
\end{myProof}