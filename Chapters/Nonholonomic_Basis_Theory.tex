\section{完整基与非完整基的概念}
在 \ref{sec:局部基}~节中,我们利用曲线坐标系 $\V{X}(\V{x})$
构造了 $\Rm$ 上的一组(局部协变)基
\begin{equation}
	\qty{\V{g}_i(\V{x})=\pdv{\V{X}}{x^i} (\V{x})}^m_{i=1}
	\subset\Rm \comma
\end{equation}
它们称为\emphA{完整基}。
与之对应,不是由曲线坐标系诱导的基,称为\emphA{非完整基}。

\begin{figure}[h]
	\centering
	\includegraphics{Images/Holonomic_Nonholonomic_Basis.PNG}
	\caption{完整基与非完整基}
	\label{fig:完整基与非完整基}
\end{figure}

如图~\ref{fig:完整基与非完整基},
$x^i$-线的\emphB{切向量}构成一组局部协变基
$\qty{\V{g}_i(\V{x})}^m_{i=1}$,
它和它的对偶 $\qty{\V{g}^i(\V{x})}^m_{i=1}$ 都是完整基。
除此以外,我们当然可以选取另外的基 $\qty{\V{g}_{(i)}(\V{x})}^m_{i=1}$
和 $\qty{\V{g}^{(i)}(\V{x})}^m_{i=1}$,它们不是由曲线坐标系诱导,
因而是非完整基。

\section{非完整基下的张量梯度}
下面我们来考察张量梯度在非完整基下的表达形式。
在 \ref{sec:张量场的梯度}~节中,我们已经推导出了张量场的(右)梯度:
\begin{equation}
	\qty\big(\T{\Phi}\tp\opGrad) (\V{x})
	\defeq \pdv{\T{\Phi}}{x^\mu} (\V{x})
		\tp \V{g}^\mu(\V{x})
	=\coD{\mu}{\tc{\Phi}{^i_j^k} (\V{x})} \,
		\V{g}_i (\V{x}) \tp \V{g}^j (\V{x})
		\tp \V{g}_k (\V{x}) \tp \V{g}^\mu (\V{x}) \fullstop
	\label{eq:张量梯度_非完整基}
\end{equation}
这是一个四阶张量,对应的张量分量可记作
\begin{equation}
	\tc{\qty\big(\T{\Phi}\tp\opGrad)}{^i_j^k_\mu} (\V{x})
	\coloneq \coD{\mu}{\tc{\Phi}{^i_j^k} (\V{x})} \fullstop
	\label{eq:张量梯度分量_非完整基}
\end{equation}
除此以外,其他的基当然也可以用来表示该张量,
比如前文提到过的 $\qty{\V{g}_{(i)}(\V{x})}^m_{i=1}$
和 $\qty{\V{g}^{(i)}(\V{x})}^m_{i=1}$,它们都是非完整基。

非完整基与完整基之间的关系,可以利用
\ref{subsec:相对不同基的张量分量之间的关系}~小节中引入的%
\emphB{坐标转换关系}来获得:
\begin{braceEq}
	\V{g}_{(i)} (\V{x})
		&= c^k_{(i)} (\V{x}) \, \V{g}_k (\V{x}) \comma \\
	\V{g}^{(i)} (\V{x})
		&= c_k^{(i)} (\V{x}) \, \V{g}^k (\V{x}) \semicolon \\
	\V{g}_i (\V{x})
		&= c^{(k)}_i (\V{x}) \, \V{g}_{(k)} (\V{x}) \comma \\
	\V{g}^i (\V{x})
		&= c_{(k)}^i (\V{x}) \, \V{g}^{(k)} (\V{x}) \fullstop
\end{braceEq}
\myPROBLEM[2017-01-20]{坐标转换关系}
其中的基转换系数都是已知量,它们的定义如下:\footnote{
	只有两个基转换系数的原因是内积具有交换律。}
\begin{braceEq}
	c^j_{(i)} (\V{x})
		&\coloneq \ipb{\V{g}_{(i)} (\V{x})}{\V{g}^j (\V{x})} \comma \\
	c_j^{(i)} (\V{x})
		&= \ipb{\V{g}^{(i)} (\V{x})}{\V{g}_j (\V{x})} \fullstop
\end{braceEq}
代入 \eqref{eq:张量梯度_非完整基}~式,可有\footnote{
	这里我们省略了“$(\V{x})$”。}
\begin{align}
	\T{\Phi}\tp\opGrad
	&=\coD{\mu}{\tc{\Phi}{^i_j^k}} \,
		\qty(\vphantom{\frac{0}{0}}
			\V{g}_i \tp \V{g}^j \tp \V{g}_k \tp \V{g}^\mu) \notag \\
	&=\coD{\mu}{\tc{\Phi}{^i_j^k} (\V{x})} \,
		\qty[\vphantom{\frac{0^0}{0^0}}
			\qty(c^{(p)}_i \, \V{g}_{(p)})
			\tp \qty(c_{(q)}^j \, \V{g}^{(q)})
			\tp \qty(c^{(r)}_k \, \V{g}_{(r)})
			\tp \qty(\vphantom{\frac{0}{0}}
				c_{(\alpha)}^\mu \, \V{g}^{(\alpha)})] \notag
	\intertext{根据线性性,提出系数:}
	&=\qty(c^{(p)}_i c_{(q)}^j c^{(r)}_k c_{(\alpha)}^\mu
			\coD{\mu}{\tc{\Phi}{^i_j^k}})
		\qty(\vphantom{\frac{0}{0}}
			\V{g}_{(p)} \tp \V{g}^{(q)} \tp \V{g}_{(r)}
			\tp \V{g}^{(\alpha)}) \notag
	\intertext{写成张量分量与简单张量“乘积”的形式,即为}
	&\eqcolon \tc{\qty\big(\T{\Phi}\tp\opGrad)}%
			{^{(p)}_{\!(q)}^{\!(r)}_{\!(\alpha)}} \,
		\qty(\vphantom{\frac{0}{0}}
			\V{g}_{(p)} \tp \V{g}^{(q)} \tp \V{g}_{(r)}
			\tp \V{g}^{(\alpha)}) \fullstop
\end{align}
这样,我们就获得了非完整基下张量梯度的表示。
再利用式~\eqref{eq:张量梯度分量_非完整基},可知
\begin{equation}
	\tc{\qty\big(\T{\Phi}\tp\opGrad)}%
		{^{(p)}_{\!(q)}^{\!(r)}_{\!(\alpha)}} \,
	=c^{(p)}_i c_{(q)}^j c^{(r)}_k c_{(\alpha)}^\mu \,
		\tc{\qty(\vphantom{0^0}
			\T{\Phi}\tp\opGrad)}{^i_j^k_\mu} \fullstop
\end{equation}
以上结果与 \ref{subsec:相对不同基的张量分量之间的关系}~小节
中的推导是完全一致的。
