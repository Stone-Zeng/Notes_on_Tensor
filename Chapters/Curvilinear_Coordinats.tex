\section{微分同胚}
\subsection{双射}
设 $f$ 是集合 $A$ 到 $B$ 的映照。
如果 $A$ 中不同的元素有不同的像,
则称 $f$ 为\emphA{单射}(也叫“一对一”);
如果 $B$ 中每个元素都是 $A$ 中元素的像,则称 $f$ 为\emphA{满射};
如果 $f$ 既是单射又是满射,
则称 $f$ 为\emphA{双射}(也叫“一一对应”)。
三种情况的示意见图~\ref{fig:单射满射双射}。

\begin{figure}[h]
	\centering
	\includegraphics{Images/Three_Mappings.PNG}
	\caption{单射、满射与双射}
	\label{fig:单射满射双射}
\end{figure}

设开集 $\domD{\V{X}},\,\domD{\V{x}}\subset\Rm$,
它们之间存在\emphB{双射},即\emphB{一一对应}关系:
\begin{equation}
	\mmap{\V{X}(\V{x})}
		{\domD{\V{x}}\ni\V{x}=\mqty[x^1 \\ \vdots \\ x^m]}
		{\V{X}(\V{x})=\mqty[X^1 \\ \vdots \\ X^m](\V{x})
			\in\domD{\V{X}}} \fullstop
\end{equation}
由于该映照实现了 $\domD{\V{x}}$ 到 $\domD{\V{X}}$ 之间的双射,
因此它存在逆映照:
\begin{equation}
	\mmap{\V{x}(\V{X})}
		{\domD{\V{X}}\ni\V{X}=\mqty[X^1 \\ \vdots \\ X^m]}
		{\V{x}(\V{X})=\mqty[x^1 \\ \vdots \\ x^m](\V{X})
			\in\domD{\V{x}}} \fullstop
\end{equation}

我们把 $\domD{\V{X}}$ 称为\emphA{物理域},它是实际物理事件
发生的区域;$\domD{\V{x}}$ 则称为\emphA{参数域}。
由于物理域通常较为复杂,因此我们常把参数域取为规整的形状,
以便之后的处理。

设物理量 $f(\V{X})$ 定义在物理域
$\domD{\V{X}}\subset\Rm$ 上\footnote{
	实际的物理事件当然只会发生在三维 Euclid 空间中
	(只就“空间”而言),但在数学上也可以推广到 $m$ 维。
},则 $f$ 就定义了一个\emphA{场}:
\begin{equation}
	\mmap{f}
		{\domD{\V{X}}\ni\V{X}}
		{f(\V{X})} \fullstop
\end{equation}
所谓的“场”,就是自变量用\emphB{位置}刻画的映照。
它可以是\emphA{标量场},如温度、压强、密度等,
此时 $f(\V{X})\in\realR$;也可以是\emphA{向量场},
如速度、加速度、力等,此时 $f(\V{X})\in\Rm$;
对于更深入的物理、力学研究,往往还需引入\emphA{张量场},
此时 $f(\V{X})\in\Tensors{r}$。

$\V{X}$ 存在于物理域 $\domD{\V{X}}$ 中,我们称它为\emphA{物理坐标}。
由于上文已经定义了 $\domD{\V{x}}$ 到 $\domD{\V{X}}$ 之间的双射
(不是 $f$!),因此 $\domD{\V{x}}$ 中就有\emphB{唯一的}
$\V{x}$ 与 $\V{X}$ 相对应,
它称为\emphA{参数坐标}(也叫\emphA{曲线坐标})。
又因为物理域 $\domD{\V{X}}$ 上已经定义了场 $f(\V{X})$,
参数域中必然\emphB{唯一}存在场 $\tilde{f}(\V{x})$ 与之对应:
\begin{equation}
	\mmap{\tilde{f}}
		{\domD{\V{x}}\ni\V{x}}
		{\tilde{f}(\V{x})=f\comp\V{X}(\V{x})
			=f\qty\big(\V{X}(\V{x}))} \fullstop
\end{equation}
$\V{x}$ 与 $\V{X}$ 是完全等价的,
因而 $\tilde{f}$ 与 $f$ 也是完全等价的,所以同样有
\begin{equation}
	f(\V{X})=\tilde{f}\qty\big(\V{x}(\V{X})) \fullstop
\end{equation}

物理域中的场要满足\emphB{守恒定律},如质量守恒、动量守恒、
能量守恒等。从数学上看,这些守恒定律就是 $f(\V{X})$
需要满足的一系列\emphB{偏微分方程}。将场变换到参数域后,
它仍要满足这些方程。但我们已经设法将参数域取得较为规整,
故在其上进行数值求解就会相当方便。

\subsection{参数域方程} \label{subsec:参数域方程}
上文已经提到,物理域中的场 $f(\V{X})$ 需满足守恒定律,
这等价于一系列偏微分方程(PDE)。
在物理学和力学中,用到的 PDE 通常是\emphB{二阶}的,它们可以写成
\begin{equation}
	\forall\,\V{X}\in\domD{\V{X}},\quad
	\sum_{\alpha=1}^{m} A_\alpha(\V{X}) \pdv{f}{X^\alpha} (\V{X})
	+\sum_{\alpha=1}^{m}\sum_{\beta=1}^{m}
		B_{\alpha\beta}(\V{X}) \pdv{f}{X^\beta}{X^\alpha} (\V{X}) = 0
\end{equation}
的形式。我们的目标是把该\emphB{物理域}方程转化为\emphB{参数域}方程,
即关于 $\tilde{f}(\V{x})$ 的 PDE。
多元微积分中已经提供了解决方案:\emphA{链式求导法则}。

考虑到
\begin{equation}
	f(\V{X})=\tilde{f}\qty\big(\V{x}(\V{X}))
	=\tilde{f}\qty(x^1(\V{X}),\,\cdots,\,x^m(\V{X})) \comma
\end{equation}
于是有
\begin{equation}
	\pdv{f}{X^\alpha} (\V{X})
	=\sum_{s=1}^{m} \pdv{\tilde{f}}{\displaystyle x^s}
		\qty\big(\V{x}(\V{X})) \cdot
		\pdv{x^s}{X^\alpha} (\V{X}) \fullstop
	\label{eq:参数域方程_一阶偏导项}
\end{equation}
这里用到的链式法则,由\emphB{复合映照可微性定理}驱动,
它要求 $\tilde{f}$ 关于 $\V{x}$ 可微,同时 $\V{x}$ 关于
$\V{X}$ 可微。

\myPROBLEM{对于更高阶的项,往往需要更强的条件。}一般地,我们要求
\begin{braceEq}
	\V{X}(\V{x})&\in\cf{\domD{\V{x}}}{\Rm} \semicolon \\
	\V{x}(\V{X})&\in\cf{\domD{\V{X}}}{\Rm} \fullstop
\end{braceEq}
这里的 $\DiffMorp$ 指\emphB{直至 $p$ 阶偏导数存在且连续}的
映照全体;$p=1$ 时,它就等价于可微。至于 $p$ 的具体取值,
则由 PDE 的阶数所决定。

通常情况下,已知条件所给定的往往都是
$\domD{\V{x}}$ 到 $\domD{\V{X}}$ 的映照
\begin{equation}
	\mmap{\V{X}(\V{x})}
		{\domD{\V{x}}\ni\V{x}=\mqty[x^1 \\ \vdots \\ x^m]}
		{\V{X}(\V{x})=\mqty[X^1 \\ \vdots \\ X^m](\V{x})
			\in\domD{\V{X}}} \comma
\end{equation}
用它不好直接得到式~\eqref{eq:参数域方程_一阶偏导项} 中的
$\pdv*{x^s}{X^\alpha}$ 项,但获得它的“倒数”
$\pdv*{X^\alpha}{x^s}$ 却很容易,只需利用 \emphA{Jacobi 矩阵}:
\begin{equation}
	\JacobiD{\V{X}}(\V{x})
	\defeq\mqty[
		\displaystyle\pdv{X^1}{\displaystyle x^1} & \cdots &
			\displaystyle\pdv{X^1}{\displaystyle x^m} \\[1ex]
		\vdots & \ddots & \vdots \\[0.5ex]
		\displaystyle\pdv{X^m}{\displaystyle x^1} & \cdots &
			\displaystyle\pdv{X^m}{\displaystyle x^m}
		]\, (\V{x}) \in\realR^{m \times m} \comma
	\label{eq:参数域方程_Jacobi矩阵}
\end{equation}
它是一个方阵。

有了 Jacobi 矩阵,施加一些手法就可以得到所需要的
$\pdv*{x^s}{X^\alpha}$ 项。考虑到
\begin{equation}
	\forall\,\V{X}\in\domD{\V{X}},\quad
	\V{X}\qty\big(\V{x}(\V{X}))=\V{X} \comma
\end{equation}
并且其中的 $\V{X}(\V{x})$ 和 $\V{x}(\V{X})$ 均可微,可以得到
\begin{equation}
	\JacobiD{\V{X}}\qty\big(\V{x}(\V{X}))
	\cdot \JacobiD{\V{x}}(\V{X})
	=\Mat{I}_m \comma
\end{equation}
其中的 $\Mat{I}_m$ 是单位阵。因此
\begin{equation}
	\JacobiD{\V{x}}(\V{X})
	\defeq\mqty[
		\displaystyle\pdv{x^1}{\displaystyle X^1} & \cdots &
			\displaystyle\pdv{x^1}{\displaystyle X^m} \\[1ex]
		\vdots & \ddots & \vdots \\[0.5ex]
		\displaystyle\pdv{x^m}{\displaystyle X^1} & \cdots &
			\displaystyle\pdv{x^m}{\displaystyle X^m} ]\, (\V{X})
	=(\JacobiD{\V{X}})^{-1}(\V{x})
	=\mqty[
		\displaystyle\pdv{X^1}{\displaystyle x^1} & \cdots &
			\displaystyle\pdv{X^1}{\displaystyle x^m} \\[1ex]
		\vdots & \ddots & \vdots \\[0.5ex]
		\displaystyle\pdv{X^m}{\displaystyle x^1} & \cdots &
			\displaystyle\pdv{X^m}{\displaystyle x^m}
		]^{-1} (\V{x}) \fullstop
\end{equation}
用代数的方法总可以求出
\begin{equation}
	\varphi^s_\alpha\coloneq\pdv{x^s}{X^\alpha}(\V{X}) \comma
	\label{eq:Jacobi矩阵的元素}
\end{equation}
它是通过求逆运算确定的函数,
即位于矩阵 $\JacobiD{\V{x}}$ 第 $s$ 行第 $\alpha$ 列的元素。这样就有
\begin{equation}
	\pdv{f}{X^\alpha} (\V{X})
	=\sum_{s=1}^{m} \pdv{\tilde{f}}{\displaystyle x^s}
		\qty\big(\V{x}(\V{X})) \cdot
		\varphi^s_\alpha \qty\big(\V{x}(\V{X})) \fullstop
\end{equation}

接下来处理二阶偏导数。由上式,
\begin{align}
	\pdv{f}{X^\beta}{X^\alpha} (\V{X})
	&=\sum_{s=1}^{m} \left[\qty\Bigg(\sum_{k=1}^{m}
			\pdv{\tilde{f}}{\displaystyle x^k}{\displaystyle x^s}
			\qty\big(\V{x}(\V{X})) \cdot \pdv{x^s}{X^\beta} (\V{X}) )
		\cdot \varphi^s_\alpha \qty\big(\V{x}(\V{X}))
		\right. \notag \\*
	&\alspace\left. \phantom{\sum_{s=1}^{m}}+
		\pdv{\tilde{f}}{\displaystyle x^s}
		\qty\big(\V{x}(\V{X})) \cdot \qty\Bigg(\sum_{k=1}^{m}
			\pdv{\displaystyle \varphi^s_\alpha}{\displaystyle x^k}
			\qty\big(\V{x}(\V{X}))
			\cdot \pdv{\displaystyle x^k}{X^\beta} (\V{X}) )
		\right] \notag
	\intertext{继续利用式~\eqref{eq:Jacobi矩阵的元素},有}
	&=\sum_{s=1}^{m} \left[\qty\Bigg(\sum_{k=1}^{m}
			\pdv{\tilde{f}}{\displaystyle x^k}{\displaystyle x^s}
			\qty\big(\V{x}(\V{X})) \cdot
			\varphi^s_\beta \qty\big(\V{x}(\V{X})) )
		\cdot \varphi^s_\alpha \qty\big(\V{x}(\V{X}))
		\right. \notag \\*
	&\alspace\left. \phantom{\sum_{s=1}^{m}}+
		\pdv{\tilde{f}}{\displaystyle x^s}
		\qty\big(\V{x}(\V{X})) \cdot \qty\Bigg(\sum_{k=1}^{m}
			\pdv{\displaystyle \varphi^s_\alpha}{\displaystyle x^k}
			\qty\big(\V{x}(\V{X}))
			\cdot \varphi^k_\beta \qty\big(\V{x}(\V{X})) )
		\right] \fullstop
\end{align}
这样,就把一阶和二阶偏导数项全部用关于 $\V{x}$ 的函数\footnote{
	当然它仍然是 $\V{X}$ 的\emphB{隐}函数:
	$\V{x}=\V{x}(\V{X})$。}表达了出来。
换句话说,我们已经把\emphB{物理域}中 $f$ 关于 $\V{X}$ 的 PDE,
转化成了\emphB{参数域}中 $\tilde{f}$ 关于 $\V{x}$ 的 PDE。
这就是上文要实现的目标。

\subsection{微分同胚的定义}
上文已经指出了 $\domD{\V{x}}$ 到 $\domD{\V{X}}$ 的映照
$\V{X}(\V{x})$ 所需满足的一些条件。这里再次罗列如下:

\begin{myEnum}
\item $\domD{\V{X}},\,\domD{\V{x}}\subset\Rm$
均为\emphA{开集}\footnote{
	用形象化的语言来说,如果在区域中的任意一点都可以吹出一个球,
	并能使球上的每个点都落在区域内,那么这个区域就是\emphA{开集}。
	这是\emphB{复合映照可微性定理}的一个要求。};

\item 存在 $\domD{\V{x}}$ 同 $\domD{\V{X}}$ 之间的\emphA{双射}
$\V{X}(\V{x})$,即存在\emphA{一一对应}关系;

\item $\V{X}(\V{x})$ 和它的逆映照 $\V{x}(\V{X})$
满足一定的\emphA{正则性}要求。
\end{myEnum}

\myPROBLEM{对第3点要稍作说明。}

如果满足这三点,则称 $\V{X}(\V{x})$ 为 $\domD{\V{x}}$ 与
$\domD{\V{X}}$ 之间的 \emphA{$\DiffMorp$-微分同胚},
记为 $\V{X}(\V{x})\in\cf{\domD{\V{x}}}{\domD{\V{X}}}$。
把物理域中的一个部分对应到参数域上的一个部分,
需要的仅仅是\emphB{双射}这一条件;而要使得物理域中所满足的
PDE 能够转换到参数域上,就需要“过去”和“回来”
都满足 $p$ 阶偏导数连续的条件(即\emphB{正则性}要求)。

有了微分同胚,物理域中的位置就可用参数域中的位置等价地进行刻画。
因此我们也把微分同胚称为\emphA{曲线坐标系}。

\section{向量值映照的可微性} \label{sec:向量值映照的可微性}
\subsection{可微性的定义}
设 $\V{x}_0$ 是参数域 $\domD{\V{x}}$ 中的一个内点。
在映照 $\V{X}(\V{x})$ 的作用下,
它对应到物理域 $\domD{\V{X}}$ 中的点 $\V{X}\qty(\V{x}_0)$。
参数域是一个\emphB{开集}。根据开集的定义,
必然存在一个实数 $\lambda>0$,使得以 $\V{x}_0$ 为球心、
$\lambda$ 为半径的球能够完全落在定义域 $\domD{\V{x}}$ 内,即
\begin{equation}
	\domB{\lambda}{\V{x}_0}\subset\domD{\V{x}} \comma
\end{equation}
其中的 $\domB{\lambda}{\V{x}_0}$ 表示 $\V{x}_0$ 的 $\lambda$ 邻域。

如果 $\exists\,\JacobiD{\V{X}}\qty(\V{x}_0)\in\LinearT{\Rm}{\Rm}$
\footnote{正如之前已经定义的,$\JacobiD{\V{X}}$
	已经用来表示 Jacobi 矩阵。这里还是请先暂时将它视为一种记号,
	其具体形式将在下一小节给出。},满足
\begin{equation}
	\forall\,\V{x}_0+\V{h}\in\domB{\lambda}{\V{x}_0},\quad
	\V{X}\qty(\V{x}_0+\V{h})-\V{X}\qty(\V{x}_0)
	=\JacobiD{\V{X}}\qty(\V{x}_0)(\V{h})+\sO{\norm{\V{h}}}
	\in\Rm \comma
	\label{eq:向量值映照可微性的定义}
\end{equation}
则称向量值映照 $\V{X}(\V{x})$ 在 $\V{x}_0$ 点\emphA{可微}。
其中,$\LinearT{\Rm}{\Rm}$ 表示从 $\Rm$ 到 $\Rm$
的\emphA{线性变换}全体。

根据这个定义,所谓\emphB{可微性},指由自变量变化所引起的因变量变化,
可以用一个\emphB{线性变换}近似,而误差为\emphB{一阶}无穷小量。
自变量可见到因变量空间最简单的映照形式就是线性映照(线性变换),
因而具有可微性的向量值映照具有至关重要的作用。

\subsection{Jacobi 矩阵}
下面我们研究 $\JacobiD{\V{X}}\qty(\V{x}_0)\in\LinearT{\Rm}{\Rm}$
的表达形式。由于 $\V{h}\in\Rm$,所以
\begin{equation}
	\V{h}=\mqty[h^1 \\ \vdots \\ h^m]
	=h^1\V{e}_1+\cdots+h^i\V{e}_i+\cdots+h^m\V{e}_m \fullstop
\end{equation}
另一方面,$\JacobiD{\V{X}}\qty(\V{x}_0)\in\LinearT{\Rm}{\Rm}$
具有\emphB{线性性}:
\begin{equation}
	\forall\,\alpha,\,\beta \in\realR
		\text{\ 和\ } \tilde{\V{h}},\,\hat{\V{h}}\in\Rm,\quad
	\JacobiD{\V{X}}\qty(\V{x}_0)
		\qty(\alpha\tilde{\V{h}}+\beta\hat{\V{h}})
	=\alpha \JacobiD{\V{X}}\qty(\V{x}_0)\qty(\tilde{h})
		+\beta \JacobiD{\V{X}}\qty(\V{x}_0)\qty(\hat{h}) \fullstop
\end{equation}
这样就有
\begin{align}
	\JacobiD{\V{X}}\qty(\V{x}_0)\qty(\V{h})
	&=\JacobiD{\V{X}}\qty(\V{x}_0)
		\qty(h^1\V{e}_1+\cdots+h^i\V{e}_i+\cdots+h^m\V{e}_m) \notag \\
	&=h^1\JacobiD{\V{X}}\qty(\V{x}_0)\qty(\V{e}_1) + \cdots
		+h^i\JacobiD{\V{X}}\qty(\V{x}_0)\qty(\V{e}_i) + \cdots
		+h^m\JacobiD{\V{X}}\qty(\V{x}_0)\qty(\V{e}_m)
		\label{eq:推导Jacobi矩阵表达形式_Part1}
	\intertext{注意到 $h^i\in\realR$ 以及
		$\JacobiD{\V{X}}\qty(\V{x}_0)\qty(\V{e}_i)\in\Rm$,
		因而该式可以用矩阵形式表述:}
	&=\mqty[\JacobiD{\V{X}}\qty(\V{x}_0)\qty(\V{e}_1),\,\cdots,\,
			\JacobiD{\V{X}}\qty(\V{x}_0)\qty(\V{e}_m)]
		\mqty[h^1 \\ \vdots \\ h^m] \fullstop
\end{align}
最后一步要用到\emphB{分块矩阵}的思想:左侧的矩阵为 1“行” $m$ 列,
每一“行”是一个 $m$ 维列向量;右侧的矩阵(向量)则为 $m$ 行 1 列。
两者相乘,得到 1“行” 1 列的矩阵(当然实际为 $m$ 行),
即之前的 \eqref{eq:推导Jacobi矩阵表达形式_Part1}~式。
在线性代数中,$m \times m$ 的矩阵
$\mqty[\JacobiD{\V{X}}\qty(\V{x}_0)\qty(\V{e}_1) & \cdots
& \JacobiD{\V{X}}\qty(\V{x}_0)\qty(\V{e}_m)]$
通常称为\emphA{变换矩阵}(也叫\emphA{过渡矩阵})。

接下来要搞清楚变换矩阵的具体形式。取
\begin{equation}
	\V{h}=\mqty[0,\,\cdots,\,\lambda,\,\cdots,\,0]\trans
	=\lambda\,\V{e}_i\in\Rm \comma
\end{equation}
即除了 $\V{h}$ 的第 $i$ 个元素为 $\lambda$ 外,其余元素均为 0
($\lambda \neq 0$)。因而有 $\norm{\V{h}}=\lambda$。
代入可微性的定义 \eqref{eq:向量值映照可微性的定义}~式,可得
\begin{align}
	&\alspace\V{X}\qty(\V{x}_0+\V{h})-\V{X}\qty(\V{x}_0)
	=\V{X}\qty(\V{x}_0+\lambda\,\V{e}_i)
		-\V{X}\qty(\V{x}_0) \notag \\
	&=\mqty[\JacobiD{\V{X}}\qty(\V{x}_0)\qty(\V{e}_1),\,\cdots,\,
			\JacobiD{\V{X}}\qty(\V{x}_0)\qty(\V{e}_i),\,\cdots,\,
			\JacobiD{\V{X}}\qty(\V{x}_0)\qty(\V{e}_m)]
		\mqty[0,\,\cdots,\,\lambda,\,\cdots,\,0]\trans
		+\sO{\lambda} \notag \\
	&=\lambda\cdot\JacobiD{\V{X}}\qty(\V{x}_0)\qty(\V{e}_i)
		+\sO{\lambda} \fullstop
\end{align}
由于 $\lambda$ 是非零实数,故可以在等式两边同时除以
$\lambda$ 并取极限:
\begin{equation}
	\lim_{\lambda\to 0}
	\frac{\V{X}\qty(\V{x}_0+\lambda\,\V{e}_i)
		-\V{X}\qty(\V{x}_0)}{\lambda}
	=\JacobiD{\V{X}}\qty(\V{x}_0)\qty(\V{e}_i) \comma
\end{equation}
这里的 $\sO{\lambda}$ 根据其定义自然趋于 0。
该式左侧极限中的分子部分,是自变量 $\V{x}$ 第 $i$ 个分量的变化
所引起因变量的变化;而分母,则是自变量第 $i$ 个分量的变化大小。
我们引入下面的记号:
\begin{equation}
	\pdv{\V{X}}{x^i}\qty(\V{x}_0)
	\coloneq \lim_{\lambda\to 0}
	\frac{\V{X}\qty(\V{x}_0+\lambda\,\V{e}_i)
		-\V{X}\qty(\V{x}_0)}{\lambda} \in\Rm \comma
\end{equation}
它表示因变量 $\V{X}\in\Rm$ 作为一个\emphB{整体},
相对于自变量 $\V{x}\in\Rm$ 第 $i$ 个\emphB{分量} $x^i\in\realR$
的“变化率”,即 $\V{X}$ 关于 $x^i$(在 $\V{x}_0$ 处)%
的\emphA{偏导数}。由于我们没有定义向量的除法,
因此自变量作为\emphB{整体}所引起因变量的变化,是没有意义的。
利用偏导数的定义,可有
\begin{align}
	&\alspace\mqty[\JacobiD{\V{X}}\qty(\V{x}_0)\qty(\V{e}_1),\,\cdots,
		\,\JacobiD{\V{X}}\qty(\V{x}_0)\qty(\V{e}_i),\,\cdots,\,
		\JacobiD{\V{X}}\qty(\V{x}_0)\qty(\V{e}_m)] \notag \\
	&=\mqty[\displaystyle \pdv{\V{X}}{x^1}\qty(\V{x}_0),\,\cdots,\,
		\displaystyle \pdv{\V{X}}{x^i}\qty(\V{x}_0),\,\cdots,\,
		\displaystyle \pdv{\V{X}}{x^m}\qty(\V{x}_0)]
		\in\realR^{m \times m} \fullstop
\end{align}

\blankline

下面给出 $\pdv*{\V{X}}{x^i}\qty(\V{x}_0)$ 的计算式。根据定义,有
\begin{align}
	\pdv{\V{X}}{x^i}\qty(\V{x}_0)
	&\coloneq \lim_{\lambda\to 0}
		\frac{\V{X}\qty(\V{x}_0+\lambda\,\V{e}_i)
		-\V{X}\qty(\V{x}_0)}{\lambda} \in\Rm \notag \\
	&=\lim_{\lambda\to 0} \frac{1}{\lambda}\cdot
		\left(\vphantom{\mqty{0\\[0.8ex]0}}\right.\!
			\mqty[X^1 \\ \vdots \\ X^m] \qty(\V{x}_0+\lambda\,\V{e}_i)
			-\mqty[X^1 \\ \vdots \\ X^m] \qty(\V{x}_0)
		\!\!\left.\vphantom{\mqty{0\\[0.8ex]0}}\right) \notag \\
	&=\lim_{\lambda\to 0}\,
		\mqty[
			\dfrac{X^1\qty(\V{x}_0+\lambda\,\V{e}_i)-X^1\qty(\V{x}_0)}
				{\lambda} \\[1ex] \vdots \\[0.5ex]
			\dfrac{X^m\qty(\V{x}_0+\lambda\,\V{e}_i)-X^m\qty(\V{x}_0)}
				{\lambda} ] \fullstop
\end{align}
向量极限存在的\emphB{充要条件}是各分量极限均存在,即存在
\begin{equation}
	\pdv{X^\alpha}{x^i}\qty(\V{x}_0) \coloneq
	\lim_{\lambda\to 0}
	\frac{X^\alpha\qty(\V{x}_0+\lambda\,\V{e}_i)-X^\alpha\qty(\V{x}_0)}
		{\lambda\vphantom{X^1\qty(\V{x}_0)}}\in\realR \comma
\end{equation}
其中的 $\alpha=1,\,\cdots,\,m$。
这其实就是我们熟知的\emphB{多元函数}偏导数的定义。
用它来表示\emphB{向量值映照}的偏导数,可有
\begin{equation}
	\pdv{\V{X}}{x^i}\qty(\V{x}_0)
	=\mqty[\displaystyle \pdv{X^1}{x^i}\qty(\V{x}_0) \\[1ex]
		\vdots \\[0.7ex] \displaystyle \pdv{X^m}{x^i}\qty(\V{x}_0)]
	=\sum_{\alpha=1}^{m} \pdv{X^\alpha}{x^i}\qty(\V{x}_0) \,
		\V{e}_\alpha \fullstop
\end{equation}

向量值映照 $\V{X}$ 关于 $x^i$ 的偏导数,
从代数的角度来看,是 Jacobi 矩阵的第 $i$ 列;
从几何的角度来看,则是物理域中 $x^i$ 线的切向量;
从计算的角度来看,又是(该映照)每个分量偏导数的组合。

\blankline

现在我们重新回到 Jacobi 矩阵。情况已经十分明了:
只需把之前获得的各列并起来,就可以得到完整的 Jacobi 矩阵。于是
\begin{align}
	\JacobiD{\V{X}}\qty(\V{x}_0)\qty(\V{h})
	&=\mqty[\displaystyle \pdv{\V{X}}{x^1},\,\cdots,\,\pdv{\V{X}}{x^m}]
		\qty(\V{x}_0)\qty(\V{h}) \notag \\
	&=\,\mqty[
		\displaystyle\pdv{X^1}{\displaystyle x^1} & \cdots &
			\displaystyle\pdv{X^1}{\displaystyle x^m} \\[1ex]
		\vdots & \ddots & \vdots \\[0.5ex]
		\displaystyle\pdv{X^m}{\displaystyle x^1} & \cdots &
			\displaystyle\pdv{X^m}{\displaystyle x^m}
		]\, \qty(\V{x}_0) \cdot
		\mqty[h^1 \\ \vdots \\ h^m] \fullstop
\end{align}
这与 \ref{subsec:参数域方程}~小节中
\eqref{eq:参数域方程_Jacobi矩阵}~式给出的定义是完全一致的。

\subsection{偏导数的几何意义} \label{subsec:偏导数的几何意义}
这一小节中,我们要回过头来,澄清向量值映照偏导数的几何意义。

如图~\ref{fig:偏导数的几何意义},$\V{X}(\V{x})$ 是定义域空间
$\domD{\V{x}}\subset\Rm$ 到值域空间 $\domD{\V{X}}\subset\Rm$
的向量值映照。在定义域空间 $\domD{\V{x}}$ 中,
过点 $\V{x}_0$ 作一条平行于 $x^i$ 轴的直线,称为 \emphB{$x^i$-线}。
$x^i$ 轴定义了向量 $\V{e}_i$,因而 $x^i$-线上的任意一点均可表示为
$\V{x}_0+\lambda\,\V{e}_i$,其中 $\lambda\in\realR$。

\begin{figure}[h]
	\centering
	\includegraphics{Images/Vector-Value_Mapping.PNG}
	\caption{向量值映照偏导数的几何意义}
	\label{fig:偏导数的几何意义}
\end{figure}

在 $\V{X}(\V{x})$ 的作用下,点 $\V{x}_0$ 被映照到
$\V{X}\qty(\V{x}_0)$,而 $\V{x}_0+\lambda\,\V{e}_i$ 则被映照到了
$\V{X}\qty(\V{x}_0+\lambda\,\V{e}_i)$。这样一来,
$x^i$-线也就被映照到了值域空间 $\domD{\V{X}}$ 中,成为一条曲线。

根据前面的定义,当 $\lambda\to 0$ 时,
\begin{equation}
	\frac{\V{X}\qty(\V{x}_0+\lambda\,\V{e}_i) - \V{X}\qty(\V{x}_0)}
	{\lambda} \to \pdv{\V{X}}{x^i}\qty(\V{x}_0) \fullstop
\end{equation}
对应到图~\ref{fig:偏导数的几何意义} 中,就是 $x^i$-线
(值域空间中)在 $\V{X}\qty(\V{x}_0)$ 处的\emphA{切向量}。

完全类似,在定义域空间 $\domD{\V{x}}$ 中,过点 $\V{x}_0$
作出 \emphB{$x^j$-线}(自然是平行于 $x^j$ 轴),
其上的点可以表示为 $\V{x}_0+\lambda\,\V{e}_j$。
映射到值域空间 $\domD{\V{X}}$ 上,
则成为 $\V{X}\qty(\V{x}_0+\lambda\,\V{e}_j)$。很显然,
\begin{equation}
	\pdv{\V{X}}{x^j}\qty(\V{x}_0)
	=\frac{\V{X}\qty(\V{x}_0+\lambda\,\V{e}_j) - \V{X}\qty(\V{x}_0)}
	{\lambda}
\end{equation}
就是 $x^j$-线在 $\V{X}\qty(\V{x}_0)$ 处的切向量。在定义域空间中,
$x^i$-线作为直线共有 $m$ 条,它们之间互相垂直。作用到值域空间后,
这样的 $x^i$-线尽管变为了曲线,但仍为 $m$ 条。相应的切向量,
自然也有 $m$ 个。

\section{局部基} \label{sec:局部基}
这里的讨论基于曲线坐标系(即微分同胚)
$\V{X}(\V{x})\in\cf{\domD{\V{x}}}{\domD{\V{X}}}$。

\subsection{局部协变基} \label{subsec:局部协变基}
我们已经知道,$\V{X}(\V{x})$ 的 Jacobi 矩阵可以表示为
\begin{equation}
	\JacobiD{\V{X}}(\V{x})
	=\mqty[\displaystyle \pdv{\V{X}}{x^1},\,\cdots,\,
		\displaystyle \pdv{\V{X}}{x^i},\,\cdots,\,
		\displaystyle \pdv{\V{X}}{x^m}] (\V{x})
		\in\realR^{m \times m} \comma
\end{equation}
式中的
\begin{equation}
	\pdv{\V{X}}{x^i} (\V{x})
	=\lim_{\lambda\to 0}
		\frac{\V{X}\qty(\V{x}+\lambda\,\V{e}_i) - \V{X}(\V{x})}
		{\lambda} \fullstop
	\label{eq:局部基_偏导数}
\end{equation}
在参数域 $\domD{\V{x}}$ 中作出 $x^i$-线。映照到物理域后,
它变成一条曲线,我们仍称之为 $x^i$-线。
\ref{subsec:偏导数的几何意义}~小节已经说明,
\eqref{eq:局部基_偏导数}~式表示物理域中 $x^i$-线的\emphB{切向量}。
在张量分析中,我们通常把它记作 $\V{g}_i(\V{x})$。

由于微分同胚要求是\emphB{双射},因而 Jacobi 矩阵
\begin{equation}
	\JacobiD{\V{X}}(\V{x})
	=\mqty[\V{g}_1,\,\cdots,\,\V{g}_i,\,\cdots,\,\V{g}_m](\V{x})
	\in\realR^{m \times m}
\end{equation}
必须是\emphB{非奇异}的。这等价于
\begin{equation}
	\qty{\V{g}_i(\V{x})=\pdv{\V{X}}{x^i} (\V{x})}^m_{i=1}
	\subset\Rm
\end{equation}
\emphB{线性无关}。由此,它们可以构成 $\Rm$ 上的一组\emphA{基}。

用任意的 $\V{x}\in\domD{\V{x}}$ 均可构建一组基。
但选取不同的 $\V{x}$,将会使所得基的取向有所不同。
因而这种基称为\emphA{局部协变基}。和之前一样,
我们用“协变”表示指标在下方。

\subsection{局部逆变基;对偶关系}
有了局部协变基 $\qty{\V{g}_i(\V{x})}^m_{i=1}$,根据
\ref{subsec:对偶基}~小节中的讨论,
必然唯一存在与之对应的\emphA{局部逆变基}
$\qty{\V{g}^i(\V{x})}^m_{i=1}$,满足
\begin{equation}
	\mqty[\V{g}^1(\V{x}),\,\cdots,\,\V{g}^m(\V{x})]\trans
		\mqty[\V{g}_1(\V{x}),\,\cdots,\,\V{g}_m(\V{x})]
	=\mqty[\qty(\V{g}^1)\trans \\ \vdots \\ \qty(\V{g}^m)\trans]\,
		(\V{x}) \cdot \JacobiD{\V{X}}(\V{x})
	=\Mat{I}_m \fullstop
\end{equation}

下面我们来寻找逆变基 $\qty{\V{g}^i(\V{x})}^m_{i=1}$ 的具体表示。
考虑到\footnote{
	这里的几步推导在 \ref{subsec:参数域方程}~小节中也有所涉及。}
\begin{equation}
	\V{X}\qty\big(\V{x}(\V{X}))=\V{X}\in\Rm \comma
\end{equation}
并利用复合映照可微性定理,可知
\begin{equation}
	\JacobiD{\V{X}}\qty\big(\V{x}(\V{X}))
	\cdot \JacobiD{\V{x}}(\V{X})
	=\Mat{I}_m \comma
	\label{eq:局部逆变基_两个Jacobi矩阵互逆}
\end{equation}
即有
\begin{equation}
	\JacobiD{\V{x}}(\V{X})
	=(\JacobiD{\V{X}})^{-1}\qty\big(\V{x}(\V{X})) \fullstop
\end{equation}
于是
\begin{equation}
	\mqty[\qty(\V{g}^1)\trans \\ \vdots \\ \qty(\V{g}^m)\trans]\,
		(\V{x})
	=(\JacobiD{\V{X}})^{-1}(\V{x})
	=\JacobiD{\V{x}}(\V{X})
	=\mqty[
		\displaystyle\pdv{x^1}{\displaystyle X^1} & \cdots &
			\displaystyle\pdv{x^1}{\displaystyle X^m} \\[1ex]
		\vdots & \ddots & \vdots \\[0.5ex]
		\displaystyle\pdv{x^m}{\displaystyle X^1} & \cdots &
			\displaystyle\pdv{x^m}{\displaystyle X^m} ]\,(\V{X}) \fullstop
\end{equation}
这样我们就得到了局部逆变基的具体表示(注意转置):
\begin{equation}
	\V{g}^i(\V{x})
	=\mqty[\displaystyle \pdv{x^i}{X^1} \\[1ex]
		\vdots \\[0.5ex] \displaystyle \pdv{x^i}{X^m}]\,(\V{X})
	=\sum_{\alpha=1}^{m} \pdv{x^i}{X^\alpha} (\V{X})\,\V{e}_\alpha
	\fullstop
\end{equation}
定义标量场 $f(\V{x})$ 的\emphA{梯度}为
\begin{equation}
	\nabla f(\V{x}) \defeq
	\sum_{\alpha=1}^{m} \pdv{f}{x^\alpha} (\V{x})\,\V{e}_\alpha \comma
\end{equation}
则局部逆变基又可以表示成
\begin{equation}
	\V{g}^i(\V{x})=\nabla x^i(\V{X}) \fullstop
\end{equation}
此处的梯度实际上就是我们熟知的三维情况在 $m$ 维下的推广。

\blankline

在 \ref{subsec:偏导数的几何意义}~小节中已经指出,
局部协变基的几何意义是 $x^i$-线的\emphB{切向量}。
现在,我们来讨论局部逆变基的几何意义。

\begin{figure}[h]
	\centering
	\includegraphics{Images/Local_Basis.PNG}
	\caption{局部逆变基的几何意义}
	\label{fig:局部逆变基的几何意义}
\end{figure}

如图~\ref{fig:局部逆变基的几何意义} 所示,在参数空间中,
过点 $\V{x}$ 作垂直于 $x^i$ 轴的平面,记为 $x^i$-面。
在 $x^i$-面上,自然有 $x^i=\const$ \ 映照到物理空间后,
$x^i$-面变为一个曲面,其上仍有 $x^i(\V{X})=\const$,
即它是一个\emphB{等值面}。
等值面的梯度方向显然与该曲面的\emphB{法向}相同。因此,局部逆变基
$\V{g}^i(\V{x})$ 的几何意义就是 $x^i$-面的\emphA{法向量}。

现在来验证一下\emphA{对偶关系}。
\begin{align}
	\ipb{\V{g}_i(\V{x})}{\V{g}^j(\V{x})}
	&=\ipb{\pdv{\V{X}}{x^i} (\V{x})}{\nabla x^j(\V{X})} \notag \\
	&=\ipb{\sum_{\alpha=1}^{m} \pdv{X^\alpha}{x^i} (\V{x})\,
			\V{e}_\alpha}
		{\sum_{\beta=1}^{m} \pdv{x^j}{X^\beta} (\V{X})\,
			\V{e}_\beta} \notag
	\intertext{利用内积的线性性,有}
	&=\sum_{\alpha=1}^{m} \sum_{\beta=1}^{m}
		\pdv{X^\alpha}{x^i} (\V{x}) \pdv{x^j}{X^\beta} (\V{X})
		\cdot \ipb{\V{e}_\alpha}{\V{e}_\beta} \notag \\
	&=\sum_{\alpha=1}^{m} \sum_{\beta=1}^{m}
		\pdv{X^\alpha}{x^i} (\V{x}) \pdv{x^j}{X^\beta} (\V{X})
		\cdot\KroneckerDelta*{\alpha\beta} \notag
	\intertext{合并掉指标 $\beta$,可得}
	&=\sum_{\alpha=1}^{m}
		\pdv{X^\alpha}{x^i} (\V{x})
		\pdv{x^j}{X^\alpha} (\V{X}) \notag \\
	&=\sum_{\alpha=1}^{m}
		\pdv{x^j}{X^\alpha} (\V{X})
		\pdv{X^\alpha}{x^i} (\V{x}) \fullstop
\end{align}
最后一步求和号中的第一项位于 Jacobi 矩阵 $\JacobiD{\V{x}}(\V{X})$
的第 $j$ 行第 $\alpha$ 列,而第二项位于 $\JacobiD{\V{X}}(\V{x})$
的第 $\alpha$ 行第 $i$ 列,因此关于 $\alpha$
的求和结果便是乘积矩阵的第 $j$ 行第 $i$ 列。
根据式~\eqref{eq:局部逆变基_两个Jacobi矩阵互逆},
这两个 Jacobi 矩阵的乘积为单位阵,所以有
\begin{equation}
	\ipb{\V{g}_i(\V{x})}{\V{g}^j(\V{x})}
	=\KroneckerDelta{j}{i} \fullstop
\end{equation}

\blankline

总结一下我们得到的结果。对于体积形态的连续介质,存在着
\begin{braceEq}
	\text{局部协变基:}\quad \qty{\V{g}_i(\V{x})
		\defeq \pdv{\V{X}}{x^i} (\V{x})}^m_{i=1} \comma \notag \\
	\text{局部逆变基:}\quad \qty{\V{g}^i(\V{x})
		\defeq \nabla x^i(\V{X})}^m_{i=1} \comma \notag
\end{braceEq}
它们满足\emphB{对偶关系}
\begin{equation}
	\ipb{\V{g}_i(\V{x})}{\V{g}^j(\V{x})}
	=\KroneckerDelta{j}{i} \fullstop
	\label{eq:局部基_对偶关系}
\end{equation}
这样,在研究连续介质中的一个点时,我们就有三种基可以使用:
局部协变基、局部逆变基,
当然还有\emphB{典则基} $\qty{\V{e}_i}^m_{i=1}$。

\section{标架运动方程}
\subsection{向量在局部基下的表示}
对于 $\Rm$ 空间中的任意一个向量 $\V{b}$,
它可以用\emphA{典则基}表示:
\begin{equation}
	\V{b}=\sum_{\alpha=1}^{m} b_\alpha \V{e}_\alpha
	=b_\alpha \V{e}_\alpha \fullstop
\end{equation}
第二步省略掉了求和号,这是根据\emphA{Einstein 求和约定}:
指标出现两次,则表示对它求和。\footnote{
	在 \ref{subsec:度量}~小节中,还要求重复指标一上一下。
	典则基不分协变、逆变,标号均在下方,可以视为一个特例。}
根据之前一小节的结论,$\V{b}$ 还可以用局部协变基和局部逆变基来表示:
\begin{align}
	\V{b} = b^i\V{g}_i(\V{x}) = b_j\V{g}^j(\V{x}) \comma
\end{align}
式中,
\begin{mySubEq}
	\begin{align}
		b^i&=\ipb{\V{b}}{\V{g}^i(\V{x})} \label{eq:活动标架_逆变分量表示}
		\intertext{和}
		b_j&=\ipb{\V{b}}{\V{g}_j(\V{x})} \label{eq:活动标架_协变分量表示}
	\end{align}
\end{mySubEq}
分别称为向量 $b$ 的\emphA{逆变分量}和\emphA{协变分量}。
注意,这里同样用到了 Einstein 求和约定。

将 $\V{b}=b^i\V{g}_i(\V{x})$ 的两边分别与
$\V{g}^j(\V{x})$ 作内积,可有
\begin{align}
	\ipb{\V{b}}{\V{g}^j(\V{x})}
	&=\ipb{b^i\V{g}_i(\V{x})}{\V{g}^j(\V{x})} \notag
	\intertext{利用内积的线性性,提出系数:}
	&=b^i\ipb{\V{g}_i(\V{x})}{\V{g}^j(\V{x})} \notag
	\intertext{利用对偶关系 \eqref{eq:局部基_对偶关系}~式,可有}
	&=b^i\KroneckerDelta{j}{i}=b^j \comma
\end{align}
这就得到了逆变分量的表示式~\eqref{eq:活动标架_逆变分量表示}。
同理,将 $\V{b}=b_j\V{g}^j(\V{x})$ 的两边分别与
$\V{g}_i(\V{x})$ 作内积,就有
\begin{equation}
	\ipb{\V{b}}{\V{g}_i(\V{x})}
	=\ipb{b_j\V{g}^j(\V{x})}{\V{g}_i(\V{x})}
	=b_j\ipb{\V{g}^j(\V{x})}{\V{g}_i(\V{x})}
	=b_j\KroneckerDelta{j}{i}=b_i \comma
\end{equation}
这便是协变分量的表示 \eqref{eq:活动标架_协变分量表示}~式。

\subsection{局部基的偏导数}
所谓局部基(或曰“活动标架”),顾名思义,它在不同的点上往往是不同的。
根据之前的定义,我们有
\begin{mySubEq}
	\begin{align}
	&\mmap{\V{g}_i(\V{x})}{\domD{\V{x}}\ni\V{x}}{\V{g}_i(\V{x})
		=\mqty[\displaystyle \pdv{X^1}{x^i} \\[1ex] \vdots \\[0.7ex]
			\displaystyle \pdv{X^m}{x^i}]\,(\V{x})\in\Rm} \comma
		\label{eq:局部协变基定义} \\
	&\mmap{\V{g}^i(\V{x})}{\domD{\V{x}}\ni\V{x}}{\V{g}^i(\V{x})
		=\mqty[\displaystyle \pdv{x^i}{X^1} \\[1ex] \vdots \\[0.5ex]
			\displaystyle \pdv{x^i}{X^m}]\,\qty\big(\V{X}(\V{x}))\in\Rm}
		\fullstop
		\label{eq:局部逆变基定义}
	\end{align}
\end{mySubEq}
从\emphB{映照}的角度来看,局部基定义了新的向量值映照,
其定义域仍为参数域,而值域则为 $\Rm$ 空间。这样一来,
我们在 \ref{sec:向量值映照的可微性}~节中所引入的操作均可完全
类似地应用在局部基上。例如,我们可以来求局部基的 Jacobi 矩阵:
\begin{braceEq}
	\JacobiD{\V{g}_i}(\V{x})
	&=\mqty[\displaystyle \pdv{\V{g}_i}{x^1},\,\cdots,\,
		\pdv{\V{g}_i}{x^m}]\,(\V{x}) \in\realR^{m \times m} \comma \\
	\JacobiD{\V{g}^i}(\V{x})
	&=\mqty[\displaystyle \pdv{\V{g}^i}{x^1},\,\cdots,\,
		\pdv{\V{g}^i}{x^m}]\,(\V{x}) \in\realR^{m \times m} \fullstop
\end{braceEq}
Jacobi 矩阵中的每一列都是局部基作为整体相对自变量第 $j$
个分量的变化率,即\emphB{偏导数}:
\begin{braceEq}
	\pdv{\V{g}_i}{x^j} (\V{x}) &\defeq \lim_{\lambda\to 0}
		\frac{\V{g}_i\qty(\V{x}+\lambda\,\V{e}_j)-\V{g}_i(\V{x})}
			{\lambda} \in\Rm \comma \\
	\pdv{\V{g}^i}{x^j} (\V{x}) &\defeq \lim_{\lambda\to 0}
		\frac{\V{g}^i\qty(\V{x}+\lambda\,\V{e}_j)-\V{g}^i(\V{x})}
			{\lambda} \in\Rm \fullstop
\end{braceEq}

下面澄清局部基偏导数的几何意义。
如图~\ref{fig:局部基偏导数的几何意义} 所示,在参数空间中,
过点 $\V{x}$ 作出 $x^j$-线,并在其上取点 $\V{x}+\lambda\,\V{e}_j$。
分别过点 $\V{x}$ 和 $\V{x}+\lambda\,\V{e}_j$ 作出 $x^i$-线,于是
$\pdv*{\V{g}_i}{x^j} (\V{x})$ 就表示 $\V{g}_i(\V{x})$
(即 $x^i$-线的切向量)沿 $x^j$-线的变化率。
同理,过点 $\V{x}$ 和 $\V{x}+\lambda\,\V{e}_j$ 作出 $x^i$-面,
则 $\pdv*{\V{g}^i}{x^j} (\V{x})$ 就表示 $\V{g}^i(\V{x})$
(即 $x^i$-面的法向量)沿 $x^j$-线的变化率。

\begin{figure}[h]
	\centering
	\includegraphics{Images/Local_Basis_PDV_1.PNG}
	\includegraphics{Images/Local_Basis_PDV_2.PNG}
	\caption{局部基偏导数的几何意义}
	\label{fig:局部基偏导数的几何意义}
\end{figure}

\subsection{Christoffel 符号}
\label{subsec:Christoffel符号}
考察 $\pdv*{\V{g}_i}{x^j} (\V{x})$,
即\emphB{协变基}的偏导数\footnote{
	以下在不引起歧义之处,将省略局部协变基、
	局部逆变基的“局部”二字。为了方便,$\V{g}_i(\V{x})$ 和
	$\V{g}^i(\V{x})$ 中的“$(\V{x})$”有时也会省略。
}。它是 $\Rm$ 空间中的一个向量,因而可以用协变基或逆变基来表示:
\begin{braceEq*}
	{\label{eq:协变基偏导数的协变与逆变表示} \pdv{\V{g}_i}{x^j} (\V{x})=}
	&\ipb{\pdv{\V{g}_i}{x^j}}{\V{g}^k} \V{g}_k \comma \\
	&\ipb{\pdv{\V{g}_i}{x^j}}{\V{g}_k} \V{g}^k \fullstop
\end{braceEq*}
引入\emphA{第一类 Christoffel 符号}
\begin{equation}
	\ChrA{j}{i}{k} \defeq \ipb{\pdv{\V{g}_i}{x^j}}{\V{g}_k}
	\label{eq:第一类Christoffel符号定义}
\end{equation}
和\emphA{第二类 Christoffel 符号}
\begin{equation}
	\ChrB{j}{i}{k} \defeq \ipb{\pdv{\V{g}_i}{x^j}}{\V{g}^k}
	\comma
	\label{eq:第二类Christoffel符号定义}
\end{equation}
则式~\eqref{eq:协变基偏导数的协变与逆变表示} 可以写成
\begin{braceEq*}
	{\pdv{\V{g}_i}{x^j} (\V{x})=}
	&\ChrB{j}{i}{k}\, \V{g}_k \comma \\
	&\ChrA{j}{i}{k}\, \V{g}^k \fullstop
\end{braceEq*}

下面我们来探讨 Christoffel 符号的基本性质——指标 $i$、
$j$ 可以交换:
\begin{braceEq}
	&\ChrB{j}{i}{k}=\ChrB{i}{j}{k} \comma
	\label{eq:第二类Christoffel符号指标交换} \\
	&\ChrA{j}{i}{k}=\ChrA{i}{j}{k} \fullstop
	\label{eq:第一类Christoffel符号指标交换}
\end{braceEq}

\begin{myProof}
根据定义 \eqref{eq:第一类Christoffel符号定义} 和
\eqref{eq:第二类Christoffel符号定义}~式,指标 $i$、$j$
来源于协变基的偏导数 $\pdv*{\V{g}_i}{x^j} (\V{x})$。
只要偏导数中的 $i$、$j$ 可以交换,Christoffel 符号中的指标 $i$、
$j$ 自然也可以。回顾协变基的定义\eqref{eq:局部协变基定义}~式:
\begin{equation}
	\V{g}_i (\V{x})
	\defeq\mqty[\displaystyle\pdv{X^1}{x^i} \\[1ex]
		\vdots \\[0.7ex] \displaystyle\pdv{X^m}{x^i}]\,(\V{x})
	\fullstop
\end{equation}
其偏导数为
\begin{equation}
	\pdv{\V{g}_i}{x^j} (\V{x})
	=\mqty[\displaystyle\pdv{X^1}{x^j}{x^i} \\[1ex]
		\vdots \\[0.7ex] \displaystyle\pdv{X^m}{x^j}{x^i}]\,(\V{x})
	=\mqty[\displaystyle\pdv{X^1}{x^i}{x^j} \\[1ex]
		\vdots \\[0.7ex] \displaystyle\pdv{X^m}{x^i}{x^j}]\,(\V{x})
	=\pdv{\V{g}_j}{x^i} (\V{x}) \fullstop
\end{equation}
注意第二个等号处交换了偏导数的次序,
其条件是\emphB{二阶}偏导数均存在且连续。
只要微分同胚达到了 $\DiffMorp[2]$,就可以满足该要求,
在一般的物理情境这都是成立的。于是我们便完成了证明。
\end{myProof}

现在再来看\emphB{逆变基}的偏导数 $\pdv*{\V{g}^i}{x^j} (\V{x})$。
它也是 $\Rm$ 空间中的向量,因此
\begin{braceEq*}
	{\label{eq:逆变基偏导数的协变与逆变表示} \pdv{\V{g}^i}{x^j} (\V{x})=}
	&\ipb{\pdv{\V{g}^i}{x^j}}{\V{g}^k} \V{g}_k \comma \\
	&\ipb{\pdv{\V{g}^i}{x^j}}{\V{g}_k} \V{g}^k \fullstop
\end{braceEq*}
利用 Christoffel 符号,可以表示出
$\ipb{\pdv*{\V{g}^i}{x^j}}{\V{g}_k}$。根据对偶关系,
\begin{equation}
	\ipb{\V{g}^i}{\V{g}_k} (\V{x})=\KroneckerDelta{i}{k} \fullstop
\end{equation}
两边对 $x^j$ 求偏导,用一下内积的求导公式,同时注意到
$\KroneckerDelta{i}{k}$ 是与 $\V{x}$ 无关的常数,因而
\begin{equation}
	\pdv{x^j} \ipb{\V{g}^i}{\V{g}_k}
	=\ipb{\pdv{\V{g}^i}{x^j}}{\V{g}_k}
		+\ipb{\V{g}^i}{\pdv{\V{g}_k}{x^j}}
	=\pdv{\KroneckerDelta{i}{k}}{x^j}=0 \fullstop
\end{equation}
所以
\begin{equation}
	\ipb{\pdv{\V{g}^i}{x^j}}{\V{g}_k}
	=-\ipb{\V{g}^i}{\pdv{\V{g}_k}{x^j}}
	=-\ipb{\pdv{\V{g}_k}{x^j}}{\V{g}^i}
	=-\ChrB{j}{k}{i} \fullstop
\end{equation}

至于 $\ipb{\pdv*{\V{g}^i}{x^j}}{\V{g}^k}$,将在以后讨论。
\myPROBLEM{你想在什么时候?}

\subsection{指标升降}
首先引入\emphA{度量}:
\begin{braceEq}
	g_{ij} (\V{x})
		&\defeq \ipb{\V{g}_i}{\V{g}_j} (\V{x}) \comma \\
	g^{ij} (\V{x})
		&\defeq \ipb{\V{g}^i}{\V{g}^j} (\V{x}) \fullstop
\end{braceEq}
由此可以获得\emphB{基向量}的指标升降
\begin{braceEq}
	\V{g}_i (\V{x}) &= g_{ij}(\V{x}) \, \V{g}^j(\V{x}) \comma \\
	\V{g}^i (\V{x}) &= g^{ij}(\V{x}) \, \V{g}_j(\V{x}) \fullstop
\end{braceEq}

如前所述,对于任意的 $\V{b}\in\Rm$,它可以表示成
\begin{equation}
	\V{b} = b^i\V{g}_i(\V{x}) = b_j\V{g}^j(\V{x}) \fullstop
\end{equation}
利用度量,同样可以获得\emphB{向量分量}的指标升降
\begin{braceEq}
	b^i&=\ipb{\V{b}}{\V{g}^i}
		=\ipb{\V{b}}{g^{ik}\,\V{g}_k}
		=g^{ik} \, \ipb{\V{b}}{\V{g}_k}
		=g^{ik} b_k \comma \\
	b_j&=\ipb{\V{b}}{\V{g}_j}
		=\ipb{\V{b}}{g_{jk}\,\V{g}^k}
		=g_{jk} \, \ipb{\V{b}}{\V{g}^k}
		=g_{jk} b^k \fullstop
\end{braceEq}

关于度量,再多说一句。由于内积的交换律,显然有
\begin{equation}
	g_{ij}(\V{x})=g_{ji}(\V{x}), \quad g^{ij}=g^{ji} \fullstop
\end{equation}

\subsection{度量的性质;Christoffel 符号的计算}
\label{subsec:度量的性质_Christoffel符号的计算}
首先,我们来澄清度量的两条性质。

\begin{myEnum}
\item 矩阵 $\qty[g_{ik}]$ 与 $\qty[g^{kj}]$ 互逆,即
\begin{equation}
	g_{ik}\,g^{kj} = \KroneckerDelta{j}{i} \fullstop
\end{equation}
证明见 \ref{subsec:度量}~小节(尽管省略了“$(\V{x})$”,
但请不要忘记这里的基是\emphB{局部基})。

\blankline

\item 第一类 Christoffel 符号满足
\begin{equation}
	\ChrA{i}{j}{k}=\frac{1}{2}\,
		\qty(\pdv{g_{jk}}{x^i}+\pdv{g_{ik}}{x^j}-\pdv{g_{ij}}{x^k})
		(\V{x}) \fullstop
	\label{eq:第一类Christoffel符号与度量的关系}
\end{equation}

\begin{myProof}
根据式~\eqref{eq:第一类Christoffel符号定义},
第一类 Christoffel 符号的定义为
\begin{equation}
	\ChrA{i}{j}{k} \defeq \ipb{\pdv{\V{g}_j}{x^i}}{\V{g}_k}
	\fullstop
\end{equation}
考虑度量的定义
\begin{equation}
	g_{ij}(\V{x})\defeq\ipb{\V{g}_i}{\V{g}_j} (\V{x}) \fullstop
\end{equation}
两边对 $x^k$ 求偏导,可得
\begin{align}
	\pdv{g_{ij}}{x^k} (\V{x})
	&=\ipb{\pdv{\V{g}_i}{x^k}}{\V{g}_j} (\V{x})
	+\ipb{\V{g}_i}{\pdv{\V{g}_j}{x^k}} (\V{x}) \notag
	\intertext{利用上面 Christoffel 符号的定义,有}
	&=\ChrA{k}{i}{j}+\ChrA{k}{j}{i} \fullstop
\end{align}
这样就获得了度量偏导数用 Christoffel 符号的表示。
但我们需要的却是 Christoffel 符号用度量偏导数的表示。
下面的工作就是完成这一“调转”。

利用指标轮换
\begin{equation*}
	i \to j, \quad j \to k, \quad k \to i \comma
\end{equation*}
可有
\begin{equation}
	\pdv{g_{jk}}{x^i} (\V{x})
	=\ChrA{i}{j}{k}+\ChrA{i}{k}{j} \fullstop
\end{equation}
再进行一次指标轮换:
\begin{equation}
	\pdv{g_{ki}}{x^j} (\V{x})
	=\ChrA{j}{k}{i}+\ChrA{j}{i}{k} \fullstop
\end{equation}
以上三式联立,就有
\begin{align}
	&\alspace\frac{1}{2}\,\qty(
		\pdv{g_{jk}}{x^i}+\pdv{g_{ki}}{x^j}-\pdv{g_{ij}}{x^k})
		(\V{x}) \notag \\
	&=\frac{1}{2}\,\qty\Big[
		\qty(\ChrA{i}{j}{k}+\ChrA{i}{k}{j})
		+\qty(\ChrA{j}{k}{i}+\ChrA{j}{i}{k})
		-\qty(\ChrA{k}{i}{j}+\ChrA{k}{j}{i})] \notag
	\intertext{利用 \eqref{eq:第一类Christoffel符号指标交换}~式
		所指出的 Christoffel 符号的指标交换性:}
	&=\frac{1}{2}\,\qty[
		\qty(\ChrA{i}{j}{k}+\hl{\ChrA{k}{i}{j}})
		+\qty(\hl[pink]{\ChrA{j}{k}{i}}
			+\ChrA{i}{j}{k})
		-\qty(\hl{\ChrA{k}{i}{j}}
			+\hl[pink]{\ChrA{j}{k}{i}}) ] \notag
	\intertext{高亮部分相互抵消,于是可得}
	&=\ChrA{i}{j}{k} \fullstop
\end{align}
\end{myProof}
\end{myEnum}

\blankline

有了这两条性质,我们就能够很容易地获取 Christoffel 符号的计算方法。

第一步从度量开始。根据 \ref{subsec:局部协变基}~小节,
在曲线坐标系(即微分同胚)
$\V{X}(\V{x})\in\cf{\domD{\V{x}}}{\domD{\V{X}}}$ 中,
Jacobi 矩阵可以用\emphB{协变基}表示为
\begin{equation}
	\JacobiD{\V{X}}(\V{x})
	=\mqty[\V{g}_1,\,\cdots,\,\V{g}_i,\,\cdots,\,\V{g}_m]
	(\V{x}) \fullstop
\end{equation}
因此协变形式的度量(矩阵形式)就可以写成
\begin{equation}
	\mqty[g_{ij}] \defeq \mqty[\ipb{\V{g}_i}{\V{g}_j}]
	=\JacobiD{\V{X}}\trans(\V{x}) \cdot
		\JacobiD{\V{X}}(\V{x}) \fullstop
\end{equation}
两种形式的度量是互逆的,于是 $g^{ij}$ 实际上也已经算出来了。

第二步,将求得的度量代入
式~\eqref{eq:第一类Christoffel符号与度量的关系}:
\begin{equation}
	\ChrA{i}{j}{k}=\frac{1}{2}\,
		\qty(\pdv{g_{jk}}{x^i}+\pdv{g_{ik}}{x^j}-\pdv{g_{ij}}{x^k})
		(\V{x}) \comma
\end{equation}
就得到了第一类 Christoffel 符号。
至于第二类 Christoffel 符号,它可以表示成
\begin{equation}
	\ChrB{i}{j}{k} \defeq \ipb{\pdv{\V{g}_j}{x^i}}{\V{g}^k}
	=\ipb{\pdv{\V{g}_j}{x^i}}{g^{kl}\,\V{g}_l}
	=g^{kl}\,\ipb{\pdv{\V{g}_j}{x^i}}{\V{g}_l}
	=g^{kl}\,\ChrA{i}{j}{l} \fullstop
	\label{eq:第二类Christoffel符号用第一类表示}
\end{equation}
这样一来,它的表示也就明确了。

\section{度量张量与 Eddington 张量}
\subsection{度量张量的定义}
在曲线坐标系(即微分同胚)
$\V{X}(\V{x})\in\cf{\domD{\V{x}}}{\domD{\V{X}}}$ 中,
可以引入\emphA{度量张量}
\begin{equation}
	\T{G}=g_{ij}\,\V{g}^i\tp\V{g}^j \in\Tensors{2} \fullstop
\end{equation}
这是用协变形式表达的。当然也可以切换成其他形式:
\begin{align}
	\T{G}&=g_{ij}\,\V{g}^i\tp\V{g}^j \notag
	\intertext{利用指标升降,有}
	&=g_{ij}\,\qty(g^{ik}\,\V{g}_k)\tp\V{g}^j \notag
	\intertext{再根据线性性提出系数:}
	&=g_{ij}\,g^{ik}\,\V{g}_k\tp\V{g}^j \notag \\
	&=\KroneckerDelta{k}{j}\,\V{g}_k\tp\V{g}^j \fullstop
\end{align}
类似地,还可以得到
\begin{align}
	\T{G}&=\KroneckerDelta{k}{j}\,\V{g}_k\tp\V{g}^j \notag \\
	&=\KroneckerDelta{k}{j}\,
		\V{g}_k\tp\qty(g^{jl}\,\V{g}_l) \notag \\
	&=\KroneckerDelta{k}{j}\,g^{jl}\,\V{g}_k\tp\V{g}_l \notag \\
	&=g^{kl}\,\V{g}_k\tp\V{g}_l \fullstop
\end{align}

综上,度量张量有三种表示:
\begin{braceEq*}{\T{G}=}
	&g_{ij}\,\V{g}^i\tp\V{g}^j \comma \\
	&g^{ij}\,\V{g}_i\tp\V{g}_j \comma \\
	&\KroneckerDelta{i}{j}\,\V{g}_i\tp\V{g}^j \comma
\end{braceEq*}
式中,协变分量 $g_{ij}=\ipb{\V{g}_i}{\V{g}_j}$,
逆变分量 $g^{ij}=\ipb{\V{g}^i}{\V{g}^j}$,
混合分量 $\KroneckerDelta{i}{j}=\ipb{\V{g}^i}{\V{g}_j}$。

\subsection{Eddington 张量的定义}
接下来引入 \emphA{Eddington 张量}
\begin{equation}
	\EdTensor=\LeviCivita{_{ijk}}\,\V{g}^i\tp\V{g}^j\tp\V{g}^k
	\in\Tensors[\realR^3]{3} \comma
\end{equation}
式中的 $\LeviCivita{_{ijk}}=\det[\V{g}_i,\,\V{g}_j,\,\V{g}_k]$。
和之前一样,仍是利用指标升降来获得等价定义:
\begin{align}
	\EdTensor
	&=\LeviCivita{_{ijk}}\,\V{g}^i\tp\V{g}^j\tp\V{g}^k \notag \\
	&=\LeviCivita{_{ijk}}\,\V{g}^i\tp\qty(g^{jl}\,\V{g}_l)
		\tp\V{g}^k \notag \\
	&=\LeviCivita{_{ijk}}\,g^{jl}\, \V{g}^i\tp\V{g}_l\tp\V{g}^k \notag
	\intertext{根据张量分量之间的关系
		(回顾 \ref{subsec:张量分量之间的关系}~小节),我们有}
	&=\LeviCivita{_i^l_k}\,\V{g}^i\tp\V{g}_l\tp\V{g}^k \fullstop
\end{align}
当然,这里的 $\LeviCivita{_i^l_k}$ 只是一个形式。
要将它显式地表达出来,需要利用行列式的线性性:
\begin{align}
	\forall\, \V{\xi},\,\hat{\V{\eta}},\,\tilde{\V{\eta}},\,\V{\zeta}
		\in\realR^3 \text{\ 以及\ }
		\alpha,\,\beta \in\realR,
	&\alspace \det\!\mqty[\V{\xi},\,\alpha\,\hat{\V{\eta}}
		+\beta\,\tilde{\V{\eta}},\,\V{\zeta}] \notag \\
	&=\alpha\det\!\mqty[\V{\xi},\,\hat{\V{\eta}},\,\V{\zeta}]
		+\beta\det\!\mqty[\V{\xi},\,\tilde{\V{\eta}},\,\V{\zeta}]
	\fullstop
\end{align}
由此可知
\begin{align}
	\LeviCivita{_i^l_k}
	&=\LeviCivita{_{ijk}}\,g^{jl} \notag \\
	&=g^{jl}\,\det\!\mqty[\V{g}_i,\,\V{g}_j,\,\V{g}_k] \notag \\
	&=\det\!\mqty[\V{g}_i,\,g^{jl}\,\V{g}_j,\,\V{g}_k] \notag \\
	&=\det\!\mqty[\V{g}_i,\,\V{g}^l,\,\V{g}_k] \fullstop
\end{align}

一般来说,张量在定义时,只需给出其分量的一种形式。
而其他的形式,则都可以通过\emphB{度量}来获得。
说得直白一些,这其实就是一套“指标升降游戏”。

\blankline

顺带一说,在 Descartes 坐标系下,$\realR^3$
空间中的叉乘可以用 Eddington 张量表示为
\begin{braceEq*}{\V{g}_i\cp\V{g}^j=}
	\LeviCivita{_i^{jk}}\,\V{g}_k \comma \\
	\LeviCivita{_i^j_k}\,\V{g}^k \fullstop
\end{braceEq*}

\begin{myProof}
利用对偶关系可以很容易地获得这一结果。$\V{g}_i\cp\V{g}^j$
仍然得到一个$\realR^3$ 空间中的向量,它自然可以用协变基来表示:
\begin{align}
	\V{g}_i\cp\V{g}^j
	&=\ipb{\V{g}_i\cp\V{g}^j}{\V{g}^k}\,\V{g}_k \notag
	\intertext{这里的内积也就是点积。根据向量三重积的知识,可以把
		$\V{A}\cp\V{B}\cdot\V{C}$ 表示成\emphB{行列式}:}
	&=\det\!\mqty[\V{g}_i,\,\V{g}^j,\,\V{g}^k]\,\V{g}_k \notag
	\intertext{根据 Eddington 张量的定义即得到}
	&=\LeviCivita{_i^{jk}}\,\V{g}_k \fullstop
\end{align}
同理,若用逆变基表示,则为
\begin{align}
	\V{g}_i\cp\V{g}^j
	&=\ipb{\V{g}_i\cp\V{g}^j}{\V{g}_k}\,\V{g}^k \notag \\
	&=\det\!\mqty[\V{g}_i,\,\V{g}^j,\,\V{g}_k]\,\V{g}^k \notag \\
	&=\LeviCivita{_i^j_k}\,\V{g}^k \fullstop
\end{align}
\end{myProof}

\subsection{两种度量的关系} \label{subsec:两种度量的关系}
两个 Eddington 张量的分量之积可以用
一个由度量张量分量所组成的行列式来表示:
\begin{equation}
	\LeviCivita{^i_j^k}\,\LeviCivita{_{pq}^r}
	=\mqty|
		\KroneckerDelta{i}{p} & \KroneckerDelta{i}{q} & g^{ir} \\
		g_{jp} & g_{jq} & \KroneckerDelta{r}{j} \\[0.8ex]
		\KroneckerDelta{k}{p} & \KroneckerDelta{k}{q} & g^{kr}
	| \fullstop
\end{equation}
类似矩阵乘法,行列式中第 $m$ 行 $n$ 列的元素,
由第一个 Eddington 张量的第 $m$ 个指标与
第二个 Eddington 张量的第 $n$ 个指标组合而成。
两个指标均在上面,则获得度量张量的\emphB{逆变分量};
两个指标均在下面,则获得\emphB{协变分量};
若是一上一下,则将得到\emphB{混合分量}(即 Kronecker δ)。

这里的 $i$、$j$、$k$ 和 $p$、$q$、$r$ 都不是\emphB{哑标},
无需考虑求和的限制,可以任意选取。
至于它们的上下位置,同样是由实际问题来确定的。

\begin{myProof}
证明思路就是化为矩阵乘法。根据定义,
\begin{align}
	\LeviCivita{^i_j^k}\,\LeviCivita{_{pq}^r}
	&=\det\!\mqty[\V{g}^i,\,\V{g}_j,\,\V{g}^k] \,
		\det\!\mqty[\V{g}_p,\,\V{g}_q,\,\V{g}^r] \notag
	\intertext{考虑行列式的性质
		$\det(\Mat{A}\Mat{B})=\det(\Mat{A})\det(\Mat{B})$ 和
		$\det(\Mat{A}\trans)=\det(\Mat{A})$,则有}
	&=\det\! \qty\Bigg(
		\mqty[\qty(\V{g}^i)\trans \\ \qty(\V{g}_j)\trans \\
			\qty(\V{g}^k)\trans]
		\mqty[\V{g}_p,\,\V{g}_q,\,\V{g}^r] ) \fullstop
\end{align}
这个矩阵可以直接算出。
\end{myProof}

如果 Eddington 张量中存在\emphB{哑标},情况就会有所不同:
\begin{equation}
	\LeviCivita{^i_j^s}\,\LeviCivita{^p_{qs}}
	=\sum_{s=1}^{3} \mqty|
		g^{ip} & \KroneckerDelta{i}{q} & \KroneckerDelta{i}{s} \\
		\KroneckerDelta{p}{j} & g_{jq} & g_{js} \\[0.6ex]
		g^{sp} & \KroneckerDelta{s}{q} & \KroneckerDelta{s}{s}
	| \fullstop
\end{equation}
由于式中的 $k$ 是哑标,因此需要对它求和。
行列式按第一行展开,可得(下面仍将根据 Einstein 约定省略求和号)
\begin{align}
	\alspace\LeviCivita{^i_j^s}\,\LeviCivita{^p_{qs}}
	&=g^{ip} \qty(g_{jq}\,\KroneckerDelta{s}{s}
			-g_{js}\,\KroneckerDelta{s}{q} )
		-\KroneckerDelta{i}{q} \, \qty(
			\KroneckerDelta{p}{j}\,\KroneckerDelta{s}{s}
			-g_{js}\,g^{sp} )
		+\KroneckerDelta{i}{s} \, \qty(
			\KroneckerDelta{p}{j}\,\KroneckerDelta{s}{q}
			-g_{jq}\,g^{sp} ) \notag \\
	&=g^{ip} \qty(3 g_{jq} - g_{jq})
		-\KroneckerDelta{i}{q} \,
			\qty(3 \KroneckerDelta{p}{j} - \KroneckerDelta{p}{j})
		+\qty(\KroneckerDelta{p}{j}\,\KroneckerDelta{i}{q}
			-g_{jq}\,g^{ip}) \notag \\
	&=g^{ip}\,g_{jq}-\KroneckerDelta{p}{j}\,\KroneckerDelta{i}{q}
	\fullstop
\end{align}
这一串稍显复杂的表达式,可以用口诀
“\emphB{前前后后,里里外外}”来记忆。
具体操作如图~\ref{fig:Eddington张量乘积口诀} 所示。

\begin{figure}[h]
	\centering
	\includegraphics[width=8cm]{Images/Eddington_Tensors_Product.PNG}
	\caption{Eddington 张量乘积口诀“前前后后,里里外外”的示意图}
	\label{fig:Eddington张量乘积口诀}
\end{figure}

下面再举两个例子来说明:
\begin{align}
	\LeviCivita{^{ij}_s}\,\LeviCivita{_{pq}^s}
	&=\KroneckerDelta{i}{p}\,\KroneckerDelta{j}{q}
		-\KroneckerDelta{j}{p}\,\KroneckerDelta{i}{q} \semicolon \\
	\LeviCivita{^{ij}_s}\,\LeviCivita{^p_q^s}
	&=g^{ip}\,\KroneckerDelta{j}{q}
		-g^{jp}\,\KroneckerDelta{i}{q} \fullstop
\end{align}
以后将会看到,这是一个相当重要的基本结构。