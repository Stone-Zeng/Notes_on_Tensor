\section{Frenet 标架}
$\Rm$ 空间中的\emphA{曲线},就是一个\emphB{单参数}的向量值映照:
\begin{equation}
	\mmap{\V{X}(\lambda)}{[\alpha,\,\beta]\ni\lambda}
		{\V{X}(\lambda)=\mqty[X^1(\lambda) \\ \vdots \\
			X^m(\lambda)]\in\Rm}
	\fullstop
\end{equation}

下面我们来研究
\begin{equation}
	\dv{\V{X}}{\lambda} (\lambda)
	=\lim_{\incr\lambda\to 0}
		\frac{\V{X}(\lambda+\incr\lambda)-\V{X}(\lambda)}{\incr\lambda}
	\eqcolon \mqty[\dot{X}^1(\lambda) \\ \vdots \\
		\dot{X}^m(\lambda)] \fullstop
\end{equation}
若该极限存在,则称 $\V{X}(\lambda)\in\Rm$ 在点 $\lambda$
处\emphA{可微}。此时,$\dv*{\V{X}}{\lambda}$ 称为曲线
$\V{X}(\lambda)$ 的\emphA{切向量}。

上述极限可以等价地表述为
\begin{equation}
	\V{X}(\lambda+\incr\lambda)
	=\V{X}(\lambda)+\dv{\V{X}}{\lambda} (\lambda) \cdot\incr\lambda
		+\sOv{\incr\lambda} \fullstop
\end{equation}
在 $\lambda_0$ 处,则可以写成
\begin{equation}
	\V{X}(\lambda)
	=\V{X}\qty(\lambda_0)+\dv{\V{X}}{\lambda} \qty(\lambda_0)
		\cdot\qty(\lambda-\lambda_0)+\sOv{\lambda-\lambda_0} \fullstop
\end{equation}
该方程表示一条直线,称为曲线 $\V{X}(\lambda)$
在 $\lambda_0$ 处的\emphA{切线}。