\section{Frenet 标架(弧长参数)}
%CODE 20161130 TeX Studio bug 无法换行
% #1898 Freeze while typing \texorpdfstring
% https://sourceforge.net/p/texstudio/bugs/1898/
\subsection{\texorpdfstring{$\Rm$}{R\^{}m} 空间中曲线的表示}
$\Rm$ 空间中的\emphA{曲线},就是一个\emphB{单参数}的向量值映照:
\begin{equation}
	\mmap{\V{X}(t)}{[\alpha,\,\beta]\ni t}
		{\V{X}(t)=\mqty[X^1(t) \\ \vdots \\ X^m(t)]\in\Rm} \fullstop
\end{equation}
考虑该向量值映照关于参数 $t$ 的变化率
\begin{equation}
	\dot{\V{X}}(t)\coloneq\dv{\V{X}}{t} (t)
	=\lim_{\incr t\to 0} \frac{\V{X}(t+\incr t)-\V{X}(t)}{\incr t}
	\eqcolon \mqty[\dot{X}^1(t) \\ \vdots \\ \dot{X}^m(t)] \comma
\end{equation}
按照物理上的习惯,我们用点表示对 $t$ 的导数。
若该极限存在,则称 $\V{X}(t)\in\Rm$ 在点 $t$
处\emphA{可微}。此时,$\dot{\V{X}}(t)$ 称为曲线
$\V{X}(t)$ 的\emphA{切向量}。上述极限可以等价地表述为
\begin{equation}
	\V{X}(t+\incr t) = \V{X}(t)+\dv{\V{X}}{t} (t) \cdot\incr t
		+\sOv{\incr t} \fullstop
\end{equation}
在 $t_0$ 处,则可以写成
\begin{equation}
	\V{X}(t) =\V{X}\qty(t_0)+\dv{\V{X}}{t} \qty(t_0)
		\cdot\qty(t-t_0)+\sOv{t-t_0} \fullstop
\end{equation}
该方程表示一条直线,称为曲线 $\V{X}(t)$
在 $t_0$ 处的\emphA{切线}。

在物理域中,\emphA{弧长} $s$ 可以表示为
\begin{equation}
	s=\int_\alpha^\beta \norm{\dv{\V{X}}{t} (t)} \dd{t} \comma
	\label{eq:弧长的定义}
\end{equation}
两边对 $t$ 求导,可有
\begin{equation}
	\dv{s}{t} (t) = \norm{\dv{\V{X}}{t} (t)} \fullstop
	\label{eq:弧长与一般参数的关系}
\end{equation}
在本节中,我们将采用弧长作为曲线的参数,即
\begin{equation}
	\mmap{\V{r}(s)}{[0,\,L]\ni s}{\V{r}(s)\in\Rm} \fullstop
\end{equation}
对应的切向量为
\begin{equation}
	\dot{\V{r}}(s)\coloneq\dv{\V{r}}{s} (s)
	=\lim_{\incr s\to 0} \frac{\V{r}(s+\incr s)-\V{r}(s)}{\incr s}
	\fullstop
\end{equation}
根据链式法则,
\begin{equation}
	\dv{\V{r}}{s} (s)
	=\dv{\V{r}}{t} (t) \cdot \dv{t}{s} (s)
	=\flatfrac{\dv{\V{r}}{t} (t)}{\dv{s}{t} (t)}
	=\flatfrac{\dv{\V{r}}{t} (t)}{\norm{\dv{\V{r}}{t} (t)}} \comma
\end{equation}
因而 $\dot{\V{r}}(s)$ 是一个单位向量,即
\begin{equation}
	\norm{\dot{\V{r}}(s)}=1 \fullstop
	\label{eq:r的一阶导数为单位向量_弧长参数}
\end{equation}
接下来继续对 $\dot{\V{r}}(s)$ 求导:
\begin{equation}
	\ddot{\V{r}}(s)\coloneq\dv{\dot{\V{r}}}{s} (s) \fullstop
\end{equation}
由于 $\norm{\dot{\V{r}}(s)}=1$,因此
\begin{equation}
	1=\norm{\dot{\V{r}}(s)}^2
	=\ipb{\vphantom{0^0} \dot{\V{r}}(s)}{\dot{\V{r}}(s)} \comma
\end{equation}
两边求导,则有
\begin{equation}
	0=\ipb{\vphantom{0^0} \ddot{\V{r}}(s)}{\dot{\V{r}}(s)}
		+\ipb{\vphantom{0^0} \dot{\V{r}}(s)}{\ddot{\V{r}}(s)}
	=2\cdot\ipb{\vphantom{0^0} \ddot{\V{r}}(s)}{\dot{\V{r}}(s)}
	\fullstop
\end{equation}
内积为零,就意味着正交:
\begin{equation}
	\ddot{\V{r}}(s) \perp \dot{\V{r}}(s) \fullstop
	\label{eq:r的一阶导数垂直二阶导数_弧长参数}
\end{equation}
采用一般参数时,\eqref{eq:r的一阶导数为单位向量_弧长参数}~式和
\eqref{eq:r的一阶导数垂直二阶导数_弧长参数} 未必成立。
解决方案见 \ref{sec:Frenet标架_一般参数}~节。

现在我们把目光限定在 $\realR^3$ 空间中。如前所述,
$\dot{\V{r}}(s)$ 已经是单位向量,我们将其记为 $\V{T}(s)$;
而 $\ddot{\V{r}}(s)$ 仍需作单位化处理,其结果记作 $\V{N}(s)$,即
\begin{equation}
	\V{N}(s)\coloneq\frac{\ddot{\V{r}}(s)}
		{\norm[\realR^3]{\ddot{\V{r}}(s)}}
	=\frac{\dv*{\dot{\V{r}}}{s}}
		{\norm[\realR^3]{\dv*{\dot{\V{r}}}{s}}}
	=\frac{\dot{\V{T}}(s)}{\norm[\realR^3]{\dot{\V{T}}(s)}} \fullstop
	\label{eq:主单位法向量定义}
\end{equation}
最后,只要再令
\begin{equation}
	\V{B}(s)\coloneq\V{T}(s)\cp\V{N}(s) \comma
	\label{eq:副单位法向量定义}
\end{equation}
我们便有了 $\realR^3$ 空间中的一组\emphB{单位正交基}:
\begin{equation}
	\qty\big{\V{T}(s),\,\V{N}(s),\,\V{B}(s)}
	\subset\realR^3 \comma
\end{equation}
它们称为\emphA{Frenet 标架}。
其中,$\V{T}(s)$、$\V{N}(s)$、$\V{B}(s)$,分别叫做%
\emphA{单位切向量}、\emphA{主单位法向量}和\emphA{副单位法向量}。

\subsection{标架运动方程}
考虑 Frenet 标架关于弧长参数 $s$ 的变化率,即\emphB{标架运动方程}:
\begin{equation}
	\qty\big{\dot{\V{T}}(s),\,\dot{\V{N}}(s),\,\dot{\V{B}}(s)}
	\subset\realR^3 \fullstop
\end{equation}
为此,我们需要先给出一个引理:设 $\qty{\V{e}_i(t)}^m_{i=1}$
是 $\Rm$ 空间中的一组\emphB{活动}单位正交基,它们满足
\begin{equation}
	\ipb{\V{e}_i(t)}{\V{e}_j(t)}=\KroneckerDelta*{ij} \fullstop
	\label{eq:活动单位正交基引理_单位正交性}
\end{equation}
这组基的导数仍位于 $\Rm$ 空间,用自身展开,可有
\begin{equation}
	\mqty[\dot{\V{e}}_1(t),\,\cdots,\,\dot{\V{e}}_m(t)]
	=\mqty[\V{e}_1(t),\,\cdots,\,\V{e}_m(t)] \Mat{P}(t) \fullstop
	\label{eq:活动单位正交基引理_导数}
\end{equation}
此时,我们有
\begin{equation}
	\Mat{P}(t)\in\Skw \comma
\end{equation}
即 $\Mat{P}(t)$ 是一个\emphB{反对称矩阵}。

\begin{myProof}
对式~\eqref{eq:活动单位正交基引理_单位正交性} 两边求导,得
\begin{equation}
	\ipb{\dot{\V{e}}_i(t)}{\V{e}_j(t)}
	+\ipb{\V{e}_i(t)}{\dot{\V{e}}_j(t)} = 0 \in\realR \comma
\end{equation}
写成矩阵形式,为
\begin{equation}
	\mqty[\dot{\V{e}}_1\trans(t) \\ \vdots \\ \dot{\V{e}}_m\trans(t)]
	\mqty[\V{e}_1(t),\,\cdots,\,\V{e}_m(t)]
	+\mqty[\V{e}_1\trans(t) \\ \vdots \\ \V{e}_m\trans(t)]
	\mqty[\dot{\V{e}}_1(t),\,\cdots,\,\dot{\V{e}}_m(t)]
	=\Mat{0}\in\realR^{m\times m} \fullstop
\end{equation}
引入矩阵 $\Mat{E}=\qty[\V{e}_1(t),\,\cdots,\,\V{e}_m(t)]$,
则上式与 \eqref{eq:活动单位正交基引理_导数}~式可以分别表示成
\begin{equation}
	\dot{\Mat{E}}\trans\Mat{E}+\Mat{E}\trans\dot{\Mat{E}}
	=\Mat{0}\in\realR^{m\times m}
\end{equation}
和
\begin{equation}
	\dot{\Mat{E}} = \Mat{E}\Mat{P}\in\realR^{m\times m} \fullstop
\end{equation}
两式联立,可有
\begin{align}
	\Mat{0}&=\dot{\Mat{E}}\trans\Mat{E}
		+\Mat{E}\trans\dot{\Mat{E}} \notag \\
	&=\qty\big(\Mat{E}\Mat{P})\trans \Mat{E}
		+\Mat{E}\trans \qty\big(\Mat{E}\Mat{P}) \notag \\
	&=\Mat{P}\trans \qty\big(\Mat{E}\trans\Mat{E})
		+\qty\big(\Mat{E}\trans\Mat{E}) \Mat{P}
	=\Mat{P}\trans+\Mat{P} \comma
\end{align}
即 $\Mat{P}\trans=-\Mat{P}$。\myPROBLEM{按照定义},便知
$\Mat{P}(t)\in\Skw$。
\end{myProof}

根据这一引理,便可有
\begin{equation}
	\mqty[\dot{\V{T}}(s),\,\dot{\V{N}}(s),\,\dot{\V{B}}(s)]
	=\mqty[\V{T}(s),\,\V{N}(s),\,\V{B}(s)] \Mat{P}(s) \comma
\end{equation}
其中的 $\Mat{P}(s)$ 是一个三阶反对称矩阵。显然,它的对角元均为零:
\begin{equation}
	\Mat{P}(s)=\mqty[0 & \ast & \ast \\
		\ast & 0 & \ast \\
		\ast & \ast & 0 ] \fullstop
\end{equation}
这里,我们用“$\ast$”表示待定元素。
由 \eqref{eq:主单位法向量定义}~式,可知
\begin{equation}
	\dot{\V{T}}(s)
	=\norm[\realR^3]{\dot{\V{T}}(s)} \V{N}(s)
	\eqcolon \kappa(s)\,\V{N}(s) \fullstop
\end{equation}
式中的 $\kappa(s)\coloneq\norm[\realR^3]{\dot{\V{T}}(s)}$。
于是,矩阵 $\Mat{P}(s)$ 的第一列就成为了
$\qty[0,\,\kappa(s),\,0]\trans$。利用\emphB{反}对称性,可有
\begin{equation}
	\Mat{P}(s)=\mqty[0 & -\kappa(s) & 0 \\
		\kappa(s) & 0 & -\tau(s) \\
		0 & \tau(s) & 0] \fullstop
\end{equation}
我们“强行”引入了 $\tau(s)$,用来取代占位符 $\ast$。
当然,它的具体形式仍然待定。

\subsection{曲率和挠率}
利用矩阵 $\Mat{P}(s)$,可以看出
\begin{equation}
	\dot{\V{B}}(s)=-\tau(s)\,\V{N}(s) \fullstop
\end{equation}
再与 $\V{N}(s)$ 做内积\footnote{%
	为了表述的清晰,本小节中用“$\vdp$”来表示内积。},便有
\begin{equation}
	\dot{\V{B}}(s)\vdp\V{N}(s)=-\tau(s) \fullstop
\end{equation}
根据定义 \eqref{eq:副单位法向量定义}~式,
\begin{equation}
	\V{B}(s)=\V{T}(s)\cp\V{N}(s) \comma
\end{equation}
于是
\begin{align}
	\dot{\V{B}}(s)
	&=\dv{s}\qty\Big[\V{T}(s)\cp\V{N}(s)]
	=\dv{s}\qty[\dot{\V{r}}(s)\cp\frac{\ddot{\V{r}}(s)}
			{\norm[\realR^3]{\ddot{\V{r}}(s)}}] \notag \\
	&=\ddot{\V{r}}(s)\cp\frac{\ddot{\V{r}}(s)}
			{\norm[\realR^3]{\ddot{\V{r}}(s)}}
		+\dot{\V{r}}(s)\cp\frac{\dddot{\V{r}}(s)}
			{\norm[\realR^3]{\ddot{\V{r}}(s)}}
		+\dv{s}\qty(\frac{1}{\norm[\realR^3]{\ddot{\V{r}}(s)}}) \,
			\dot{\V{r}}(s)\cp\ddot{\V{r}}(s) \fullstop
\end{align}
显然,该式中的第一项为零。考虑 $\dot{\V{B}}(s)\vdp\V{N}(s)$,
注意到 $\V{N}(s)$ 与 $\ddot{\V{r}}(s)$ 平行,因此与
$\dot{\V{r}}(s)\cp\ddot{\V{r}}(s)$ 垂直,所以第三项在点乘
$\V{N}(s)$ 后也为零。这样便有
\begin{align}
	\tau(s)&=-\dot{\V{B}}(s)\vdp\V{N}(s) \notag \\
	&=-\dot{\V{r}}(s)\cp\frac{\dddot{\V{r}}(s)}
			{\norm[\realR^3]{\ddot{\V{r}}(s)}}
		\vdp\frac{\dddot{\V{r}}(s)}
			{\norm[\realR^3]{\ddot{\V{r}}(s)}} \notag \\
	&=-\frac{1}{\norm[\realR^3]{\ddot{\V{r}}(s)}^2}
		\qty\Big[\dot{\V{r}}(s) \cp \dddot{\V{r}}(s)
				\vdp \ddot{\V{r}}(s)] \notag
	\intertext{利用向量三重积的性质,再把负号移进来,可得}
	&=\frac{1}{\norm[\realR^3]{\ddot{\V{r}}(s)}^2}
		\det\!\mqty[\dot{\V{r}}(s),\,\ddot{\V{r}}(s),\,\dddot{\V{r}}(s)]
	\fullstop
\end{align}

\blankline

至此,我们就得到了 $\realR^3$ 空间中以\emphB{弧长}%
为参数的\emphB{Frenet 标架}:
\begin{braceEq*}{\label{eq:Frenet标架_弧长参数}}
	\V{T}(s) &= \dot{\V{r}}(s) \comma
	\label{eq:Frenet标架定义_T} \\
	\V{N}(s) &= \frac{\dot{\V{T}}(s)}{\norm[\realR^3]{\dot{\V{T}}(s)}}
		=\frac{\ddot{\V{r}}(s)}{\norm[\realR^3]{\ddot{\V{r}}(s)}} \comma
	\label{eq:Frenet标架定义_N} \\
	\V{B}(s) &= \V{T}(s)\cp\V{N}(s)
		=\frac{\dot{\V{r}}(s)\cp\ddot{\V{r}}(s)}
			{\norm[\realR^3]{\ddot{\V{r}}(s)}} \fullstop
	\label{eq:Frenet标架定义_B}
\end{braceEq*}
以及对应的标架运动方程
\begin{braceEq*}{\label{eq:Frenet标架运动方程_弧长参数}}
	\dot{\V{T}}(s)&=\kappa(s)\,\V{N}(s) \comma
	\label{eq:Frenet标架的导数_T} \\
	\dot{\V{N}}(s)&=-\kappa(s)\,\V{T}(s)+\tau(s)\,\V{B}(s) \comma
	\label{eq:Frenet标架的导数_N} \\
	\dot{\V{B}}(s)&=-\tau(s)\,\V{N}(s) \comma
	\label{eq:Frenet标架的导数_B}
\end{braceEq*}
其中,
\begin{equation}
	\kappa(s)\defeq\norm[\realR^3]{\dot{\V{T}}(s)}
	=\norm[\realR^3]{\ddot{\V{r}}(s)}
	\label{eq:曲率的定义}
\end{equation}
称为\emphA{曲率},
\begin{equation}
	\tau(s)\defeq\frac{1}{\norm[\realR^3]{\ddot{\V{r}}(s)}^2}
		\det\!\mqty[\dot{\V{r}}(s),\,\ddot{\V{r}}(s),\,\dddot{\V{r}}(s)]
	=\frac{1}{\kappa^2(s)}
		\det\!\mqty[\dot{\V{r}}(s),\,\ddot{\V{r}}(s),\,\dddot{\V{r}}(s)]
	\label{eq:挠率的定义}
\end{equation}
称为\emphA{挠率}。

\subsection{Frenet 标架的几何意义}
利用 Taylor 公式,把 $\V{r}(s_0+\incr s)$ 展开至三阶,可得
\begin{equation}
	\V{r}(s_0+\incr s)
	=\V{r}(s_0)+\dot{\V{r}}(s_0)\cdot(\incr s)
		+\frac{1}{2}\,\ddot{\V{r}}(s_0)\cdot(\incr s)^2
		+\frac{1}{6}\,\dddot{\V{r}}(s_0)\cdot(\incr s)^3
		+\sOv{(\incr s)^3} \fullstop
\end{equation}
其中,一阶和二阶导数可以分别表示为
\begin{equation}
	\dot{\V{r}}(s_0)=\V{T}(s_0)
\end{equation}
和
\begin{equation}
	\ddot{\V{r}}(s_0)=\norm[\realR^3]{\ddot{\V{r}}(s_0)} \V{N}(s_0)
	=\kappa(s_0)\,\V{N}(s_0) \fullstop
\end{equation}
至于三阶导数,可用一组单位正交基(此处当然要用 Frenet 标架)展开:
\begin{equation}
	\dddot{\V{r}}(s_0)
	=\qty[\dddot{\V{r}}(s_0)\vdp\V{T}(s_0)]\,\V{T}(s_0)
	+\qty[\dddot{\V{r}}(s_0)\vdp\V{N}(s_0)]\,\V{N}(s_0)
	+\qty[\dddot{\V{r}}(s_0)\vdp\V{B}(s_0)]\,\V{B}(s_0) \fullstop
\end{equation}
关于 $\V{T}(s_0)$、$\V{N}(s_0)$、$\V{B}(s_0)$ 合并同类项,可得
\begin{align}
	\V{r}(s_0+\incr s)
	&=\V{r}(s_0)
	+\qty[\incr s + \frac{\dddot{\V{r}}(s_0)\vdp\V{T}(s_0)}{6}\,
		(\incr s)^3]\,\V{T}(s_0) \notag \\*
	&\alspace\phantom{\V{r}(s_0)}
		+\qty[\frac{\kappa(s_0)}{2}\,(\incr s)^2
			+\frac{\dddot{\V{r}}(s_0)\vdp\V{N}(s_0)}{6}\,(\incr s)^3]\,
		\V{N}(s_0) \notag \\*
	&\alspace\phantom{\V{r}(s_0)}
		+\qty[\frac{\dddot{\V{r}}(s_0)\vdp\V{B}(s_0)}{6}\,(\incr s)^3]\,
		\V{B}(s_0) + \sOv{(\incr s)^3}
\end{align}

\subsection{应用:速度与加速度}
三维空间中,质点的运动\emphA{轨迹}可以用 $\realR^3$ 中的曲线
$\V{r}(t)$ 来表示,其中的参数 $t$ 为时间。
质点运动的\emphA{速度} $\V{v}(t)$,定义为
\begin{equation}
	\V{v}(t)\defeq\dot{\V{r}}(t)
	\coloneq\dv{\V{r}}{t} (t)
	=\dv{\V{r}}{s} (s) \cdot \dv{s}{t} (t)
	=\dot{\V{r}}(s)\cdot\dv{s}{t} (t) \fullstop
\end{equation}
由式~\eqref{eq:弧长的定义},弧长的定义为
\begin{equation}
	s(t)=\int_{t_0}^{t} \norm{\dot{\V{r}}(\xi)}\dd{\xi} \comma
\end{equation}
求导,可得
\begin{equation}
	\dv{s}{t} (t) = \norm{\dot{\V{r}}(t)} \fullstop
	\label{eq:弧长的导数}
\end{equation}
再代入 Frenet 标架的定义 \eqref{eq:Frenet标架定义_T}~式,便有
\begin{equation}
	\V{v}(t)=\norm{\dot{\V{r}}(t)}\,\V{T}(s) \fullstop
\end{equation}
可见,速度 $\V{v}(t)$ 的方向与 $\V{T}(t)$ 平行。另外由于 $\V{T}(t)$
是单位向量,因此速度的大小
\begin{equation}
	\norm{\V{v}(t)}=\norm{\dot{\V{r}}(t)} \comma
\end{equation}
我们称之为\emphA{速率}。

速度相对时间的变化率称为\emphA{加速度}:
\begin{align}
	\V{a}(t)\defeq\dot{\V{v}}(t)
	&=\dv{t}\norm{\dot{\V{r}}(t)} \cdot \V{T}(s)
		+\norm{\dot{\V{r}}(t)} \cdot \dv{\V{T}}{t} (s) \notag \\
	&=\dv{t}\norm{\dot{\V{r}}(t)} \cdot \V{T}(s)
		+\norm{\dot{\V{r}}(t)} \cdot \dot{\V{T}}(s) \cdot
		\dv{s}{t} (t) \notag
	\intertext{代入标架运动方程 \eqref{eq:Frenet标架的导数_T}~式以及
		弧长的导数 \eqref{eq:弧长的导数}~式,可有}
	&=\dv{t}\norm{\dot{\V{r}}(t)} \cdot \V{T}(s)
		+\norm{\dot{\V{r}}(t)} \cdot \kappa(s)\,\V{N}(s)
			\cdot \norm{\dot{\V{r}}(t)} \notag
	\intertext{换成速率,则为}
	&=\dv{t}\norm{\V{v}(t)} \cdot \V{T}(s)
	+\norm{\V{v}(t)}^2\,\kappa(s)\,\V{N}(s) \fullstop
\end{align}
由此可知,加速度只有 $\V{T}$ 分量和 $\V{N}$ 分量;
当质点做\emphB{匀变速}运动时,则只有 $\V{N}$ 分量。

\section{Frenet 标架(一般参数)}
\label{sec:Frenet标架_一般参数}
本节我们重回\emphB{一般参数}下的映照形式,即
\begin{equation}
	\mmap{\V{r}(t)}{[\alpha,\,\beta]\ni t}{\V{r}(t)\in\realR^3} \comma
\end{equation}
其中的参数 $t$ 与弧长 $s$
的关系同式~\eqref{eq:弧长与一般参数的关系}:
\begin{equation}
	\dv{s}{t} (t)=\norm[\realR^3]{\dot{\V{r}}(t)} \fullstop
	\label{eq:弧长与一般参数的关系2}
\end{equation}

后文的推导需要用到向量的\emphA{内蕴正交分解}:
$\forall\,\V{\xi},\,\V{e}\in\realR^3$ 且满足
$\norm[\realR^3]{\V{e}}=1$,有
\begin{equation}
	\V{\xi}=\qty\big(\V{\xi}\vdp\V{e})\,\V{e}
		-\qty\big(\V{\xi}\cp\V{e})\cp\V{e} \fullstop
	\label{eq:内蕴正交分解_Chapter6}
\end{equation}
证明见\myPROBLEM{内蕴正交分解}。

\subsection{标架的形式}
首先来处理单位切向量:\footnote{
	此时我们把 $s$ 作为参数,只不过 $s=s(t)$。所以 $\V{T}(s)$
	实际上和 $\V{T}(t)$ 相等。
	\label{fn:T(s)=T(t)}}
\begin{equation}
	\V{T}(s)\defeq\dot{\V{r}}(s)=\dot{\V{r}}(t)\cdot\dv{t}{s} (s)
	=\flatfrac{\dot{\V{r}}(t)}{\dv{s}{t} (t)}
	=\frac{\dot{\V{r}}(t)}{\norm[\realR^3]{\dot{\V{r}}(t)}} \fullstop
	\label{eq:T_一般参数}
\end{equation}
接下来计算它的导数 $\dot{\V{T}}(s)$:
\begin{align}
	\dot{\V{T}}(s)
	&=\dv{s}\qty[\frac{\dot{\V{r}}(t)}
		{\norm[\realR^3]{\dot{\V{r}}(t)}}] \notag \\
	&=\dv{t}\qty[\frac{\dot{\V{r}}(t)}
		{\norm[\realR^3]{\dot{\V{r}}(t)}}] \cdot \dv{t}{s} (s) \notag
	\intertext{代入 \eqref{eq:弧长与一般参数的关系2}~式,得}
	&=\frac{1}{\norm[\realR^3]{\dot{\V{r}}(t)}}
		\dv{t}\qty[\frac{\dot{\V{r}}(t)}
			{\norm[\realR^3]{\dot{\V{r}}(t)}}] \notag \\
	&=\frac{1}{\norm[\realR^3]{\dot{\V{r}}(t)}^2} \qty[\ddot{\V{r}}(t)
			-\frac{\dot{\V{r}}(t)}{\norm[\realR^3]{\dot{\V{r}}(t)}}
			\cdot\dv{t}\norm[\realR^3]{\dot{\V{r}}(t)}] \fullstop
	\label{eq:T的导数_一般参数_Part1}
\end{align}
此处涉及到了模的导数。考虑
\begin{equation}
	\dv{t}\qty\Big[\norm[\realR^3]{\dot{\V{r}}(t)}^2]
	=2\,\norm[\realR^3]{\dot{\V{r}}(t)}
		\cdot \dv{t}\norm[\realR^3]{\dot{\V{r}}(t)} \semicolon
\end{equation}
另一方面,
\begin{equation}
	\dv{t}\qty\Big[\norm[\realR^3]{\dot{\V{r}}(t)}^2]
	=\dv{t}\qty\big[\dot{\V{r}}(t)\vdp\dot{\V{r}}(t)]
	=2\,\dot{\V{r}}(t)\vdp\ddot{\V{r}}(t) \fullstop
\end{equation}
联立两式,便有
\begin{equation}
	\dv{t}\norm[\realR^3]{\dot{\V{r}}(t)}
	=\ddot{\V{r}}(t) \vdp
		\frac{\dot{\V{r}}(t)}{\norm[\realR^3]{\dot{\V{r}}(t)}} \fullstop
\end{equation}
代回 \eqref{eq:T的导数_一般参数_Part1}~式,继续推导:
\begin{align}
	\dot{\V{T}}(s)
	&=\frac{1}{\norm[\realR^3]{\dot{\V{r}}(t)}^2} \qty[\ddot{\V{r}}(t)
		-\qty(\ddot{\V{r}}(t) \vdp
			\frac{\dot{\V{r}}(t)}{\norm[\realR^3]{\dot{\V{r}}(t)}})
		\frac{\dot{\V{r}}(t)}{\norm[\realR^3]{\dot{\V{r}}(t)}}] \notag
	\intertext{式中的 $\flatfrac{\dot{\V{r}}(t)}{\norm{\dot{\V{r}}(t)}}$
		是一个单位向量,它相当于式~\eqref{eq:内蕴正交分解_Chapter6}
		中的 $\V{e}$,而 $\ddot{\V{r}}(t)$ 则相当于 $\V{\xi}$。因此,
		利用向量的内蕴正交分解,有}
	&=-\frac{1}{\norm[\realR^3]{\dot{\V{r}}(t)}^2}
		\qty[\qty(\ddot{\V{r}}(t)
			\cp \frac{\dot{\V{r}}(t)}{\norm[\realR^3]{\dot{\V{r}}(t)}})
		\cp \frac{\dot{\V{r}}(t)}{\norm[\realR^3]{\dot{\V{r}}(t)}}]
		\notag \\
	&=-\frac{\qty\big(\ddot{\V{r}}(t)\cp\dot{\V{r}}(t))
			\cp\dot{\V{r}}(t)}{\norm[\realR^3]{\dot{\V{r}}(t)}^4} \fullstop
	\label{eq:T的导数_一般参数_Part2}
\end{align}

根据曲率的定义 \eqref{eq:曲率的定义}~式,
\begin{align}
	\kappa(s)\defeq\norm[\realR^3]{\dot{\V{T}}(s)}
	=\frac{\norm[\realR^3]{\qty\big(\ddot{\V{r}}(t)\cp\dot{\V{r}}(t))
		\cp\dot{\V{r}}(t)}}{\norm[\realR^3]{\dot{\V{r}}(t)}^4} \fullstop
\end{align}
注意到 $\qty\big(\ddot{\V{r}}(t)\cp\dot{\V{r}}(t))
\perp\dot{\V{r}}(t)$,所以
\begin{equation}
	\norm[\realR^3]{\qty\big(\ddot{\V{r}}(t)\cp\dot{\V{r}}(t))
		\cp\dot{\V{r}}(t)}
	=\norm[\realR^3]{\ddot{\V{r}}(t)\cp\dot{\V{r}}(t)}
		\cdot\norm[\realR^3]{\dot{\V{r}}(t)} \fullstop
\end{equation}
这样,曲率就能够写成
\begin{equation}
	\kappa(s)=\frac{\norm[\realR^3]{\ddot{\V{r}}(t)\cp\dot{\V{r}}(t)}}
		{\norm[\realR^3]{\dot{\V{r}}(t)}^3} \fullstop
	\label{eq:曲率_一般参数}
\end{equation}
利用定义 \eqref{eq:Frenet标架定义_N}~式,
$\V{N}(s)$ 也便可以易如反掌地写出来了:
\begin{equation}
	\V{N}(s)\defeq\frac{\dot{\V{T}}(s)}
		{\norm[\realR^3]{\dot{\V{T}}(s)}}
	=\frac{\dot{\V{T}}(s)}{\kappa(s)}
	=-\frac{\qty\big(\ddot{\V{r}}(t)\cp\dot{\V{r}}(t))
			\cp\dot{\V{r}}(t)}
		{\norm[\realR^3]{\ddot{\V{r}}(t)\cp\dot{\V{r}}(t)}
			\norm[\realR^3]{\dot{\V{r}}(t)}} \fullstop
\end{equation}

最后轮到 $\V{B}(s)$ 了:
\begin{align}
	\V{B}(s)=\V{T}(s)\cp\V{N}(s)
	&=\frac{\dot{\V{r}}(t)}{\norm[\realR^3]{\dot{\V{r}}(t)}}
		\cp\qty[-\frac{\qty\big(\ddot{\V{r}}(t)\cp\dot{\V{r}}(t))
				\cp\dot{\V{r}}(t)}
			{\norm[\realR^3]{\ddot{\V{r}}(t)\cp\dot{\V{r}}(t)}
				\norm[\realR^3]{\dot{\V{r}}(t)}}] \notag \\
	&=\frac{\dot{\V{r}}(t)}
			{\norm[\realR^3]{\ddot{\V{r}}(t)\cp\dot{\V{r}}(t)}}
		\cp\qty[-\qty(\ddot{\V{r}}(t)
				\cp\frac{\dot{\V{r}}(t)}{\norm[\realR^3]{\dot{\V{r}}(t)}})
			\cp\frac{\dot{\V{r}}(t)}
				{\norm[\realR^3]{\dot{\V{r}}(t)}}] \notag
	\intertext{这样处理是为了倒过来应用\emphB{内蕴正交分解}:}
	&=\frac{\dot{\V{r}}(t)}
			{\norm[\realR^3]{\ddot{\V{r}}(t)\cp\dot{\V{r}}(t)}}
		\cp\qty[\ddot{\V{r}}(t)-\qty(\ddot{\V{r}}(t) \vdp
				\frac{\dot{\V{r}}(t)}{\norm[\realR^3]{\dot{\V{r}}(t)}})
			\frac{\dot{\V{r}}(t)}{\norm[\realR^3]{\dot{\V{r}}(t)}}] \notag
	\intertext{由于 $\dot{\V{r}}(t)\cp\dot{\V{r}}(t)=0$,
		因而方括号中的第二项可以略去,使得结果大为简化:}
	&=\frac{\dot{\V{r}}(t)\cp\ddot{\V{r}}(t)}
			{\norm[\realR^3]{\ddot{\V{r}}(t)\cp\dot{\V{r}}(t)}} \fullstop
	\label{eq:B_一般参数}
\end{align}

\subsection{曲率和挠率}
曲率在计算 $\V{N}$ 的时候已经顺带求过了,我们现在来求挠率。
根据式~\eqref{eq:挠率的定义},有
\begin{equation}
	\tau(s)=\frac{1}{\kappa^2(s)}
		\det\!\mqty[\dot{\V{r}}(s),\,\ddot{\V{r}}(s),\,\dddot{\V{r}}(s)]
	\fullstop
\end{equation}
因此首先需要知道 $\V{r}$ 关于 $s$ 的一至三阶导数。
由 \eqref{eq:T_一般参数}~式,可知
\begin{equation}
	\dot{\V{r}}(s)=\V{T}(s)
	=\frac{\dot{\V{r}}(t)}{\norm[\realR^3]{\dot{\V{r}}(t)}} \semicolon
\end{equation}
二阶导数则要利用式~\eqref{eq:T的导数_一般参数_Part2}:
\begin{equation}
	\ddot{\V{r}}(s)=\dot{\V{T}}(s)
	=-\frac{\qty\big(\ddot{\V{r}}(t)\cp\dot{\V{r}}(t))
		\cp\dot{\V{r}}(t)}{\norm[\realR^3]{\dot{\V{r}}(t)}^4} \semicolon
\end{equation}
进而又可得到三阶导数:
\begin{equation}
	\dddot{\V{r}}(s)
	=\dv{\ddot{\V{r}}}{s} (s)
	=-\dv{\ddot{\V{r}}}{t} (s) \cdot \dv{t}{s}
	=-\flatfrac{\dv{\ddot{\V{r}}}{t} (s)}{\dv{s}{t} (t)}
	=-\dv{t}\qty[\frac{\qty\big(\ddot{\V{r}}(t)\cp\dot{\V{r}}(t))
			\cp\dot{\V{r}}(t)}{\norm[\realR^3]{\dot{\V{r}}(t)}^4}]
		\cdot\frac{1}{\norm[\realR^3]{\dot{\V{r}}(t)}} \fullstop
\end{equation}
最后一步仍然利用了 \eqref{eq:弧长与一般参数的关系2}~式。
我们知道,
\begin{equation*}
	\det\!\mqty[\dot{\V{r}}(s),\,\ddot{\V{r}}(s),\,\dddot{\V{r}}(s)]
	=\dot{\V{r}}(s)\cp\ddot{\V{r}}(s)\vdp\dddot{\V{r}}(s) \fullstop
\end{equation*}
首先考察叉乘项:
\begin{align}
	\dot{\V{r}}(s)\cp\ddot{\V{r}}(s)
	&=\frac{\dot{\V{r}}(t)}{\norm[\realR^3]{\dot{\V{r}}(t)}}
		\cp \qty[-\frac{\qty\big(\ddot{\V{r}}(t)\cp\dot{\V{r}}(t))
			\cp\dot{\V{r}}(t)}{\norm[\realR^3]{\dot{\V{r}}(t)}^4}] \notag
	\intertext{注意到该式的结构与 \eqref{eq:B_一般参数}~式的
		第二步非常相似,因此我们采用同样的办法处理,即先调整系数,
		再反向运用\emphB{内蕴正交分解}:}
	&=\frac{\dot{\V{r}}(t)}{\norm[\realR^3]{\dot{\V{r}}(t)}^3}
		\cp\qty[-\qty(\ddot{\V{r}}(t)
				\cp\frac{\dot{\V{r}}(t)}{\norm[\realR^3]{\dot{\V{r}}(t)}})
			\cp\frac{\dot{\V{r}}(t)}
				{\norm[\realR^3]{\dot{\V{r}}(t)}}] \notag \\
	&=\frac{\dot{\V{r}}(t)}{\norm[\realR^3]{\dot{\V{r}}(t)}^3}
		\cp\qty[\ddot{\V{r}}(t)-\qty(\ddot{\V{r}}(t) \vdp
				\frac{\dot{\V{r}}(t)}{\norm[\realR^3]{\dot{\V{r}}(t)}})
			\frac{\dot{\V{r}}(t)}
				{\norm[\realR^3]{\dot{\V{r}}(t)}}] \notag
	\intertext{同样,因为 $\dot{\V{r}}(t)\cp\dot{\V{r}}(t)=0$,
		所以又只剩下了第一项,即}
	&=\frac{\dot{\V{r}}(t)\cp\ddot{\V{r}}(t)}
			{\norm[\realR^3]{\dot{\V{r}}(t)}^3} \fullstop
\end{align}
再来看 $\dddot{\V{r}}(t)$。
\begin{align}
	\dddot{\V{r}}(t)
	&=-\dv{t}\qty[\frac{\qty\big(\ddot{\V{r}}(t)\cp\dot{\V{r}}(t))
			\cp\dot{\V{r}}(t)}{\norm[\realR^3]{\dot{\V{r}}(t)}^4}]
		\cdot\frac{1}{\norm[\realR^3]{\dot{\V{r}}(t)}} \notag \\
	&=-\frac{\displaystyle
			\dv{t}\qty\Big[\qty\big(\ddot{\V{r}}(t)\cp\dot{\V{r}}(t))
				\cp\dot{\V{r}}(t)] \cdot \norm[\realR^3]{\dot{\V{r}}(t)}^4
			+\qty\Big[\qty\big(\ddot{\V{r}}(t)\cp\dot{\V{r}}(t))
				\cp\dot{\V{r}}(t)]
				\cdot \dv{t}\qty\Big[\norm[\realR^3]{\dot{\V{r}}(t)}^4]}
		{\norm[\realR^3]{\dot{\V{r}}(t)}^8}
		\cdot\frac{1}{\norm[\realR^3]{\dot{\V{r}}(t)}} \notag \\
	&=-\frac{\displaystyle
			\dv{t}\qty\big(\ddot{\V{r}}(t)\cp\dot{\V{r}}(t))
				\cp\dot{\V{r}}(t)
			+\qty\big(\ddot{\V{r}}(t)\cp\dot{\V{r}}(t))
				\cp\dv{\V{\ddot{r}}}{t} (t)}
			{\norm[\realR^3]{\dot{\V{r}}(t)}^5}
		-\frac{\displaystyle
				\dv{t}\qty\Big[\norm[\realR^3]{\dot{\V{r}}(t)}^4]}
			{\norm[\realR^3]{\dot{\V{r}}(t)}^9}
			\cdot\qty\Big[\qty\big(\ddot{\V{r}}(t)\cp\dot{\V{r}}(t))
				\cp\dot{\V{r}}(t)] \notag \\
	&=-\frac{\qty\big(\dddot{\V{r}}(t)\cp\dot{\V{r}}(t)
				+\ddot{\V{r}}(t)\cp\ddot{\V{r}}(t)) \cp \dot{\V{r}}(t)
			+\qty\big(\ddot{\V{r}}(t)\cp\dot{\V{r}}(t))
				\cp \dddot{\V{r}}(t)}{\norm[\realR^3]{\dot{\V{r}}(t)}^5}
		-\frac{\displaystyle
				\dv{t}\qty\Big[\norm[\realR^3]{\dot{\V{r}}(t)}^4]}
			{\norm[\realR^3]{\dot{\V{r}}(t)}^9}
			\cdot\qty\Big[\qty\big(\ddot{\V{r}}(t)\cp\dot{\V{r}}(t))
				\cp\dot{\V{r}}(t)] \notag
	\intertext{第一项中,$\ddot{\V{r}}(t)\cp\ddot{\V{r}}(t)=0$。所以}
	&=-\frac{\qty\big(\dddot{\V{r}}(t)\cp\dot{\V{r}}(t))
				\cp\dot{\V{r}}(t)
			+\hl{\qty\big(\ddot{\V{r}}(t)\cp\dot{\V{r}}(t))}
				\cp\dddot{\V{r}}(t)}{\norm[\realR^3]{\dot{\V{r}}(t)}^5}
		-\frac{\displaystyle
				\dv{t}\qty\Big[\norm[\realR^3]{\dot{\V{r}}(t)}^4]}
			{\norm[\realR^3]{\dot{\V{r}}(t)}^9}
			\cdot\qty\Big[\hl{\qty\big(\ddot{\V{r}}(t)\cp\dot{\V{r}}(t))}
				\cp\dot{\V{r}}(t)] \fullstop
\end{align}
式中的高亮部分与另一个向量做叉乘后,垂直于
$\ddot{\V{r}}(t)\cp\dot{\V{r}}(t)$;
而另一方面,$\dot{\V{r}}(s)\cp\ddot{\V{r}}(s)$ 又平行于
$\dot{\V{r}}(t)\cp\ddot{\V{r}}(t)$。因此二者做点乘后即为零。
这样,我们就有
\begin{align}
	\dot{\V{r}}(s)\cp\ddot{\V{r}}(s)\vdp\dddot{\V{r}}(s)
	&=\frac{\dot{\V{r}}(t)\cp\ddot{\V{r}}(t)}
			{\norm[\realR^3]{\dot{\V{r}}(t)}^3}
		\vdp\qty[-\frac{\qty\big(\dddot{\V{r}}(t)\cp\dot{\V{r}}(t))
				\cp\dot{\V{r}}(t)}
			{\norm[\realR^3]{\dot{\V{r}}(t)}^5}] \notag \\
	&=\frac{\dot{\V{r}}(t)\cp\ddot{\V{r}}(t)}
			{\norm[\realR^3]{\dot{\V{r}}(t)}^6}
		\vdp\qty[-\qty(\dddot{\V{r}}(t)
				\cp\frac{\dot{\V{r}}(t)}{\norm[\realR^3]{\dot{\V{r}}(t)}})
			\cp\frac{\dot{\V{r}}(t)}
				{\norm[\realR^3]{\dot{\V{r}}(t)}}] \notag
	\intertext{照例,使用\emphB{内蕴正交分解}:}
	&=\frac{\dot{\V{r}}(t)\cp\ddot{\V{r}}(t)}
			{\norm[\realR^3]{\dot{\V{r}}(t)}^6}
		\vdp\qty[\dddot{\V{r}}(t)-\qty(\dddot{\V{r}}(t) \vdp
				\frac{\dot{\V{r}}(t)}{\norm[\realR^3]{\dot{\V{r}}(t)}})
			\frac{\dot{\V{r}}(t)}
				{\norm[\realR^3]{\dot{\V{r}}(t)}}] \notag
	\intertext{第二项点乘后为零:}
	&=\frac{\dot{\V{r}}(t)\cp\ddot{\V{r}}(t)\vdp\dddot{\V{r}}(t)}
			{\norm[\realR^3]{\dot{\V{r}}(t)}^6} \fullstop
\end{align}
此即
\begin{equation}
	\det\!\mqty[\dot{\V{r}}(s),\,\ddot{\V{r}}(s),\,\dddot{\V{r}}(s)]
	=\frac{1}{\norm[\realR^3]{\dot{\V{r}}(t)}^6}
		\det\!\mqty[\dot{\V{r}}(t),\,\ddot{\V{r}}(t),\,\dddot{\V{r}}(t)]
	\fullstop
\end{equation}
再代入一般参数下曲率的表达式~\eqref{eq:曲率_一般参数},
就可得到挠率:
\begin{align}
	\tau(s)&=\frac{1}{\kappa^2(s)}
		\cdot\frac{1}{\norm[\realR^3]{\dot{\V{r}}(t)}^6}
		\det\!\mqty[\dot{\V{r}}(t),\,\ddot{\V{r}}(t),\,\dddot{\V{r}}(t)]
		\notag \\
	&=\frac{\norm[\realR^3]{\dot{\V{r}}(t)}^6}
			{\norm[\realR^3]{\ddot{\V{r}}(t)\cp\dot{\V{r}}(t)}^2}
		\cdot\frac{1}{\norm[\realR^3]{\dot{\V{r}}(t)}^6}
		\det\!\mqty[\dot{\V{r}}(t),\,\ddot{\V{r}}(t),\,\dddot{\V{r}}(t)]
		\notag \\
	&=\frac{1}{\norm[\realR^3]{\ddot{\V{r}}(t)\cp\dot{\V{r}}(t)}^2}
		\det\!\mqty[\dot{\V{r}}(t),\,\ddot{\V{r}}(t),\,\dddot{\V{r}}(t)]
		\fullstop
\end{align}

\blankline

现在来总结一下\emphB{一般参数}下的 Frenet 标架:
\begin{braceEq*}{\label{eq:Frenet标架_一般参数}}
	\V{T}(t)&=\frac{\dot{\V{r}}(t)}{\norm[\realR^3]{\dot{\V{r}}(t)}}
		\comma \\
	\V{N}(t)&=-\frac{\qty\big(\ddot{\V{r}}(t)\cp\dot{\V{r}}(t))
			\cp\dot{\V{r}}(t)}
		{\norm[\realR^3]{\ddot{\V{r}}(t)\cp\dot{\V{r}}(t)}
			\norm[\realR^3]{\dot{\V{r}}(t)}} \comma \\
	\V{B}(t)&=\frac{\dot{\V{r}}(t)\cp\ddot{\V{r}}(t)}
		{\norm[\realR^3]{\ddot{\V{r}}(t)\cp\dot{\V{r}}(t)}} \fullstop
\end{braceEq*}
曲率和挠率分别为
\begin{equation}
	\kappa(t)=\frac{\norm[\realR^3]{\ddot{\V{r}}(t)\cp\dot{\V{r}}(t)}}
		{\norm[\realR^3]{\dot{\V{r}}(t)}^3}
\end{equation}
和
\begin{equation}
	\tau(t)=
	\frac{1}{\norm[\realR^3]{\ddot{\V{r}}(t)\cp\dot{\V{r}}(t)}^2}
	\det\!\mqty[\dot{\V{r}}(t),\,\ddot{\V{r}}(t),\,\dddot{\V{r}}(t)]
	\fullstop
\end{equation}
注意我们把参数全部换成了 $t$。\footnote{
	根据第~\pageref{fn:T(s)=T(t)}~页的脚注~\ref{fn:T(s)=T(t)},
	$\V{T}(s)=\V{T}(t)$。类似地,还有 $\kappa(s)=\kappa(t)$ 等。
实际上,它们是同一个量在不同参数下的表示。
但 $\dot{\V{T}}(s)\neq\dot{\V{T}}(t)$。这是因为前者是对 $s$ 求导,
而后者则是对 $t$ 求导。不要被符号迷惑。}

至于标架运动方程,则可直接利用
\eqref{eq:Frenet标架运动方程_弧长参数}~式:
\begin{braceEq*}{\label{eq:Frenet标架运动方程_一般参数}}
	\dot{\V{T}}(t)&=\dot{\V{T}}(s)\cdot\dv{s}{t} (t)
		=\norm[\realR^3]{\dot{\V{r}}(t)} \cdot
			\qty\big[\kappa(t)\,\V{N}(t)] \comma \\
	\dot{\V{N}}(t)&=\dot{\V{N}}(s)\cdot\dv{s}{t} (t)
		=\norm[\realR^3]{\dot{\V{r}}(t)} \cdot
			\qty\big[-\kappa(t)\,\V{T}(t)+\tau(t)\,\V{B}(t)] \comma \\
	\dot{\V{B}}(t)&=\dot{\V{B}}(s)\cdot\dv{s}{t} (t)
		=\norm[\realR^3]{\dot{\V{r}}(t)} \cdot
			\qty\big[-\tau(t)\,\V{N}(t)] \fullstop
\end{braceEq*}

若 $t$ 取为 $s$,有
\begin{equation}
	\norm[\realR^3]{\dot{\V{r}}(t)}
	=\norm[\realR^3]{\dot{\V{r}}(s)}=1 \fullstop
\end{equation}
根据式~\eqref{eq:r的一阶导数垂直二阶导数_弧长参数},
$\ddot{\V{r}}(s) \perp \dot{\V{r}}(s)$,因此
\begin{equation}
	\norm[\realR^3]{\ddot{\V{r}}(t)\cp\dot{\V{r}}(t)}
	=\norm[\realR^3]{\ddot{\V{r}}(s)\cp\dot{\V{r}}(s)}
	=\norm[\realR^3]{\ddot{\V{r}}(t)}
		\cdot \norm[\realR^3]{\dot{\V{r}}(t)}
	=\norm[\realR^3]{\ddot{\V{r}}(t)} \fullstop
\end{equation}
利用内蕴正交分解,稍做计算,还可知
\begin{equation}
	\qty\big(\ddot{\V{r}}(t)\cp\dot{\V{r}}(t))\cp\dot{\V{r}}(t)
	=-\ddot{\V{r}}(t)+
		\qty\big(\ddot{\V{r}}(t)\vdp\dot{\V{r}}(t))\,\dot{\V{r}}(t)
	=-\ddot{\V{r}}(t) \fullstop
\end{equation}
此时,把 \eqref{eq:Frenet标架_一般参数}~\linktilde
~\eqref{eq:Frenet标架运动方程_一般参数}~式
与 \eqref{eq:Frenet标架_弧长参数}~\linktilde
~\eqref{eq:挠率的定义}~式进行比较,可以发现它们是完全一样的。