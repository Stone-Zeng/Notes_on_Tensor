\section{曲面的定义;切空间}
\subsection{曲面的定义}
自变量维数比因变量维数低一维的向量值映照,都可以称为\emphA{曲面}。
常见的三维曲面,其自变量是二维的,正符合了该定义。

如图~\ref{fig:曲面的定义},
$m+1$ 维 Euclid 空间中的 $m$ 维曲面 $\V{\Sigma}(\V{x})$,
都可以用如下的向量值映照表示:
\begin{equation}
	\mmap{\V{\Sigma}(\V{x})}
		{\domD{\V{x}}\ni\V{x}=\mqty[x^1 \\ \vdots \\ x^m]}
		{\V{\Sigma}(\V{x})
			=\mqty[\Sigma^1 \\ \vdots \\ \Sigma^m \\ \Sigma^{m+1}](\V{x})
			\in\Rm*} \fullstop
\end{equation}
式中的 $\domD{\V{x}}$ 代表\emphB{参数域}。

\begin{figure}[h]
	\centering
	\includegraphics{Images/Surface_Definition.PNG}
	\caption{$m+1$ 维 Euclid 空间中 $m$ 维曲面的定义}
	\label{fig:曲面的定义}
\end{figure}

\subsection{切向量与切空间}
循着与体积上张量场场论(见第\ref{chap:微分同胚}章)相同的思路,
我们先来计算 $\V{\Sigma}(\V{x})$ 的 Jacobi 矩阵:
\begin{equation}
	\JacobiD{\V{\Sigma}}(\V{x})
	=\mqty[\displaystyle \pdv{\V{\Sigma}}{x^1},\,
			\cdots,\,\pdv{\V{\Sigma}}{x^m}] (\V{x})
	\defeq\mqty[\V{g}_1,\,\cdots,\,\V{g}_m] (\V{x})
		\in\realR^{(m+1)\times m} \comma
	\label{eq:曲面映照的Jacobi矩阵}
\end{equation}
式中,
\begin{equation}
	\V{g}_\mu(\V{x}) \defeq \pdv{\V{\Sigma}}{x^\mu} (\V{x})
	=\lim_{\lambda\to 0}
		\frac{\V{\Sigma}\qty(\V{x}+\lambda\,\V{e}_\mu)-\V{\Sigma}(\V{x})}
		{\lambda} \in\Rm* \fullstop
\end{equation}
需要注意,与微分同胚不同,此处的 Jacobi 矩阵\emphB{不是}方阵。
为了表述的清晰,本章中我们将用拉丁字母 $i$、$j$
代表指标 1 到 $m+1$,而用希腊字母 $\mu$、$\nu$ 代表指标 1 到 $m$。

为了考察 $\V{g}_\mu(\V{x})$ 的几何意义,
我们在定义域空间 $\domD{\V{x}}\in\Rm$ 中过点 $\V{x}$ 作出
$x^\mu$-线,其上的任意一点均可表示为 $\V{x}+\lambda\,\V{e}_\mu$。
在映照 $\V{\Sigma}(\V{x})$ 的作用下,
点 $\V{x}$ 和 $\V{x}+\lambda\,\V{e}_\mu$ 分别被映照到值域空间
$\domD{\V{\Sigma}}$ 中的点 $\Sigma(\V{x})$ 和
$\Sigma\qty(\V{x}+\lambda\,\V{e}_\mu)$。同时,$x^\mu$-线也被映照到
$\domD{\V{\Sigma}}$ 中,成为一条曲线。

当 $\lambda\to 0$ 时,
\begin{equation}
	\frac{\V{\Sigma}\qty(\V{x}+\lambda\,\V{e}_\mu)-\V{\Sigma}(\V{x})}
	{\lambda}
	\to \pdv{\V{\Sigma}}{x^\mu} (\V{x}) = \V{g}_\mu(\V{x}) \comma
\end{equation}
这便是值域空间中 $x^\mu$-线的\emphA{切向量}。
切向量张成了 $\Rm*$ 空间的一个子空间:
\begin{equation}
	\vspan\qty{\V{g}_\mu(\V{x})}^m_{\mu=1} \subset\Rm* \fullstop
\end{equation}

如果要构造一组基,必须使其满足\emphB{线性无关}的要求。
由此,我们引出\emphA{正则点}的定义,
它指能够使 Jacobi 矩阵 $\JacobiD{\V{\Sigma}}(\V{x})$ 列满秩,即
\begin{equation*}
	\rank\JacobiD{\V{\Sigma}}(\V{x})=m
\end{equation*}
的点 $\V{x}\in\domD{\V{x}}$ 或 $\V{\Sigma}(\V{x})\in\Rm*$。
在正则点处,便有 $\qty{\V{g}_\mu(\V{x})}^m_{\mu=1} \subset\Rm*$
线性无关。此时,切向量张成的空间称为\emphA{切空间}
(或\emphA{切平面}),记作
\begin{equation}
	\Tspace{\V{\Sigma}}{\V{x}}
	\defeq\vspan\qty{\V{g}_\mu(\V{x})}^m_{\mu=1} \subset\Rm*
	\fullstop
\end{equation}
切空间 $\Tspace{\V{\Sigma}}{\V{x}}$ 的维度是 $m$。

\subsection{局部基}
设位于点 $\V{x}$ 处的 $m+1$ 维单位向量 $\V{n}(\V{x})$ 满足
$\norm[\Rm*]{\V{n}(\V{x})}=1$ 且
\begin{equation}
	(\JacobiD{\V{\Sigma}})\trans(\V{x}) \vdp \V{n}(\V{x})
	=\mqty[\V{g}_1,\,\cdots,\,\V{g}_m]\trans (\V{x}) \vdp \V{n}(\V{x})
	=\mqty[\qty(\V{g}_1)\trans \\ \vdots \\ \qty(\V{g}_m)\trans]
		\vdp \V{n}(\V{x}) = \V{0}\in\Rm \fullstop
	\label{eq:曲面协变基_法向量}
\end{equation}
当 $\JacobiD{\V{\Sigma}}(\V{x})$ 列满秩(即 $\V{x}$ 是正则点)时,
根据线性代数中的\emphB{基扩张定理}
\myPROBLEM[2017-02-13]{基扩张定理},
这样的单位向量 $\V{n}(\V{x})$ 是唯一存在的,称为\emphA{法向量}。

令 $\V{g}_{m+1}\defeq\V{n}(\V{x})$,则
\begin{equation}
	\qty{\V{g}_i(\V{x})}^{m+1}_{i=1}
	=\qty{\pdv{\V{\Sigma}}{x^1} (\V{x}),\,\cdots,\,
		\pdv{\V{\Sigma}}{x^m} (\V{x}),\,\V{n}(\V{x})}
\end{equation}
是 $\Rm*$ 空间中的一组基。
它的指标都在下面,因此是一组\emphA{局部协变基}。

\blankline

接下来研究对应的\emphA{局部逆变基}
$\qty{\V{g}^i(\V{x})}^{m+1}_{i=1}$,
它与局部协变基共同满足\emphA{对偶关系}:\footnote{
	以下在不引起混淆的地方将省去“$(\V{x})$”,
	但仍不要忘记正则点的要求。}
\begin{equation}
	\ipb[\Rm*]{\V{g}_i}{\V{g}^j} = \KroneckerDelta{j}{i} \fullstop
	\label{eq:曲面局部基_对偶关系}
\end{equation}
写成矩阵形式,为
\begin{equation}
	\qty[\begin{array}{@{}c@{}}
		\qty(\V{g}^1)\trans \\ \vdots \\ \qty\big(\V{g}^m)\trans \\[3pt]
		\hdashline \\[-12pt]
		\qty(\V{g}^{m+1})\trans
	\end{array}] \,
	\qty[\begin{array}{@{}c:c@{}}
		\V{g}_1,\,\cdots,\,\V{g}_m & \V{n}
	\end{array}]
	=\qty[\begin{array}{@{}c:c@{}}
		\mqty[\qty(\V{g}^\nu)\trans\vdp\V{g}_\mu]^m_{\mu,\,\nu=1} &
			\mqty[\qty(\V{g}^1)\trans \\ \vdots \\ \qty(\V{g}^m)\trans]
			\vdp \V{n} \\[3pt]
		\hdashline \\[-12pt]
		\qty(\V{g}^{m+1})\trans \vdp \mqty[\V{g}_1,\,\cdots,\,\V{g}_m] &
		\qty(\V{g}^{m+1})\trans \vdp \V{n}
	\end{array}]
	=\Mat{I}_{m+1} \fullstop
	\label{eq:曲面局部基_对偶关系_矩阵形式}
\end{equation}

按照分块矩阵的计算法则,显然有
$\qty(\V{g}^{m+1})\trans \vdp \V{n}=1$。于是
\begin{equation}
	\V{g}^{m+1} = \V{n}+\sum_{\mu=1}^{m} a_\mu\,\V{g}_\mu \comma
	\label{eq:曲面逆变基m+1_完整形式}
\end{equation}
式中的 $a_\mu$ 是待定系数。

考虑式~\eqref{eq:曲面局部基_对偶关系_矩阵形式} 中矩阵的左下角,有
\begin{align}
	\qty(\V{g}^{m+1})\trans \vdp \mqty[\V{g}_1,\,\cdots,\,\V{g}_m]
	&=\qty(\V{n}+\sum_{\mu=1}^{m} a_\mu\,\V{g}_\mu)\trans
		\vdp \mqty[\V{g}_1,\,\cdots,\,\V{g}_m] \notag
	\intertext{根据 \eqref{eq:曲面协变基_法向量}~式,
		$\V{n}$ 与 $\V{g}_\mu$ 正交:}
	&=\sum_{\mu=1}^{m} a_\mu\,\qty(\V{g}_\mu)\trans
		\vdp \mqty[\V{g}_1,\,\cdots,\,\V{g}_m] \notag \\
	&=\mqty[\displaystyle
			\sum_{\mu=1}^{m} a_\mu\,\qty(\V{g}_\mu)\trans
				\vdp\V{g}_1,\,\cdots,\,
			\sum_{\mu=1}^{m} a_\mu\,\qty(\V{g}_\mu)\trans
				\vdp\V{g}_m] \notag \\
	&\defeq \mqty[\displaystyle
			\sum_{\mu=1}^{m} a_\mu\,g_{1\mu},\,\cdots,\,
			\sum_{\mu=1}^{m} a_\mu\,g_{m\mu}] \notag \\
	&=\V{0}\in\realR^{1\times m} \fullstop
\end{align}
式中的 $g_{\mu\nu}\defeq\qty(\V{g}_\nu)\trans\vdp\V{g}_\mu
	=\ipb[\Rm*]{\V{g}_\mu}{\V{g}_\nu}$。转置一下,可得
\begin{align}
	\mqty[\displaystyle \sum_{\mu=1}^{m} a_\mu\,g_{1\mu} \\ \vdots \\
		\displaystyle \sum_{\mu=1}^{m} a_i\,g_{m\mu}]
	&=\mqty[g_{11} & \cdots & g_{1m} \\
			\vdots & \ddots & \vdots \\
			g_{m1} & \cdots & g_{mm}]\,
		\mqty[a_1 \\ \vdots \\ a_m] \notag \\
	&=\mqty[\qty(\V{g}_1)\trans \\ \vdots \\ \qty(\V{g}_m)\trans]
		\mqty[\V{g}_1,\,\cdots,\,\V{g}_m] \,
		\mqty[a_1 \\ \vdots \\ a_m] \notag \\
	&=\mqty[\V{g}_1,\,\cdots,\,\V{g}_m]\trans
		\mqty[\V{g}_1,\,\cdots,\,\V{g}_m] \,
		\mqty[a_1 \\ \vdots \\ a_m]
	\defeq\qty(\Mat{A}\trans\Mat{A}) \, \mqty[a_1 \\ \vdots \\ a_m]
	=\V{0}\in\Rm \fullstop
\end{align}
其中,矩阵 $\Mat{A}\defeq\qty[\V{g}_1,\,\cdots,\,\V{g}_m]
	\in\realR^{(m+1)\times m}$。
由于处在\emphB{正则点} $\V{x}$ 处,
$\qty{\V{g}_\mu}^m_{\mu=1}$ 线性无关,因此 $\rank\Mat{A}=m$。
根据线性代数的知识,
\begin{equation}
	\rank\qty(\Mat{A}\trans\Mat{A}) = \rank\Mat{A} = m \comma
\end{equation}
所以矩阵 $\Mat{A}\trans\Mat{A}$ 非奇异。
这样就必然有 $\qty[a_1,\,\cdots,\,a_m]\trans=\V{0}\in\Rm$,
即 $a_\mu=0$。代回到 \eqref{eq:曲面逆变基m+1_完整形式}~式,可知
\begin{equation}
	\V{g}^{m+1}=\V{n} \fullstop
\end{equation}

再来看矩阵的右上角:
\begin{equation}
	\mqty[\qty(\V{g}^1)\trans \\ \vdots \\ \qty(\V{g}^m)\trans]
		\vdp \V{n}
	=\mqty[\qty(\V{g}^1)\trans\vdp\V{n} \\ \vdots \\
		\qty(\V{g}^m)\trans\vdp\V{n}]
	=\V{0}\in\Rm \comma
\end{equation}
因此 $\V{g}^\mu\perp\V{n}$。又因为 $\V{n}$ 和
$\vspan\qty{\V{g}_i(\V{x})}^m_{\mu=1}$ 共同构成了基,所以
\begin{equation}
	\V{g}^\mu\in\vspan\qty{\V{g}_\mu(\V{x})}^m_{\mu=1} \fullstop
\end{equation}

最后轮到左上角:
\begin{equation}
	\mqty[\qty(\V{g}^\nu)\trans\vdp\V{g}_\mu]^m_{\mu,\,\nu=1}=\Mat{I}_m
	\iff \ipb[\Rm*]{\V{g}_\mu}{\V{g}^\nu}
		=\KroneckerDelta{\nu}{\mu} \fullstop
	\label{eq:曲面局部基_切空间_对偶关系}
\end{equation}
这是一个与式~\eqref{eq:曲面局部基_对偶关系} 类似的“对偶关系”,
不过请注意指标取值的不同。此处的“对偶关系”仅存在于\emphB{切空间}
$\Tspace{\V{\Sigma}}{\V{x}}
	=\vspan\qty{\V{g}_\mu(\V{x})}^m_{\mu=1}$ 中。

\begin{figure}[h]
	\centering
	\includegraphics{Images/Surface_Local_Basis.PNG}
	\caption{$m+1$ 维 Euclid 空间中 $m$ 维曲面上的局部基}
	\label{fig:曲面上的局部基}
\end{figure}

总结一下前面得到的结果。如图~\ref{fig:曲面上的局部基} 所示,
对于 $m+1$ 维空间中的 $m$ 维曲面而言,
其协变基由切向量与法向量共同组成:
\begin{equation}
	\qty{\V{g}_i(\V{x})}^m_{i=1}
	\defeq\qty{\pdv{\V{\Sigma}}{x^1} (\V{x}),\,\cdots,\,
		\pdv{\V{\Sigma}}{x^m} (\V{x}),\,\V{n}(\V{x})} \semicolon
\end{equation}
至于逆变基,它的前 $m$ 个向量由切空间上的对偶关系
\eqref{eq:曲面局部基_切空间_对偶关系}~式决定,而第 $m+1$ 个向量则为
\begin{equation}
	\V{g}^{m+1}(\V{x})=\V{g}_{m+1}(\V{x})=\V{n}(\V{x}) \fullstop
\end{equation}

\subsection{曲面上的曲线}
首先要在参数域 $\domD{\V{x}}$ 中定义曲线:
\begin{equation}
	\mmap{\V{\Gamma}_\V{x}(t)}{[a,\,b]\ni t}
		{\V{\Gamma}_\V{x}(t)
			=\mqty[\Gamma_\V{x}^1(t) \\ \vdots \\ \Gamma_\V{x}^m(t)]
			=\mqty[x^1(t) \\ \vdots \\ x^m(t)]
				\in\domD{\V{x}}\in\Rm} \fullstop
\end{equation}
这里用下标 $\V{x}$ 表示参数域。
于是,曲面(物理域)上的曲线就可以表示为
$\V{\Gamma}_\V{x}$ 与 $\V{\Sigma}$ 的复合:
\begin{equation}
	\mmap{\V{\Gamma}_\V{\Sigma}(t)}{[a,\,b]\ni t}
		{\V{\Gamma}_\V{\Sigma}(t)
			\defeq\V{\Sigma}\comp\V{\Gamma}_\V{x}(t)
			=\V{\Sigma}\qty\big[\V{x}(t)] \in\Rm*} \fullstop
\end{equation}
类似地,下标 $\V{\Sigma}$ 表示曲面。

下面计算曲面上曲线的切向量:
\begin{align}
	\dv{\V{\Gamma}_\V{\Sigma}}{t} (t)
	=\JacobiD{\V{\Gamma}_\V{\Sigma}} (t)
	&=\JacobiD{\V{\Sigma}} \qty\big(\V{x}(t))
		\vdp \JacobiD{\V{\Gamma}_\V{x}} (t) \notag
	\intertext{上面一个等号利用了\emphB{复合映照的可微性定理}。
		再根据式~\eqref{eq:曲面映照的Jacobi矩阵},
		$\V{\Sigma}(\V{x})$ 的 Jacobi 矩阵由(前 $m$ 个)
		局部协变基组成,而 $\V{\Gamma}_\V{x} (t)$ 的 Jacobi
		矩阵又可以直接写出来,于是}
	&=\mqty[\V{g}_1 \qty\big(\V{x}(t)),\,\cdots,\,
			\V{g}_m \qty\big(\V{x}(t))] \,
		\mqty[\dot{x}^1(t) \\ \vdots \\ \dot{x}^m(t)] \notag \\
	&=\dot{x}^\mu(t) \, \V{g}_\mu\qty\big(\V{x}(t)) \fullstop
\end{align}
不用忘记按照 Einstein 求和约定对哑标 $\mu$ 进行求和。可以发现,
\begin{equation}
	\dv{\V{\Gamma}_\V{\Sigma}}{t} (t)
	=\dot{x}^\mu(t) \, \V{g}_\mu\qty\big(\V{x}(t))
		\in\Tspace{\V{\Sigma}}{\V{x}}
			=\vspan\qty{\V{g}_\mu(\V{x})}^m_{\mu=1} \comma
\end{equation}
说明曲面上所有曲线的切向量都落在该曲面的切空间中。

在\emphB{体积上}的张量场场论中,如果非要使用典则基,也并非不可;
但在\emphB{曲面上}的张量场场论中,曲面上的局部基却是完全无法
被典则基所取代的。如果使用典则基,自变量的微小误差,
就会使得像点落在曲面之外。这在工业设计领域将是不可接受的。
\myPROBLEM[2017-02-25]{见 12-曲面定义及其切空间-Part 02.flv 最后}

\section{度量张量与曲率张量}
\subsection{两种基本形式}
曲面的\emphA{第一基本形式}指切空间中的内积,即
\begin{equation}
	g_{\mu\nu}\defeq\ipb[\Rm*]{\V{g}_\mu}{\V{g}_\nu} \comma
\end{equation}
其中的 $\mu$、$\nu$ 表示 1 到 $m$。$g_{\mu\nu}$ 构成的矩阵为
\begin{equation}
	\Mat{G}\coloneq\mqty[g_{\mu\nu}]
	=\mqty[g_{11} & \cdots & g_{1m} \\
		\vdots & \ddots & \vdots \\
		g_{m1} & \cdots & g_{mm}]
	=\mqty[\qty(\V{g}_1)\trans \\ \vdots \\ \qty(\V{g}_m)\trans]\,
		\mqty[\V{g}_1,\,\cdots,\,\V{g}_m]
	=(\JacobiD{\V{\Sigma}})\trans \vdp (\JacobiD{\V{\Sigma}})
	\fullstop
\end{equation}
根据内积的对称性,$g_{\mu\nu}=g_{\nu\mu}$,于是矩阵 $\Mat{G}$
是一个对称阵。除此以外,在正则点处,它还是一个\emphB{正定}矩阵。

\begin{myProof}
对于任意的向量 $\V{\xi}\in\Rm$,有
\begin{align}
	\V{\xi}\trans\,\Mat{G}\,\V{\xi}
	&=\V{\xi}\trans \,
		\qty\Big[(\JacobiD{\V{\Sigma}})\trans
			\vdp (\JacobiD{\V{\Sigma}})] \,
		\V{\xi} \notag \\
	&=\qty\Big[\V{\xi}\trans (\JacobiD{\V{\Sigma}})\trans]
		\vdp \qty\Big[\qty(\JacobiD{\V{\Sigma}}) \, \V{\xi}]
	=\qty\Big[\qty(\JacobiD{\V{\Sigma}}) \, \V{\xi}]\trans
		\vdp \qty\Big[\qty(\JacobiD{\V{\Sigma}}) \, \V{\xi}] \notag \\
	&=\norm[\Rm*]{\vphantom{0^0}
			\qty(\JacobiD{\V{\Sigma}}) \, \V{\xi}}^2
	=\norm[\Rm*]{\xi^\mu\,\V{g}_\mu}^2 \geqslant 0 \fullstop
\end{align}
这样,若有 $\V{\xi}\trans\,\Mat{G}\,\V{\xi}=0$,则必有
$\xi^\mu\,\V{g}_\mu=0$。而在正则点处,
$\qty{\V{g}_\mu(\V{x})}^m_{\mu=1} \subset\Rm*$ 线性无关。
于是 $\V{g}_\mu$ 的系数就全都为零,即
$\xi^\mu=0\in\realR$,$\V{\xi}=\V{0}\in\Rm$。
这就证明了 $\Mat{G}$ 的正定性。
\end{myProof}

以后我们的讨论都只考虑正则点的情况,
这相当于体积上张量场场论中\emphB{微分同胚}的条件。

\blankline

曲面的\emphA{第二基本形式}定义为
\begin{equation}
	b_{\mu\nu}\defeq\ipb[\Rm*]{\pdv{\V{g}_\nu}{x^\mu}}{\V{n}} \fullstop
\end{equation}
与第一基本形式类似,矩阵 $\Mat{B}\coloneq\mqty[b_{\mu\nu}]$
也是对称的(但未必正定)。

\begin{myProof}
对称性的根源在于二阶偏导数可以交换次序:
\begin{align}
	\pdv{\V{g}_\nu}{x^\mu}
	=\pdv{x^\mu} \qty(\pdv{\V{\Sigma}}{x^\nu})
	=\pdv{\V{\Sigma}}{x^\mu}{x^\nu}
	=\pdv{\V{\Sigma}}{x^\nu}{x^\mu}
	=\pdv{x^\nu} \qty(\pdv{\V{\Sigma}}{x^\mu})
	=\pdv{\V{g}_\mu}{x^\nu} \fullstop
\end{align}
于是 $b_{\mu\nu}=b_{\nu\mu}$。
\end{myProof}

\blankline

根据线性代数中的\emphA{同时对角化}定理,
\myPROBLEM[2017-02-25]{同时对角化}
由于第一、第二基本形式的表示矩阵 $\Mat{G}$ 和 $\Mat{B}$
均是对称矩阵,并且 $\Mat{G}$ 还具有正定性,因此必然唯一存在
一个非奇异的矩阵 $\Mat{S}\in\realR^{m\times m}$,使得
\begin{equation}
	\Mat{S}\trans\Mat{G}\Mat{S}=\Mat{I}_m \quad\text{且}\quad
	\Mat{S}\trans\Mat{B}\Mat{S}
	=\mqty[\dmat{\lambda_1,\ddots,\lambda_m}] \comma
\end{equation}
其中的 $\lambda_\mu$ 满足
\begin{equation}
	\det(\Mat{B}-\lambda_\mu\,\Mat{G})=0 \fullstop
\end{equation}
根据行列式的性质,可知
\begin{equation}
	\det(\Mat{G}^{-1}\Mat{B}-\lambda_\mu\,\Mat{I}_m)
	=\det\qty\Big[\Mat{G}^{-1} \qty(\Mat{B}-\lambda_\mu\,\Mat{G})]
	=\det(\Mat{G}^{-1}) \cdot \det(\Mat{B}-\lambda_\mu\,\Mat{G})
	=\det(\Mat{G}^{-1}) \cdot 0 = 0 \fullstop
\end{equation}
这说明 $\lambda_i$ 是矩阵 $\Mat{G}^{-1}\Mat{B}$ 的特征值。

定义\emphA{Gauss 曲率} $K_G$ 为这些特征值之积,它等于矩阵
$\Mat{G}^{-1}\Mat{B}$ 的行列式:
\begin{equation}
	K_G\defeq\prod_{\mu=1}^{m} \lambda_\mu = \det(\Mat{G}^{-1}\Mat{B})
	=\frac{\det\Mat{B}}{\det\Mat{G}} \semicolon
\end{equation}
定义\emphA{平均曲率} $H$ 为特征值之和的平均值,
它等于矩阵的迹除以 $m$:
\begin{equation}
	H\defeq\frac{1}{m}\sum_{\mu=1}^{m} \lambda_\mu
	=\frac{\tr(\Mat{G}^{-1}\Mat{B})}{m} \fullstop
\end{equation}

\subsection{度量张量与曲率张量}
曲面上的\emphA{度量张量} $\T{G}$ 定义为
\begin{equation}
	\T{G} \defeq g_{\mu\nu}\,\V{g}^\mu\tp\V{g}^\nu
	\in\Tensors[\Rm*]{2} \comma
\end{equation}
式中的 $g_{\mu\nu}\defeq\ipb[\Rm*]{\V{g}_\mu}{\V{g}_\nu}$。

把 $\V{g}_\nu$ 用局部\emphB{逆变}基展开,可有
\begin{align}
	\V{g}_\nu &=\ipb[\Rm*]{\V{g}_\nu}{\V{g}_i}\,\V{g}^i \notag \\
	&=\ipb[\Rm*]{\V{g}_\nu}{\V{g}_\mu}\,\V{g}^\mu
		+\ipb[\Rm*]{\V{g}_\nu}{\V{n}}\,\V{n} \notag
	\intertext{由于 $\V{g}_\nu$ 与 $\V{n}$ 正交,因此}
	&=\ipb[\Rm*]{\V{g}_\nu}{\V{g}_\mu}\,\V{g}^\mu
	=g_{\mu\nu}\,\V{g}^\mu \fullstop
	\label{eq:曲面局部基_指标升降_1}
\end{align}
同理,还可以知道
\begin{equation}
	\V{g}^\nu=g^{\mu\nu}\,\V{g}_\mu \fullstop
	\label{eq:曲面局部基_指标升降_2}
\end{equation}
此处的 $g^{\mu\nu}$ 自然等于 $\ipb[\Rm*]{\V{g}^\mu}{\V{g}^\nu}$。

于是我们便可以获得度量张量的其他形式:
\begin{mySubEq}
	\begin{align}
		\T{G} &\defeq g_{\mu\nu}\,\V{g}^\mu\tp\V{g}^\nu \notag
		\intertext{代入 \eqref{eq:曲面局部基_指标升降_1}~式,
			并利用 Krnonecker δ,可得}
		&=\V{g}_\nu\tp\V{g}^\nu
		=\KroneckerDelta{\mu}{\nu}\,\V{g}_\mu\tp\V{g}^\nu \fullstop
		\intertext{这是\emphB{混合分量}。代入
			\eqref{eq:曲面局部基_指标升降_2}~式,又有}
		\T{G} &= \V{g}_\nu\tp\V{g}^\nu
		=g^{\mu\nu}\,\V{g}_\mu\tp\V{g}_\nu \fullstop
	\end{align}
\end{mySubEq}
这是\emphB{逆变分量}。