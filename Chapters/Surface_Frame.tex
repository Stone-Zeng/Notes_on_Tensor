\section{曲面的定义;切空间}
\subsection{曲面的定义}
自变量维数比因变量维数低一维的向量值映照,都可以称为\emphA{曲面}。
常见的三维曲面,其自变量是二维的,正符合了该定义。

如图~\ref{fig:曲面的定义},
$m+1$ 维 Euclid 空间中的 $m$ 维曲面 $\V{\Sigma}(\V{x})$,
都可以用如下的向量值映照表示:
\begin{equation}
	\mmap{\V{\Sigma}(\V{x})}
		{\domD{\V{x}}\ni\V{x}=\mqty[x^1 \\ \vdots \\ x^m]}
		{\V{\Sigma}(\V{x})
			=\mqty[\Sigma^1 \\ \vdots \\ \Sigma^m \\ \Sigma^{m+1}](\V{x})
			\in\Rm*} \fullstop
\end{equation}

\begin{figure}[h]
	\centering
	\includegraphics{Images/Surface_Definition.PNG}
	\caption{$m+1$ 维 Euclid 空间中 $m$ 维曲面的定义}
	\label{fig:曲面的定义}
\end{figure}

\subsection{切向量与切空间}
循着与体积上张量场场论(见第\ref{chap:微分同胚}章)相同的思路,
我们先来计算 $\V{\Sigma}(\V{x})$ 的 Jacobi 矩阵:
\begin{equation}
	\JacobiD{\V{\Sigma}}(\V{x})
	=\mqty[\displaystyle \pdv{\V{\Sigma}}{x^1},\,
			\cdots,\,\pdv{\V{\Sigma}}{x^m}] (\V{x})
	\defeq\mqty[\V{g}_1,\,\cdots,\,\V{g}_m] (\V{x})
		\in\realR^{(m+1)\times m} \comma
\end{equation}
式中,
\begin{equation}
	\V{g}_i(\V{x}) \defeq \pdv{\V{\Sigma}}{x^i} (\V{x})
	=\lim_{\lambda\to 0}
		\frac{\V{\Sigma}\qty(\V{x}+\lambda\,\V{e}_i)-\V{\Sigma}(\V{x})}
		{\lambda} \in\Rm* \fullstop
\end{equation}
需要注意,与微分同胚不同,此处的 Jacobi 矩阵\emphB{不是}方阵。

为了考察 $\V{g}_i(\V{x})$ 的几何意义,
我们在定义域空间 $\domD{\V{x}}\in\Rm$ 中过点 $\V{x}$ 作出
$x^i$-线,其上的任意一点均可表示为 $\V{x}+\lambda\,\V{e}_i$。
在映照 $\V{\Sigma}(\V{x})$ 的作用下,
点 $\V{x}$ 和 $\V{x}+\lambda\,\V{e}_i$ 分别被映照到值域空间
$\domD{\V{\Sigma}}$ 中的点 $\Sigma(\V{x})$ 和
$\Sigma\qty(\V{x}+\lambda\,\V{e}_i)$。同时,$x^i$-线也被映照到
$\domD{\V{\Sigma}}$ 中,成为一条曲线。

当 $\lambda\to 0$ 时,
\begin{equation}
	\frac{\V{\Sigma}\qty(\V{x}+\lambda\,\V{e}_i)-\V{\Sigma}(\V{x})}
	{\lambda}
	\to \pdv{\V{\Sigma}}{x^i} (\V{x}) = \V{g}_i(\V{x}) \comma
\end{equation}
这便是值域空间中 $x^i$-线的\emphA{切向量}。
切向量张成了 $\Rm*$ 空间的一个子空间:
\begin{equation}
	\vspan\qty{\V{g}_i(\V{x})}^m_{i=1} \subset\Rm* \fullstop
\end{equation}

如果要构造一组基,必须使其满足\emphB{线性无关}的要求。
由此,我们引出\emphA{正则点}的定义,
它指能够使 Jacobi 矩阵 $\JacobiD{\V{\Sigma}}(\V{x})$ 列满秩,即
\begin{equation*}
	\rank\JacobiD{\V{\Sigma}}(\V{x})=m
\end{equation*}
的点 $\V{x}\in\domD{\V{x}}$ 或 $\V{\Sigma}(\V{x})\in\Rm*$。
在正则点处,便有 $\qty{\V{g}_i(\V{x})}^m_{i=1} \subset\Rm*$
线性无关。此时,切向量张成的空间称为\emphA{切空间}
(或\emphA{切平面}),记作
\begin{equation}
	\Tspace{\V{\Sigma}}{\V{x}}
	\defeq\vspan\qty{\V{g}_i(\V{x})}^m_{i=1} \subset\Rm*
	\fullstop
\end{equation}
切空间 $\Tspace{\V{\Sigma}}{\V{x}}$ 的维度是 $m$。

\subsection{局部基}
设位于点 $\V{x}$ 处的 $m+1$ 维单位向量 $\V{n}(\V{x})$ 满足
$\norm[\Rm*]{\V{n}(\V{x})}=1$ 且
\begin{equation}
	(\JacobiD{\V{\Sigma}})\trans(\V{x}) \vdp \V{n}(\V{x})
	=\mqty[\V{g}_1,\,\cdots,\,\V{g}_m]\trans (\V{x}) \vdp \V{n}(\V{x})
	=\mqty[\qty(\V{g}_1)\trans \\ \vdots \\ \qty(\V{g}_m)\trans]
		\vdp \V{n}(\V{x}) = \V{0}\in\Rm \fullstop
	\label{eq:曲面协变基_法向量}
\end{equation}
当 $\JacobiD{\V{\Sigma}}(\V{x})$ 列满秩(即 $\V{x}$ 是正则点)时,
根据线性代数中的\emphB{基扩张定理}
\myPROBLEM[2017-02-13]{基扩张定理},
这样的单位向量 $\V{n}(\V{x})$ 是唯一存在的,称为\emphA{法向量}。

令 $\V{g}_{m+1}\defeq\V{n}(\V{x})$,则
\begin{equation}
	\qty{\V{g}_i(\V{x})}^{m+1}_{i=1}
	=\qty{\pdv{\V{\Sigma}}{x^1} (\V{x}),\,\cdots,\,
		\pdv{\V{\Sigma}}{x^m} (\V{x}),\,\V{n}(\V{x})}
\end{equation}
是 $\Rm*$ 空间中的一组基。
它的指标都在下面,因此是一组\emphA{局部协变基}。

\blankline

接下来研究对应的\emphA{局部逆变基}
$\qty{\V{g}^i(\V{x})}^{m+1}_{i=1}$,
它与局部协变基共同满足\emphA{对偶关系}:\footnote{
	以下在不引起混淆的地方将省去“$(\V{x})$”,
	但仍不要忘记正则点的要求。}
\begin{equation}
	\ipb[\Rm*]{\V{g}_i}{\V{g}^j} = \KroneckerDelta{j}{i} \fullstop
	\label{eq:曲面局部基_对偶关系}
\end{equation}
写成矩阵形式,为
\begin{equation}
	\qty[\begin{array}{@{}c@{}}
		\qty(\V{g}^1)\trans \\ \vdots \\ \qty\big(\V{g}^m)\trans \\[3pt]
		\hdashline \\[-12pt]
		\qty(\V{g}^{m+1})\trans
	\end{array}] \,
	\qty[\begin{array}{@{}c:c@{}}
		\V{g}_1,\,\cdots,\,\V{g}_m & \V{n}
	\end{array}]
	=\qty[\begin{array}{@{}c:c@{}}
		\mqty[\qty(\V{g}^\nu)\trans\vdp\V{g}_\mu]^m_{\mu,\,\nu=1} &
			\mqty[\qty(\V{g}^1)\trans \\ \vdots \\ \qty(\V{g}^m)\trans]
			\vdp \V{n} \\[3pt]
		\hdashline \\[-12pt]
		\qty(\V{g}^{m+1})\trans \vdp \mqty[\V{g}_1,\,\cdots,\,\V{g}_m] &
		\qty(\V{g}^{m+1})\trans \vdp \V{n}
	\end{array}]
	=\Mat{I}_{m+1} \fullstop
	\label{eq:曲面局部基_对偶关系_矩阵形式}
\end{equation}
为了表述的清晰,在这里以及之后的一小段我们将用拉丁字母 $i$、$j$
代表指标 1 到 $m+1$,而用希腊字母 $\mu$、$\nu$ 代表指标 1 到 $m$。

按照分块矩阵的计算法则,显然有
$\qty(\V{g}^{m+1})\trans \vdp \V{n}=1$。于是
\begin{equation}
	\V{g}^{m+1} = \V{n}+\sum_{\mu=1}^{m} a_\mu\,\V{g}_\mu \comma
	\label{eq:曲面逆变基m+1_完整形式}
\end{equation}
式中的 $a_\mu$ 是待定系数。

考虑式~\eqref{eq:曲面局部基_对偶关系_矩阵形式} 中矩阵的左下角,有
\begin{align}
	\qty(\V{g}^{m+1})\trans \vdp \mqty[\V{g}_1,\,\cdots,\,\V{g}_m]
	&=\qty(\V{n}+\sum_{\mu=1}^{m} a_\mu\,\V{g}_\mu)\trans
		\vdp \mqty[\V{g}_1,\,\cdots,\,\V{g}_m] \notag
	\intertext{根据 \eqref{eq:曲面协变基_法向量}~式,
		$\V{n}$ 与 $\V{g}_\mu$ 正交:}
	&=\sum_{\mu=1}^{m} a_\mu\,\qty(\V{g}_\mu)\trans
		\vdp \mqty[\V{g}_1,\,\cdots,\,\V{g}_m] \notag \\
	&=\mqty[\displaystyle
			\sum_{\mu=1}^{m} a_\mu\,\qty(\V{g}_\mu)\trans
				\vdp\V{g}_1,\,\cdots,\,
			\sum_{\mu=1}^{m} a_\mu\,\qty(\V{g}_\mu)\trans
				\vdp\V{g}_m] \notag \\
	&\defeq \mqty[\displaystyle
			\sum_{\mu=1}^{m} a_\mu\,g_{1\mu},\,\cdots,\,
			\sum_{\mu=1}^{m} a_\mu\,g_{m\mu}] \notag \\
	&=\V{0}\in\realR^{1\times m} \fullstop
\end{align}
式中的 $g_{\mu\nu}\defeq\qty(\V{g}_\nu)\trans\vdp\V{g}_\mu
	=\ipb[\Rm*]{\V{g}_\mu}{\V{g}_\nu}$。转置一下,可得
\begin{align}
	\mqty[\displaystyle \sum_{\mu=1}^{m} a_\mu\,g_{1\mu} \\ \vdots \\
		\displaystyle \sum_{\mu=1}^{m} a_i\,g_{m\mu}]
	&=\mqty[g_{11} & \cdots & g_{1m} \\
			\vdots & \ddots & \vdots \\
			g_{m1} & \cdots & g_{mm}]\,
		\mqty[a_1 \\ \vdots \\ a_m] \notag \\
	&=\mqty[\qty(\V{g}_1)\trans \\ \vdots \\ \qty(\V{g}_m)\trans]
		\mqty[\V{g}_1,\,\cdots,\,\V{g}_m] \,
		\mqty[a_1 \\ \vdots \\ a_m] \notag \\
	&=\mqty[\V{g}_1,\,\cdots,\,\V{g}_m]\trans
		\mqty[\V{g}_1,\,\cdots,\,\V{g}_m] \,
		\mqty[a_1 \\ \vdots \\ a_m]
	\defeq\qty(\Mat{A}\trans\Mat{A}) \, \mqty[a_1 \\ \vdots \\ a_m]
	=\V{0}\in\Rm \fullstop
\end{align}
其中,矩阵 $\Mat{A}\defeq\qty[\V{g}_1,\,\cdots,\,\V{g}_m]
	\in\realR^{(m+1)\times m}$。
由于处在\emphB{正则点} $\V{x}$ 处,
$\qty{\V{g}_\mu}^m_{\mu=1}$ 线性无关,因此 $\rank\Mat{A}=m$。
根据线性代数的知识,
\begin{equation}
	\rank\qty(\Mat{A}\trans\Mat{A}) = \rank\Mat{A} = m \comma
\end{equation}
所以矩阵 $\Mat{A}\trans\Mat{A}$ 非奇异。
这样就必然有 $\qty[a_1,\,\cdots,\,a_m]\trans=\V{0}\in\Rm$,
即 $a_\mu=0$。代回到 \eqref{eq:曲面逆变基m+1_完整形式}~式,可知
\begin{equation}
	\V{g}^{m+1}=\V{n} \fullstop
\end{equation}

再来看矩阵的右上角:
\begin{equation}
	\mqty[\qty(\V{g}^1)\trans \\ \vdots \\ \qty(\V{g}^m)\trans]
		\vdp \V{n}
	=\mqty[\qty(\V{g}^1)\trans\vdp\V{n} \\ \vdots \\
		\qty(\V{g}^m)\trans\vdp\V{n}]
	=\V{0}\in\Rm \comma
\end{equation}
因此 $\V{g}^\mu\perp\V{n}$。又因为 $\V{n}$ 和
$\vspan\qty{\V{g}_i(\V{x})}^m_{\mu=1}$ 共同构成了基,所以
\begin{equation}
	\V{g}^\mu\in\vspan\qty{\V{g}_\mu(\V{x})}^m_{\mu=1} \fullstop
\end{equation}

最后轮到左上角:
\begin{equation}
	\mqty[\qty(\V{g}^\nu)\trans\vdp\V{g}_\mu]^m_{\mu,\,\nu=1}=\Mat{I}_m
	\iff \ipb[\Rm*]{\V{g}_\mu}{\V{g}^\nu}
		=\KroneckerDelta{\nu}{\mu} \fullstop
	\label{eq:曲面局部基_切空间_对偶关系}
\end{equation}
这是一个与式~\eqref{eq:曲面局部基_对偶关系} 类似的“对偶关系”,
不过请注意指标取值的不同。此处的“对偶关系”仅存在于\emphB{切空间}
$\Tspace{\V{\Sigma}}{\V{x}}
	=\vspan\qty{\V{g}_\mu(\V{x})}^m_{\mu=1}$ 中。

\begin{figure}[h]
	\centering
	\includegraphics{Images/Surface_Local_Basis.PNG}
	\caption{$m+1$ 维 Euclid 空间中 $m$ 维曲面上的局部基}
	\label{fig:曲面上的局部基}
\end{figure}

总结一下前面得到的结果。如图~\ref{fig:曲面上的局部基} 所示,
对于 $m+1$ 维空间中的 $m$ 维曲面而言,
其协变基由切向量与法向量共同组成:
\begin{equation}
	\qty{\V{g}_i(\V{x})}^m_{i=1}
	\defeq\qty{\pdv{\V{\Sigma}}{x^1} (\V{x}),\,\cdots,\,
		\pdv{\V{\Sigma}}{x^m} (\V{x}),\,\V{n}(\V{x})} \semicolon
\end{equation}
至于逆变基,它的前 $m$ 个向量由切空间上的对偶关系
\eqref{eq:曲面局部基_切空间_对偶关系}~式决定,而第 $m+1$ 个向量则为
\begin{equation}
	\V{g}^{m+1}(\V{x})=\V{g}_{m+1}(\V{x})=\V{n}(\V{x}) \fullstop
\end{equation}