\section{张量的范数}
\subsection{赋范线性空间}
对于一个\emphA{线性空间} $\SPACE{V}$,它总是定义了\emphA{线性结构}:
\begin{equation}
	\forall\, \V{x},\,\V{y}\in\SPACE{V}
	\text{\ 和\ } \forall\, \alpha,\,\beta\in\realR,\quad
	\alpha\,\V{x}+\beta\,\V{y} \in \SPACE{V} \fullstop
\end{equation}
为了进一步研究的需要,我们还要引入\emphA{范数}的概念。
所谓“范数”,就是对线性空间中任意元素\emphB{大小}的一种刻画。
举个我们熟悉的例子, $m$ 维 Euclid 空间 $\Rm$ 中某个向量的范数,
就定义为该向量在 Descartes 坐标下各分量的平方和的平方根。

一般而言,线性空间 $\SPACE{V}$ 中的范数
$\norm[\SPACE{V}]{\cdotord}$ 是从 $\SPACE{V}$ 到 $\realR$
的一个映照,并且需要满足以下三个条件:

\begin{myEnum}
\item \emphA{非负性}
\begin{equation}
	\forall\,\V{x}\in\SPACE{V},\quad
	\norm[\SPACE{V}]{\V{x}} \geqslant 0
\end{equation}
以及\emphA{非退化性}
\begin{equation}
	\forall\,\V{x}\in\SPACE{V},\quad
	\norm[\SPACE{V}]{\V{x}}=0
	\iff \V{x}=\V{0}\in\SPACE{V} \comma
\end{equation}
这里的 $\V{0}$ 是线性空间 $\SPACE{V}$ 中的\emphA{零元素},
它是唯一存在的。

\blankline

\item 零元是唯一的,线性空间中的元素 $\V{x}$
与从 $\V{0}$ 指向它的向量一一对应。
因此,线性空间中的元素也常被称为“向量”。

考虑线性空间中的数乘运算。从几何上看, $\V{x}$ 乘上 $\lambda$,
就是将 $\V{x}$ 沿着原来的指向进行伸缩。显然有
\begin{equation}
	\forall\,\V{x}\in\SPACE{V}
	\text{\ 和\ } \forall\,\lambda\in\realR,\quad
	\norm[\SPACE{V}]{\lambda\,\V{x}}
	=\abs{\lambda}\cdot\norm[\SPACE{V}]{\V{x}} \comma
\end{equation}
这称为\emphA{正齐次性}。

\myPROBLEM{想要图吗?}

\item 线性空间中的加法满足\emphB{平行四边形法则}。直观地看,就有
\begin{equation}
	\forall\, \V{x},\,\V{y}\in\SPACE{V},\quad
	\norm[\SPACE{V}]{\V{x}+\V{y}} \leqslant
	\norm[\SPACE{V}]{\V{x}}+\norm[\SPACE{V}]{\V{y}} \comma
\end{equation}
这称为\emphA{三角不等式}。
\end{myEnum}

定义了范数的线性空间称为\emphA{赋范线性空间}。

\subsection{张量范数的定义}
考虑 $p$ 阶张量 $\T{\Phi}\in\Tensors{p}$,
它可以用\emphB{逆变分量}或\emphB{协变分量}来表示:
\begin{braceEq*}{\T{\Phi}=}
	\Phi^{i_1 \cdots i_p}\,
		\V{g}_{i_1}\tp\cdots\tp\V{g}_{i_p} \comma \\
	\Phi_{i_1 \cdots i_p}\,
		\V{g}^{i_1}\tp\cdots\tp\V{g}^{i_p} \comma
\end{braceEq*}
其中
\begin{braceEq}
	\Phi^{i_1 \cdots i_p}&=
		\T{\Phi}\qty(\V{g}^{i_1},\,\cdots,\,\V{g}^{i_p}) \fullstop \\
	\Phi_{i_1 \cdots i_p}&=
		\T{\Phi}\qty\big(\V{g}_{i_1},\,\cdots,\,\V{g}_{i_p}) \comma
\end{braceEq}
张量的\emphA{范数}定义为
\begin{equation}
	\norm[\Tensors{p}]{\T{\Phi}}\defeq
	\sqrt{\Phi^{i_1 \cdots i_p} \, \Phi_{i_1 \cdots i_p}}
	\in\realR \fullstop
\end{equation}
$i_1 \cdots i_p$ 可独立取值,每个又有 $m$ 种取法,
所以根号下共有 $m^p$ 项。
注意 $\Phi^{i_1 \cdots i_p}$ 与 $\Phi_{i_1 \cdots i_p}$
未必相等,因而根号下的部分未必是平方和,
这与 Euclid 空间中向量的模是不同的。

复习一下 \ref{subsec:相对不同基的张量分量之间的关系}~小节,
我们可以用另一组(带括号的)基表示张量 $\T{\Phi}$:
\begin{braceEq}
	\Phi^{i_1 \cdots i_p} &=
		c^{i_1}_{(\xi_1)} \cdots c^{i_p}_{(\xi_p)} \,
		\Phi^{(\xi_1)\cdots(\xi_p)} \comma \\
	\Phi_{i_1 \cdots i_p} &=
		c^{(\eta_1)}_{i_1} \cdots c^{(\eta_p)}_{i_p} \,
		\Phi_{(\eta_1)\cdots(\eta_p)} \comma
\end{braceEq}
其中的 $c^i_{(\xi)}=\ipb{\V{g}_{(\xi)}}{\V{g}^i}$,
$c^{(\eta)}_i=\ipb{\V{g}^{(\eta)}}{\V{g}_i}$,它们满足
\begin{equation}
	c^i_{(\xi)}\,c^{(\eta)}_i
	=\KroneckerDelta{(\eta)}{(\xi)} \fullstop
\end{equation}
于是
\begin{align}
	&\alspace\Phi^{i_1 \cdots i_p} \, \Phi_{i_1 \cdots i_p} \notag \\
	&=\qty(c^{i_1}_{(\xi_1)} \cdots c^{i_p}_{(\xi_p)} \,
			\Phi^{(\xi_1)\cdots(\xi_p)})
		\qty(c^{(\eta_1)}_{i_1} \cdots c^{(\eta_p)}_{i_p} \,
			\Phi_{(\eta_1)\cdots(\eta_p)}) \notag \\
	&=\qty(c^{i_1}_{(\xi_1)} c^{(\eta_1)}_{i_1}) \cdots
		\qty(c^{i_p}_{(\xi_p)} c^{(\eta_p)}_{i_p}) \,
		\Phi^{(\xi_1)\cdots(\xi_p)} \Phi_{(\eta_1)\cdots(\eta_p)}
		\notag \\
	&=\KroneckerDelta{(\eta_1)}{(\xi_1)} \cdots
		\KroneckerDelta{(\eta_p)}{(\xi_p)} \,
		\Phi^{(\xi_1)\cdots(\xi_p)} \Phi_{(\eta_1)\cdots(\eta_p)}
		\notag \\
	&=\Phi^{(\xi_1)\cdots(\xi_p)} \Phi_{(\xi_1)\cdots(\xi_p)}
	\fullstop
\end{align}
它是 $\T{\Phi}$ 在另一组基下的逆变分量与协变分量乘积之和。

以上结果说明,张量的范数不依赖于基的选取,
这就好比用不同的秤来称同一个人的体重,都将获得相同的结果。
既然如此,不妨采用\emphB{单位正交基}来表示张量的范数:
\begin{align}
	\norm[\Tensors{p}]{\T{\Phi}}
	&\defeq\sqrt{\Phi^{i_1 \cdots i_p} \, \Phi_{i_1 \cdots i_p}}
	\notag \\
	&=\sqrt{\Phi^{\orthIdx{i_1}\cdots\orthIdx{i_p}} \,
		\Phi_{\orthIdx{i_1}\cdots\orthIdx{i_p}}} \notag \\
	&\eqcolon\sqrt{\sum\nolimits_{i_1,\,\cdots,\,i_p=1}^{m}
		\qty\big(\Phi\midscript{\orthIdx{i_1,\,\cdots,\,i_p}})^2}
	\fullstop
\end{align}
这里的 $\Phi\midscript{\orthIdx{i_1,\,\cdots,\,i_p}}$
表示张量 $\T{\Phi}$ 在单位正交基下的分量,它的指标不区分上下。

有了这样的表示,很容易就可以验证张量范数符合之前的三个要求。
一组数的平方和开根号,必然是\emphB{非负}的。
至于\emphB{非退化性},若范数为零,则所有分量均为零,自然成为零张量;
反之,对于零张量,所有分量为零,范数也为零。
将 $\T{\Phi}$ 乘上 $\lambda$,则有
\begin{align}
	\norm[\Tensors{p}]{\lambda\,\T{\Phi}}
	&=\sqrt{\sum\nolimits_{i_1,\,\cdots,\,i_p=1}^{m}
		\qty\big(\lambda\,
			\Phi\midscript{\orthIdx{i_1,\,\cdots,\,i_p}})^2} \notag \\
	&=\sqrt{\lambda^2 \sum\nolimits_{i_1,\,\cdots,\,i_p=1}^{m}
			\qty\big(\Phi\midscript{\orthIdx{i_1,\,\cdots,\,i_p}})^2}
		\notag \\
	&=\abs{\lambda} \sqrt{\sum\nolimits_{i_1,\,\cdots,\,i_p=1}^{m}
			\qty\big(\Phi\midscript{\orthIdx{i_1,\,\cdots,\,i_p}})^2}
		\notag \\
	&=\abs{\lambda}\cdot\norm[\Tensors{p}]{\T{\Phi}} \comma
\end{align}
于是\emphB{正齐次性}也得以验证。
最后,利用 Cauchy--Schwarz 不等式,可有
\begin{align}
	&\alspace\norm[\Tensors{p}]{\T{\Phi}+\T{\Psi}}^2 \notag \\
	&=\sum \qty\Big(\Phi\midscript{\orthIdx{i_1,\,\cdots,\,i_p}}
			+\Psi\midscript{\orthIdx{i_1,\,\cdots,\,i_p}})^2 \notag \\
	&=\sum \qty[
			\qty\big(\Phi\midscript{\orthIdx{i_1,\,\cdots,\,i_p}})^2
			+2\, \Phi\midscript{\orthIdx{i_1,\,\cdots,\,i_p}} \,
				\Psi\midscript{\orthIdx{i_1,\,\cdots,\,i_p}}
			+\qty\big(\Psi\midscript{\orthIdx{i_1,\,\cdots,\,i_p}})^2]
		\notag \\
	&=\sum \qty\big(\Phi\midscript{\orthIdx{i_1,\,\cdots,\,i_p}})^2
		+2\sum \Phi\midscript{\orthIdx{i_1,\,\cdots,\,i_p}} \,
			\Psi\midscript{\orthIdx{i_1,\,\cdots,\,i_p}}
		+\sum \qty\big(\Psi\midscript{\orthIdx{i_1,\,\cdots,\,i_p}})^2
		\notag \\
	&\leqslant\norm[\Tensors{p}]{\T{\Phi}}^2
		+2 \sqrt{\sum \qty\big(
				\Phi\midscript{\orthIdx{i_1,\,\cdots,\,i_p}})^2}
			\sqrt{\sum \qty\big(
				\Psi\midscript{\orthIdx{i_1,\,\cdots,\,i_p}})^2}
		+\norm[\Tensors{p}]{\T{\Phi}}^2 \notag \\
	&=\norm[\Tensors{p}]{\T{\Phi}}^2
		+2\norm[\Tensors{p}]{\T{\Phi}}\cdot\norm[\Tensors{p}]{\T{\Psi}}
		+\norm[\Tensors{p}]{\T{\Phi}}^2 \notag \\
	&=\qty\Big(\norm[\Tensors{p}]{\T{\Phi}}
		+\norm[\Tensors{p}]{\T{\Phi}})^2 \fullstop
\end{align}
两边开方,即为\emphB{三角不等式}。

\blankline

由此,我们就完整地给出了张量大小的刻画手段。
可以看出,它实际上就是 Euclid 空间中向量模的直接推广。

\subsection{简单张量的范数}
根据 \ref{subsec:张量的表示与简单张量}~小节中的定义,
简单张量是形如 $\V{\xi}\tp\V{\eta}\tp\V{\zeta}$ 的张量,
其中的 $\V{\xi},\,\V{\eta},\,\V{\zeta}\in\Rm$,
它是三个向量的张量积。
简单张量的范数为
\begin{equation}
	\norm[\Tensors{3}]{\V{\xi}\tp\V{\eta}\tp\V{\zeta}}
	=\norm{\V{\xi}}\cdot\norm{\V{\eta}}\cdot\norm{\V{\zeta}}
	\fullstop \label{eq:简单张量的范数}
\end{equation}

\begin{myProof}
$\V{\xi}\tp\V{\eta}\tp\V{\zeta}$ 的逆变分量为
\begin{equation}
	\qty(\V{\xi}\tp\V{\eta}\tp\V{\zeta})^{ijk}
	\defeq \V{\xi}\tp\V{\eta}\tp\V{\zeta}
		\qty(\V{g}^i,\,\V{g}^j,\,\V{g}^k)
	=\xi^i \eta^j \zeta^k \fullstop
\end{equation}
同理,它的协变分量为
\begin{equation}
	\qty(\V{\xi}\tp\V{\eta}\tp\V{\zeta})_{ijk}
	\defeq \V{\xi}\tp\V{\eta}\tp\V{\zeta}
		\qty(\V{g}_i,\,\V{g}_j,\,\V{g}_k)
	=\xi_i \eta_j \zeta_k \fullstop
\end{equation}
二者相乘,有
\begin{align}
	&\alspace\qty(\V{\xi}\tp\V{\eta}\tp\V{\zeta})^{ijk}
		\cdot \qty(\V{\xi}\tp\V{\eta}\tp\V{\zeta})_{ijk} \notag \\
	&=\qty(\xi^i \eta^j \zeta^k)
		\cdot \qty(\xi_i \eta_j \zeta_k) \notag \\
	&=\qty(\xi^i \xi_i) \cdot \qty(\eta^j \eta_j)
		\cdot \qty(\zeta^k \zeta_k) \fullstop
\end{align}
注意到
\begin{align}
	\norm{\xi}^2
	&=\ipb{\xi}{\xi} \notag
	\intertext{分别把二者用协变和逆变分量表示:}
	&=\ipb{\xi^i\,\V{g}_i}{\xi_j\,\V{g}^j} \notag \\
	&=\xi^i \xi_j \ipb{\V{g}_i}{\V{g}^j} \notag \\
	&=\xi^i \xi_j \KroneckerDelta{j}{i}
	=\xi^i \xi_i \comma
\end{align}
于是
\begin{equation}
	\qty(\V{\xi}\tp\V{\eta}\tp\V{\zeta})^{ijk}
		\cdot \qty(\V{\xi}\tp\V{\eta}\tp\V{\zeta})_{ijk}
	=\norm{\xi}^2 \cdot \norm{\V{\eta}}^2 \cdot \norm{\V{\zeta}}^2
	\fullstop
\end{equation}
两边开方,即得 \eqref{eq:简单张量的范数}~式。
\end{myProof}

\section{张量场的偏导数;协变导数}
在区域 $\domD{\V{x}}\subset\Rm$ 上,
若存在一个自变量用\emphB{位置}刻画的映照
\begin{equation}
	\mmap{\T{\Phi}}{\domD{\V{x}}\ni\V{x}}
		{\T{\Phi}(\V{x})\in\Tensors{r}} \comma
\end{equation}
则称张量 $\T{\Phi}(\V{x})$ \footnote{%
	类似“$\T{\Phi}(\V{x})$”的记号在前文也表示张量 $\T{\Phi}$
	\emphB{作用}在向量 $\V{x}$ 上(“吃掉”了 $\V{x}$),此时有
	$\T{\Phi}(\V{x})\in\realR$,注意不要混淆。
	符号有限,难免如此,还望诸位体谅。}是定义在 $\domD{\V{x}}$
上的一个\emphA{张量场}。

下面我们以三阶张量为例。设在物理域 $\domD{\V{X}}\subset\Rm$
和参数域 $\domD{\V{x}}\subset\Rm$ 之间已经建立了微分同胚
$\V{X}(\V{x})\in\cf{\domD{\V{x}}}{\domD{\V{X}}}$。
在 $\V{X}(\V{x})$ 处,张量场 $\T{\Phi}(\V{x})$
可以用分量形式表示为
\begin{equation}
	\T{\Phi}(\V{x})=\tc{\Phi}{^i_j^k}(\V{x})\,
		\V{g}_i(\V{x})\tp\V{g}^j(\V{x})\tp\V{g}_k(\V{x})
	\in\Tensors{3} \comma
\end{equation}
其中的 $\V{g}_i(\V{x}),\,\V{g}^j(\V{x}),\,\V{g}_k(\V{x})$
都是\emphB{局部}基,而张量分量则定义为\footnote{%
	请注意,下式 $\T{\Phi}$ 之后的第一个圆括号表示\emphB{位于}
	$\V{x}$ 处;而后面的方括号则表示\emphB{作用在}这几个向量上。}
\begin{equation}
	\tc{\Phi}{^i_j^k}(\V{x})
	\defeq \T{\Phi}(\V{x})
		\qty[\V{g}_i(\V{x}),\,\V{g}^j(\V{x}),\,\V{g}_k(\V{x})]
	\in\realR \fullstop
\end{equation}
类似地,当点沿着 $x^\mu$-线运动到
$\V{X}\qty(\V{x}+\lambda\,\V{e}_\mu)$ 处时,有
\begin{equation}
	\T{\Phi}\qty(\V{x}+\lambda\,\V{e}_\mu)
	=\tc{\Phi}{^i_j^k}\qty(\V{x}+\lambda\,\V{e}_\mu)\,
		\V{g}_i \qty(\V{x}+\lambda\,\V{e}_\mu)
		\tp\V{g}^j \qty(\V{x}+\lambda\,\V{e}_\mu)
		\tp\V{g}_k \qty(\V{x}+\lambda\,\V{e}_\mu) \fullstop
	\label{eq:x+lambda_e_mu处的张量场}
\end{equation}

现在研究 $\lambda\to 0 \in\realR$ 时的极限
\begin{equation}
	\lim_{\lambda\to 0} \frac{\T{\Phi}\qty(\V{x}+\lambda\,\V{e}_\mu)
		-\T{\Phi}(\V{x})} {\lambda}
	\eqcolon \pdv{\T{\Phi}}{x^\mu} (\V{x})
	\in\Tensors{3} \fullstop
	\label{eq:张量场偏导数的极限定义}
\end{equation}
与之前的向量值映照类似,该极限表示张量场 $\T{\Phi}(\V{x})$
作为一个\emphB{整体},相对于自变量第 $\mu$ 个分量的变化率,
即 $\T{\Phi}$ 关于 $x^\mu$(在 $\V{x}$ 处)的\emphA{偏导数}。
式中,$\T{\Phi}\qty(\V{x}+\lambda\,\V{e}_\mu)$
已由 \eqref{eq:x+lambda_e_mu处的张量场}~式给出。
注意到张量分量实际上就是一个\emphB{多元函数},于是
\begin{equation}
	\tc{\Phi}{^i_j^k}\qty(\V{x}+\lambda\,\V{e}_\mu)
	=\tc{\Phi}{^i_j^k}(\V{x})
	+\pdv{\tc{\Phi}{^i_j^k}}{x^\mu} (\V{x}) \cdot\lambda
	+\sO*{^i_j^k}{\lambda} \in\realR \fullstop
\end{equation}
另外,三个基向量作为\emphB{向量值映照},同样可以展开:
\begin{braceEq}
	\V{g}_i\qty(\V{x}+\lambda\,\V{e}_\mu)
	&=\V{g}_i(\V{x})+\pdv{\V{g}_i}{x^\mu} (\V{x}) \cdot\lambda
		+\sOv*{_i}{\lambda} \in\Rm \comma \\
	\V{g}^j\qty(\V{x}+\lambda\,\V{e}_\mu)
	&=\V{g}^j(\V{x})+\pdv{\V{g}^j}{x^\mu} (\V{x}) \cdot\lambda
		+\sOv*{^j}{\lambda} \in\Rm \comma \\
	\V{g}_k\qty(\V{x}+\lambda\,\V{e}_\mu)
	&=\V{g}_k(\V{x})+\pdv{\V{g}_k}{x^\mu} (\V{x}) \cdot\lambda
		+\sOv*{_k}{\lambda} \in\Rm \fullstop
\end{braceEq}
如果直接展开,一共有 81 项,显然过于繁杂,不便操作。
我们将按 $\lambda$ 的次数逐次展开。首先看 $\lambda$ 的零次项:
\begin{equation}
	\tc{\Phi}{^i_j^k}(\V{x})\,
	\V{g}_i(\V{x})\tp\V{g}^j(\V{x})\tp\V{g}_k(\V{x}) \comma
\end{equation}
这就是 $\T{\Phi}(\V{x})$。然后是 $\lambda$ 的一次项:
\begin{align}
	\lambda &\cdot\left[\pdv{\tc{\Phi}{^i_j^k}}{x^\mu} (\V{x})\,
		\V{g}_i(\V{x})\tp\V{g}^j(\V{x})\tp\V{g}_k(\V{x})
	+\tc{\Phi}{^i_j^k} (\V{x})\,
		\pdv{\V{g}_i}{x^\mu} (\V{x})\tp\V{g}^j(\V{x})\tp\V{g}_k(\V{x})
	\right. \notag \\*
	&\phantom{M}\left.\phantom{}+\tc{\Phi}{^i_j^k} (\V{x})\,
		\V{g}_i(\V{x})\tp\pdv{\V{g}^j}{x^\mu} (\V{x})\tp\V{g}_k(\V{x})
	+\tc{\Phi}{^i_j^k} (\V{x})\,
		\V{g}_i(\V{x})\tp\V{g}^j(\V{x})\tp\pdv{\V{g}_k}{x^\mu} (\V{x})
	\vphantom{\pdv{\tc{\Phi}{^i_j^k}}{x^\mu}} \right] \fullstop
\end{align}
剩下的至少是 $\lambda$ 的二次项,
我们将其统一写作“$\res$”(余项)。

现在回头看之前的极限 \eqref{eq:张量场偏导数的极限定义}~式。
$\lambda$ 的零次项与 $\T{\Phi}(\V{x})$ 相互抵消,
而一次项就只剩下了系数部分。至于余项 $\res$,则要证明它趋于零。以
\begin{equation}
	\tc{\Phi}{^i_j^k} (\V{x})\,
	\sOv*{_i}{\lambda}\tp\V{g}^j(\V{x})\tp\V{g}_k(\V{x})
\end{equation}
为例,我们需要证明它等于 $\sOv{\lambda}\in\Tensors{3}$,即
\begin{equation}
	\lim_{\lambda\to 0}
	\frac{\norm[\Tensors{3}]{\tc{\Phi}{^i_j^k} (\V{x})\,
			\sOv*{_i}{\lambda}\tp\V{g}^j(\V{x})\tp\V{g}_k(\V{x})}}
		{\lambda}=0\in\realR \fullstop
	\label{eq:余项证明举例}
\end{equation}

\begin{myProof}
这里为了叙述方便,我们将暂时不使用 Einstein 求和约定,
而是把求和号显式地写出来。于是分子部分可以写成
\begin{align}
	&\alspace\norm[\Tensors{3}]{\sum_{i,\,j,\,k=1}^{m}
		\tc{\Phi}{^i_j^k} (\V{x})\,
		\sOv*{_i}{\lambda}\tp\V{g}^j(\V{x})\tp\V{g}_k(\V{x})} \notag
	\intertext{根据范数的\emphB{三角不等式},有}
	&\leqslant\sum_{i,\,j,\,k=1}^{m}
		\norm[\Tensors{3}]{\tc{\Phi}{^i_j^k} (\V{x})\,
			\sOv*{_i}{\lambda}\tp\V{g}^j(\V{x})\tp\V{g}_k(\V{x})} \notag
	\intertext{再利用\emphB{正齐次性},可得}
	&=\sum_{i,\,j,\,k=1}^{m} \abs{\tc{\Phi}{^i_j^k} (\V{x})}
		\cdot \norm[\Tensors{3}]
			{\sOv*{_i}{\V{x}}\tp\V{g}^j(\V{x})\tp\V{g}_k(\V{x})} \notag
	\intertext{代入简单张量的范数,便有}
	&=\sum_{i,\,j,\,k=1}^{m} \abs{\tc{\Phi}{^i_j^k} (\V{x})}
		\cdot \norm{\sOv*{_i}{\lambda}}
		\cdot \norm{\V{g}^j(\V{x})}
		\cdot \norm{\V{g}_k(\V{x})} \fullstop
\end{align}
这几项中只有 $\norm[\Tensors{3}]{\sOv*{_i}{\lambda}}$ 与
$\lambda$ 有关。于是
\begin{align}
	&\alspace\lim_{\lambda\to 0}
	\frac{\norm[\Tensors{3}]{\tc{\Phi}{^i_j^k} (\V{x})\,
			\sOv*{_i}{\lambda}\tp\V{g}^j(\V{x})\tp\V{g}_k(\V{x})}}
		{\lambda} \notag \\
	&=\sum_{i,\,j,\,k=1}^{m} \abs{\tc{\Phi}{^i_j^k} (\V{x})}
		\cdot \norm{\V{g}^j(\V{x})}
		\cdot \norm{\V{g}_k(\V{x})}
		\cdot \lim_{\lambda\to 0}
		\frac{\norm{\sOv*{_i}{\lambda}}}{\lambda} \notag
	\intertext{根据定义,最后的极限为零,因此}
	&=0 \fullstop
\end{align}
\end{myProof}

类似地,其他七十多项也都是 $\lambda$ 的一阶无穷小量。
而有限个无穷小量之和仍为无穷小量,于是 $\res\to 0$。

\blankline

综上,我们有
\begin{align}
	&\alspace \pdv{\T{\Phi}}{x^\mu} (\V{x})  \coloneq
	\lim_{\lambda\to 0} \frac{\T{\Phi}\qty(\V{x}+\lambda\,\V{e}_\mu)
		-\T{\Phi}(\V{x})} {\lambda} \notag \\
	&=\qty(\pdv{\tc{\Phi}{^i_j^k}}{x^\mu} \,
		\V{g}_i\tp\V{g}^j\tp\V{g}_k
	+\tc{\Phi}{^i_j^k} \,
		\pdv{\V{g}_i}{x^\mu} \tp\V{g}^j\tp\V{g}_k
	+\tc{\Phi}{^i_j^k} \,
		\V{g}_i\tp\pdv{\V{g}^j}{x^\mu} \tp\V{g}_k
	+\tc{\Phi}{^i_j^k} \,
		\V{g}_i\tp\V{g}^j\tp\pdv{\V{g}_k}{x^\mu}) \, (\V{x})
	\fullstop \label{eq:张量场偏导数展开式}
\end{align}
式中,$\pdv*{\V{g}_i}{x^\mu} (\V{x})$ 可以用 Christoffel 符号表示:
\begin{equation}
	\pdv{\V{g}_i}{x^\mu} (\V{x})
	=\ChristoffelB{\mu}{i}{s} \V{g}_s (\V{x}) \fullstop
\end{equation}
因此 \eqref{eq:张量场偏导数展开式}~式的第二项
\begin{align}
	&\alspace\tc{\Phi}{^i_j^k} (\V{x}) \,
		\pdv{\V{g}_i}{x^\mu} (\V{x})
		\tp\V{g}^j(\V{x})\tp\V{g}_k(\V{x}) \notag \\
	&=\ChristoffelB{\mu}{i}{s} \tc{\Phi}{^i_j^k}(\V{x}) \,
		\V{g}_s(\V{x})\tp\V{g}^j(\V{x})\tp\V{g}_k(\V{x}) \notag
	\intertext{$i$ 和 $s$ 都是哑标,不妨进行一下交换:}
	&=\ChristoffelB{\mu}{s}{i} \tc{\Phi}{^s_j^k}(\V{x}) \,
		\V{g}_i(\V{x})\tp\V{g}^j(\V{x})\tp\V{g}_k(\V{x}) \fullstop
\end{align}
同样,后面的两项也可进行类似的处理。这样便有
\begin{align}
	&\alspace \pdv{\T{\Phi}}{x^\mu} (\V{x}) \coloneq
	\lim_{\lambda\to 0} \frac{\T{\Phi}\qty(\V{x}+\lambda\,\V{e}_\mu)
		-\T{\Phi}(\V{x})} {\lambda} \notag \\
	&=\qty[\qty\Bigg(\pdv{\tc{\Phi}{^i_j^k}}{x^\mu}
		+\ChristoffelB{\mu}{s}{i} \tc{\Phi}{^s_j^k}
		-\ChristoffelB{\mu}{j}{s} \tc{\Phi}{^i_s^k}
		+\ChristoffelB{\mu}{s}{k} \tc{\Phi}{^i_j^s}) \,
		\V{g}_i\tp\V{g}^j\tp\V{g}_k] (\V{x}) \notag \\
	&\eqcolon \coD{\mu}{\tc{\Phi}{^i_j^k} (\V{x})} \,
		\V{g}_i(\V{x})\tp\V{g}^j(\V{x})\tp\V{g}_k(\V{x}) \comma
\end{align}
式中,我们称 $\coD{\mu}{\tc{\Phi}{^i_j^k}(\V{x})}\in\realR$
为张量分量 $\tc{\Phi}{^i_j^k}(\V{x})$ 相对于 $x^\mu$
的\emphA{协变导数},其定义为:
\begin{equation}
	\coD{\mu}{\tc{\Phi}{^i_j^k} (\V{x})} \defeq
	\pdv{\tc{\Phi}{^i_j^k}}{x^\mu} (\V{x})
	+\ChristoffelB{\mu}{s}{i} \tc{\Phi}{^s_j^k}(\V{x})
	-\ChristoffelB{\mu}{j}{s} \tc{\Phi}{^i_s^k}(\V{x})
	+\ChristoffelB{\mu}{s}{k} \tc{\Phi}{^i_j^s}(\V{x})
	\fullstop
\end{equation}

\section{张量场的梯度}
\subsection{梯度;可微性}
我们在参数域中的内点 $\V{x}_0$ 处取一个半径为 $\delta$ 的邻域
$\domB{\delta}{\V{x}_0}$。若使自变量变化到 $\V{x}_0+\V{h}$,
则在物理域中,对应的点就将从 $\V{X}\qty(\V{x}_0)$ 变化到
$\V{X}\qty(\V{x}_0+\V{h})$。考察定义在参数域 $\domD{x}$ 上的张量场
$\T{\Phi}(\V{x})$,它的变化为
\begin{align}
	&\alspace\T{\Phi}\qty(\V{x}_0+\V{h})-\T{\Phi}\qty(\V{x}_0)
		\notag \\
	&=\tc{\Phi}{^i_j^k}\qty(\V{x}_0+\V{h}) \,
		\V{g}_i \qty(\V{x}_0+\V{h})
		\tp\V{g}^j \qty(\V{x}_0+\V{h})
		\tp\V{g}_k \qty(\V{x}_0+\V{h}) \notag \\
	&\alspace\phantom{}
		-\tc{\Phi}{^i_j^k}\qty(\V{x}_0) \,
		\V{g}_i \qty(\V{x}_0)
		\tp\V{g}^j \qty(\V{x}_0)
		\tp\V{g}_k \qty(\V{x}_0) \fullstop
	\label{eq:张量场沿任意方向的变化_1}
\end{align}
与之前一样将第一部分逐项展开,有
\begin{braceEq}
	&\tc{\Phi}{^i_j^k}\qty(\V{x}_0+\V{h})
	=\tc{\Phi}{^i_j^k}\qty(\V{x}_0)
	+\pdv{\tc{\Phi}{^i_j^k}}{x^\mu} \qty(\V{x}_0) \cdot h^\mu
	+\sO*{^i_j^k}{\norm{\V{h}}} \in\realR \comma \\
	&\V{g}_i \qty(\V{x}_0+\V{h})
	=\V{g}_i \qty(\V{x}_0)
		+\pdv{\V{g}_i}{x^\mu} \qty(\V{x}_0) \cdot h^\mu
		+\sOv*{_i}{\norm{\V{h}}} \in\Rm \comma \\
	&\V{g}^j \qty(\V{x}_0+\V{h})
	=\V{g}^j \qty(\V{x}_0)
		+\pdv{\V{g}^j}{x^\mu} \qty(\V{x}_0) \cdot h^\mu
		+\sOv*{^j}{\norm{\V{h}}} \in\Rm \comma \\
	&\V{g}_k \qty(\V{x}_0+\V{h})
	=\V{g}_k \qty(\V{x}_0)
		+\pdv{\V{g}_k}{x^\mu} \qty(\V{x}_0) \cdot h^\mu
		+\sOv*{_k}{\norm{\V{h}}} \in\Rm \fullstop
\end{braceEq}
不要忘记对哑标 $\mu$ 进行求和。

不显含 $\V{h}$ 的只有每行的第一项;它们组合起来,
与式~\eqref{eq:张量场沿任意方向的变化_1} 的第二部分相互抵消。
再看 $\V{h}$ 的一次项 :\footnote{%
	实际是 $h^\mu$ 的一次项。别忘了求和。}
\begin{equation}
	\qty[
		\pdv{\tc{\Phi}{^i_j^k}}{x^\mu}\,
		\V{g}_i \tp \V{g}^j \tp \V{g}_k
	+\tc{\Phi}{^i_j^k} \qty(
		\pdv{\V{g}_i}{x^\mu}\tp\V{g}^j\tp\V{g}_k
		+\V{g}_i\tp\pdv{\V{g}^j}{x^\mu}\tp\V{g}_k
		+\V{g}_i\tp\V{g}^j\tp\pdv{\V{g}_k}{x^\mu} )]
	\qty(\V{x}_0) \cdot h^\mu \fullstop
\end{equation}
利用 Christoffel 符号又可以把它写成
\begin{align}
	&\alspace \left[\pdv{\tc{\Phi}{^i_j^k}}{x^\mu}\,
			\V{g}_i \tp \V{g}^j \tp \V{g}_k
		+\tc{\Phi}{^i_j^k}\, \qty\Big(\ChristoffelB{\mu}{i}{s}\V{g}_s)
			\tp\V{g}^j\tp\V{g}_k \right.
		\notag \\*
	&\phantom{MM} \left.\phantom{}-\tc{\Phi}{^i_j^k} \,
			\V{g}_i\tp\qty\Big(\ChristoffelB{\mu}{s}{j}\V{g}^s)\tp\V{g}_k
		+\tc{\Phi}{^i_j^k} \,
			\V{g}_i\tp\V{g}^j\tp\qty\Big(\ChristoffelB{\mu}{k}{s}\V{g}_s)
		\vphantom{\pdv{\tc{\Phi}{^i_j^k}}{x^\mu}} \right]
		\qty(\V{x}_0) \cdot h^\mu \notag \\
	%%
	&=\left(\pdv{\tc{\Phi}{^i_j^k}}{x^\mu}\,
			\V{g}_i \tp \V{g}^j \tp \V{g}_k
		+\ChristoffelB{\mu}{s}{i} \tc{\Phi}{^s_j^k} \,
			\V{g}_i\tp\V{g}^j\tp\V{g}_k \right.
		\notag \\*
	&\phantom{MM} \left.\phantom{}
		-\ChristoffelB{\mu}{j}{s} \tc{\Phi}{^i_s^k} \,
			\V{g}_i\tp\V{g}^j\tp\V{g}_k
		+\ChristoffelB{\mu}{s}{k} \tc{\Phi}{^i_j^s} \,
			\V{g}_i\tp\V{g}^j\tp\V{g}_k
		\vphantom{\pdv{\tc{\Phi}{^i_j^k}}{x^\mu}} \right)
		\qty(\V{x}_0) \cdot h^\mu \notag \\
	%%
	&=\qty[\qty\Bigg(\pdv{\tc{\Phi}{^i_j^k}}{x^\mu}
		+\ChristoffelB{\mu}{s}{i} \tc{\Phi}{^s_j^k}
		-\ChristoffelB{\mu}{j}{s} \tc{\Phi}{^i_s^k}
		+\ChristoffelB{\mu}{s}{k} \tc{\Phi}{^i_j^s}) \,
		\V{g}_i\tp\V{g}^j\tp\V{g}_k] \qty(\V{x}_0)
		\cdot h^\mu \fullstop
\end{align}
至于高阶项,它们都等于 $\sOv{\norm{\V{h}}}\in\Tensors{3}$,并且满足
\begin{equation}
	\lim_{\V{h}\to\V{0}\,\in\,\Rm}
		\frac{\norm[\Tensors{3}]{\sOv{\norm{\V{h}}}}}{\norm{\V{h}}}
	=0 \in\realR \fullstop
\end{equation}
证明与 \eqref{eq:余项证明举例}~式类似。

整理一下,我们有
\begin{align}
	&\alspace\T{\Phi}\qty(\V{x}_0+\V{h})-\T{\Phi}\qty(\V{x}_0)
		\notag \\
	&=\qty[\qty\Bigg(\pdv{\tc{\Phi}{^i_j^k}}{x^\mu}
			+\ChristoffelB{\mu}{s}{i} \tc{\Phi}{^s_j^k}
			-\ChristoffelB{\mu}{j}{s} \tc{\Phi}{^i_s^k}
			+\ChristoffelB{\mu}{s}{k} \tc{\Phi}{^i_j^s}) \,
			\V{g}_i\tp\V{g}^j\tp\V{g}_k] \qty(\V{x}_0) \cdot h^\mu
		+\sOv{\norm{\V{h}}} \notag
	\intertext{利用协变导数,有}
	&\eqcolon \qty[\vphantom{\frac{0^0}{0^0}}
		\coD{\mu}{\tc{\Phi}{^i_j^k} \qty(\V{x}_0)} \,
		\V{g}_i \qty(\V{x}_0)
		\tp\V{g}^j \qty(\V{x}_0)
		\tp\V{g}_k \qty(\V{x}_0)] h^\mu
		+\sOv{\norm{\V{h}}} \fullstop
	\label{eq:张量场沿任意方向的变化_2}
\end{align}
至此,从微分学的角度来看,任务已经完成。
但对于张量分析而言,我们还需要再做一点微小的工作。
简单张量部分再并上一个 $\V{g}^\mu$,从而使张量升一阶;
后面则改成 $h^\nu\,\V{g}_\nu\qty(\V{x}_0)$,并利用点乘保持总阶数不变:
\begin{align}
	&\alspace\qty[\vphantom{\frac{0^0}{0^0}}
		\coD{\mu}{\tc{\Phi}{^i_j^k} \qty(\V{x}_0)} \,
		\V{g}_i \qty(\V{x}_0) \tp \V{g}^j \qty(\V{x}_0)
		\tp\V{g}_k \qty(\V{x}_0)] h^\mu \notag \\
	&=\qty[\vphantom{\frac{0^0}{0^0}}
			\coD{\mu}{\tc{\Phi}{^i_j^k} \qty(\V{x}_0)} \,
			\V{g}_i \qty(\V{x}_0) \tp \V{g}^j \qty(\V{x}_0)
			\tp\V{g}_k \qty(\V{x}_0) \tp \V{g}^\mu \qty(\V{x}_0)]
		\cdot \qty[\vphantom{\frac{0}{0}} h^\nu\V{g}_\nu\qty(\V{x}_0)]
	\fullstop
\end{align}
所谓“点乘”,其实就是 \emphB{$e$ 点积}在 $e=1$ 时的情况。实际上,
\begin{equation}
	\V{g}^\mu\qty(\V{x}_0) \cdot
		\qty[\vphantom{\frac{0}{0}} h^\nu\V{g}_\nu\qty(\V{x}_0)]
	=h^\nu\,\KroneckerDelta{\mu}{\nu}
	=h^\mu \fullstop
\end{equation}
这里用到了局部基的\emphB{对偶关系} \eqref{eq:局部基_对偶关系}~式。

此时, 我们获得了一个四阶张量与
$h^\nu\V{g}_\nu\qty(\V{x}_0)$ 的点积。接下来讨论该项的意义。
参数域中 $\V{x}_0$ 发生 $\V{h}=h^i\,\V{e}_i$ 的变化时,
根据向量值映照 $\V{X}(\V{x})$ 的\emphB{可微性},
对应物理域中的变化为
\begin{align}
	\V{X}\qty(\V{x}_0+\V{h})-\V{X}\qty(\V{x}_0)
	&=\JacobiD{\V{X}}\qty(\V{x}_0)(\V{h})
		+\sOv{\norm{\V{h}}} \notag \\
	&\eqcolon\pdv{\V{X}}{x^\nu} \qty(\V{x}_0) \, h^\nu
		+\sOv{\norm{\V{h}}} \notag
	\intertext{代入 \ref{subsec:局部协变基}~小节中%
		\emphB{局部协变基}的定义,可有}
	&\eqcolon h^\nu\V{g}_\nu\qty(\V{x}_0)
		+\sOv{\norm{\V{h}}} \fullstop
\end{align}
代入式~\eqref{eq:张量场沿任意方向的变化_2},有
\begin{align}
	&\alspace\T{\Phi}\qty(\V{x}_0+\V{h})-\T{\Phi}\qty(\V{x}_0)
		\notag \\
	&=\qty[\vphantom{\frac{0^0}{0^0}}
			\coD{\mu}{\tc{\Phi}{^i_j^k} \qty(\V{x}_0)} \,
			\V{g}_i \qty(\V{x}_0) \tp \V{g}^j \qty(\V{x}_0)
			\tp\V{g}_k \qty(\V{x}_0) \tp \V{g}^\mu \qty(\V{x}_0)]
		\notag \\*
	&\alspace\phantom{} \cdot \qty[\vphantom{\frac{0}{0}}
			\V{X}\qty(\V{x}_0+\V{h})-\V{X}\qty(\V{x}_0)
			+\sOv{\norm{\V{h}}}]+\sOv{\norm{\V{h}}} \notag
	\intertext{合并掉一阶无穷小量\footnotemark{},可得}
	&=\qty[\vphantom{\frac{0^0}{0^0}}
			\coD{\mu}{\tc{\Phi}{^i_j^k} \qty(\V{x}_0)} \,
			\V{g}_i \qty(\V{x}_0) \tp \V{g}^j \qty(\V{x}_0)
			\tp\V{g}_k \qty(\V{x}_0) \tp \V{g}^\mu \qty(\V{x}_0)]
		\notag \\*
	&\alspace\phantom{} \cdot \qty[\vphantom{\frac{0}{0}}
			\V{X}\qty(\V{x}_0+\V{h})-\V{X}\qty(\V{x}_0)]
		+\sOv{\norm{\V{h}}} \notag \\
	&=\qty[\pdv{\T{\Phi}}{x^\mu} \qty(\V{x}_0)
		\tp \V{g}^\mu\qty(\V{x}_0)]
		\cdot \qty[\vphantom{\frac{0}{0}}
			\V{X}\qty(\V{x}_0+\V{h})-\V{X}\qty(\V{x}_0)]
		+\sOv{\norm{\V{h}}} \in\Tensors{3} \fullstop
\end{align}
\footnotetext{式中的两个 $\sOv{\norm{\V{h}}}$ 是不同的,
	前者属于 $\Rm$,后者属于 $\Tensors{3}$。}%
引入记号
\begin{equation}
	\T{\Phi} \qty(\V{x}_0)
	\tp\qty[\V{g}^\mu \pdv{x^\mu} \qty(\V{x}_0)]
	\coloneq \pdv{\T{\Phi}}{x^\mu} \qty(\V{x}_0)
		\tp \V{g}^\mu\qty(\V{x}_0) \comma
\end{equation}
再引入\emphA{梯度算子}
\begin{equation}
	\opGrad \coloneq
	\V{g}^\mu \pdv{x^\mu} \qty(\V{x}_0) \comma
\end{equation}
我们得到的结论就可以表述为
\begin{equation}
	\T{\Phi}\qty(\V{x}_0+\V{h})-\T{\Phi}\qty(\V{x}_0)
	=\qty\big(\T{\Phi}\tp\opGrad) \qty(\V{x}_0)
	\cdot \qty[\vphantom{\frac{0}{0}}
		\V{X}\qty(\V{x}_0+\V{h})-\V{X}\qty(\V{x}_0)]
	+\sOv{\norm{\V{h}}} \in\Tensors{3} \comma
\end{equation}
式中的“$\cdot$”表示点乘。以上结果称之为张量场的\emphA{可微性},
它表明,由一点处的位置移动所引起张量场的变化,
可以用该点处张量场的\emphA{梯度}(即 $\T{\Phi}\tp\opGrad$)
点乘物理空间中的位置差别来近似,误差为一阶无穷小量。
以上分析基于三阶张量。但显然,对于 $p$ 阶张量,
将会有完全一致的结果。

\subsection{方向导数}
现在来研究张量场沿 $\V{e}$ 方向的变化率(设 $\norm{\V{e}}=1$)。
取一个与 $\V{e}$ 平行的向量 $\lambda\,\V{e}$。
注意到 $\lambda\,\V{e}$ 其实就是物理空间中的位置变化,
于是根据张量场的可微性,我们有
\begin{equation}
	\T{\Phi}\qty(\V{x}_0+\lambda\,\V{e})-\T{\Phi}\qty(\V{x}_0)
	=\qty\big(\T{\Phi}\tp\opGrad) \qty(\V{x}_0)
	\cdot \qty\big(\lambda\,\V{e}) + \sOv{\lambda} \comma
\end{equation}
该式等价于
\begin{equation}
	\lim_{\lambda\to 0} \frac{\T{\Phi}\qty(\V{x}_0+\lambda\,\V{e})
		-\T{\Phi}\qty(\V{x}_0)}{\lambda}
	=\qty\big(\T{\Phi}\tp\opGrad) \qty(\V{x}_0) 
		\cdot \V{e} \fullstop
\end{equation}
我们把它定义为张量场 $\T{\Phi}(\V{x})$ 沿 $\V{e}$ 方向
的\emphA{方向导数}:
\begin{equation}
	\pdv{\T{\Phi}}{\V{e}} (\V{x}_0)
	\defeq \qty\big(\T{\Phi}\tp\opGrad) \qty(\V{x}_0) 
		\cdot \V{e} \fullstop
\end{equation}

\subsection{左梯度与右梯度} \label{subsec:左梯度与右梯度}
我们已经知道,利用梯度算子
\begin{equation}
	\opGrad \coloneq
	\V{g}^\mu \pdv{x^\mu} \qty(\V{x}_0) \comma
\end{equation}
可以把张量场的梯度表示为
\begin{equation}
	\qty\big(\T{\Phi}\tp\opGrad) \qty(\V{x}_0)
	=\T{\Phi}\qty(\V{x}_0)
		\tp\qty[\V{g}^\mu \pdv{x^\mu} \qty(\V{x}_0)]
	\coloneq \pdv{\T{\Phi}}{x^\mu} \qty(\V{x}_0)
		\tp \V{g}^\mu \qty(\V{x}_0) \fullstop
\end{equation}
“$\opGrad$”在右边,故称之为\emphA{右梯度}(简称\emphA{梯度})。
相应地,自然会有\emphA{左梯度}:
\begin{equation}
	\qty\big(\opGrad\tp\T{\Phi}) \qty(\V{x}_0)
	=\qty[\V{g}^\mu \pdv{x^\mu} \qty(\V{x}_0)]
		\tp \T{\Phi}\qty(\V{x}_0)
	\coloneq \V{g}^\mu \qty(\V{x}_0)
		\tp \pdv{\T{\Phi}}{x^\mu} \qty(\V{x}_0) \fullstop
\end{equation}
张量积不存在交换律,因而这两者是不同的。
注意,梯度运算将使张量的阶数增加一阶。

张量场的可微性可以用\emphB{左梯度}来等价表述:
\begin{equation}
	\T{\Phi}\qty(\V{x}_0+\V{h})-\T{\Phi}\qty(\V{x}_0)
	=\qty[\vphantom{\frac{0}{0}}
		\V{X}\qty(\V{x}_0+\V{h})-\V{X}\qty(\V{x}_0)]
	\cdot \qty\big(\opGrad\tp\T{\Phi}) \qty(\V{x}_0)
	+\sOv{\norm{\V{h}}} \fullstop
\end{equation}
类似地,还有方向导数:
\begin{equation}
	\pdv{\T{\Phi}}{\V{e}} (\V{x}_0)
	\defeq \qty\big(\T{\Phi}\tp\opGrad) \qty(\V{x}_0) \cdot \V{e}
	=\V{e} \cdot \qty\big(\opGrad\tp\T{\Phi}) \qty(\V{x}_0)
	\fullstop
\end{equation}

\section{场论恒等式(一)}
为了给下一节做好铺垫,本节将证明几个重要引理。

\subsection{Ricci 引理}
首先来证明两个结论:
\begin{braceEq}
	\pdv{\T{G}}{x^\mu} (\V{x})
		&=\T{0}\in\Tensors{2} \comma \\
	\pdv{\EdTensor}{x^\mu} (\V{x})
		&=\T{0}\in\Tensors[\realR^3]{3} \comma
\end{braceEq}
其中的 $\T{G}$ 和 $\EdTensor$ 分别是度量张量和 Eddington 张量。

\begin{myProof}
为方便起见,证明中我们将省去“$(\V{x})$”。

先考察度量张量的偏导数:
\begin{align}
	\pdv{\T{G}}{x^\mu}
	&=\pdv{x^\mu} \qty(g_{ij}\,\V{g}^i\tp\V{g}^j) \notag \\
	&=\coD{\mu}{g_{ij}} \, \V{g}^i\tp\V{g}^j \comma
\end{align}
式中,协变导数定义为
\begin{equation}
	\coD{\mu}{g_{ij}} \defeq \pdv{g_{ij}}{x^\mu}
		-\ChristoffelB{\mu}{i}{s} g_{sj}
		-\ChristoffelB{\mu}{j}{s} g_{is} \fullstop
\end{equation}
以下有两种方法证明 $\coD{\mu}{g_{ij}}=0$。

方法一利用度量的定义:
\begin{align}
	\pdv{g_{ij}}{x^\mu}
	&\defeq \pdv{x^\mu} \ipb{\V{g}_i}{\V{g}_j} \notag \\
	&=\ipb{\pdv{\V{g}_i}{x^\mu}}{\V{g}_j}
		+\ipb{\V{g}_i}{\pdv{\V{g}_j}{x^\mu}} \notag
	\intertext{根据 Christoffel 符号的定义
		(见 \ref{subsec:Christoffel符号}~小节),有}
	&=\ChristoffelA{\mu}{i}{j}+\ChristoffelA{\mu}{j}{i} \fullstop
\end{align}
另一方面,回忆 \eqref{eq:第二类Christoffel符号用第一类表示}~式:
\begin{equation}
	\ChristoffelB{i}{j}{k}=\ChristoffelA{i}{j}{l} g^{kl}\ \comma
\end{equation}
可有
\begin{braceEq}
	\ChristoffelB{\mu}{i}{s} g_{sj}
		&=\ChristoffelA{\mu}{i}{k} g^{sk} g_{sj}
		=\ChristoffelA{\mu}{i}{k} \KroneckerDelta{k}{j}
		=\ChristoffelA{\mu}{i}{j} \comma \\
	\ChristoffelB{\mu}{j}{s} g_{is}
		&=\ChristoffelA{\mu}{j}{k} g^{sk} g_{is}
		=\ChristoffelA{\mu}{j}{k} \KroneckerDelta{k}{i}
		=\ChristoffelA{\mu}{j}{i} \fullstop
\end{braceEq}
于是
\begin{equation}
	\coD{\mu}{g_{ij}}
	=\ChristoffelA{\mu}{i}{j}+\ChristoffelA{\mu}{j}{i}
	-\ChristoffelA{\mu}{i}{j}-\ChristoffelA{\mu}{j}{i}
	=0 \fullstop
\end{equation}

方法二则利用第一类 Christoffel 符号的性质
\eqref{eq:第一类Christoffel符号与度量的关系}~式:
\begin{equation}
	\ChristoffelA{i}{j}{k}=\frac{1}{2}\, \qty(
		\pdv{g_{jk}}{x^i}+\pdv{g_{ik}}{x^j}-\pdv{g_{ij}}{x^k})
	\fullstop
\end{equation}
因而
\begin{align}
	&\alspace \ChristoffelB{\mu}{i}{s} g_{sj}
		+\ChristoffelB{\mu}{j}{s} g_{is} \notag \\
	&=\ChristoffelA{\mu}{i}{j}+\ChristoffelA{\mu}{j}{i} \notag \\
	&=\frac{1}{2}\, \qty(\pdv{g_{ij}}{x^\mu}
			+\pdv{g_{\mu j}}{x^i}-\pdv{g_{\mu i}}{x^j})
		+\frac{1}{2}\, \qty(\pdv{g_{ji}}{x^\mu}
			+\pdv{g_{\mu i}}{x^j}-\pdv{g_{\mu j}}{x^i}) \notag \\
	&=\frac{1}{2}\, \qty(\pdv{g_{ij}}{x^\mu}+\pdv{g_{ji}}{x^\mu})
	=\pdv{g_{ij}}{x^\mu} \fullstop
\end{align}
显然,立刻就有
\begin{equation}
	\coD{\mu}{g_{ij}}
	=\pdv{g_{ij}}{x^\mu}-\pdv{g_{ij}}{x^\mu} = 0 \fullstop
\end{equation}

综上,因为 $\coD{\mu} g_{ij}=0\in\realR$,所以
\begin{equation}
	\pdv{\T{G}}{x^\mu} (\V{x})=\T{0}\in\Tensors{2} \fullstop
\end{equation}
如果用其他形式的分量来表述这一结果,我们便有
\begin{equation}
	\coD{\mu}{g_{ij}} = \coD{\mu}{g^{ij}}
	=\coD{\mu}{\KroneckerDelta{i}{j}} = 0 \fullstop
\end{equation}
此结论称为\emphA{Ricci引理}。

\blankline

再来看 Eddington 张量的偏导数:
\begin{align}
	\pdv{\EdTensor}{x^\mu}
	&=\pdv{x^\mu} \qty(\LeviCivita{^i_j^k}\,
		\V{g}_i\tp\V{g}^j\tp\V{g}_k) \notag \\
	&=\coD{\mu}{\LeviCivita{^i_j^k}}\,
		\V{g}_i\tp\V{g}^j\tp\V{g}_k \comma
\end{align}
式中,
\begin{align}
	\coD{\mu}{\LeviCivita{^i_j^k}}
	\defeq \pdv{\LeviCivita{^i_j^k}}{x^\mu}
		+\ChristoffelB{\mu}{s}{i} \LeviCivita{^s_j^k}
		-\ChristoffelB{\mu}{j}{s} \LeviCivita{^i_s^k}
		+\ChristoffelB{\mu}{s}{k} \LeviCivita{^i_j^s} \fullstop
	\label{eq:Eddington张量偏导数推导}
\end{align}
根据定义,$\LeviCivita{^i_j^k}=\det[\V{g}^i,\,\V{g}_j,\,\V{g}^k]$。
因此
\begin{align}
	\pdv{\LeviCivita{^i_j^k}}{x^\mu}
	&=\pdv{x^\mu} \qty\bigg(
		\det[\V{g}^i,\,\V{g}_j,\,\V{g}^k]) \notag \\
	&=\det[\pdv{\V{g}^i}{x^\mu},\,\V{g}_j,\,\V{g}^k]
		+\det[\V{g}^i,\,\pdv{\V{g}_j}{x^\mu},\,\V{g}^k]
		+\det[\V{g}^i,\,\V{g}_j,\,\pdv{\V{g}^k}{x^\mu}] \notag
	\intertext{利用标架运动方程,有}
	&=\det[-\ChristoffelB{\mu}{s}{i}\V{g}^s,\,\V{g}_j,\,\V{g}^k]
		+\det[\V{g}^i,\,\ChristoffelB{\mu}{j}{s}\V{g}_s,\,\V{g}^k]
		+\det[\V{g}^i,\,\V{g}_j,\,-\ChristoffelB{\mu}{s}{k}\V{g}^s]
		\notag
	\intertext{再利用行列式的线性性,提出系数:}
	&=-\ChristoffelB{\mu}{s}{i}\det[\V{g}^s,\,\V{g}_j,\,\V{g}^k]
		+\ChristoffelB{\mu}{j}{s}\det[\V{g}^i,\,\V{g}_s,\,\V{g}^k]
		-\ChristoffelB{\mu}{s}{k}\det[\V{g}^i,\,\V{g}_j,\,\V{g}^s]
		\notag
	\intertext{代回 Eddington 张量的定义,可得}
	&=-\ChristoffelB{\mu}{s}{i} \LeviCivita{^s_j^k}
		+\ChristoffelB{\mu}{j}{s} \LeviCivita{^i_s^k}
		-\ChristoffelB{\mu}{s}{k} \LeviCivita{^i_j^s} \fullstop
\end{align}
这与式~\eqref{eq:Eddington张量偏导数推导} 的后三项恰好抵消。
于是便有 $\coD{\mu}{\LeviCivita{^i_j^k}}=0$。进而
\begin{equation}
	\pdv{\EdTensor}{x^\mu} (\V{x})=\T{0}\in\Tensors[\realR^3]{3}
	\fullstop
\end{equation}

和度量张量类似,Eddington 张量其他分量的偏导数,
如 $\coD{\mu}{\LeviCivita{_{ijk}}}$、
$\coD{\mu}{\LeviCivita{^{ijk}}}$ 等,也都等于零。
此结论同样称为\emphA{Ricci 引理}。
\end{myProof}

\subsection{Leibniz 法则}
协变导数满足\emphA{Leibniz 法则}:
\begin{equation}
	\coD{\mu}{\qty\Big(\tc{\Phi}{^i_j^k}\,\tc{\Psi}{_p^q})}
	=\qty\Big(\coD{\mu}{\tc{\Phi}{^i_j^k}})\,\tc{\Psi}{_p^q}
	+\tc{\Phi}{^i_j^k}\,\qty\Big(\coD{\mu}{\tc{\Psi}{_p^q}})
	\fullstop
\end{equation}
式中,张量分量的形式可以是任意的。

\begin{myProof}
显然,$\tc{\Phi}{^i_j^k}\,\tc{\Psi}{_p^q}\in\Tensors{5}$。不妨令
\begin{equation}
	\tc{\Omega}{^i_j^k_p^q}
	=\tc{\Phi}{^i_j^k}\,\tc{\Psi}{_p^q} \fullstop
\end{equation}
则
\begin{align}
	&\alspace \coD{\mu}{\qty\Big(\tc{\Phi}{^i_j^k}\,\tc{\Psi}{_p^q})}
	=\coD{\mu}{\tc{\Omega}{^i_j^k_p^q}} \notag \\
	&\defeq \pdv{\tc{\Omega}{^i_j^k_p^q}}{x^\mu}
		+\ChristoffelB{\mu}{s}{i} \tc{\Omega}{^s_j^k_p^q}
		-\ChristoffelB{\mu}{j}{s} \tc{\Omega}{^i_s^k_p^q}
		+\ChristoffelB{\mu}{s}{k} \tc{\Omega}{^i_j^s_p^q}
		-\ChristoffelB{\mu}{p}{s} \tc{\Omega}{^i_j^k_s^q}
		+\ChristoffelB{\mu}{s}{q} \tc{\Omega}{^i_j^k_p^s} \notag \\
	&=\pdv{x^\mu} \qty\Big(\tc{\Phi}{^i_j^k}\,\tc{\Psi}{_p^q})
		+\qty\Big(\ChristoffelB{\mu}{s}{i} \tc{\Phi}{^s_j^k}
			-\ChristoffelB{\mu}{j}{s} \tc{\Phi}{^i_s^k}
			+\ChristoffelB{\mu}{s}{k} \tc{\Phi}{^i_j^s})\,\tc{\Psi}{_p^q}
		+\tc{\Phi}{^i_j^k}\,\qty\Big(
			-\ChristoffelB{\mu}{p}{s} \tc{\Psi}{_s^q}
			+\ChristoffelB{\mu}{s}{q} \tc{\Psi}{_p^s}) \fullstop
\end{align}
第一项偏导数自然满足乘积法则:
\begin{equation}
	\pdv{x^\mu} \qty\Big(\tc{\Phi}{^i_j^k}\,\tc{\Psi}{_p^q})
	=\pdv{\tc{\Phi}{^i_j^k}}{x^\mu}\,\tc{\Psi}{_p^q}
		+\tc{\Phi}{^i_j^k}\,\pdv{\tc{\Psi}{_p^q}}{x^\mu} \fullstop
\end{equation}
代回前一式,即有
\begin{align}
	\coD{\mu}{\qty\Big(\tc{\Phi}{^i_j^k}\,\tc{\Psi}{_p^q})}
	&=\qty\Bigg(\pdv{\tc{\Phi}{^i_j^k}}{x^\mu}
			+\ChristoffelB{\mu}{s}{i} \tc{\Phi}{^s_j^k}
			-\ChristoffelB{\mu}{j}{s} \tc{\Phi}{^i_s^k}
			+\ChristoffelB{\mu}{s}{k} \tc{\Phi}{^i_j^s})\,\tc{\Psi}{_p^q}
		+\tc{\Phi}{^i_j^k}\,\qty\Bigg(\pdv{\tc{\Psi}{_p^q}}{x^\mu}
			-\ChristoffelB{\mu}{p}{s} \tc{\Psi}{_s^q}
			+\ChristoffelB{\mu}{s}{q} \tc{\Psi}{_p^s}) \notag \\
	&\defeq \qty\Big(\coD{\mu}{\tc{\Phi}{^i_j^k}})\,\tc{\Psi}{_p^q}
		+\tc{\Phi}{^i_j^k}\,\qty\Big(\coD{\mu}{\tc{\Psi}{_p^q}})
		\fullstop
\end{align}
\end{myProof}

现在来考虑 $\coD{\mu}{\qty(\tc{\Phi}{^i_j^k}\,\tc{\Psi}{_k^q})}$,
注意其中的 $k$ 是哑标。若按照 Leibniz 法则,似乎有
\begin{align}
	\coD{\mu}{\qty\Big(\tc{\Phi}{^i_j^k}\,\tc{\Psi}{_k^q})}
	&=\qty\Big(\coD{\mu}{\tc{\Phi}{^i_j^k}})\,\tc{\Psi}{_k^q}
		+\tc{\Phi}{^i_j^k}\,\qty\Big(\coD{\mu}{\tc{\Psi}{_k^q}})
		\notag \\
	&=\qty\Bigg(\pdv{\tc{\Phi}{^i_j^k}}{x^\mu}
			+\ChristoffelB{\mu}{s}{i} \tc{\Phi}{^s_j^k}
			-\ChristoffelB{\mu}{j}{s} \tc{\Phi}{^i_s^k}
			+\ChristoffelB{\mu}{s}{k} \tc{\Phi}{^i_j^s})\,\tc{\Psi}{_k^q}
		+\tc{\Phi}{^i_j^k}\,\qty\Bigg(\pdv{\tc{\Psi}{_k^q}}{x^\mu}
			-\ChristoffelB{\mu}{k}{s} \tc{\Psi}{_s^q}
			+\ChristoffelB{\mu}{s}{q} \tc{\Psi}{_k^s}) \notag \\
	&=\pdv{x^\mu} \qty\Big(\tc{\Phi}{^i_j^k}\,\tc{\Psi}{_k^q})
		+\qty\Big(\ChristoffelB{\mu}{s}{i} \tc{\Phi}{^s_j^k}
			-\ChristoffelB{\mu}{j}{s}\tc{\Phi}{^i_s^k})\,\tc{\Psi}{_k^q}
		+\tc{\Phi}{^i_j^k}\,
			\qty\Big(\ChristoffelB{\mu}{s}{q}\tc{\Psi}{_k^s}) \notag \\*
	&\alspace\phantom{}
		+\ChristoffelB{\mu}{s}{k} \tc{\Phi}{^i_j^s}\,\tc{\Psi}{_k^q}
		-\ChristoffelB{\mu}{k}{s} \tc{\Phi}{^i_j^k}\,\tc{\Psi}{_s^q}
	\fullstop \label{eq:带哑标的Leibniz法则}
\end{align}
而根据定义,则
\begin{equation}
	\coD{\mu}{\qty\Big(\tc{\Phi}{^i_j^k}\,\tc{\Psi}{_k^q})}
	\defeq \pdv{x^\mu} \qty\Big(\tc{\Phi}{^i_j^k}\,\tc{\Psi}{_k^q})
		+\qty\Big(\ChristoffelB{\mu}{s}{i} \tc{\Phi}{^s_j^k}
			-\ChristoffelB{\mu}{j}{s} \tc{\Phi}{^i_s^k})\,\tc{\Psi}{_k^q}
		+\tc{\Phi}{^i_j^k}\,
			\qty\Big(\ChristoffelB{\mu}{s}{q} \tc{\Psi}{_k^s})
	\fullstop
\end{equation}
很明显,式~\eqref{eq:带哑标的Leibniz法则} 中多了两项。
不过稍作计算,就可知道
\begin{align}
	&\alspace \ChristoffelB{\mu}{s}{k}
		\tc{\Phi}{^i_j^s}\,\tc{\Psi}{_k^q}
	-\ChristoffelB{\mu}{k}{s}
		\tc{\Phi}{^i_j^k}\,\tc{\Psi}{_s^q} \notag
	\intertext{$k$ 和 $s$ 都是哑标,不妨在第二项中将二者交换:}
	&=\ChristoffelB{\mu}{s}{k} \tc{\Phi}{^i_j^s}\,\tc{\Psi}{_k^q}
		-\ChristoffelB{\mu}{s}{k} \tc{\Phi}{^i_j^s}\,\tc{\Psi}{_k^q}
	=0 \fullstop
\end{align}
可见,Leibniz 法则经受住了考验。

\blankline

把 Ricci 引理和 Leibniz 法则联合起来,便有
\begin{braceEq}
	\coD{\mu}{\qty\Big(g_{ij}\,\tc{\Psi}{^p_q})}
	&=\qty\Big(\coD{\mu}{g_{ij}})\,\tc{\Psi}{^p_q}
		+g_{ij} \qty\Big(\coD{\mu}{\tc{\Psi}{^p_q}})
	=g_{ij}\,\coD{\mu}{\tc{\Psi}{^p_q}} \comma \\
	\coD{\mu}{\qty\Big(\LeviCivita{^i_j^k}\,\tc{\Psi}{^p_q})}
	&=\qty\Big(\coD{\mu}{\LeviCivita{^i_j^k}})\,\tc{\Psi}{^p_q}
		+\LeviCivita{^i_j^k} \qty\Big(\coD{\mu}{\tc{\Psi}{^p_q}})
	=\LeviCivita{^i_j^k}\,\coD{\mu}{\tc{\Psi}{^p_q}} \fullstop
\end{braceEq}
这说明度量张量和 Eddington 张量类似常数,可以提到协变导数的外面。

\subsection{混合协变导数}
与混合偏导数定理类似,协变导数满足
\begin{equation}
	\coD{\nu}{\coD{\mu}{\tc{\Phi}{^i_j^k}}}
	=\coD{\mu}{\coD{\nu}{\tc{\Phi}{^i_j^k}}} \fullstop
	\label{eq:混合协变导数定理}
\end{equation}
不必多说,张量分量依然可以任意选取。
只是需要注意,该定理只在\myPROBLEM{体积上张量场场论}成立。

\begin{myProof}
首先计算张量场\emphB{整体}的一阶偏导数:
\begin{equation}
	\pdv{\T{\Phi}}{x^\mu}
	=\pdv{x^\mu} \qty\Big(
		\tc{\Phi}{^i_j^k} \V{g}_i\tp\V{g}^j\tp\V{g}_k)
	=\coD{\mu}{\tc{\Phi}{^i_j^k}} \, \V{g}_i\tp\V{g}^j\tp\V{g}_k
	\in\Tensors{3} \fullstop
\end{equation}
再求一次偏导数,可有
\begin{align}
	\pdv{\T{\Phi}}{x^\nu}{x^\mu}
	\coloneq \pdv{x^\nu} \qty(\pdv{\T{\Phi}}{x^\mu})
	=\pdv{x^\nu} \qty\Big(\coD{\mu}{\tc{\Phi}{^i_j^k}} \,
		\V{g}_i\tp\V{g}^j\tp\V{g}_k) \fullstop
\end{align}
请注意,括号里的张量带有一个\emphB{独立指标} $\mu$。
按照极限分析,有
\begin{align}
	&\alspace \pdv{x^\nu} \qty\Big(\coD{\mu}{\tc{\Phi}{^i_j^k}} \,
		\V{g}_i\tp\V{g}^j\tp\V{g}_k) \notag \\
	&=\pdv{x^\nu}\qty\Big(\coD{\mu}{\tc{\Phi}{^i_j^k}}) \,
		\V{g}_i\tp\V{g}^j\tp\V{g}_k
		+\coD{\mu}{\tc{\Phi}{^i_j^k}} \qty(
			\pdv{\V{g}_i}{x^\nu} \tp\V{g}^j\tp\V{g}_k
			+\V{g}_i\tp\pdv{\V{g}^j}{x^\nu} \tp\V{g}_k
			+\V{g}_i\tp\V{g}^j\tp\pdv{\V{g}_k}{x^\nu}) \notag \\
	&=\pdv{x^\nu}\qty\Big(\coD{\mu}{\tc{\Phi}{^i_j^k}}) \,
		\V{g}_i\tp\V{g}^j\tp\V{g}_k
		+\coD{\mu}{\tc{\Phi}{^i_j^k}} \qty\Big(
			\ChristoffelB{\nu}{i}{s} \V{g}_s\tp\V{g}^j\tp\V{g}_k
			-\ChristoffelB{\nu}{s}{j} \V{g}_i\tp\V{g}^s\tp\V{g}_k
			+\ChristoffelB{\nu}{k}{s} \V{g}_i\tp\V{g}^j\tp\V{g}_s)
		\notag \\
	&=\qty[\pdv{x^\nu} \qty\Big(\coD{\mu}{\tc{\Phi}{^i_j^k}})
			+\ChristoffelB{\nu}{s}{i} \coD{\mu}{\tc{\Phi}{^s_j^k}}
			-\ChristoffelB{\nu}{j}{s} \coD{\mu}{\tc{\Phi}{^i_s^k}}
			+\ChristoffelB{\nu}{s}{k} \coD{\mu}{\tc{\Phi}{^i_j^s}}] \,
		\V{g}_i\tp\V{g}^j\tp\V{g}_k \fullstop
	\label{eq:混合协变导数推导}
\end{align}
但是 $\coD{\mu}{\tc{\Phi}{^i_j^k}}$ 本身带有 4 个指标,因而
\begin{equation}
	\coD{\nu}{\qty\Big(\coD{\mu}{\tc{\Phi}{^i_j^k}})}
	\defeq \pdv{x^\nu} \qty\Big(\coD{\mu}{\tc{\Phi}{^i_j^k}})
		+\ChristoffelB{\nu}{s}{i} \coD{\mu}{\tc{\Phi}{^s_j^k}}
		-\ChristoffelB{\nu}{j}{s} \coD{\mu}{\tc{\Phi}{^i_s^k}}
		+\ChristoffelB{\nu}{s}{k} \coD{\mu}{\tc{\Phi}{^i_j^s}}
		-\ChristoffelB{\nu}{\mu}{s} \coD{s}{\tc{\Phi}{^i_j^k}}
	\fullstop
\end{equation}
代入 \eqref{eq:混合协变导数推导}~式,可得
\begin{equation}
	\pdv{\T{\Phi}}{x^\nu}{x^\mu}
	=\pdv{x^\nu} \qty\Big(\coD{\mu}{\tc{\Phi}{^i_j^k}} \,
		\V{g}_i\tp\V{g}^j\tp\V{g}_k)
	=\qty\Big(\coD{\nu}{\coD{\mu}{\tc{\Phi}{^i_j^k}}}
		+\ChristoffelB{\nu}{\mu}{s} \coD{s}{\tc{\Phi}{^i_j^k}}) \,
		\V{g}_i\tp\V{g}^j\tp\V{g}_k \fullstop
\end{equation}
同理,
\begin{equation}
	\pdv{\T{\Phi}}{x^\mu}{x^\nu}
	=\qty\Big(\coD{\mu}{\coD{\nu}{\tc{\Phi}{^i_j^k}}}
		+\ChristoffelB{\mu}{\nu}{s} \coD{s}{\tc{\Phi}{^i_j^k}}) \,
		\V{g}_i\tp\V{g}^j\tp\V{g}_k \fullstop
\end{equation}

根据 Christoffel 符号的性质,
\begin{equation}
	\ChristoffelB{\nu}{\mu}{s}=\ChristoffelB{\mu}{\nu}{s}
	\semicomma
\end{equation}
而按照\myPROBLEM{一般赋范线性空间上的微分学}, 
当张量场具有足够正则性时, 成立
\begin{equation}
	\pdv{\T{\Phi}}{x^\nu}{x^\mu}=\pdv{\T{\Phi}}{x^\mu}{x^\nu}
	\fullstop
\end{equation}
这样就可得到
\begin{equation}
	\coD{\nu}{\coD{\mu}{\tc{\Phi}{^i_j^k}}}
	=\coD{\mu}{\coD{\nu}{\tc{\Phi}{^i_j^k}}} \fullstop
\end{equation}
\end{myProof}

\section{场论恒等式(二)}
本节将给出微分形式张量场场论中的若干恒等式,以及它们的推演过程。

\subsection{微分算子}
在 \ref{subsec:左梯度与右梯度}~小节中,
我们已经定义了\emphA{左梯度}
\begin{mySubEq}
	\begin{align}
		\qty\big(\opGrad\tp\T{\Phi}) (\V{x})
			&\defeq \qty[\V{g}^\mu\pdv{x^\mu} (\V{x})]\tp\T{\Phi}(\V{x})
			\coloneq \V{g}^\mu(\V{x})\tp\pdv{\T{\Phi}}{x^\mu} (\V{x})
		\intertext{和\emphA{(右)梯度}}
		\qty\big(\T{\Phi}\tp\opGrad) (\V{x})
			&\defeq \T{\Phi}(\V{x})\tp\qty[\V{g}^\mu\pdv{x^\mu} (\V{x})]
			\coloneq \pdv{\T{\Phi}}{x^\mu} (\V{x})\tp\V{g}^\mu(\V{x})
			\fullstop
	\end{align}
\end{mySubEq}
如果 $\T{\Phi}(\V{x})$ 是 $r$ 阶张量,则左右梯度都是 $r+1$ 阶张量。
%
%\blankline

类似地,我们还可以定义\emphA{左散度}
\begin{mySubEq}
	\begin{align}
		\qty\big(\opGrad\cdot\T{\Phi}) (\V{x})
			&\defeq \qty[\V{g}^\mu\pdv{x^\mu} (\V{x})]\cdot\T{\Phi}(\V{x})
			\coloneq \V{g}^\mu(\V{x})\cdot\pdv{\T{\Phi}}{x^\mu} (\V{x})
		\intertext{和\emphA{右散度}}
		\qty\big(\T{\Phi}\cdot\opGrad) (\V{x})
			&\defeq \T{\Phi}(\V{x})\cdot\qty[\V{g}^\mu\pdv{x^\mu} (\V{x})]
			\coloneq \pdv{\T{\Phi}}{x^\mu} (\V{x})\cdot\V{g}^\mu(\V{x})
			\fullstop
	\end{align}
\end{mySubEq}
如果 $\T{\Phi}(\V{x})$ 是 $r$ 阶张量,则左右散度都是 $r-1$ 阶张量。
%
%\blankline

当然,如果底空间是 $\realR^3$,
即 $\T{\Phi}(\V{x})\in\Tensors[\realR^3]{r}$,
还不能忘了定义\emphA{左旋度}
\begin{mySubEq}
	\begin{align}
		\qty\big(\opGrad\cp\T{\Phi}) (\V{x})
			&\defeq \qty[\V{g}^\mu\pdv{x^\mu} (\V{x})]\cp\T{\Phi}(\V{x})
			\coloneq \V{g}^\mu(\V{x})\cp\pdv{\T{\Phi}}{x^\mu} (\V{x})
		\intertext{和\emphA{右旋度}}
		\qty\big(\T{\Phi}\cp\opGrad) (\V{x})
			&\defeq \T{\Phi}(\V{x})\cp\qty[\V{g}^\mu\pdv{x^\mu} (\V{x})]
			\coloneq \pdv{\T{\Phi}}{x^\mu} (\V{x})\cp\V{g}^\mu(\V{x})
			\fullstop
	\end{align}
\end{mySubEq}
旋度不改变张量的阶数,即左右旋度仍属于 $\Tensors[\realR^3]{r}$。

\blankline

左右梯度、散度和梯度都是张量场中常用的微分算子。
向量微积分中的梯度、散度和梯度,其实就是一阶张量的特殊情况。

\subsection{推演举例}
首先是为人熟知的“梯度场无旋,旋度场无源”:
\begin{braceEq*}
	{\forall\,\T{\Phi}\in\Tensors[\realR^3]{r},\quad\text{有\ }}
	&\opGrad \cp \qty\big(\opGrad\tp\T{\Phi})=\T{0}
		\in\Tensors[\realR^3]{r+1} \comma \\
	&\opGrad \cdot \qty\big(\opGrad\cp\T{\Phi})=\T{0}
		\in\Tensors[\realR^3]{r-1} \fullstop
\end{braceEq*}

\begin{myProof}
不失一般性,我们设 $\T{\Phi}$ 是一个三阶张量。
代入上一小节中梯度和旋度的定义,可有
\begin{align}
	\opGrad \cp \qty\big(\opGrad\tp\T{\Phi})
	&=\qty(\V{g}^\nu\pdv{x^\nu})
		\cp \qty(\V{g}^\mu\tp\pdv{\T{\Phi}}{x^\mu}) \notag \\
	&=\qty(\V{g}^\nu\pdv{x^\nu})
		\cp \qty\Big(\coD{\mu}{\tc{\Phi}{^i_j^k}}
			\V{g}^\mu\tp\V{g}_i\tp\V{g}^j\tp\V{g}_k) \notag \\
	&=\V{g}^\nu\cp\pdv{x^\nu}
		\qty\Big(\coD{\mu}{\tc{\Phi}{^i_j^k}}
			\V{g}^\mu\tp\V{g}_i\tp\V{g}^j\tp\V{g}_k) \notag
	\intertext{与 \eqref{eq:混合协变导数推导}~式不同,第二个括号中的 
		$\mu$、$i$、$j$、$k$ 都是\emphB{哑标},
		所以偏导数可以直接用协变导数表示,而不会出现多余的
		Christoffel 符号:}
	&=\V{g}^\nu\cp \qty[ \vphantom{\frac{0}{0}}
			\qty(\coD{\nu}{\coD{\mu}{\tc{\Phi}{^i_j^k}}}) \,
			\V{g}^\mu\tp\V{g}_i\tp\V{g}^j\tp\V{g}_k] \notag
	\intertext{按照叉乘的定义(见 \ref{sec:叉乘}~节),
		$\V{g}^\nu$ 将与构成简单张量的第一个基向量相乘,即}
	&=\qty(\coD{\nu}{\coD{\mu}{\tc{\Phi}{^i_j^k}}})
		\qty(\V{g}^\nu\cp\V{g}^\mu)
		\tp\V{g}_i\tp\V{g}^j\tp\V{g}_k \notag
	\intertext{利用 Levi-Civita 记号展开叉乘项,有}
	&=\LeviCivita{^{\mu\nu s}}
		\qty(\coD{\nu}{\coD{\mu}{\tc{\Phi}{^i_j^k}}}) \,
		\V{g}_s\tp\V{g}_i\tp\V{g}^j\tp\V{g}_k \fullstop
\end{align}
考虑交换哑标 $\mu$、$\nu$,结果必然保持不变。
但是 Eddington 张量关于指标 $\mu\nu$ 反对称\footnote{
	根据式~\eqref{eq:Levi-Civita记号的定义},
	Levi-Civita 记号由行列式定义,而行列式交换两列将改变符号。},
即 $\LeviCivita{^{\mu\nu s}}=-\LeviCivita{^{\nu\mu s}}$;
另一方面,混合协变导数却又满足
$\coD{\nu}{\coD{\mu}{\tc{\Phi}{^i_j^k}}}
=\coD{\mu}{\coD{\nu}{\tc{\Phi}{^i_j^k}}}$,
因此总的结果将变为其相反数。这样,结果只可能为零,即
\begin{equation}
	\opGrad \cp \qty\big(\opGrad\tp\T{\Phi})=\T{0} \fullstop
\end{equation}

同理,也可证明“旋度场无源”:
\begin{align}
	\opGrad \cdot \qty\big(\opGrad\cp\T{\Phi})
	&=\qty(\V{g}^\nu\pdv{x^\nu})
		\cdot \qty(\V{g}^\mu\cp\pdv{\T{\Phi}}{x^\mu}) \notag \\
	&=\qty(\V{g}^\nu\pdv{x^\nu})
		\cdot \qty[ \vphantom{\frac{0}{0}}
			\V{g}^\mu\cp \qty(\coD{\mu}{\tc{\Phi}{^i_j^k}} \,
				\V{g}_i\tp\V{g}^j\tp\V{g}_k)] \notag \\
	&=\V{g}^\nu \cdot \pdv{x^\nu} \qty[ \vphantom{\frac{0}{0}}
			\V{g}^\mu\cp \qty(\coD{\mu}{\tc{\Phi}{^i_j^k}} \,
				\V{g}_i\tp\V{g}^j\tp\V{g}_k)] \notag \\
	&=\V{g}^\nu \cdot \pdv{x^\nu} \qty[ \vphantom{\frac{0}{0}}
			\coD{\mu}{\tc{\Phi}{^i_j^k}} \,
			\qty(\V{g}^\mu\cp\V{g}_i)\tp\V{g}^j\tp\V{g}_k] \notag \\
	&=\V{g}^\nu \cdot \pdv{x^\nu} \qty( \vphantom{\frac{0}{0}}
			\coD{\mu}{\tc{\Phi}{^i_j^k} \LeviCivita{^\mu_i^s}} \,
			\V{g}_s\tp\V{g}^j\tp\V{g}_k) \notag
	\intertext{$\mu$、$s$、$i$、$j$、$k$ 全部是哑标,因此}
	&=\V{g}^\nu \cdot \qty[ \vphantom{\frac{0}{0}}
		\qty(\coD{\nu}{
			\coD{\mu}{\tc{\Phi}{^i_j^k} \LeviCivita{^\mu_i^s}}}) \,
		\V{g}_s\tp\V{g}^j\tp\V{g}_k] \notag
	\intertext{点乘之后出来 Kronecker δ:}
	&=\qty(\coD{\nu}{
			\coD{\mu}{\tc{\Phi}{^i_j^k} \LeviCivita{^\mu_i^s}}}) \,
		\KroneckerDelta{\nu}{s} \, \V{g}^j\tp\V{g}_k \notag \\
	&=\LeviCivita{^\mu_i^\nu} \qty(\coD{\nu}{
			\coD{\mu}{\tc{\Phi}{^i_j^k}}}) \,
		\V{g}^j\tp\V{g}_k \fullstop
\end{align}
交换哑标 $\mu$、$\nu$,仿上,便有
\begin{equation}
	\opGrad \cdot \qty\big(\opGrad\cp\T{\Phi})=\T{0} \fullstop
\end{equation}
\end{myProof}

\blankline

接下来回忆一下\emphB{向量场}旋度的复合
\begin{equation}
	\forall\,\V{A}\in\realR^3,\quad
	\opGrad\cp \qty(\opGrad\cp\V{A})
	=\opGrad\qty\big(\opGrad\cdot\V{A})-\opLap\V{A}
	\in\realR^3 \comma
\end{equation}
式中,$\opLap$ 称为\emphA{Laplace 算子},其定义为
\begin{equation}
	\opLap\V{A} \defeq
	\opGrad\cdot\qty\big(\opGrad\tp\V{A}) \fullstop
\end{equation}
推广到张量场上,我们有如下两式:
\begin{braceEq*}
	{\forall\,\T{\Phi}\in\Tensors[\realR^3]{r},\quad}
	\opGrad\cp\qty\big(\opGrad\cp\T{\Phi})
	&=\opGrad\tp\qty\big(\opGrad\cdot\T{\Phi})
		-\opLap\T{\Phi} \comma \label{eq:张量场旋度复合_1} \\
	\qty\big(\T{\Phi}\cp\opGrad)\cp\opGrad
	&=\qty\big(\T{\Phi}\cdot\opGrad)\tp\opGrad
		-\T{\Phi}\opLap \comma \label{eq:张量场旋度复合_2}
\end{braceEq*}
其中,Laplace 算子的定义为
\begin{braceEq}
	\opLap\T{\Phi} &\defeq
		\opGrad\cdot\qty\big(\opGrad\tp\T{\Phi}) \comma \\
	\T{\Phi}\opLap &\defeq
		\qty\big(\T{\Phi}\tp\opGrad)\cdot\opGrad \fullstop
\end{braceEq}

\begin{myProof}
同样,我们以三阶张量为例进行计算。
代入旋度的定义,有
\begin{align}
	\opGrad \cp \qty\big(\opGrad\cp\T{\Phi})
	&=\qty(\V{g}^\nu\pdv{x^\nu})
		\cp \qty(\V{g}^\mu\cp\pdv{\T{\Phi}}{x^\mu}) \notag \\
	&=\V{g}^\nu \cp \pdv{x^\nu} \qty[ \vphantom{\frac{0}{0}}
			\V{g}^\mu\cp \qty(\coD{\mu}{\tc{\Phi}{^i_j^k}} \,
				\V{g}_i\tp\V{g}^j\tp\V{g}_k)] \notag \\
	&=\V{g}^\nu \cp \pdv{x^\nu} \qty( \vphantom{\frac{0}{0}}
			\coD{\mu}{\tc{\Phi}{^i_j^k} \LeviCivita{^\mu_i^s}} \,
			\V{g}_s\tp\V{g}^j\tp\V{g}_k) \notag \\
	&=\V{g}^\nu \cp \qty[ \vphantom{\frac{0}{0}}
		\qty(\coD{\nu}{
			\coD{\mu}{\tc{\Phi}{^i_j^k} \LeviCivita{^\mu_i^s}}}) \,
		\V{g}_s\tp\V{g}^j\tp\V{g}_k] \notag \\
	&=\LeviCivita{^\mu_i^s}\LeviCivita{^\nu_s^t} \qty(\coD{\nu}{
			\coD{\mu}{\tc{\Phi}{^i_j^k}}}) \,
		\V{g}_t\tp\V{g}^j\tp\V{g}_k \notag
	\intertext{为了使用“\emphB{前前后后,里里外外}”之法则
		(见 \ref{subsec:两种度量的关系}~小节),
		需要交换第二个 Eddington 张量的指标 $s$ 挪到最后,
		不要忘了添上负号:}
	&=-\LeviCivita{^\mu_i^s}\LeviCivita{^\nu^t_s} \qty(\coD{\nu}{
			\coD{\mu}{\tc{\Phi}{^i_j^k}}}) \,
		\V{g}_t\tp\V{g}^j\tp\V{g}_k \notag
	\intertext{这样就可以顺利用上口诀:}
	&=-\qty(g^{\mu\nu}\KroneckerDelta{t}{i}
			-\KroneckerDelta{\nu}{i}g^{\mu t})
		\qty(\coD{\nu}{
			\coD{\mu}{\tc{\Phi}{^i_j^k}}}) \,
		\V{g}_t\tp\V{g}^j\tp\V{g}_k \fullstop
\end{align}
接下来,形式上引入\emphA{逆变导数}
\begin{equation}
	\ctrD{\mu}{} \coloneq g^{\mu\nu}\coD{\nu}{} \comma
\end{equation}
则上式可化为
\begin{align}
	\opGrad \cp \qty\big(\opGrad\cp\T{\Phi})
	&=\KroneckerDelta{\nu}{i}g^{\mu t} \qty(\coD{\nu}{
			\coD{\mu}{\tc{\Phi}{^i_j^k}}}) \,
		\V{g}_t\tp\V{g}^j\tp\V{g}_k
		-g^{\mu\nu}\KroneckerDelta{t}{i} \qty(\coD{\nu}{
			\coD{\mu}{\tc{\Phi}{^i_j^k}}}) \,
		\V{g}_t\tp\V{g}^j\tp\V{g}_k \notag \\
	&=\qty(\coD{i}{\ctrD{t}{\tc{\Phi}{^i_j^k}}}) \,
			\V{g}_t\tp\V{g}^j\tp\V{g}_k
		-\qty(\ctrD{\mu}{\coD{\mu}{\tc{\Phi}{^i_j^k}}}) \,
			\V{g}_i\tp\V{g}^j\tp\V{g}_k \fullstop
	\label{eq:张量场旋度复合推导}
\end{align}
等式右边的第一项为
\begin{align}
	\opGrad \tp \qty\big(\opGrad\cdot\T{\Phi})
	&=\qty(\V{g}^\nu\pdv{x^\nu})
		\tp \qty(\V{g}^\mu\cdot\pdv{\T{\Phi}}{x^\mu}) \notag \\
	&=\V{g}^\nu \tp \pdv{x^\nu} \qty[ \vphantom{\frac{0}{0}}
			\V{g}^\mu \cdot \qty(\coD{\mu}{\tc{\Phi}{^i_j^k}} \,
				\V{g}_i\tp\V{g}^j\tp\V{g}_k)] \notag \\
	&=\V{g}^\nu \tp \pdv{x^\nu} \qty( \vphantom{\frac{0}{0}}
			\coD{\mu}{\tc{\Phi}{^i_j^k} \KroneckerDelta{\mu}{i}} \,
			\V{g}^j\tp\V{g}_k) \notag \\
	&=\V{g}^\nu \tp \qty[ \vphantom{\frac{0}{0}}
		\qty(\coD{\nu}{
			\coD{\mu}{\tc{\Phi}{^i_j^k} \KroneckerDelta{\mu}{i}}}) \,
		\V{g}^j\tp\V{g}_k] \notag \\
	&=\KroneckerDelta{\mu}{i} g^{\nu t} \qty(\coD{\nu}{
			\coD{\mu}{\tc{\Phi}{^i_j^k}}}) \,
		\V{g}_t\tp\V{g}^j\tp\V{g}_k \notag \\
	&=\qty(\ctrD{t}{\coD{i}{\tc{\Phi}{^i_j^k}}}) \,
			\V{g}_t\tp\V{g}^j\tp\V{g}_k \notag
	\intertext{根据式~\eqref{eq:混合协变导数定理},
		协变导数可以交换顺序;而逆变导数无非就是利用度量玩了下
		“指标升降游戏”,理应可以交换:}
	&=\qty(\coD{i}{\ctrD{t}{\tc{\Phi}{^i_j^k}}}) \,
		\V{g}_t\tp\V{g}^j\tp\V{g}_k \comma
\end{align}
而第二项为
\begin{align}
	\opLap\T{\Phi}
	\defeq \opGrad \cdot \qty\big(\opGrad\tp\T{\Phi})
	&=\qty(\V{g}^\nu\pdv{x^\nu})
		\cdot \qty(\V{g}^\mu\tp\pdv{\T{\Phi}}{x^\mu}) \notag \\
	&=\V{g}^\nu \cdot \pdv{x^\nu} \qty[ \vphantom{\frac{0}{0}}
			\V{g}^\mu \tp \qty(\coD{\mu}{\tc{\Phi}{^i_j^k}} \,
				\V{g}_i\tp\V{g}^j\tp\V{g}_k)] \notag \\
	&=\V{g}^\nu \cdot \pdv{x^\nu} \qty( \vphantom{\frac{0}{0}}
			\coD{\mu}{\tc{\Phi}{^i_j^k}} \,
			\V{g}^\mu\tp\V{g}_i\tp\V{g}^j\tp\V{g}_k) \notag \\
	&=\V{g}^\nu \cdot \qty[ \vphantom{\frac{0}{0}}
		\qty(\coD{\nu}{
			\coD{\mu}{\tc{\Phi}{^i_j^k}}}) \,
		\V{g}^\mu\tp\V{g}_i\tp\V{g}^j\tp\V{g}_k] \notag \\
	&=g^{\mu\nu} \qty(\coD{\nu}{
			\coD{\mu}{\tc{\Phi}{^i_j^k}}}) \,
		\V{g}_i\tp\V{g}^j\tp\V{g}_k \notag \\
	&=\qty(\ctrD{\mu}{\coD{\mu}{\tc{\Phi}{^i_j^k}}}) \,
			\V{g}_u\tp\V{g}^j\tp\V{g}_k \fullstop
\end{align}
将它们与 \eqref{eq:张量场旋度复合推导}~式比较,可得
\begin{equation}
	\opGrad\cp\qty\big(\opGrad\cp\T{\Phi})
	=\opGrad\tp\qty\big(\opGrad\cdot\T{\Phi}) 
		-\opLap\T{\Phi} \fullstop
\end{equation}

式~\eqref{eq:张量场旋度复合_2} 可以完全类似地证明,此处不再赘述。
\end{myProof}